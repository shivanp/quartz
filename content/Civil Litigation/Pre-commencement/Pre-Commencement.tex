% Options for packages loaded elsewhere
\PassOptionsToPackage{unicode}{hyperref}
\PassOptionsToPackage{hyphens}{url}
%
\documentclass[
]{article}
\usepackage{amsmath,amssymb}
\usepackage{lmodern}
\usepackage{iftex}
\ifPDFTeX
  \usepackage[T1]{fontenc}
  \usepackage[utf8]{inputenc}
  \usepackage{textcomp} % provide euro and other symbols
\else % if luatex or xetex
  \usepackage{unicode-math}
  \defaultfontfeatures{Scale=MatchLowercase}
  \defaultfontfeatures[\rmfamily]{Ligatures=TeX,Scale=1}
\fi
% Use upquote if available, for straight quotes in verbatim environments
\IfFileExists{upquote.sty}{\usepackage{upquote}}{}
\IfFileExists{microtype.sty}{% use microtype if available
  \usepackage[]{microtype}
  \UseMicrotypeSet[protrusion]{basicmath} % disable protrusion for tt fonts
}{}
\makeatletter
\@ifundefined{KOMAClassName}{% if non-KOMA class
  \IfFileExists{parskip.sty}{%
    \usepackage{parskip}
  }{% else
    \setlength{\parindent}{0pt}
    \setlength{\parskip}{6pt plus 2pt minus 1pt}}
}{% if KOMA class
  \KOMAoptions{parskip=half}}
\makeatother
\usepackage{xcolor}
\usepackage[margin=1in]{geometry}
\usepackage{color}
\usepackage{fancyvrb}
\newcommand{\VerbBar}{|}
\newcommand{\VERB}{\Verb[commandchars=\\\{\}]}
\DefineVerbatimEnvironment{Highlighting}{Verbatim}{commandchars=\\\{\}}
% Add ',fontsize=\small' for more characters per line
\newenvironment{Shaded}{}{}
\newcommand{\AlertTok}[1]{\textcolor[rgb]{1.00,0.00,0.00}{\textbf{#1}}}
\newcommand{\AnnotationTok}[1]{\textcolor[rgb]{0.38,0.63,0.69}{\textbf{\textit{#1}}}}
\newcommand{\AttributeTok}[1]{\textcolor[rgb]{0.49,0.56,0.16}{#1}}
\newcommand{\BaseNTok}[1]{\textcolor[rgb]{0.25,0.63,0.44}{#1}}
\newcommand{\BuiltInTok}[1]{#1}
\newcommand{\CharTok}[1]{\textcolor[rgb]{0.25,0.44,0.63}{#1}}
\newcommand{\CommentTok}[1]{\textcolor[rgb]{0.38,0.63,0.69}{\textit{#1}}}
\newcommand{\CommentVarTok}[1]{\textcolor[rgb]{0.38,0.63,0.69}{\textbf{\textit{#1}}}}
\newcommand{\ConstantTok}[1]{\textcolor[rgb]{0.53,0.00,0.00}{#1}}
\newcommand{\ControlFlowTok}[1]{\textcolor[rgb]{0.00,0.44,0.13}{\textbf{#1}}}
\newcommand{\DataTypeTok}[1]{\textcolor[rgb]{0.56,0.13,0.00}{#1}}
\newcommand{\DecValTok}[1]{\textcolor[rgb]{0.25,0.63,0.44}{#1}}
\newcommand{\DocumentationTok}[1]{\textcolor[rgb]{0.73,0.13,0.13}{\textit{#1}}}
\newcommand{\ErrorTok}[1]{\textcolor[rgb]{1.00,0.00,0.00}{\textbf{#1}}}
\newcommand{\ExtensionTok}[1]{#1}
\newcommand{\FloatTok}[1]{\textcolor[rgb]{0.25,0.63,0.44}{#1}}
\newcommand{\FunctionTok}[1]{\textcolor[rgb]{0.02,0.16,0.49}{#1}}
\newcommand{\ImportTok}[1]{#1}
\newcommand{\InformationTok}[1]{\textcolor[rgb]{0.38,0.63,0.69}{\textbf{\textit{#1}}}}
\newcommand{\KeywordTok}[1]{\textcolor[rgb]{0.00,0.44,0.13}{\textbf{#1}}}
\newcommand{\NormalTok}[1]{#1}
\newcommand{\OperatorTok}[1]{\textcolor[rgb]{0.40,0.40,0.40}{#1}}
\newcommand{\OtherTok}[1]{\textcolor[rgb]{0.00,0.44,0.13}{#1}}
\newcommand{\PreprocessorTok}[1]{\textcolor[rgb]{0.74,0.48,0.00}{#1}}
\newcommand{\RegionMarkerTok}[1]{#1}
\newcommand{\SpecialCharTok}[1]{\textcolor[rgb]{0.25,0.44,0.63}{#1}}
\newcommand{\SpecialStringTok}[1]{\textcolor[rgb]{0.73,0.40,0.53}{#1}}
\newcommand{\StringTok}[1]{\textcolor[rgb]{0.25,0.44,0.63}{#1}}
\newcommand{\VariableTok}[1]{\textcolor[rgb]{0.10,0.09,0.49}{#1}}
\newcommand{\VerbatimStringTok}[1]{\textcolor[rgb]{0.25,0.44,0.63}{#1}}
\newcommand{\WarningTok}[1]{\textcolor[rgb]{0.38,0.63,0.69}{\textbf{\textit{#1}}}}
\usepackage{longtable,booktabs,array}
\usepackage{calc} % for calculating minipage widths
% Correct order of tables after \paragraph or \subparagraph
\usepackage{etoolbox}
\makeatletter
\patchcmd\longtable{\par}{\if@noskipsec\mbox{}\fi\par}{}{}
\makeatother
% Allow footnotes in longtable head/foot
\IfFileExists{footnotehyper.sty}{\usepackage{footnotehyper}}{\usepackage{footnote}}
\makesavenoteenv{longtable}
\setlength{\emergencystretch}{3em} % prevent overfull lines
\providecommand{\tightlist}{%
  \setlength{\itemsep}{0pt}\setlength{\parskip}{0pt}}
\setcounter{secnumdepth}{-\maxdimen} % remove section numbering
\usepackage{xcolor}
\definecolor{aliceblue}{HTML}{F0F8FF}
\definecolor{antiquewhite}{HTML}{FAEBD7}
\definecolor{aqua}{HTML}{00FFFF}
\definecolor{aquamarine}{HTML}{7FFFD4}
\definecolor{azure}{HTML}{F0FFFF}
\definecolor{beige}{HTML}{F5F5DC}
\definecolor{bisque}{HTML}{FFE4C4}
\definecolor{black}{HTML}{000000}
\definecolor{blanchedalmond}{HTML}{FFEBCD}
\definecolor{blue}{HTML}{0000FF}
\definecolor{blueviolet}{HTML}{8A2BE2}
\definecolor{brown}{HTML}{A52A2A}
\definecolor{burlywood}{HTML}{DEB887}
\definecolor{cadetblue}{HTML}{5F9EA0}
\definecolor{chartreuse}{HTML}{7FFF00}
\definecolor{chocolate}{HTML}{D2691E}
\definecolor{coral}{HTML}{FF7F50}
\definecolor{cornflowerblue}{HTML}{6495ED}
\definecolor{cornsilk}{HTML}{FFF8DC}
\definecolor{crimson}{HTML}{DC143C}
\definecolor{cyan}{HTML}{00FFFF}
\definecolor{darkblue}{HTML}{00008B}
\definecolor{darkcyan}{HTML}{008B8B}
\definecolor{darkgoldenrod}{HTML}{B8860B}
\definecolor{darkgray}{HTML}{A9A9A9}
\definecolor{darkgreen}{HTML}{006400}
\definecolor{darkgrey}{HTML}{A9A9A9}
\definecolor{darkkhaki}{HTML}{BDB76B}
\definecolor{darkmagenta}{HTML}{8B008B}
\definecolor{darkolivegreen}{HTML}{556B2F}
\definecolor{darkorange}{HTML}{FF8C00}
\definecolor{darkorchid}{HTML}{9932CC}
\definecolor{darkred}{HTML}{8B0000}
\definecolor{darksalmon}{HTML}{E9967A}
\definecolor{darkseagreen}{HTML}{8FBC8F}
\definecolor{darkslateblue}{HTML}{483D8B}
\definecolor{darkslategray}{HTML}{2F4F4F}
\definecolor{darkslategrey}{HTML}{2F4F4F}
\definecolor{darkturquoise}{HTML}{00CED1}
\definecolor{darkviolet}{HTML}{9400D3}
\definecolor{deeppink}{HTML}{FF1493}
\definecolor{deepskyblue}{HTML}{00BFFF}
\definecolor{dimgray}{HTML}{696969}
\definecolor{dimgrey}{HTML}{696969}
\definecolor{dodgerblue}{HTML}{1E90FF}
\definecolor{firebrick}{HTML}{B22222}
\definecolor{floralwhite}{HTML}{FFFAF0}
\definecolor{forestgreen}{HTML}{228B22}
\definecolor{fuchsia}{HTML}{FF00FF}
\definecolor{gainsboro}{HTML}{DCDCDC}
\definecolor{ghostwhite}{HTML}{F8F8FF}
\definecolor{gold}{HTML}{FFD700}
\definecolor{goldenrod}{HTML}{DAA520}
\definecolor{gray}{HTML}{808080}
\definecolor{green}{HTML}{008000}
\definecolor{greenyellow}{HTML}{ADFF2F}
\definecolor{grey}{HTML}{808080}
\definecolor{honeydew}{HTML}{F0FFF0}
\definecolor{hotpink}{HTML}{FF69B4}
\definecolor{indianred}{HTML}{CD5C5C}
\definecolor{indigo}{HTML}{4B0082}
\definecolor{ivory}{HTML}{FFFFF0}
\definecolor{khaki}{HTML}{F0E68C}
\definecolor{lavender}{HTML}{E6E6FA}
\definecolor{lavenderblush}{HTML}{FFF0F5}
\definecolor{lawngreen}{HTML}{7CFC00}
\definecolor{lemonchiffon}{HTML}{FFFACD}
\definecolor{lightblue}{HTML}{ADD8E6}
\definecolor{lightcoral}{HTML}{F08080}
\definecolor{lightcyan}{HTML}{E0FFFF}
\definecolor{lightgoldenrodyellow}{HTML}{FAFAD2}
\definecolor{lightgray}{HTML}{D3D3D3}
\definecolor{lightgreen}{HTML}{90EE90}
\definecolor{lightgrey}{HTML}{D3D3D3}
\definecolor{lightpink}{HTML}{FFB6C1}
\definecolor{lightsalmon}{HTML}{FFA07A}
\definecolor{lightseagreen}{HTML}{20B2AA}
\definecolor{lightskyblue}{HTML}{87CEFA}
\definecolor{lightslategray}{HTML}{778899}
\definecolor{lightslategrey}{HTML}{778899}
\definecolor{lightsteelblue}{HTML}{B0C4DE}
\definecolor{lightyellow}{HTML}{FFFFE0}
\definecolor{lime}{HTML}{00FF00}
\definecolor{limegreen}{HTML}{32CD32}
\definecolor{linen}{HTML}{FAF0E6}
\definecolor{magenta}{HTML}{FF00FF}
\definecolor{maroon}{HTML}{800000}
\definecolor{mediumaquamarine}{HTML}{66CDAA}
\definecolor{mediumblue}{HTML}{0000CD}
\definecolor{mediumorchid}{HTML}{BA55D3}
\definecolor{mediumpurple}{HTML}{9370DB}
\definecolor{mediumseagreen}{HTML}{3CB371}
\definecolor{mediumslateblue}{HTML}{7B68EE}
\definecolor{mediumspringgreen}{HTML}{00FA9A}
\definecolor{mediumturquoise}{HTML}{48D1CC}
\definecolor{mediumvioletred}{HTML}{C71585}
\definecolor{midnightblue}{HTML}{191970}
\definecolor{mintcream}{HTML}{F5FFFA}
\definecolor{mistyrose}{HTML}{FFE4E1}
\definecolor{moccasin}{HTML}{FFE4B5}
\definecolor{navajowhite}{HTML}{FFDEAD}
\definecolor{navy}{HTML}{000080}
\definecolor{oldlace}{HTML}{FDF5E6}
\definecolor{olive}{HTML}{808000}
\definecolor{olivedrab}{HTML}{6B8E23}
\definecolor{orange}{HTML}{FFA500}
\definecolor{orangered}{HTML}{FF4500}
\definecolor{orchid}{HTML}{DA70D6}
\definecolor{palegoldenrod}{HTML}{EEE8AA}
\definecolor{palegreen}{HTML}{98FB98}
\definecolor{paleturquoise}{HTML}{AFEEEE}
\definecolor{palevioletred}{HTML}{DB7093}
\definecolor{papayawhip}{HTML}{FFEFD5}
\definecolor{peachpuff}{HTML}{FFDAB9}
\definecolor{peru}{HTML}{CD853F}
\definecolor{pink}{HTML}{FFC0CB}
\definecolor{plum}{HTML}{DDA0DD}
\definecolor{powderblue}{HTML}{B0E0E6}
\definecolor{purple}{HTML}{800080}
\definecolor{red}{HTML}{FF0000}
\definecolor{rosybrown}{HTML}{BC8F8F}
\definecolor{royalblue}{HTML}{4169E1}
\definecolor{saddlebrown}{HTML}{8B4513}
\definecolor{salmon}{HTML}{FA8072}
\definecolor{sandybrown}{HTML}{F4A460}
\definecolor{seagreen}{HTML}{2E8B57}
\definecolor{seashell}{HTML}{FFF5EE}
\definecolor{sienna}{HTML}{A0522D}
\definecolor{silver}{HTML}{C0C0C0}
\definecolor{skyblue}{HTML}{87CEEB}
\definecolor{slateblue}{HTML}{6A5ACD}
\definecolor{slategray}{HTML}{708090}
\definecolor{slategrey}{HTML}{708090}
\definecolor{snow}{HTML}{FFFAFA}
\definecolor{springgreen}{HTML}{00FF7F}
\definecolor{steelblue}{HTML}{4682B4}
\definecolor{tan}{HTML}{D2B48C}
\definecolor{teal}{HTML}{008080}
\definecolor{thistle}{HTML}{D8BFD8}
\definecolor{tomato}{HTML}{FF6347}
\definecolor{turquoise}{HTML}{40E0D0}
\definecolor{violet}{HTML}{EE82EE}
\definecolor{wheat}{HTML}{F5DEB3}
\definecolor{white}{HTML}{FFFFFF}
\definecolor{whitesmoke}{HTML}{F5F5F5}
\definecolor{yellow}{HTML}{FFFF00}
\definecolor{yellowgreen}{HTML}{9ACD32}
\usepackage[most]{tcolorbox}

\usepackage{ifthen}
\provideboolean{admonitiontwoside}
\makeatletter%
\if@twoside%
\setboolean{admonitiontwoside}{true}
\else%
\setboolean{admonitiontwoside}{false}
\fi%
\makeatother%

\newenvironment{env-1be3f683-7ee7-4449-b5e2-32b9f7058bc7}
{
    \savenotes\tcolorbox[blanker,breakable,left=5pt,borderline west={2pt}{-4pt}{firebrick}]
}
{
    \endtcolorbox\spewnotes
}
                

\newenvironment{env-df4c9afc-0c9c-44b2-8fff-78e0836b938c}
{
    \savenotes\tcolorbox[blanker,breakable,left=5pt,borderline west={2pt}{-4pt}{blue}]
}
{
    \endtcolorbox\spewnotes
}
                

\newenvironment{env-29b7bea7-455a-403e-9de2-6bcc936e22b8}
{
    \savenotes\tcolorbox[blanker,breakable,left=5pt,borderline west={2pt}{-4pt}{green}]
}
{
    \endtcolorbox\spewnotes
}
                

\newenvironment{env-b404abbb-77a5-4e20-a4fc-39be2fe470e5}
{
    \savenotes\tcolorbox[blanker,breakable,left=5pt,borderline west={2pt}{-4pt}{aquamarine}]
}
{
    \endtcolorbox\spewnotes
}
                

\newenvironment{env-d1322b6f-e720-4039-acb9-657b01a72e1f}
{
    \savenotes\tcolorbox[blanker,breakable,left=5pt,borderline west={2pt}{-4pt}{orange}]
}
{
    \endtcolorbox\spewnotes
}
                

\newenvironment{env-04a5a820-7151-4242-8b9c-78fb2bfa21fc}
{
    \savenotes\tcolorbox[blanker,breakable,left=5pt,borderline west={2pt}{-4pt}{blue}]
}
{
    \endtcolorbox\spewnotes
}
                

\newenvironment{env-2cd3add5-9f53-45ce-948b-9662836fd753}
{
    \savenotes\tcolorbox[blanker,breakable,left=5pt,borderline west={2pt}{-4pt}{yellow}]
}
{
    \endtcolorbox\spewnotes
}
                

\newenvironment{env-c60d3340-0952-4389-8be4-3e6123c6a270}
{
    \savenotes\tcolorbox[blanker,breakable,left=5pt,borderline west={2pt}{-4pt}{darkred}]
}
{
    \endtcolorbox\spewnotes
}
                

\newenvironment{env-47c93edd-8052-435c-a49c-15534da8760e}
{
    \savenotes\tcolorbox[blanker,breakable,left=5pt,borderline west={2pt}{-4pt}{pink}]
}
{
    \endtcolorbox\spewnotes
}
                

\newenvironment{env-21b3be99-5b2c-43a8-a0c7-c16dc6e0848b}
{
    \savenotes\tcolorbox[blanker,breakable,left=5pt,borderline west={2pt}{-4pt}{cyan}]
}
{
    \endtcolorbox\spewnotes
}
                

\newenvironment{env-79d59d75-b01c-4159-a247-ef67554443b0}
{
    \savenotes\tcolorbox[blanker,breakable,left=5pt,borderline west={2pt}{-4pt}{cyan}]
}
{
    \endtcolorbox\spewnotes
}
                

\newenvironment{env-363ddab5-8977-4c04-ba88-58db13aeb1ef}
{
    \savenotes\tcolorbox[blanker,breakable,left=5pt,borderline west={2pt}{-4pt}{purple}]
}
{
    \endtcolorbox\spewnotes
}
                

\newenvironment{env-8439f77b-ddc7-4e7a-bd7e-d8818ada3795}
{
    \savenotes\tcolorbox[blanker,breakable,left=5pt,borderline west={2pt}{-4pt}{darksalmon}]
}
{
    \endtcolorbox\spewnotes
}
                

\newenvironment{env-664b3530-0d1f-43e2-9df9-b47c950b81a8}
{
    \savenotes\tcolorbox[blanker,breakable,left=5pt,borderline west={2pt}{-4pt}{gray}]
}
{
    \endtcolorbox\spewnotes
}
                
\ifLuaTeX
  \usepackage{selnolig}  % disable illegal ligatures
\fi
\IfFileExists{bookmark.sty}{\usepackage{bookmark}}{\usepackage{hyperref}}
\IfFileExists{xurl.sty}{\usepackage{xurl}}{} % add URL line breaks if available
\urlstyle{same} % disable monospaced font for URLs
\hypersetup{
  pdftitle={Introduction},
  hidelinks,
  pdfcreator={LaTeX via pandoc}}

\title{Introduction}
\author{}
\date{}

\begin{document}
\maketitle

{
\setcounter{tocdepth}{3}
\tableofcontents
}
\begin{Shaded}
\begin{Highlighting}[]

\end{Highlighting}
\end{Shaded}

\hypertarget{civil-litigation}{%
\section{Civil Litigation}\label{civil-litigation}}

\hypertarget{woolf-and-jackson-reforms}{%
\subsection{Woolf and Jackson Reforms}\label{woolf-and-jackson-reforms}}

The nature of civil litigation dramatically changed on 26/04/99, when
Civil Procedure Rules 1998 (CPR 1998) (SI 1998/3132) came into force.
This is the court's attempt to implement the `Woolf Reforms', set out in
Lord Woolf's report, Access to Justice.

The aim was to simplify the legal system and cut costs. Further changes
occurred in Apr 2013 following a review by Lord Justice Jackson.

\hypertarget{overriding-objective}{%
\subsubsection{Overriding Objective}\label{overriding-objective}}

A civil justice system that was just, fair and easily understood, as
well as being reasonable in cost and timescales. Woolf thought it
necessary to transfer control of litigation from the parties to the
court. The court then imposes strict timetables, backed by a system of
sanctions.

\begin{Shaded}
\begin{Highlighting}[]
\NormalTok{title: r 1.1 CPR 1998}
\NormalTok{(1) These Rules are a new procedural code with the overriding objective of enabling the court to deal with cases justly.}

\NormalTok{(2) Dealing with a case justly includes, so far as is practicable—}

\NormalTok{{-} (a) ensuring that the parties are on an equal footing;}

\NormalTok{{-} (b) saving expense;}

\NormalTok{{-} (c) dealing with the case in ways which are proportionate—}

\NormalTok{    {-} (i) to the amount of money involved;}
    
\NormalTok{    {-} (ii) to the importance of the case;}
    
\NormalTok{    {-} (iii) to the complexity of the issues; and}
    
\NormalTok{    {-} (iv) to the financial position of each party;}

\NormalTok{{-} (d) ensuring that it is dealt with expeditiously and fairly; and}

\NormalTok{{-} (e) allotting to it an appropriate share of the court’s resources, while taking into account the need to allot resources to other cases.}
\end{Highlighting}
\end{Shaded}

\hypertarget{unequal-representation}{%
\subsubsection{Unequal Representation}\label{unequal-representation}}

In {[}{[}Maltez v Lewis (1999) The Times, 4 May{]}{]}, the court held
that it was a fundamental right of citizens to be represented by counsel
or solicitors of their own choice. But the court would intervene subtly
to achieve a level playing field; e.g., by giving a small law firm more
time, having a larger firm prepare trial bundles. So the court would
ensure compliance with the overriding objective when representatives
were unequal.

\hypertarget{conditionality-of-procedures}{%
\paragraph{Conditionality of
Procedures}\label{conditionality-of-procedures}}

In addition, note that in his Final Report, Lord Woolf suggested that:

\begin{quote}
Where one of the parties is unable to afford a particular procedure, the
court, if it decides that that procedure is to be followed, should be
entitled to make its order conditional upon the other side meeting the
difference in costs of the weaker party, whatever the outcome.
\end{quote}

\hypertarget{applicability}{%
\paragraph{Applicability}\label{applicability}}

The overriding objective must be kept in mind at all times when
conducting civil litigation by the court:

\begin{Shaded}
\begin{Highlighting}[]
\NormalTok{title: r 1.2 CPR}
\NormalTok{The court must seek to give effect to the overriding objective when it—}
\NormalTok{{-} (a) exercises any power given to it by the Rules; or}
\NormalTok{{-} (b) interprets any rule.}
\end{Highlighting}
\end{Shaded}

and by the parties:

\begin{Shaded}
\begin{Highlighting}[]
\NormalTok{title: r 1.3 CPR}
\NormalTok{The parties are required to help the court to further the overriding objective. }
\end{Highlighting}
\end{Shaded}

\hypertarget{parties-duty}{%
\paragraph{Parties' Duty}\label{parties-duty}}

All other rules are designed to try to achieve the overriding objective.
Observe that r 1.3 imposes a positive duty on solicitors and their
clients to help the court further the overriding objective. This applies
to dealings (including correspondence) between legal representatives as
well as dealings with the court.

\begin{Shaded}
\begin{Highlighting}[]
\NormalTok{title: Does this mean a party to litigation owes a duty to the other party?}
\NormalTok{No {-} [[Woodward v Phoenix Healthcare Distribution Ltd [2018] EWHC 2152 (Ch)]]. HHJ Hodge: }
\NormalTok{\textgreater{} In my judgment, the culture introduced by the CPR does not require a solicitor who has in no way contributed to a mistake on the part of his opponent, or his opponent’s solicitors, to draw attention to that mistake. That is, in my judgment, not required by CPR 1.3; and it does not amount to ‘technical game playing\textquotesingle{}.}
\end{Highlighting}
\end{Shaded}

See also {[}{[}Hannigan v Hannigan {[}2000{]} 2 FCR 650{]}{]}.

\hypertarget{professional-conduct}{%
\paragraph{Professional Conduct}\label{professional-conduct}}

So each party is required to help the court further the overriding
objective, but neither has an obligation nor duty to their opponent.

A solicitor is also bound by the standards of professional conduct set
by the SRA -- see {[}{[}Professional Conduct and Regulation
1\#SRA{]}{]}.

\hypertarget{judicial-case-management}{%
\paragraph{Judicial Case Management}\label{judicial-case-management}}

Under CPR 1998, the court has a duty to manage cases and will determine
the pace of litigation. Governed by Rule 1.4

\begin{Shaded}
\begin{Highlighting}[]
\NormalTok{title: r 1.4 CPR}
\NormalTok{(1) The court must further the overriding objective by actively managing cases.}

\NormalTok{(2) Active case management includes —}

\NormalTok{{-} (a) encouraging the parties to co{-}operate with each other in the conduct of the proceedings;}

\NormalTok{{-} (b) identifying the issues at an early stage;}

\NormalTok{{-} (c) deciding promptly which issues need full investigation and trial and accordingly disposing summarily of the others;}

\NormalTok{{-} (d) deciding the order in which issues are to be resolved;}

\NormalTok{{-} (e) encouraging the parties to use an alternative dispute resolution(GL) procedure if the court considers that appropriate and facilitating the use of such procedure;}

\NormalTok{{-} (f) helping the parties to settle the whole or part of the case;}

\NormalTok{{-} (g) fixing timetables or otherwise controlling the progress of the case;}

\NormalTok{{-} (h) considering whether the likely benefits of taking a particular step justify the cost of taking it;}

\NormalTok{{-} (i) dealing with as many aspects of the case as it can on the same occasion;}

\NormalTok{{-} (j) dealing with the case without the parties needing to attend at court;}

\NormalTok{{-} (k) making use of technology; and}

\NormalTok{{-} (l) giving directions to ensure that the trial of a case proceeds quickly and efficiently.}
\end{Highlighting}
\end{Shaded}

\hypertarget{vulnerable-parties-or-witnesses}{%
\paragraph{Vulnerable Parties or
Witnesses}\label{vulnerable-parties-or-witnesses}}

Practice Direction 1A makes provision for how the court gives effect to
the Overriding Objective in relation to vulnerable parties or witnesses.

\begin{itemize}
\tightlist
\item
  Try to identify vulnerability at earliest possible stage
\item
  Court should consider ordering `ground rules' before a vulnerable
  witness gives evidence.
\end{itemize}

\hypertarget{rules}{%
\subsection{Rules}\label{rules}}

\hypertarget{scope}{%
\subsubsection{Scope}\label{scope}}

The CPR 1998 apply to all proceedings in the County Court, High Court
and the Civil Division of the Court of Appeal, except:

\begin{itemize}
\item
  \begin{enumerate}
  \def\labelenumi{(\alph{enumi})}
  \tightlist
  \item
    insolvency proceedings;
  \end{enumerate}
\item
  \begin{enumerate}
  \def\labelenumi{(\alph{enumi})}
  \setcounter{enumi}{1}
  \tightlist
  \item
    family proceedings;
  \end{enumerate}
\item
  \begin{enumerate}
  \def\labelenumi{(\alph{enumi})}
  \setcounter{enumi}{2}
  \tightlist
  \item
    adoption proceedings;
  \end{enumerate}
\item
  \begin{enumerate}
  \def\labelenumi{(\alph{enumi})}
  \setcounter{enumi}{3}
  \tightlist
  \item
    proceedings before the Court of Protection;
  \end{enumerate}
\item
  \begin{enumerate}
  \def\labelenumi{(\alph{enumi})}
  \setcounter{enumi}{4}
  \tightlist
  \item
    non-contentious probate proceedings;
  \end{enumerate}
\item
  \begin{enumerate}
  \def\labelenumi{(\alph{enumi})}
  \setcounter{enumi}{5}
  \tightlist
  \item
    proceedings where the High Court acts as a Prize Court (e.g.,
    Admiralty proceedings);
  \end{enumerate}
\item
  \begin{enumerate}
  \def\labelenumi{(\alph{enumi})}
  \setcounter{enumi}{6}
  \tightlist
  \item
    election petitions in the High Court.
  \end{enumerate}
\end{itemize}

\hypertarget{practice-directions}{%
\subsubsection{Practice Directions}\label{practice-directions}}

These flesh out the rules and supplement them.

\hypertarget{civil-claim-overview}{%
\subsection{Civil Claim Overview}\label{civil-claim-overview}}

5 stages of a civil claim:

{[}{[}5-stages-civil.png{]}{]}

\hypertarget{stage-1-pre-commencement}{%
\subsubsection{Stage 1:
Pre-commencement}\label{stage-1-pre-commencement}}

\begin{itemize}
\tightlist
\item
  Identify client objectives

  \begin{itemize}
  \tightlist
  \item
    Legal or otherwise
  \item
    Compensation? Apology?
  \end{itemize}
\item
  Identify prospective parties

  \begin{itemize}
  \tightlist
  \item
    General rule: ``all persons to be sued should be sued at the same
    time and in the same action'' -- {[}{[}Morris v Wentworth-Stanley
    {[}1999{]} QB 1004{]}{]}
  \item
    Consider whether each D is worth suing
  \end{itemize}
\item
  Evidence

  \begin{itemize}
  \tightlist
  \item
    Never delay taking a statement.
  \end{itemize}
\item
  Costs

  \begin{itemize}
  \tightlist
  \item
    Client needs to know from outset
  \end{itemize}
\item
  Check limitation period and jurisdiction (is it England and Wales? Are
  there alternatives?)
\item
  Dispute resolution

  \begin{itemize}
  \tightlist
  \item
    Ensure client weighs up many factors including costs, time and
    resources and effect on business.
  \end{itemize}
\item
  Pre-action protocols

  \begin{itemize}
  \tightlist
  \item
    Govern the steps parties should take before commencing a court case.
  \item
    Failure to follow a protocol step or its spirit will incur a
    sanction if litigation is commenced.
  \item
    Many protocols have been approved by MoJ, setting out how parties
    should behave.
  \item
    Where none apply, should follow Practice Direction on Pre-action
    Conduct and Protocols.
  \end{itemize}
\item
  Alternative Dispute Resolution (ADR)

  \begin{itemize}
  \tightlist
  \item
    Parties and solicitors encouraged to enter into discussions/
    negotiations before proceedings
  \item
    By r 1.4(2)(e), active case management by the court will include
    encouraging them to use an ADR procedure the court considers
    appropriate.
  \end{itemize}
\item
  Standard letter before claim

  \begin{itemize}
  \tightlist
  \item
    If the claim is one to which an approved protocol applies (e.g.,
    professional negligence),

    \begin{itemize}
    \tightlist
    \item
      Information to be included will be specified in the protocol
    \end{itemize}
  \item
    Else:

    \begin{itemize}
    \tightlist
    \item
      Enough information must be given so that the prospective D can
      commence investigations, or at least put a broad valuation on the
      claim.
    \item
      See
      \href{https://www.justice.gov.uk/courts/procedure-rules/civil/rules/pd_pre-action_conduct}{Practice
      Direction - Pre-action Conduct and Protocols}.
    \end{itemize}
  \item
    Letter of response

    \begin{itemize}
    \tightlist
    \item
      Prospective D should:

      \begin{itemize}
      \tightlist
      \item
        Acknowledge safe receipt of the letter of claim
      \item
        State whether or not liability is admitted
      \item
        If not admitted, say why
      \item
        If contributory negligence alleged, give details
      \item
        Question of ADR should be addressed.
      \end{itemize}
    \end{itemize}
  \end{itemize}
\end{itemize}

\hypertarget{stage-2-commencement-of-claim}{%
\subsubsection{Stage 2: Commencement of
Claim}\label{stage-2-commencement-of-claim}}

The client should appreciate:

\begin{itemize}
\tightlist
\item
  that the court will impose a strict timetable of steps that must be
  taken;
\item
  that any unfavourable documents will have to be shown to the other
  side;
\item
  that they might have to attest at trial;
\item
  that if they want to stop litigation at any point, they will have to
  pay the opponent's costs (unless otherwise negotiated).
\end{itemize}

\hypertarget{mechanics}{%
\paragraph{Mechanics}\label{mechanics}}

\begin{longtable}[]{@{}
  >{\raggedright\arraybackslash}p{(\columnwidth - 2\tabcolsep) * \real{0.1364}}
  >{\raggedright\arraybackslash}p{(\columnwidth - 2\tabcolsep) * \real{0.8636}}@{}}
\toprule()
\begin{minipage}[b]{\linewidth}\raggedright
Track
\end{minipage} & \begin{minipage}[b]{\linewidth}\raggedright
Details
\end{minipage} \\
\midrule()
\endhead
Small claims & Typically concern consumer disputes, parties not expected
to be represented. \\
Fast track & Usually legal representation, court tightly controls costs
and types of evidence. Single joint expert should be used by parties
where necessary, and trial must be conducted within one day. \\
\bottomrule()
\end{longtable}

\hypertarget{stage-3-interim-matters}{%
\paragraph{Stage 3: Interim Matters}\label{stage-3-interim-matters}}

Careful court management of a case:

\begin{itemize}
\tightlist
\item
  Directions given to the parties as to the steps that must be taken to
  prepare for trial.
\item
  Strict timetable impose as to when each step should be taken.

  \begin{itemize}
  \tightlist
  \item
    In small claims and fast tracks, these directions can be given
    without any court hearings
  \item
    For multi-track, parties usually meet with a judge at case
    management conference.
  \end{itemize}
\end{itemize}

The most common case management directions

are for:

\begin{itemize}
\item
  \begin{enumerate}
  \def\labelenumi{(\alph{enumi})}
  \tightlist
  \item
    standard disclosure (i.e., the parties list the documents in their
    possession that they intend to rely on, or which are adverse to
    their case, or support an opponent's case ; and
  \end{enumerate}
\item
  \begin{enumerate}
  \def\labelenumi{(\alph{enumi})}
  \setcounter{enumi}{1}
  \tightlist
  \item
    the exchange of evidence before trial that the parties intend to
    rely on (e.g., experts' reports and statements, known as `witness
    statements', of non-expert witnesses.
  \end{enumerate}
\end{itemize}

Parties may apply to the court for any specific orders that may be
required.

\hypertarget{stage-4-trial}{%
\paragraph{Stage 4: Trial}\label{stage-4-trial}}

\begin{itemize}
\tightlist
\item
  Small claims is v informal and at the direction of the judge
\item
  Fast track and multi-track: formal rules of evidence.
\item
  Judge decides on costs

  \begin{itemize}
  \tightlist
  \item
    General rule is that loser pays winner's costs
  \item
    If parties cannot agree on amount of costs, determined post-trial by
    a costs judge.
  \end{itemize}
\end{itemize}

\hypertarget{stage-5-post-trial}{%
\paragraph{Stage 5: Post-trial}\label{stage-5-post-trial}}

\begin{itemize}
\tightlist
\item
  A party may decide to appeal all or part of the trial judge's
  decision.
\item
  A party awarded costs/ damages will expect to be paid by a date set by
  the court.

  \begin{itemize}
  \tightlist
  \item
    If not paid, apply to court to enforce the judgment.
  \item
    Often involves court officials going to premises and taking
    belongings to be sold at public auction.
  \end{itemize}
\end{itemize}

\hypertarget{case-analysis}{%
\subsection{Case Analysis}\label{case-analysis}}

\hypertarget{causes-of-action}{%
\subsubsection{Causes of Action}\label{causes-of-action}}

\begin{Shaded}
\begin{Highlighting}[]
\NormalTok{title: Cause of action}
\NormalTok{A cause of action is the legal basis of a claim. e.g., breach of contract, negligence, negligenct misstatement, misrepresentation, nuisance... }
\end{Highlighting}
\end{Shaded}

\begin{Shaded}
\begin{Highlighting}[]
\NormalTok{title: Breach of contract}
\NormalTok{To succeed in a claim for breach of contract, need to establish:}
\NormalTok{{-} (a) contract (its formation – parties, date, written or oral, subject matter (goods, goods and}
\NormalTok{{-} services/materials and work, services), consideration);}
\NormalTok{{-} (b) terms relied on (express and/or implied);}
\NormalTok{{-} (c) breach of those terms;}
\NormalTok{{-} (d) factual consequences of the breach of those terms;}
\NormalTok{{-} (e) damage and loss.}
\end{Highlighting}
\end{Shaded}

Often, solicitors record their analysis in a table. Continually review
issues in dispute and how they will be proved.

\begin{Shaded}
\begin{Highlighting}[]
\NormalTok{Key case analysis points are:}
\NormalTok{1. Viability}
\NormalTok{2. Liability}
\NormalTok{3. Quantum}
\end{Highlighting}
\end{Shaded}

\hypertarget{first-interview-considerations}{%
\section{First Interview
Considerations}\label{first-interview-considerations}}

\hypertarget{purpose}{%
\subsection{Purpose}\label{purpose}}

Solicitor needs to:

\begin{itemize}
\tightlist
\item
  Identify client objectives
\item
  Explain issues and options
\item
  Agree next steps
\end{itemize}

\begin{Shaded}
\begin{Highlighting}[]
\NormalTok{The scope of the work to be done for the client is formally known as the solicitor\textquotesingle{}s "retainer".  }
\end{Highlighting}
\end{Shaded}

\hypertarget{professional-conduct-1}{%
\subsection{Professional Conduct}\label{professional-conduct-1}}

\hypertarget{duty-of-confidentiality}{%
\subsubsection{Duty of Confidentiality}\label{duty-of-confidentiality}}

\begin{Shaded}
\begin{Highlighting}[]
\NormalTok{title: Para 6.3 Code for Solicitors}
\NormalTok{A solicitor is under a duty to maintain the confidentiality of their client\textquotesingle{}s affairs except the client\textquotesingle{}s prior authority is obtained to disclose particular information, or if the solicitor is required or permitted by law to do so. }
\end{Highlighting}
\end{Shaded}

If a solicitor holds confidential information on (former) client A, they
should not risk breaching confidentiality by acting, or continuing to
act, for another client (B) on a matter where:

\begin{itemize}
\tightlist
\item
  that information might reasonably be expected to be material and
\item
  client B has an interest adverse to client A.
\end{itemize}

See {[}{[}Professional Conduct and Regulation 1\#Ch 12 LF
Confidentiality{]}{]}.

\hypertarget{conflict-of-interest}{%
\subsubsection{Conflict of Interest}\label{conflict-of-interest}}

A solicitor should not act for two or more clients where this would
cause a conflict of interest.

\begin{Shaded}
\begin{Highlighting}[]
\NormalTok{A conflict of interests exists if the}
\NormalTok{solicitor owes separate duties to act in the best interests of two or more clients in relation to the same or related matters, and those duties conflict or there is a significant risk that those duties may conflict.}
\end{Highlighting}
\end{Shaded}

See {[}{[}Professional Conduct and Regulation 1\#Ch 13 LF Conflicts of
Interest{]}{]}.

\hypertarget{money-laundering}{%
\subsubsection{Money Laundering}\label{money-laundering}}

See {[}{[}Professional Conduct and Regulation 2\#Ch 15 Money
Laundering{]}{]}.

\begin{Shaded}
\begin{Highlighting}[]
\NormalTok{title: A solicitor should ask: }
\NormalTok{Who is my client and am I authorised to act?}
\end{Highlighting}
\end{Shaded}

\hypertarget{officer-of-court}{%
\subsubsection{Officer of Court}\label{officer-of-court}}

The solicitor has an overriding duty not to mislead the court (para 1.4
Code for Solicitors). The solicitor must disclose all relevant legal
authorities to the court, even if these are not favourable to their
case.

A solicitor must be mindful of their core duties and abide by SRA
principles.

\hypertarget{funding}{%
\subsection{Funding}\label{funding}}

Discuss costs with the clients, including advising the client of
different types of funding available.

The solicitor should explain potential costs given potential steps of
litigation and agree on a ceiling figure and review dates. It is unusual
for there to be a fixed fee -- this can be the case for taking some
specific step (e.g., attending a hearing).

If a solicitor fixes their fee too low, tough. Still obliged to complete
work to the ordinary standard of care ({[}{[}Inventors Friend Ltd v
Leathes Prior (a firm) {[}2011{]} EWHC 711{]}{]}).

If you don't give clients regular cost information, can be bound by
original costs estimate ({[}{[}Reynolds v Stone Rowe Brewer {[}2008{]}
EWHC 497{]}{]}).

\hypertarget{types-of-costs}{%
\subsubsection{Types of Costs}\label{types-of-costs}}

The solicitor should explain the distinction between solicitor and
client costs, and the costs that may be awarded between parties in
litigation.

\begin{Shaded}
\begin{Highlighting}[]
\NormalTok{title: r 44.2(2)(a) CPR 1998}
\NormalTok{As a general rule the unsuccessful party in litigation will be ordered to pay the costs of the successful party.}
\end{Highlighting}
\end{Shaded}

If a client wins the case, they will still have to pay their own
solicitor's costs, and then they will receive their costs from the
opponent. But it is normal for the costs awarded to be less than the
actual legal cost paid.

Even if successful, the court has the power to reduce costs to reflect
any unreasonable conduct on the part of the successful party
({[}{[}Benyatov v Credit Suisse Securities (Europe) Ltd {[}2022{]} WL
00509179{]}{]}).

\hypertarget{contingency-fee-arrangements}{%
\subsubsection{Contingency Fee
Arrangements}\label{contingency-fee-arrangements}}

\begin{Shaded}
\begin{Highlighting}[]
\NormalTok{title: Contingency fee arrangement}
\NormalTok{The client only pays a fee if they are successful.}
\end{Highlighting}
\end{Shaded}

These used to be completely illegal -- solicitors could only charge
clients on a time basis. Currently, only two types are legal:

\begin{itemize}
\tightlist
\item
  Conditional fee arrangement
\item
  Damages-based agreement.
\end{itemize}

\hypertarget{conditional-fee-arrangements}{%
\subsubsection{Conditional Fee
Arrangements}\label{conditional-fee-arrangements}}

\begin{Shaded}
\begin{Highlighting}[]
\NormalTok{title: Conditional fee arrangement (CFA)}
\NormalTok{‘An agreement with a person providing advocacy or litigation services which provides for his fees and expenses, or any part of them, to be payable only in specified circumstances’ {-} s 58(2)(a) of the Courts and Legal Services Act 1990 (CLSA 1990)}
\end{Highlighting}
\end{Shaded}

(Circumstances are winning the case).

Note they can be `no win, no fee', but also `no win, less fee'. Any fee
payable is based on the solicitor's usual hourly charging rates, up to a
maximum of 100\% extra. So maximum you can charge is 200\% of normal
fee.

A CFA is enforceable only if it meets the requirements of ss 58 and 58A
of the Courts and Legal Services Act 1990. These provide that a CFA:

\begin{itemize}
\item
  \begin{enumerate}
  \def\labelenumi{(\alph{enumi})}
  \tightlist
  \item
    may be entered into in relation to any civil litigation matter,
    except family proceedings;
  \end{enumerate}
\item
  \begin{enumerate}
  \def\labelenumi{(\alph{enumi})}
  \setcounter{enumi}{1}
  \tightlist
  \item
    must be in writing; and
  \end{enumerate}
\item
  \begin{enumerate}
  \def\labelenumi{(\alph{enumi})}
  \setcounter{enumi}{2}
  \tightlist
  \item
    must state the percentage by which the amount of the fee that would
    be payable if it were not a CFA is to be increased (the success
    fee).
  \end{enumerate}
\end{itemize}

Where it is agreed that the solicitor should receive higher than normal
payment if the case is won, the success fee cannot exceed 100\% of the
solicitor's normal charges. This limit is set by the CFA Order 2013 (SI
2013/689), reg 3 (although note that different provisions apply to
personal injury claims).

The solicitor should explain the circumstances in which the client may
be liable for their own legal costs (and when the solicitor would seek
payment) and their right to an assessment of those costs. Also check if
the client has before-the-event legal insurance.

The CFA is usually worded, so the solicitors' firm has a right to take
enforcement action in the client's name.

\hypertarget{drafting-cfa}{%
\paragraph{Drafting CFA}\label{drafting-cfa}}

Draft carefully, including a clear definition of `win'. Does there need
to be a threshold level of damages obtained?

\hypertarget{success-fee}{%
\paragraph{Success Fee}\label{success-fee}}

If the opponent is ordered to pay costs, these cannot include the
success fee, which will be payable by the client. Make this crystal.

The solicitor is taking a risk here, so they should perform a thorough
risk assessment (likely damages, chances of client succeeding, likely
time taken etc.). This might take time and research, and the success fee
should not be set arbitrarily.

\begin{Shaded}
\begin{Highlighting}[]
\NormalTok{title: Can you accept a client suggested success fee which is exorbitant?}
\NormalTok{No:}
\NormalTok{{-} principle 2 {-} upholding public trust and confidence, }
\NormalTok{{-} principle 4 {-} acting with honesty, }
\NormalTok{{-} principle 5 {-} acting with integrity, }
\NormalTok{{-} principle 6 {-} acting in best interests of client,}
\NormalTok{{-} para 1.2 {-} solicitor must not abuse their position by taking unfair advantage of the client, }
\NormalTok{{-} para 8.7 {-} ensuring clients receive the best possible information about how their matter will be priced.}
\end{Highlighting}
\end{Shaded}

\hypertarget{liability-for-costs}{%
\paragraph{Liability for Costs}\label{liability-for-costs}}

If a CFA client loses their case, they will not usually have to pay
their own solicitor's fees, but will be liable for their opponent's
costs including disbursements. They will also have to fund disbursements
such as barrister and expert witness fees.

Can mitigate by purchasing after-the-event insurance (AEI). The
solicitor should discuss with the client whether this is appropriate.

\hypertarget{damages-based-agreement}{%
\paragraph{Damages-based Agreement}\label{damages-based-agreement}}

Key statute:

\begin{Shaded}
\begin{Highlighting}[]
\NormalTok{title: s 58AA(3)(a) CLSA 1990}
\NormalTok{An agreement between a person providing advocacy services, litigation services or claims management services and the recipient of those services which provides that—}
\NormalTok{{-} (i) the recipient is to make a payment to the person providing the services if the recipient obtains a specified financial benefit in connection with the matter in relation to which the services are provided, and}
\NormalTok{{-} (ii) the amount of that payment is to be determined by reference to the amount of the financial benefit obtained.}
\end{Highlighting}
\end{Shaded}

It means that if the client receives damages, the solicitor's fee is an
agreed percentage of those damages.

A DBA is only enforceable if it meets the requirements of s 58AA(4) of
the CLSA 1990. These are that the agreement:

\begin{itemize}
\item
  \begin{enumerate}
  \def\labelenumi{(\alph{enumi})}
  \tightlist
  \item
    must be in writing;
  \end{enumerate}
\item
  \begin{enumerate}
  \def\labelenumi{(\alph{enumi})}
  \setcounter{enumi}{1}
  \tightlist
  \item
    must not provide for a payment above a prescribed amount or for a
    payment above an amount calculated in a prescribed manner;
  \end{enumerate}
\item
  \begin{enumerate}
  \def\labelenumi{(\alph{enumi})}
  \setcounter{enumi}{2}
  \tightlist
  \item
    must comply with such other requirements as to its terms and
    conditions as are prescribed; and
  \end{enumerate}
\item
  \begin{enumerate}
  \def\labelenumi{(\alph{enumi})}
  \setcounter{enumi}{3}
  \tightlist
  \item
    must be made only after the person providing services under the
    agreement has provided prescribed information.
  \end{enumerate}
\end{itemize}

As per Damages-Based Agreements Regulations 2013 (SI 2013/609):

\begin{itemize}
\tightlist
\item
  the DBA must not provide for a payment above an amount which,
  including VAT, is equal to 50\% of the sums ultimately recovered by
  the client.
\item
  This cap is inclusive of counsel fees, but not disbursements.
\item
  Cap does not apply to any appeal proceedings.
\end{itemize}

Regulation 4 provides that the client cannot be required to pay an
amount other than the agreed fee net of the following:

\begin{itemize}
\item
  \begin{enumerate}
  \def\labelenumi{(\roman{enumi})}
  \tightlist
  \item
    any costs (including fixed costs) and counsel's fees, that have been
    paid or are payable by another party to the proceedings by agreement
    or order; and
  \end{enumerate}
\item
  \begin{enumerate}
  \def\labelenumi{(\roman{enumi})}
  \setcounter{enumi}{1}
  \tightlist
  \item
    any expenses incurred by the solicitor after accounting for any
    amount which has been paid or is payable by another party to the
    proceedings by agreement or order. Expenses in this context would
    include disbursements such as experts' fees and court fees.
  \end{enumerate}
\end{itemize}

The terms and conditions of a DBA must specify:

\begin{itemize}
\item
  \begin{enumerate}
  \def\labelenumi{(\roman{enumi})}
  \tightlist
  \item
    the claim or proceedings or parts of them to which the agreement
    relates;
  \end{enumerate}
\item
  \begin{enumerate}
  \def\labelenumi{(\roman{enumi})}
  \setcounter{enumi}{1}
  \tightlist
  \item
    the circumstances in which the representative's payment, expenses
    and costs, or part of them, are payable; and
  \end{enumerate}
\item
  \begin{enumerate}
  \def\labelenumi{(\roman{enumi})}
  \setcounter{enumi}{2}
  \tightlist
  \item
    the reason for setting the amount of the payment at the level
    agreed.
  \end{enumerate}
\end{itemize}

\begin{Shaded}
\begin{Highlighting}[]
\NormalTok{title: Is a DBA valid if it contains a clause which provides that the client must pay the solicitors’ normal fees and disbursements if they terminate the retainer prematurely?}
\NormalTok{Yes, see [[Zuberi v Lexlaw Ltd [2021] EWCA Civ 16]]. The case is also authority for the use of so{-}called ‘hybrid DBAs’. For example, a law firm may receive concurrent funding via both a DBA and some other form of retainer; this might consist of the DBA in the event of the claim’s success and discounted hourly rate fees in the event of the claim’s failure. Alternatively, a DBA might comprise one or other of the methods of funding for different stages of the legal proceedings.}
\end{Highlighting}
\end{Shaded}

\begin{Shaded}
\begin{Highlighting}[]
\NormalTok{Slightly different provisions for the DBA for personal injury claims/ employment matters.}
\end{Highlighting}
\end{Shaded}

\hypertarget{insurance}{%
\paragraph{Insurance}\label{insurance}}

The solicitor should check if the client has before-the-event (BEI)
insurance, which might fund the litigation. This is commonly part of
household/ motor insurance.

Discuss with the client whether litigation is covered on existing
policy, or if AEI should be purchased. Often a very good option. A
solicitor who fails to discuss the possibility of such insurance may be
negligent and in breach of professional insurance.

\hypertarget{third-party-funding}{%
\paragraph{Third Party Funding}\label{third-party-funding}}

Trade union or organisational funding available? Historically,
commercial funding of litigation has been unlawful, but in recent years
has become accepted.

\hypertarget{public-funding}{%
\paragraph{Public Funding}\label{public-funding}}

There are \emph{very} limited circumstances in which public funding
(legal aid) is available for civil litigation. But this usually doesn't
apply (budget cuts cri).

\begin{itemize}
\tightlist
\item
  Administered by the Legal Aid Agency (LAA).
\item
  Usually not available for cases that could be financed by a CFA.
\item
  Claims in negligence for personal injury, death or damage to property
  are excluded, as are claims brought or defended by sole traders.
\item
  Only available if financially eligible and a cost-benefit analysis
  will be carried out.
\item
  A contribution from income could be required towards costs. This is
  payable monthly for as long as the claim is funded by the LAA.
\end{itemize}

In {[}{[}David Truex, Solicitor (a firm) v Kitchin {[}2007{]} EWCA Civ
618{]}{]}, the Court of Appeal held that a solicitor must from the
outset of a case consider whether a client might be eligible for legal
aid.

\hypertarget{statutory-charge}{%
\subparagraph{Statutory Charge}\label{statutory-charge}}

Where a publicly-funded client recovers money from proceedings, they may
have to pay some or all of their legal costs to the LAA out of the money
recovered. Known as the statutory charge, and will apply to the extent
costs are not recovered from the opponent.

The same principle applies to property as well a money. A solicitor must
ensure that the client understands.

\hypertarget{case-analysis-1}{%
\subsection{Case Analysis}\label{case-analysis-1}}

Recall the need to assess viability, liability, and quantum.

\hypertarget{liability}{%
\subsubsection{Liability}\label{liability}}

\begin{itemize}
\tightlist
\item
  What is the cause of action?
\item
  Is there more than one?
\item
  Establish legal elements for the claim to succeed.
\end{itemize}

\hypertarget{limitation}{%
\subsubsection{Limitation}\label{limitation}}

A solicitor must ascertain when the limitation period began and when it
will expire. Continually review this.

\begin{Shaded}
\begin{Highlighting}[]
\NormalTok{title: What if claim forms are sent to the court very close to the expiration date?}
\NormalTok{Per [[St Helens Metropolitan BC v Barnes [2006] EWCA Civ 1372]], }
\NormalTok{\textgreater{} Proceedings are started when the court issues a claim form at the request of the claimant (see rule 7.2) but where the claim form as issued was received in the court office on a date earlier than the date on which it was issued by the court, the claim is ‘brought’ for the purposes of the Limitation Act 1980 and}
\NormalTok{\textgreater{} any other relevant statute on that earlier date.}
\end{Highlighting}
\end{Shaded}

The Limitation Act 1980 (LA 1980) (as amended) prescribes fixed periods
of time for issuing

various types of proceedings. This is important to a client because if
this period of time elapses without proceedings being issued, the case
becomes `statute-barred'. The claimant can still commence their claim,
but the defendant will have an impregnable defence. If the defendant
wishes to rely on this, it must be stated specifically in their defence.

\hypertarget{contractual-tortious-claims}{%
\subsubsection{Contractual/ Tortious
Claims}\label{contractual-tortious-claims}}

\begin{itemize}
\tightlist
\item
  The basic rule is that the claimant has six years from the date when
  the cause of action accrued to commence their proceedings.

  \begin{itemize}
  \tightlist
  \item
    In contract, the cause of action accrues as soon as the breach of
    contract occurs.
  \item
    With an anticipatory breach of contract, the cause of action accrues
    when the intention not to perform the contract is made clear, and
    not at the later date when performance was due to occur
    ({[}{[}Hochester v de la Tour (1853) 2 E \& B 678{]}{]})
  \end{itemize}
\item
  In tort, the cause of action accrues when the tort is committed.

  \begin{itemize}
  \tightlist
  \item
    In negligence, the cause of action accrues only when some damage
    occurs (this may be a later date than the breach of duty itself --
    think misrepresentation claims)
  \end{itemize}
\item
  The basic rule is modified for certain claims such as personal injury.
\end{itemize}

\hypertarget{latent-damage}{%
\subsubsection{Latent Damage}\label{latent-damage}}

In a non-personal injury claim based on negligence, where the damage is
latent at the date when the cause of action accrued, s 14A of the LA
1980 provides that the limitation period expires either:

\begin{itemize}
\item
  \begin{enumerate}
  \def\labelenumi{(\alph{enumi})}
  \tightlist
  \item
    six years from the date on which the cause of action accrued; or
  \end{enumerate}
\item
  \begin{enumerate}
  \def\labelenumi{(\alph{enumi})}
  \setcounter{enumi}{1}
  \tightlist
  \item
    three years from the date of knowledge of certain material facts
    about the damage, if this period expires after the period mentioned
    in (a).
  \end{enumerate}
\end{itemize}

Additionally, there is a long-stop limitation period of 15 years from
the date of the alleged breach of duty (LA 1980, s 14B). It may bar a
cause of action at a date earlier than the claimant's knowledge, or even
before a cause of action has accrued.

\hypertarget{person-under-disability}{%
\subsubsection{Person Under Disability}\label{person-under-disability}}

\begin{Shaded}
\begin{Highlighting}[]
\NormalTok{Either a child (under 18) or a protected party (a person of unsound mind under the meaning of the Mental Capacity Act 2005 who is incapable of managing and administering their property and affairs).}
\end{Highlighting}
\end{Shaded}

When the claimant is a person under disability when a right of action
accrues, the limitation

period does not begin to run until the claimant ceases to be under that
disability.

\hypertarget{contractual-limitation}{%
\subsubsection{Contractual Limitation}\label{contractual-limitation}}

Check if any limitation period is specified in the contract.

\begin{longtable}[]{@{}
  >{\raggedright\arraybackslash}p{(\columnwidth - 2\tabcolsep) * \real{0.5000}}
  >{\raggedright\arraybackslash}p{(\columnwidth - 2\tabcolsep) * \real{0.5000}}@{}}
\toprule()
\begin{minipage}[b]{\linewidth}\raggedright
Type of claim
\end{minipage} & \begin{minipage}[b]{\linewidth}\raggedright
Statutory limitation period
\end{minipage} \\
\midrule()
\endhead
Contract (excluding personal injury) & 6 years (LA 1980,s 5) \\
Tort (excluding personal injury and latent damage) & 6 years (LA 1980, s
2) \\
Latent damage & 6 years or 3 years from date of knowledge (LA 1980,s
14A) \\
\bottomrule()
\end{longtable}

\hypertarget{remedies}{%
\subsubsection{Remedies}\label{remedies}}

Different remedies available -- remember contract law. The most common
one is damages.

\begin{Shaded}
\begin{Highlighting}[]
\NormalTok{title: Quantum}
\NormalTok{How much D will pay to V if liability is proved (or vice versa if not).}
\end{Highlighting}
\end{Shaded}

\hypertarget{damages}{%
\subsubsection{Damages}\label{damages}}

Rule as to quantum depends on the type of claim being pursued.

\hypertarget{contract}{%
\subsubsection{Contract}\label{contract}}

A claim for damages arises when one party to the contract has failed to
perform an obligation under the contract. Purpose: to put the injured
party in the position they would have been in if the contract had been
performed properly.

Damages must not be too remote from the breach to be recoverable.

\hypertarget{tort}{%
\subsubsection{Tort}\label{tort}}

Injury, loss or damage is caused to the claimant or their property. Aim
is to put V, as far as possible, in the position they would have been in
if the damage had not occurred.

\hypertarget{duty-to-mitigate}{%
\subsubsection{Duty to Mitigate}\label{duty-to-mitigate}}

Remember claimants have a duty to try to mitigate their losses. In
{[}{[}Frost v Knight (1872) LR 7 Ex 111{]}{]}

the court observed that this duty means looking at what the claimant

\begin{quote}
`has done, or has had the means of doing, and, as a prudent man, ought
in reason to have done, whereby his loss has been, or would have been,
diminished'.
\end{quote}

\hypertarget{debt}{%
\subsubsection{Debt}\label{debt}}

A debt action is a particular type of contract claim. The claimant is
claiming a sum which D promised to pay under the contract. Note that in
debt actions, V has no duty to mitigate their loss.

\begin{Shaded}
\begin{Highlighting}[]
\NormalTok{If the buyer takes delivery for goods but fails to pay, the action is for debt.}
\end{Highlighting}
\end{Shaded}

\hypertarget{quantum}{%
\subsubsection{Quantum}\label{quantum}}

Think about the evidence you have to prove the loss. Figures or
estimates?

\hypertarget{viability-and-burden-of-proof}{%
\subsection{Viability and Burden of
Proof}\label{viability-and-burden-of-proof}}

\hypertarget{viability}{%
\subsubsection{Viability}\label{viability}}

Viability of a claim depends on a number of questions.

\begin{Shaded}
\begin{Highlighting}[]
\NormalTok{title: Considerations}
\NormalTok{{-} Who is the prospective defendant?}
\NormalTok{{-} Is there more than one possible defendant?}
\NormalTok{{-} Where is the defendant?}
\NormalTok{{-} Is the defendant solvent?}
\NormalTok{{-} Where are the defendant’s assets?}
\NormalTok{{-} What are those assets worth?}
\NormalTok{{-} Will the defendant be able to pay any judgment?}
\NormalTok{{-} What can the client afford to pay?}
\NormalTok{{-} Does the client have any suitable BEI?}
\NormalTok{{-} Is the case suitable for a CFA or DBA and/or AEI cover?}
\NormalTok{{-} Does the client qualify for public funding or require third{-}party funding?}
\NormalTok{{-} How much time and resources will the client have to commit to investigate and deal}
\NormalTok{with the case?}
\NormalTok{{-} Does a cost–benefit analysis suggest the desirability of a quicker and cheaper solution}
\NormalTok{than litigation can offer?}
\end{Highlighting}
\end{Shaded}

\hypertarget{defendants-and-status}{%
\subsubsection{Defendants and Status}\label{defendants-and-status}}

The case analysis must identify against whom each cause of action lies.
Recall the general rule that all persons to be sued should be sued at
the same time and in the same claim.

Must ensure that each of the prospective D's are sued in their correct
capacity (personally/ as a partnership/ as a director etc.).

\hypertarget{ds-solvency}{%
\paragraph{D's Solvency}\label{ds-solvency}}

\ldots{} check D is solvent. Do this through a company search/
bankruptcy search, Facebook stalking etc.

\hypertarget{whereabouts}{%
\paragraph{Whereabouts}\label{whereabouts}}

D must be traceable. An inquiry agent may be helpful.

\hypertarget{the-claim}{%
\paragraph{The Claim}\label{the-claim}}

Involves balancing the merits of the claim against the overall cost of
pursuing it and the prospects of a successful outcome. Tell the client
about the overriding objective and r 1.3 that parties must help the
court to further it.

\hypertarget{alternative-remedies}{%
\paragraph{Alternative Remedies}\label{alternative-remedies}}

Advise the client on, e.g., alternative dispute resolution.

\hypertarget{burden-of-proof}{%
\paragraph{Burden of Proof}\label{burden-of-proof}}

\begin{itemize}
\tightlist
\item
  Legal burden

  \begin{itemize}
  \tightlist
  \item
    The party asserting a fact must prove it unless admitted by their
    component.

    \begin{itemize}
    \tightlist
    \item
      e.g., claimant alleging breach of contract must prove a contract
      existed, that D broke terms, and loss suffered.
    \item
      D must prove any allegation of failure to mitigate loss or
      contributory negligence.
    \end{itemize}
  \end{itemize}
\item
  Balance of probabilities

  \begin{itemize}
  \tightlist
  \item
    Standard of proof.
  \end{itemize}
\end{itemize}

\hypertarget{interest}{%
\subsection{Interest}\label{interest}}

\hypertarget{specified-and-unspecified-claims}{%
\subsubsection{Specified and Unspecified
Claims}\label{specified-and-unspecified-claims}}

A claim for a fixed amount of money is a specified amount. A specified
claim is in the nature of a debt, i.e., a fixed amount of money due and
payable under and by virtue of a contract.

Here, the amount known will be known and fixed, or mathematically
determinable.

If the court will have to conduct an investigation to decide on the
amount of money payable, the claim is best seen as being for an
unspecified amount, even if the claimant puts some figures forward for
the amount claimed.

If mixed, classified as unspecified.

\hypertarget{entitlement-to-interest}{%
\subsubsection{Entitlement to Interest}\label{entitlement-to-interest}}

Pre-action, a prospective claimant can demand interest on a claim only
if entitled to it under

any contractual provision (including any provision implied by the
\href{https://www.legislation.gov.uk/ukpga/1998/20/contents}{Late
Payments of Commercial Debts (Interest) Act 1998}).

Where the remedy sought by the claimant is damages or the repayment of a
debt, the court may award interest on the sum outstanding.

\hypertarget{breach-of-contract-and-debt-claims}{%
\paragraph{Breach of Contract and Debt
Claims}\label{breach-of-contract-and-debt-claims}}

There are 3 alternative claims to interest:

\begin{enumerate}
\def\labelenumi{\arabic{enumi}.}
\tightlist
\item
  The contract may specify a rate of interest payable on any outstanding
  sum -- the rate negotiated by the parties.
\item
  May be possible to claim under the Late Payment of Commercial Debts
  (Interest) Act 1998.
\item
  In all other cases, the court has a discretion to award interest
  either under s 35A of the Senior Courts Act 1981 (SCA 1981) in respect
  of High Court cases, or under s 69 of the County Courts Act 1984 (CCA
  1984) in respect of County Court cases. The current rate of interest
  payable under either statute is 8\% pa in non-commercial cases and,
  generally, 1\% or 2\% pa over base rate in commercial cases.
\end{enumerate}

In a debt claim, interest must be claimed precisely. Give a lump sum for
accrued interest from breach of contract until date of issue of
proceedings, and a daily rate thereafter. In a debt claim, where s 35A
of the SCA 1981 or s 69 of the CCA 1984 applies, the convention is to
claim interest from and including the day after the last day payment was
due.

\hypertarget{late-payment-of-commercial-debts-interest-act-1998}{%
\paragraph{Late Payment of Commercial Debts (Interest) Act
1998}\label{late-payment-of-commercial-debts-interest-act-1998}}

This Act (as amended by the Late Payment of Commercial Debt Regulations
2013 (SI 2013/395) and the Late Payment of Commercial Debts (Amendment)
Regulations 2015 (SI 2015/ 1336)) gives a statutory right to interest on
commercial debts that are paid late if the contract itself does not
provide for interest in the event of late payment.

\hypertarget{scope-1}{%
\subparagraph{Scope}\label{scope-1}}

The term `commercial debt' includes debts arising from the supply of
goods and services. As the Act is only concerned with commercial debt,
it does not apply to unspecified claims or a specified amount owed by a
consumer.

\hypertarget{details}{%
\subparagraph{Details}\label{details}}

Interest under the Act can be claimed at a rate of 8\% pa above the Bank
of England's reference rate on the date the debt became due for payment.
The reference rate is the base rate applicable on 31 December and 30
June each year and will apply for the following six months.

The interest accrues from the expiry of any period of credit under the
contract. If the contract does not provide for any such period, interest
can be claimed from 30 days after the latest of:

\begin{itemize}
\item
  \begin{enumerate}
  \def\labelenumi{(\alph{enumi})}
  \tightlist
  \item
    delivery of the bill;
  \end{enumerate}
\item
  \begin{enumerate}
  \def\labelenumi{(\alph{enumi})}
  \setcounter{enumi}{1}
  \tightlist
  \item
    delivery of the goods;
  \end{enumerate}
\item
  \begin{enumerate}
  \def\labelenumi{(\alph{enumi})}
  \setcounter{enumi}{2}
  \tightlist
  \item
    performance of the service.
  \end{enumerate}
\end{itemize}

The parties may agree to extend the period from 30 days up to a maximum
of 60 days. Any further extension must not be grossly unfair to the
supplier.

In addition to the debt and interest, the Act also provides for payment
of a fixed sum of between £40 and £100 compensation for late payment,
the amount varying according to the size of the debt. In addition, the
supplier may also claim as compensation any `reasonable' costs of
recovering

the debt that exceed the fixed sum.

\hypertarget{tort-1}{%
\subparagraph{Tort}\label{tort-1}}

The court has a general discretion to award interest on damages in any
negligence claim. This power is derived from SCA 1981, s 35A in respect
of High Court claims, and from CCA 1984, s 69 in respect of County Court
claims. Generally speaking, if interest has been claimed properly, the
court will normally exercise its discretion to award interest for such
period as it considers appropriate.

\hypertarget{specifics}{%
\subparagraph{Specifics}\label{specifics}}

Though theoretically damages can be awarded from when the cause of
action first arose, in practice it is normally made from when the loss
is sustained. See {[}{[}Kaines (UK) Ltd v Osterreichische {[}1993{]} 2
Lloyd's Rep 1{]}{]}.

Awards of interest are compensatory in measure. Judges are a bit fluffy
about them and the circumstances of the claimant; not simply the cost of
finance at the time.

In {[}{[}Tate \& Lyle Food and Distribution Ltd v Greater London Council
{[}1982{]} 1 WLR 149{]}{]}, Forbes J said:

\begin{quote}
One looks, therefore, not at the profit which the defendant wrongly made
out of the money he withheld -- this would indeed involve a scrutiny of
the defendant's financial position -- but at the cost to the
{[}claimant{]} of being deprived of the money which he should have had.
I feel satisfied that in commercial cases, the interest is intended to
reflect the rate at which the {[}claimant{]} would have had to borrow
money to supply the place of that which was withheld.
\end{quote}

\begin{Shaded}
\begin{Highlighting}[]
\NormalTok{title: Commercial case}
\NormalTok{Generally, one where all parties are in business and the claim is based on the transaction of trade and commerce. }
\end{Highlighting}
\end{Shaded}

In non-commercial cases, the courts award interest at their discretion
under s 35A of the SCA 1981 or s 69 of the CCA 1984 at 8\% pa.
Traditionally, in commercial cases, the award of interest has been at
1\% or 2\% over base rate. However, in J{[}{[}aura v Ahmed {[}2002{]}
EWCA Civ 210{]}{]}, the Court of Appeal held that it is permissible for
a judge to depart from this conventional rate if it is necessary to
reflect the higher rate at which the claimant had to borrow, e.g., if
the claimant was a small businessperson who could only borrow from a
bank at 3\% pa over base rate.

Generally, a person of the claimant's general position will be
considered, but no regard will be given to the claimant's particular
attributes.

Generally, simple interest rather than compound interest is awarded
(unless there is a contractual term providing for this), or in
exceptional cases (see {[}{[}Sempra Metals Ltd v Inland Revenue
Commissioners {[}2007{]} UKHL 34{]}{]}).

{[}{[}Interest-flowchart.png{]}{]}

Key questions:

\begin{itemize}
\tightlist
\item
  Is the claim based on a written contract providing for interest?
\item
  Is the claim for the payment of a commercial debt?
\end{itemize}

Remember D might be in a position to make a claim that includes
interest.

\hypertarget{foreign-element}{%
\subsection{Foreign Element}\label{foreign-element}}

In international cases, consider:

\begin{itemize}
\tightlist
\item
  Which countries can the claim be brought in?
\item
  English courts can hear any proceedings if the claim form was served
  on D whilst they were present in England and Wales.
\item
  D could object to the proceedings continuing in England, on the basis
  that a court abroad would be the most appropriate forum.
\end{itemize}

\hypertarget{litigation-alternatives}{%
\subsection{Litigation Alternatives}\label{litigation-alternatives}}

\hypertarget{arbitration}{%
\subsubsection{Arbitration}\label{arbitration}}

\begin{Shaded}
\begin{Highlighting}[]
\NormalTok{title: Arbitration}
\NormalTok{Adjudication operating outside the normal court process, by which a third party reaches a decision binding on the parties in dispute. }
\end{Highlighting}
\end{Shaded}

Many business contracts specify using arbitration rather than
litigation.

Advantages:

\begin{itemize}
\tightlist
\item
  Speed
\item
  Less formal
\item
  Private
\item
  Often more practical than the solutions a court has power to order
\item
  Decisions binding on parties.
\end{itemize}

\begin{Shaded}
\begin{Highlighting}[]
\NormalTok{The winning party to an arbitration can apply to the High Court for permission to enforce the arbitration award as if it were a court judgment (Arbitration Act 1996, s 66).}
\end{Highlighting}
\end{Shaded}

Disadvantages:

\begin{itemize}
\tightlist
\item
  Certain remedies like injunctions not available
\item
  Can be less in depth than a court investigation
\item
  Not always cheaper.
\end{itemize}

\hypertarget{alternative-dispute-resolution}{%
\subsubsection{Alternative Dispute
Resolution}\label{alternative-dispute-resolution}}

Independent third parties helps parties reach a solution. The third
party will often suggest a solution to the parties but not impose one
(known as `non-determinative ADR'). But sometimes the third party will
impose a solution (known as `determinative ADR'). The decision to use
ADR is voluntary; the parties choose the process and either of them can
withdraw at any time before a settlement is reached.

\begin{Shaded}
\begin{Highlighting}[]
\NormalTok{title: ADR examples}
\NormalTok{{-} mediation}
\NormalTok{{-} expert appraisal}
\NormalTok{{-} expert determination}
\end{Highlighting}
\end{Shaded}

Under the CPR 1998, the courts actively encourage parties to use some
form of ADR.

\begin{Shaded}
\begin{Highlighting}[]
\NormalTok{title: Can a court order that parties must use an ADR method?}

\NormalTok{No, held the Court of Appeal in [[Halsey v Milton Keynes General NHS Trust [2004] EWCA Civ 576]]. }
\end{Highlighting}
\end{Shaded}

Distinguish this from Early Neutral Evaluation, which the court can
order parties to engage with since it is part of the court process
(unlike mediation etc.).

\hypertarget{trade-schemes}{%
\subsubsection{Trade Schemes}\label{trade-schemes}}

Some professional bodies and trade associations have their own dispute
resolution schemes

\hypertarget{negotiating-settlements}{%
\subsubsection{Negotiating Settlements}\label{negotiating-settlements}}

Negotiations should be commenced as soon as possible. Once reasonable
attempts at settling have been exhausted, there may be no option but to
go to court.

\hypertarget{insurance-1}{%
\subsubsection{Insurance}\label{insurance-1}}

Many D's are insured. The existence of insurers does not affect the
conduct of proceedings, though the majority of policies require the
insured to notify the insurer of any potential claim.

\(\exists\) circumstances where a judgement against an insured defendant
can be enforced against the insurers. Notice of the proceedings must be
served on the insurers either before or within seven days of commencing
proceedings to invoke these provisions.

\hypertarget{motor-insurance-bureau-mib}{%
\subsubsection{Motor Insurance Bureau
(MIB)}\label{motor-insurance-bureau-mib}}

Two schemes allowing the victims of uninsured/ untraced drivers to
recover losses. It is set up by insurance companies and financed by
them.

\hypertarget{criminal-injuries-compensation-authority-cica}{%
\subsubsection{Criminal Injuries Compensation Authority
(CICA)}\label{criminal-injuries-compensation-authority-cica}}

Body set up by the Government to provide the victims of criminal acts
with ex gratia compensation for any personal injuries sustained as a
result of those acts.

\hypertarget{criminal-compensation-order}{%
\subsubsection{Criminal Compensation
Order}\label{criminal-compensation-order}}

Criminal courts have powers to order compensation in respect of any
personal injury, loss or damage resulting from a criminal offence when
imposing a sentence at the end of criminal proceedings.

\hypertarget{early-action}{%
\section{Early Action}\label{early-action}}

\begin{itemize}
\tightlist
\item
  After the initial interview, a number of practical preliminary steps
  should be taken by the solicitor.
\item
  Bear in mind CPR 1998 pre-action protocols if appropriate.
\item
  If not, comply with the `spirit' of the protocols and the Practice
  Direction on the Pre-action Conduct and Protocols.
\end{itemize}

\hypertarget{writing-to-the-client}{%
\subsection{Writing to the Client}\label{writing-to-the-client}}

Advise client in writing. By now you should have identified the cause of
action, undertaken any necessary legal research and assessed the
available evidence. If a proof of evidence has been taken from the
client, this should be sent to the client for approval and signature. If
there are any `gaps' in the case -- factual issues in respect of which
there is no evidence or weak evidence -- you must explain this to the
client and outline the options.

Either include details of funding and costs, or put these in a separate
letter. Cost advice should include:

\begin{itemize}
\item
  \begin{enumerate}
  \def\labelenumi{(\alph{enumi})}
  \tightlist
  \item
    advising the client of the basis and terms of the firm's charges;
  \end{enumerate}
\item
  \begin{enumerate}
  \def\labelenumi{(\alph{enumi})}
  \setcounter{enumi}{1}
  \tightlist
  \item
    advising the client of likely payments to third parties (eg court
    fees, barristers' fees, experts' fees);
  \end{enumerate}
\item
  \begin{enumerate}
  \def\labelenumi{(\alph{enumi})}
  \setcounter{enumi}{2}
  \tightlist
  \item
    setting out how the client has agreed to pay; and
  \end{enumerate}
\item
  \begin{enumerate}
  \def\labelenumi{(\alph{enumi})}
  \setcounter{enumi}{3}
  \tightlist
  \item
    advising the client of their potential liability for another party's
    costs.
  \end{enumerate}
\end{itemize}

\hypertarget{interviewing-witnesses}{%
\subsection{Interviewing Witnesses}\label{interviewing-witnesses}}

\hypertarget{proof-of-evidence}{%
\subsubsection{Proof of Evidence}\label{proof-of-evidence}}

Take a proof of evidence from a witness asap. The solicitor may request
an interview with anyone who may have information about the case.
However, there is nothing that can be done if a witness absolutely
refuses to give a statement.

\hypertarget{professional-conduct-2}{%
\subsubsection{Professional Conduct}\label{professional-conduct-2}}

It is permissible for a solicitor to interview and take a proof of
evidence (statement) from a prospective witness where that witness has
already been interviewed by another party.

However, there is a risk that the solicitor will be exposed to the
allegation that they have improperly tampered with evidence. This can be
overcome by offering to conduct the interview in the presence of a
representative of the other party.

A client can through their solicitor pay reasonable expenses to a
witness and reasonable compensation for their time attending court.

\hypertarget{reluctant-witness}{%
\subsubsection{Reluctant Witness}\label{reluctant-witness}}

Important to persuade potential witnesses to give a statement if at all
possible -- otherwise at trial there is no guarantee they will say
anything useful.

\hypertarget{taking-proof-of-evidence}{%
\subsubsection{Taking Proof of
Evidence}\label{taking-proof-of-evidence}}

Interview may be at the solicitor's office or wherever the witness
prefers (e.g., their home).

\begin{itemize}
\tightlist
\item
  Take personal details
\item
  Note relationship to client
\item
  Ask open and closed questions
\item
  Want to record the witness's own words
\item
  Obtain relevant documents -- probe about this
\item
  Picture painting -- need the witness to tell a story which is clear.
\item
  Set witness statement out in numbered paragraphs.
\item
  Judge credibility of witness: are there any inconsistencies? Spot and
  clarify these now rather than have them come up in trial.
\item
  Pursue other lines of inquiry opened up by interview.
\item
  Be wary of witness being `too helpful' - better to spot weaknesses now
\item
  Summarise what you have grasped
\item
  Type up proof of evidence and send it to the witness for correction/
  approval.
\end{itemize}

\hypertarget{preserving-documents}{%
\subsection{Preserving Documents}\label{preserving-documents}}

Ask client to bring all relevant documents with them as soon as
possible.

\begin{itemize}
\tightlist
\item
  A solicitor is under an obligation, both as a matter of professional
  conduct and under the CPR 1998, to ensure that the client understands
  the rules relating to disclosure of documents.
\item
  Explain that documents includes digital recordings, laptops, etc.
\item
  Explain that the duty extends to documents that the client might
  previously have had in their physical possession, even if they have
  now been, say, lost, destroyed or given to someone else.
\end{itemize}

Read documents to ensure there are no nasty surprises (e.g., limitation
clauses in contracts).

\hypertarget{obtaining-expert-evidence}{%
\subsection{Obtaining Expert Evidence}\label{obtaining-expert-evidence}}

\hypertarget{instructing-expert}{%
\subsubsection{Instructing Expert}\label{instructing-expert}}

\begin{itemize}
\tightlist
\item
  Firm probably has contacts
\item
  If counsel are involved from an early stage, they may be able to
  recommend suitable experts.
\item
  Look through The Law Society, or Academy of Expert Witnesses.
\end{itemize}

Expert must:

\begin{itemize}
\tightlist
\item
  Have expertise
\item
  Be able to present a convincing and easily understood report
\item
  To perform well as a witness, particularly under cross-examination.
\end{itemize}

\hypertarget{paying-expert}{%
\subsubsection{Paying Expert}\label{paying-expert}}

Clear expert's fees with the clients.

\begin{Shaded}
\begin{Highlighting}[]
\NormalTok{title: Can an expert be instructed on a conditional or contingency fee basis?}
\NormalTok{No {-} para 88 Guidance for the Instruction of Experts in Civil Claims. Incompatible with an expert\textquotesingle{}s duty of independence and impartiality. }
\end{Highlighting}
\end{Shaded}

The use of expert evidence during proceedings requires permission from
the court. So the fee paid to the expert may not be recoverable from the
opponent even if the client is successful in litigation.

\hypertarget{expert-reports}{%
\subsubsection{Expert Reports}\label{expert-reports}}

Check the report for errors. Send to the client to do the same. Be open
to returning to the expert for clarification.

\hypertarget{opinion}{%
\subsubsection{Opinion}\label{opinion}}

An expert can give opinion evidence (while an ordinary witness cannot).

\begin{Shaded}
\begin{Highlighting}[]
\NormalTok{title: Section 3(1) Civil Evidence Act 1972}
\NormalTok{Where a person is called as a witness in any civil proceedings, his opinion on any relevant matter on which he is qualified to give expert evidence shall be admissible in evidence.}
\end{Highlighting}
\end{Shaded}

\hypertarget{restriction-on-use-of-expert-evidence}{%
\subsubsection{Restriction on Use of Expert
Evidence}\label{restriction-on-use-of-expert-evidence}}

CPR 1998 introduced significant restrictions on the use of expert
evidence.

Although a party to proceedings is free to obtain as much expert
evidence as they wish, the extent to which such evidence may be used in
court is strictly controlled. By r 35.1 of the CPR 1998, expert evidence
is restricted to that which is reasonably required to resolve the
proceedings. The court can therefore limit the number of expert
witnesses who can give evidence, or order that a single joint expert be
appointed, or restrict expert evidence to a written report rather than
oral evidence in court.

A solicitor advising a client on whether to obtain expert evidence
should always bear in mind that the costs of doing so will usually be
recoverable from the opponent (assuming the case is won) only if the
court gives permission for the expert evidence to be used.

\hypertarget{site-visits}{%
\subsection{Site Visits}\label{site-visits}}

May be needed for the purpose of taking photographs or making plans.
Though unlikely to be disputed, the person who took the photo/ made the
plan may have to give evidence, so should be someone who is not the
solicitor. For a formal plan, hire a surveyor. If an inspection will be
expensive, obtain prior authorisation from the client.

\hypertarget{instructing-counsel}{%
\subsection{Instructing Counsel}\label{instructing-counsel}}

\hypertarget{use-of-counsel}{%
\subsubsection{Use of Counsel}\label{use-of-counsel}}

It is not necessary to instruct a barrister in every case.

\begin{Shaded}
\begin{Highlighting}[]
\NormalTok{title:}
\NormalTok{As a highly trained lawyer, the solicitor should have confidence in their own knowledge and ability!!}
\end{Highlighting}
\end{Shaded}

Judicious use of counsel is sensible. If issues are difficult, sensible
to instruct counsel to advise on liability. Counsel's opinion on quantum
may b needed at an early stage if the case is not straightforward.

\hypertarget{method}{%
\subsubsection{Method}\label{method}}

Instructing counsel requires the preparation of a formal document
(called `Instructions to counsel'). It will bear the heading of the
claim (or proposed claim) and should contain a list of the enclosures
being forwarded to counsel. The enclosures will obviously vary with the
case but will typically include copies of the client's statement, any
other proofs of evidence, any existing statements of case, any experts'
reports, and any relevant correspondence.

The body of the instructions to counsel will identify the client and set
out briefly both sides of the case. Counsel can refer to the enclosures
for detail, but the instructions should contain sufficient information
to enable the barrister to identify the major issues. The solicitor
should indicate their own view of the case and draw counsel's attention
to those areas on which particular advice is required. The instructions
will end with a formal request to counsel to carry out the required
task.

The instructions must carry a back sheet endorsed with the title of the
claim, what the instructions are (eg, `Instructions to counsel to advise
on quantum'), counsel's name and chambers, and the solicitor's firm's
name, address and reference.

Sometimes, counsel may not be able to proceed without a conference (the
name given to a meeting with counsel) with the solicitor and the client.
If a written opinion is required, this should be made clear in the
instructions, but the costs of both will not be recoverable from the
other side unless the court thinks it was reasonable to seek a written
opinion.

Instructions are traditionally prepared using the third person. Some law
firms scrap this.

\hypertarget{professional-conduct-3}{%
\subsubsection{Professional Conduct}\label{professional-conduct-3}}

A solicitor can rely on counsel's advice, but should still exercise some
judgment -- if the advice is glaringly wrong, under a duty to reject it.
Where a solicitor relies on the advice of counsel and subsequently both
are sued by the client in negligence, the solicitor is likely to seek an
indemnity or a contribution from counsel (see,

for example, {[}{[}Percy v Merriman White {[}2021{]} EWHC 22
(Ch){]}{]}).

\hypertarget{pre-action-protocols}{%
\subsection{Pre-action Protocols}\label{pre-action-protocols}}

\begin{Shaded}
\begin{Highlighting}[]
\NormalTok{title: Protocols}
\NormalTok{There are approved protocols for debt claims, personal injury, clinical disputes, construction and engineering, judicial review, media and communications, disease and illness, package travel claims, possession claims by social landlords, possession claims for mortgage arrears, housing condition cases (England); Housing Disrepair Cases (Wales), low value personal injury claims in road traffic accidents, personal injury claims below the small claims limit in road traffic accidents, dilapidations (commercial property), low value employers’ and public liability claims, and professional negligence claims.}
\end{Highlighting}
\end{Shaded}

\hypertarget{aims}{%
\subsubsection{Aims}\label{aims}}

The main aim is efficiency! More specifically,

\begin{itemize}
\item
  \begin{enumerate}
  \def\labelenumi{(\alph{enumi})}
  \tightlist
  \item
    to initiate and increase pre-action contact between the parties;
  \end{enumerate}
\item
  \begin{enumerate}
  \def\labelenumi{(\alph{enumi})}
  \setcounter{enumi}{1}
  \tightlist
  \item
    to encourage better and earlier exchange of information;
  \end{enumerate}
\item
  \begin{enumerate}
  \def\labelenumi{(\alph{enumi})}
  \setcounter{enumi}{2}
  \tightlist
  \item
    to encourage better pre-action investigation by both sides;
  \end{enumerate}
\item
  \begin{enumerate}
  \def\labelenumi{(\alph{enumi})}
  \setcounter{enumi}{3}
  \tightlist
  \item
    to put the parties in a position where they may be able to settle
    cases fairly and early without litigation;
  \end{enumerate}
\item
  \begin{enumerate}
  \def\labelenumi{(\alph{enumi})}
  \setcounter{enumi}{4}
  \tightlist
  \item
    to enable proceedings to run to the court's timetable and
    efficiently, if litigation does become necessary.
  \end{enumerate}
\end{itemize}

\hypertarget{steps}{%
\subsubsection{Steps}\label{steps}}

The Practice Direction and protocols deal with matters such as
notification to the defendant of a possible claim as soon as possible,
the form of the letter before claim, disclosure of documents and the
instruction of experts, where relevant.

\begin{Shaded}
\begin{Highlighting}[]
\NormalTok{title: Para 3 Practice Direction}
\NormalTok{Before commencing proceedings, the court will expect parties to have exchanged sufficient information to:}
\NormalTok{{-} (a) understand each other’s position;}
\NormalTok{{-} (b) make decisions about how to proceed;}
\NormalTok{{-} (c) try to settle the issues without proceedings;}
\NormalTok{{-} (d) consider a form of Alternative Dispute Resolution (ADR) to assist with settlement;}
\NormalTok{{-} (e) support the efficient management of those proceedings; and}
\NormalTok{{-} (f) reduce the costs of resolving the dispute.}
\end{Highlighting}
\end{Shaded}

Para 8 of the Practice Direction states that the parties should consider
whether some form of ADR procedure would enable them to settle their
dispute without commencing proceedings. The courts take the view that
litigation should be a last resort, and that claims should not be issued
prematurely when a settlement is still actively being explored. Parties
are warned that if this provision is not followed, then the court must
have regard to such conduct when determining costs.

\hypertarget{sanctions-for-non-compliance}{%
\subsubsection{Sanctions for
Non-compliance}\label{sanctions-for-non-compliance}}

Where non-compliance has led to proceedings that might otherwise not
have been commenced, or has led to unnecessary costs being incurred, the
court may impose sanctions.

Includes:

\begin{itemize}
\tightlist
\item
  Party at fault pay some or all of opponent's costs,
\item
  Depriving a claimant who is at fault of some or all of the interest
  they may be awarded on damages they recover,
\item
  Requiring D who is at fault to pay interest on some or all of any
  damages that are subsequently awarded to the claimant.
\end{itemize}

The court should aim to place the innocent party in no worse a position
than they would have been in had the Practice Direction or approved
protocol been complied with (see {[}{[}Straker v Tudor Rose (a firm)
{[}2007{]} EWCA Civ 368{]}{]}). And if proceedings occur, the party
awarded costs should usually be able to recover reasonable pre-action
costs.

\hypertarget{pre-action-correspondence}{%
\subsection{Pre-action Correspondence}\label{pre-action-correspondence}}

\hypertarget{letter-before-claim}{%
\subsubsection{Letter Before Claim}\label{letter-before-claim}}

Where the solicitor is satisfied that the client has a valid claim, they
should advise the client and obtain instructions to send a letter to the
prospective defendant setting out full details of the claim. This is
called a `letter before claim'.

\begin{Shaded}
\begin{Highlighting}[]
\NormalTok{Called a "letter of claim" in professional negligence pre{-}action protocols and a "letter before claim" in PD of pre{-}action conduct. }
\end{Highlighting}
\end{Shaded}

\hypertarget{professional-conduct-4}{%
\subsubsection{Professional Conduct}\label{professional-conduct-4}}

The letter is normally addressed to the potential defendant in person,
or to their solicitor if they already have one. If the potential
defendant is likely to be insured, the solicitor should ask that the
letter is passed on to the insurers.

\hypertarget{practice-direction-guidance}{%
\subsubsection{Practice Direction
Guidance}\label{practice-direction-guidance}}

The Practice Direction provides at para 6(a) that the claimant's letter
before claim should give\\
concise details about the matter.

\begin{Shaded}
\begin{Highlighting}[]
\NormalTok{title: Suggested to include}
\NormalTok{1. state the claimant’s full name and address;}
\NormalTok{2. state the basis on which the claim is made (ie why the claimant says the defendant is liable);}
\NormalTok{3. provide a clear summary of the facts on which the claim is based;}
\NormalTok{4. state what the claimant wants from the defendant;}
\NormalTok{5. if financial loss is claimed, provide an explanation of how the amount has been calculated;}
\NormalTok{6. list the essential documents on which the claimant intends to rely;}
\NormalTok{7. set out the form of ADR (if any) that the claimant considers the most suitable and invite the defendant to agree to this;}
\NormalTok{8. state the date by which the claimant considers it reasonable for a full response to be provided by the defendant; and}
\NormalTok{9. identify and ask for copies of any relevant documents not in the claimant’s possession and which the claimant wishes to see.}
\end{Highlighting}
\end{Shaded}

Straightforward claim: give 14 days for acknowledgement of receipt and/
or full response. For more tricky stuff involving insurers, give a month
or even three.

Also include some helpful suggestions:

\begin{enumerate}
\def\labelenumi{\arabic{enumi}.}
\tightlist
\item
  refer the defendant to the Practice Direction and in particular draw
  attention to paragraph 16 concerning the court's powers to impose
  sanctions for failure to comply with the Practice Direction;
\item
  inform the defendant that ignoring the letter before claim may lead to
  the claimant starting proceedings and may increase the defendant's
  liability for costs;
\item
  warn the defendant of any claim for interest in proceedings that may
  be commenced; and
\item
  suggest that the defendant takes independent legal advice.
\end{enumerate}

{[}{[}pre-action-flowchart.png{]}{]}

{[}{[}prof-neg-flowchart.png{]}{]}

\hypertarget{letter-of-acknowledgement}{%
\subsubsection{Letter of
Acknowledgement}\label{letter-of-acknowledgement}}

Acknowledgement should state whether an insurer is or may be involved
and the date by which D/ insurer will provide a full written response.

\hypertarget{letter-of-response-under-practice-direction}{%
\subsubsection{Letter of Response Under Practice
Direction}\label{letter-of-response-under-practice-direction}}

Paragraph 6(b) of the Practice Direction provides that the defendant's
letter of response to the\\
letter of claim should include:

\begin{enumerate}
\def\labelenumi{\arabic{enumi}.}
\tightlist
\item
  confirmation as to whether the claim is accepted and, if it is not
  accepted, the reason why;
\item
  an explanation as to which facts and parts of the claim are disputed;
  and
\item
  whether the defendant is making a counterclaim and, if so, its
  details.
\end{enumerate}

When negotiating with the other side, try to resolve the matter without
needing to start proceedings. At the very least, it should be possible,
as para 12 of the Practice Direction points out, to establish what
issues remain outstanding so as to narrow the scope of any subsequent
court proceedings, and therefore limit potential costs.

\hypertarget{experts}{%
\subsubsection{Experts}\label{experts}}

Paragraph 7 of the Practice Direction reminds the parties that many
matters can and should\\
be resolved without the need for advice or evidence from an expert.

Where the parties go ahead and instruct their own expert or experts
pre-action, it will be for the court to determine later if any party can
rely on any particular expert's evidence and how expert evidence should
be given at a trial.

\hypertarget{false-statements}{%
\subsubsection{False Statements}\label{false-statements}}

Paragraph 2 of the Practice Direction reminds the parties that any
person who knowingly makes a false statement in a pre-action protocol
letter or other document prepared in anticipation of legal proceedings
may be subject to proceedings for contempt of court.

\hypertarget{pre-action-disclosure}{%
\subsection{Pre-action Disclosure}\label{pre-action-disclosure}}

An application for disclosure of documents prior to the start of
proceedings is permitted under s 33 of the SCA 1981, or s 52 of the CCA
1984. The application must be supported by evidence, and the procedure
is dealt with in r 31.16(3) of the CPR 1998.

The court may make an order for disclosure only where:

\begin{itemize}
\item
  \begin{enumerate}
  \def\labelenumi{(\alph{enumi})}
  \tightlist
  \item
    the respondent is likely to be a party to subsequent proceedings;
  \end{enumerate}
\item
  \begin{enumerate}
  \def\labelenumi{(\alph{enumi})}
  \setcounter{enumi}{1}
  \tightlist
  \item
    the applicant is also likely to be a party to the proceedings;
  \end{enumerate}
\item
  \begin{enumerate}
  \def\labelenumi{(\alph{enumi})}
  \setcounter{enumi}{2}
  \tightlist
  \item
    if proceedings had started, the respondent's duty by way of standard
    disclosure set out in rule 31.6, would extend to the documents or
    classes of documents of which the applicant seeks disclosure; and
  \end{enumerate}
\item
  \begin{enumerate}
  \def\labelenumi{(\alph{enumi})}
  \setcounter{enumi}{3}
  \tightlist
  \item
    disclosure before proceedings have started is desirable in order
    to--
  \end{enumerate}

  \begin{itemize}
  \item
    \begin{enumerate}
    \def\labelenumi{(\roman{enumi})}
    \tightlist
    \item
      dispose fairly of the anticipated proceedings;
    \end{enumerate}
  \item
    \begin{enumerate}
    \def\labelenumi{(\roman{enumi})}
    \setcounter{enumi}{1}
    \tightlist
    \item
      assist the dispute to be resolved without proceedings; or
    \end{enumerate}
  \item
    \begin{enumerate}
    \def\labelenumi{(\roman{enumi})}
    \setcounter{enumi}{2}
    \tightlist
    \item
      save costs.
    \end{enumerate}
  \end{itemize}
\end{itemize}

An order under this rule must specify the documents or class of
documents which the

respondent must disclose, and require them, when making such disclosure,
to specify any of those documents which they no longer have, or which
they claim the right or duty to withhold from inspection. The order may
also specify the time and place for disclosure and inspection to take
place.

Common examples are in personal injury litigation and significant
commercial cases. The court will not allow applications for pre-action
disclosure where there is no sufficient evidence that a claim exists and
the application is purely speculative (see {[}{[}Hunt v Caddick (Mill
Harbour) Ltd {[}2019{]} EWHC 2933{]}{]}).

\hypertarget{settlement}{%
\subsection{Settlement}\label{settlement}}

\hypertarget{negotiations-without-prejudice}{%
\subsubsection{Negotiations `without
Prejudice'}\label{negotiations-without-prejudice}}

Any negotiations that take place as a part of a genuine attempt to
settle a claim are impliedly

`without prejudice'. However, it is preferable to mark any
correspondence accordingly, or to clarify at the start of a
meeting/telephone negotiation that this is the basis on which you are
proceeding. If `without prejudice' negotiations take place, neither
party may rely upon anything said or written in the course of the
negotiations for the purpose of proving liability and/or quantum at
trial.

The rule exists to encourage litigants to reach a settlement. It means
that all negotiations which are genuinely aimed at a settlement are
excluded from being given in evidence.

\hypertarget{genuine-attempt}{%
\paragraph{Genuine Attempt}\label{genuine-attempt}}

Although as a matter of good practice the words `without prejudice'
should appear on this type of correspondence, the presence or absence of
the words is not conclusive. What is important is that the letter is a
genuine attempt to settle the case. If there is a dispute as to whether
or not a communication is protected in this way, the court can examine
the document (obviously in advance of the trial by someone other than
the trial judge) to see whether or not its purpose was to settle the
dispute.

The court must consider the circumstances of the communications from an
objective standpoint: {[}{[}Sang Kook Suh v Mace (UK) Ltd {[}2016{]}
EWCA Civ 4{]}{]}.

\hypertarget{post-settlement}{%
\paragraph{Post-settlement}\label{post-settlement}}

Once a settlement is concluded, any `without prejudice' correspondence
can be produced in court to show the terms agreed between the parties.
This might be necessary if, for example, a dispute arose as to
enforcement of an agreed settlement or the true terms of the agreement
reached (see {[}{[}Oceanbulk Shipping \& Trading SA v TMT Asia Ltd
{[}2010{]} UKSC 44{]}{]}).

\begin{Shaded}
\begin{Highlighting}[]
\NormalTok{If a party wishes to reserve the right to draw the trial judge’s attention to a without prejudice offer to settle a case on the question of costs, they should mark the offer ‘without prejudice save as to costs’}
\end{Highlighting}
\end{Shaded}

A without prejudice or without prejudice save as to costs offer of
settlement that has no time limit for acceptance can be accepted even
after the trial has started; such an offer does not lapse at the
commencement of the hearing (see {[}{[}MEF (A Protected Party, by his
Mother and Litigation Friend, FEM) v St George's Healthcare NHS Trust
{[}2020{]} EWHC 1300 (QB){]}{]}).

\hypertarget{pre-action-offers-under-part-36}{%
\paragraph{Pre-action Offers Under Part
36}\label{pre-action-offers-under-part-36}}

Parties are encouraged to negotiate and settle the claim. Part 36
formally recognises this and lends weight to such offers.

Any party can offer to settle a monetary claim for a specific sum, or on
express terms for any non-monetary claim.

\begin{itemize}
\tightlist
\item
  If litigation occurs and the claimant fails to obtain a judgment more
  advantageous than a defendant's Part 36 offer, the claimant will
  usually suffer severe financial penalties.
\item
  If a claimant makes a Part 36 offer and the judgment against the
  defendant is at least as advantageous to the claimant as the proposals
  contained in the claimant's Part 36 offer, the defendant will usually
  suffer severe financial penalties.
\end{itemize}

Take care when drafting Part 36 offers to ensure that the stated terms
are consistent with Part 36. Else, there is a risk that the court may
not apply Part 36 financial penalties. The court may take the offer into
account when exercising its general discretion with regard to costs (see
{[}{[}James v James {[}2018{]} EWHC 242 (Ch){]}{]}).

\hypertarget{researching-the-law}{%
\paragraph{Researching the Law}\label{researching-the-law}}

This will often not be necessary, but do double check current
authorities. For procedural points, check CPR 1998, Practice Directions
and relevant case law.

\hypertarget{cost-benefit-analysis}{%
\paragraph{Cost-benefit Analysis}\label{cost-benefit-analysis}}

Discuss potential outcomes of any legal case with the client, including
the risk of having to pay opponent's costs. Court is hella expensive --
usually tens of thousands for cases that make it to the senior courts.

\hypertarget{alternative-dispute-resolution-1}{%
\section{Alternative Dispute
Resolution}\label{alternative-dispute-resolution-1}}

\hypertarget{nature-of-adr}{%
\subsection{Nature of ADR}\label{nature-of-adr}}

A means of resolving disputes using an independent third party.

\begin{longtable}[]{@{}ll@{}}
\toprule()
Type of ADR & Description \\
\midrule()
\endhead
Determinative ADR & Can impose a solution \\
Non-determinative ADR & Cannot impose a solution. \\
\bottomrule()
\end{longtable}

\hypertarget{alternatives}{%
\subsubsection{Alternatives}\label{alternatives}}

Litigation:

\begin{itemize}
\tightlist
\item
  Not voluntary
\item
  Once case starts, neither party can withdraw without paying opponent's
  costs
\item
  Court imposes a solution if parties cannot come to a settlement.
\end{itemize}

Arbitration:

\begin{itemize}
\tightlist
\item
  Voluntary in the sense that parties voluntarily entered into an
  arbitration agreement.
\item
  When a dispute arises, one party can force the other to arbitrate
  against their will because of the original contractual agreement.
\item
  Arbitrator will impose a solution which the winner can enforce.
\item
  An arbitrator's award is confidential but binding on the parties.
\end{itemize}

Strictly speaking, negotiation is a form of ADR. It is both voluntary
and non-binding. However, there is no independent third party and the
negotiators will normally be identified with their respective sides.

\hypertarget{independent-third-party}{%
\subsubsection{Independent Third Party}\label{independent-third-party}}

\begin{itemize}
\tightlist
\item
  Independence of the third party is an essential feature.
\item
  Likely to be more accommodating.
\item
  TP trained to be neutral and an industry expert.
\end{itemize}

\hypertarget{advantages-of-adr}{%
\subsection{Advantages of ADR}\label{advantages-of-adr}}

The CPR 1998 specifically recognise the advantages of ADR.

\begin{Shaded}
\begin{Highlighting}[]
\NormalTok{title: Rule 1.4(2)(e)}
\NormalTok{The court may further the overriding objective of dealing with cases justly by encouraging the parties to use an alternative dispute resolution procedure if the court considers that appropriate and facilitating the use of such procedure.}
\end{Highlighting}
\end{Shaded}

Arbitration and ADR procedures are confidential. Nothing said can be
referred to in any later court proceedings unless all parties agree to
waive confidentiality.

It can be significantly cheaper than both arbitration and litigation.
This is because it is quicker. A skilled neutral can, in most cases that
are suitable for ADR, help the parties to resolve their dispute in a
relatively short period of time.

\hypertarget{flexibility}{%
\subsubsection{Flexibility}\label{flexibility}}

Parties can choose the form of ADR and can choose a procedure to be
followed. There is no case law limiting what parties or the neutral can
do.

\hypertarget{preserving-business-relationship}{%
\subsubsection{Preserving Business
Relationship}\label{preserving-business-relationship}}

\begin{itemize}
\tightlist
\item
  Virtue of privacy
\item
  Non-confrontational method
\end{itemize}

\hypertarget{commercial-reality}{%
\subsubsection{Commercial Reality}\label{commercial-reality}}

Parties may just want realistic and workable terms of settlement.

\hypertarget{disadvantages-of-adr}{%
\subsection{Disadvantages of ADR}\label{disadvantages-of-adr}}

\hypertarget{does-not-bind-parties}{%
\subsubsection{Does Not Bind Parties}\label{does-not-bind-parties}}

No one can be forced to resolve a dispute this way. Most agreements
allow a party to withdraw at any stage.

The court can stay litigation that has been commenced in breach of an
agreed method of

resolving disputes. This is the case even if that method is not
technically an arbitration agreement under the Arbitration Act 1996.
Indeed, the courts have increasingly stayed proceedings for ADR to take
place, whether or not pursuant to a contractual agreement. For example,
in {[}{[}Cable \& Wireless v IBM UK Ltd {[}2002{]} BLR 89{]}{]}, the
parties were directed to pursue a previously agreed ADR method.

\hypertarget{awards-not-easily-enforceable}{%
\subsubsection{Awards Not Easily
Enforceable}\label{awards-not-easily-enforceable}}

There is no equivalent of s 66 Arbitration Act 1996 enabling ADR awards
to be enforced as if court judgments.

But if parties do agree to terms suggested as a result of
non-determinative ADR, they have entered into a contract. A
non-compliant party could then be sued for breach of contract, and
usually obtain summary judgment (i.e., without a full trial) under Part
24 CPR 1998.

Usually, ADR forms state that no agreement is binding unless in signed
writing. A party who has commenced court proceedings but then resolved
the dispute by ADR, can record the agreement reached in a consent order,
which can be enforced by the usual methods.

\hypertarget{facts-may-not-be-fully-disclosed}{%
\subsubsection{Facts May Not Be Fully
Disclosed}\label{facts-may-not-be-fully-disclosed}}

There is no disclosure stage, so less thorough. But in business this can
be worth it for efficiency.

\hypertarget{not-always-appropriate}{%
\subsubsection{Not Always Appropriate}\label{not-always-appropriate}}

Such as:

\begin{itemize}
\tightlist
\item
  Where the client needs an injunction or security for costs (requiring
  a party to pay money into court as security for the costs of the other
  party, should they lose)
\item
  Where there is no dispute

  \begin{itemize}
  \tightlist
  \item
    e.g., for a simple debt collection matter, just fill a claim for and
    a summary judgment application, or consider insolvency proceedings.
  \end{itemize}
\item
  Where the client needs to resolve a point of law.
\end{itemize}

If the client reasonably believes that they have a watertight case, that
might well be sufficient justification for a refusal to mediate (see
{[}{[}Swain Mason v Mills \& Reeve {[}2012{]} EWCA Civ 498{]}{]}).

A standard direction in proceedings requires a party who rejects a
proposal for ADR to file a witness statement detailing their reasons for
doing so. This will be available to the trial judge when the issue of
costs is considered.

\hypertarget{types-of-adr}{%
\subsection{Types of ADR}\label{types-of-adr}}

\hypertarget{mediation-and-conciliation}{%
\subsubsection{Mediation and
Conciliation}\label{mediation-and-conciliation}}

(Terms used interchangeably).

\begin{itemize}
\tightlist
\item
  Third party receives written statements from both parties
\item
  Mediator discusses case with parties
\item
  Doesn't need to be face to face
\end{itemize}

\hypertarget{med-arb}{%
\subsubsection{Med-arb}\label{med-arb}}

Parties agree to submit their dispute to mediation and, if this does not
work, that they will refer the matter to arbitration.

Possible to use the mediator as arbitrator. This will save costs, though
there is a risk their position of being privy to confidential
information known to one party will compromise their independent
position.

\hypertarget{mini-trial}{%
\subsubsection{Mini-trial}\label{mini-trial}}

Also known as `structured settlement procedure'. Parties appoint a
neutral party who will sit as chair of a tribunal comprised of a chair
and senior representatives of each party. The representatives have
authority to reach such compromise as they see fit.

\hypertarget{expert-appraisal}{%
\subsubsection{Expert Appraisal}\label{expert-appraisal}}

Some or all of the dispute can be referred to an expert in the disputed
field for an opinion. The opinion is not binding on the parties, but
could influence their approach/ negotiations.

\hypertarget{judicial-appraisal}{%
\subsubsection{Judicial Appraisal}\label{judicial-appraisal}}

The Centre for Effective Dispute Resolution (CEDR) has a scheme whereby
former judges and senior counsel are available to give a quick
preliminary view on the legal position, having heard representations
from both parties. It is a matter for agreement between the parties as
to whether this opinion is to be binding on them or not.

\hypertarget{expert-determination}{%
\subsubsection{Expert Determination}\label{expert-determination}}

Halfway house between arbitration and ADR. Parties select an expert to
decide the case for them and agree to accept the expert's decision.

If the other party does not accept, it can be sued for breach of
contract. But the expert's decision cannot be enforced as a court order,
and they do not have the power of an arbitrator under Arbitration Act
1996. Also, unlike an arbitrator, they can be sued in negligence by a
party who thinks their decision was wrong.

\hypertarget{final-offer-arbitration}{%
\subsubsection{Final Offer Arbitration}\label{final-offer-arbitration}}

The parties can instruct their chosen neutral that they will both make
an offer of the terms on which they will settle, and that the neutral
must choose one of those two offers and no other solution. Neither party
can afford to make an unrealistic offer, because that will mean that the
neutral will choose the opponent's offer.

\hypertarget{early-neutral-evaluation}{%
\subsubsection{Early Neutral
Evaluation}\label{early-neutral-evaluation}}

Allows the parties to instruct their chosen neutral to make a
preliminary assessment of the facts at an early stage in the dispute.
Normally, the parties

submit written case summaries and supporting documents. The evaluator
then makes a recommendation. This very often helps the parties to
negotiate a settlement (or move to another ADR method), avoiding the
expense of litigation.

\begin{Shaded}
\begin{Highlighting}[]
\NormalTok{title: Rule 3.1(2)(m) CPR 1998}
\NormalTok{The court may take any step or make any order for the purpose of managing the case and furthering the overriding objective, including hearing an early neutral evaluation (ENE) with the aim of helping the parties to settle the case.}
\end{Highlighting}
\end{Shaded}

\begin{Shaded}
\begin{Highlighting}[]
\NormalTok{title: Can the court order that a judicial appraisal take place by way of ENE against the wishes of the parties?}
\NormalTok{Yes, see [[Lomax v Lomax [2019] EWCA Civ 1467]], because it is part of the court process which can assist with the fair and sensible resolution of a case.}
\end{Highlighting}
\end{Shaded}

The outcome of Judicial ENE is normally `without prejudice' unless
privilege is mutually waived, and is not normally binding unless the
parties agree.

\hypertarget{ombudsman}{%
\subsubsection{Ombudsman}\label{ombudsman}}

The number of ombudsman schemes have grown over the years. Some service
and goods providers offer similar schemes.

\hypertarget{organisations-providing-adr}{%
\subsection{Organisations Providing
ADR}\label{organisations-providing-adr}}

Two groups have pioneered:

\begin{itemize}
\tightlist
\item
  CEDR: independent non-profit promoting ADR
\item
  ADR Group: private company which undertakes mediation and training.
\end{itemize}

See also Chartered Institute of Arbitrators, Academy of Experts and
Royal Institution of Chartered Surveyors.

\hypertarget{using-adr}{%
\subsection{Using ADR}\label{using-adr}}

Prudent to agree in the original contract that if any dispute arises, it
will be resolved by a specified form of ADR.

Existing contracts which provide for arbitration should probably be
amended to have some form of mediation as a step.

\hypertarget{disclosure}{%
\subsubsection{Disclosure}\label{disclosure}}

Include clauses dealing with potential pitfalls of ADR. Parties should
decide whether to have a clause requiring full disclosure. This takes
longer and is more expensive, but makes it possible to set aside any ADR
settlement reached on discovering that the other party has concealed
information.

\hypertarget{confidentiality}{%
\subsubsection{Confidentiality}\label{confidentiality}}

A confidentiality clause encourages full disclosure. Parties will be
more likely to disclose information to each other if they know the other
party has agreed not to divulge this information to anyone else.

\hypertarget{other-matters}{%
\subsubsection{Other Matters}\label{other-matters}}

Specify procedures, appointment of mediator, and that the
representatives who attend ADR process must have full authority to
settle the dispute there and then.

\hypertarget{choosing-adr}{%
\subsection{Choosing ADR}\label{choosing-adr}}

Discuss the possibility of ADR whenever a dispute arises. If a client is
willing to use ADR, use it unless obviously inappropriate.

At the first sign of non-cooperation/ lack of trust, commence
arbitration or litigation. You can still continue with ADR too.

Failure to respond to a reasonable proposal to attempt settlement by ADR
may have a significant impact on any subsequent order for costs.

See also {[}{[}Final Preparations, Trial and Costs\#Conduct and
ADR{]}{]}.

\hypertarget{considerations}{%
\subsection{Considerations}\label{considerations}}

\begin{longtable}[]{@{}
  >{\raggedright\arraybackslash}p{(\columnwidth - 2\tabcolsep) * \real{0.1538}}
  >{\raggedright\arraybackslash}p{(\columnwidth - 2\tabcolsep) * \real{0.8462}}@{}}
\toprule()
\begin{minipage}[b]{\linewidth}\raggedright
Consideration
\end{minipage} & \begin{minipage}[b]{\linewidth}\raggedright
Details
\end{minipage} \\
\midrule()
\endhead
Contract & The contract may specify/ exclude a particular ADR method. \\
Privacy & Litigation is very public; ``airing dirty laundry'' \\
Publicity & But client may want to clear their name/ prove that they
will take action. \\
Precedent & Novel point of law? Prevent the possibility of further
claims by forming a precedent/ don't want the precedent \\
Jurisdiction & Issues could be avoided if using mediation and
arbitration \\
Cost/ time & Litigation is time-consuming and expensive. \\
Expert determination & Useful if there is a technical issue \\
Relief required & May require an injunction, which cannot be obtained
out of court. \\
Future relationships & Preserving relationship with the opponent;
consider using mediation. Can be used to find a middle ground. \\
\bottomrule()
\end{longtable}

\end{document}
