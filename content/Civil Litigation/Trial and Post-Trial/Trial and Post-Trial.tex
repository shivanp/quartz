% Options for packages loaded elsewhere
\PassOptionsToPackage{unicode}{hyperref}
\PassOptionsToPackage{hyphens}{url}
%
\documentclass[
]{article}
\usepackage{amsmath,amssymb}
\usepackage{lmodern}
\usepackage{iftex}
\ifPDFTeX
  \usepackage[T1]{fontenc}
  \usepackage[utf8]{inputenc}
  \usepackage{textcomp} % provide euro and other symbols
\else % if luatex or xetex
  \usepackage{unicode-math}
  \defaultfontfeatures{Scale=MatchLowercase}
  \defaultfontfeatures[\rmfamily]{Ligatures=TeX,Scale=1}
\fi
% Use upquote if available, for straight quotes in verbatim environments
\IfFileExists{upquote.sty}{\usepackage{upquote}}{}
\IfFileExists{microtype.sty}{% use microtype if available
  \usepackage[]{microtype}
  \UseMicrotypeSet[protrusion]{basicmath} % disable protrusion for tt fonts
}{}
\makeatletter
\@ifundefined{KOMAClassName}{% if non-KOMA class
  \IfFileExists{parskip.sty}{%
    \usepackage{parskip}
  }{% else
    \setlength{\parindent}{0pt}
    \setlength{\parskip}{6pt plus 2pt minus 1pt}}
}{% if KOMA class
  \KOMAoptions{parskip=half}}
\makeatother
\usepackage{xcolor}
\usepackage[margin=1in]{geometry}
\usepackage{color}
\usepackage{fancyvrb}
\newcommand{\VerbBar}{|}
\newcommand{\VERB}{\Verb[commandchars=\\\{\}]}
\DefineVerbatimEnvironment{Highlighting}{Verbatim}{commandchars=\\\{\}}
% Add ',fontsize=\small' for more characters per line
\newenvironment{Shaded}{}{}
\newcommand{\AlertTok}[1]{\textcolor[rgb]{1.00,0.00,0.00}{\textbf{#1}}}
\newcommand{\AnnotationTok}[1]{\textcolor[rgb]{0.38,0.63,0.69}{\textbf{\textit{#1}}}}
\newcommand{\AttributeTok}[1]{\textcolor[rgb]{0.49,0.56,0.16}{#1}}
\newcommand{\BaseNTok}[1]{\textcolor[rgb]{0.25,0.63,0.44}{#1}}
\newcommand{\BuiltInTok}[1]{#1}
\newcommand{\CharTok}[1]{\textcolor[rgb]{0.25,0.44,0.63}{#1}}
\newcommand{\CommentTok}[1]{\textcolor[rgb]{0.38,0.63,0.69}{\textit{#1}}}
\newcommand{\CommentVarTok}[1]{\textcolor[rgb]{0.38,0.63,0.69}{\textbf{\textit{#1}}}}
\newcommand{\ConstantTok}[1]{\textcolor[rgb]{0.53,0.00,0.00}{#1}}
\newcommand{\ControlFlowTok}[1]{\textcolor[rgb]{0.00,0.44,0.13}{\textbf{#1}}}
\newcommand{\DataTypeTok}[1]{\textcolor[rgb]{0.56,0.13,0.00}{#1}}
\newcommand{\DecValTok}[1]{\textcolor[rgb]{0.25,0.63,0.44}{#1}}
\newcommand{\DocumentationTok}[1]{\textcolor[rgb]{0.73,0.13,0.13}{\textit{#1}}}
\newcommand{\ErrorTok}[1]{\textcolor[rgb]{1.00,0.00,0.00}{\textbf{#1}}}
\newcommand{\ExtensionTok}[1]{#1}
\newcommand{\FloatTok}[1]{\textcolor[rgb]{0.25,0.63,0.44}{#1}}
\newcommand{\FunctionTok}[1]{\textcolor[rgb]{0.02,0.16,0.49}{#1}}
\newcommand{\ImportTok}[1]{#1}
\newcommand{\InformationTok}[1]{\textcolor[rgb]{0.38,0.63,0.69}{\textbf{\textit{#1}}}}
\newcommand{\KeywordTok}[1]{\textcolor[rgb]{0.00,0.44,0.13}{\textbf{#1}}}
\newcommand{\NormalTok}[1]{#1}
\newcommand{\OperatorTok}[1]{\textcolor[rgb]{0.40,0.40,0.40}{#1}}
\newcommand{\OtherTok}[1]{\textcolor[rgb]{0.00,0.44,0.13}{#1}}
\newcommand{\PreprocessorTok}[1]{\textcolor[rgb]{0.74,0.48,0.00}{#1}}
\newcommand{\RegionMarkerTok}[1]{#1}
\newcommand{\SpecialCharTok}[1]{\textcolor[rgb]{0.25,0.44,0.63}{#1}}
\newcommand{\SpecialStringTok}[1]{\textcolor[rgb]{0.73,0.40,0.53}{#1}}
\newcommand{\StringTok}[1]{\textcolor[rgb]{0.25,0.44,0.63}{#1}}
\newcommand{\VariableTok}[1]{\textcolor[rgb]{0.10,0.09,0.49}{#1}}
\newcommand{\VerbatimStringTok}[1]{\textcolor[rgb]{0.25,0.44,0.63}{#1}}
\newcommand{\WarningTok}[1]{\textcolor[rgb]{0.38,0.63,0.69}{\textbf{\textit{#1}}}}
\usepackage{longtable,booktabs,array}
\usepackage{calc} % for calculating minipage widths
% Correct order of tables after \paragraph or \subparagraph
\usepackage{etoolbox}
\makeatletter
\patchcmd\longtable{\par}{\if@noskipsec\mbox{}\fi\par}{}{}
\makeatother
% Allow footnotes in longtable head/foot
\IfFileExists{footnotehyper.sty}{\usepackage{footnotehyper}}{\usepackage{footnote}}
\makesavenoteenv{longtable}
\setlength{\emergencystretch}{3em} % prevent overfull lines
\providecommand{\tightlist}{%
  \setlength{\itemsep}{0pt}\setlength{\parskip}{0pt}}
\setcounter{secnumdepth}{-\maxdimen} % remove section numbering
\usepackage{xcolor}
\definecolor{aliceblue}{HTML}{F0F8FF}
\definecolor{antiquewhite}{HTML}{FAEBD7}
\definecolor{aqua}{HTML}{00FFFF}
\definecolor{aquamarine}{HTML}{7FFFD4}
\definecolor{azure}{HTML}{F0FFFF}
\definecolor{beige}{HTML}{F5F5DC}
\definecolor{bisque}{HTML}{FFE4C4}
\definecolor{black}{HTML}{000000}
\definecolor{blanchedalmond}{HTML}{FFEBCD}
\definecolor{blue}{HTML}{0000FF}
\definecolor{blueviolet}{HTML}{8A2BE2}
\definecolor{brown}{HTML}{A52A2A}
\definecolor{burlywood}{HTML}{DEB887}
\definecolor{cadetblue}{HTML}{5F9EA0}
\definecolor{chartreuse}{HTML}{7FFF00}
\definecolor{chocolate}{HTML}{D2691E}
\definecolor{coral}{HTML}{FF7F50}
\definecolor{cornflowerblue}{HTML}{6495ED}
\definecolor{cornsilk}{HTML}{FFF8DC}
\definecolor{crimson}{HTML}{DC143C}
\definecolor{cyan}{HTML}{00FFFF}
\definecolor{darkblue}{HTML}{00008B}
\definecolor{darkcyan}{HTML}{008B8B}
\definecolor{darkgoldenrod}{HTML}{B8860B}
\definecolor{darkgray}{HTML}{A9A9A9}
\definecolor{darkgreen}{HTML}{006400}
\definecolor{darkgrey}{HTML}{A9A9A9}
\definecolor{darkkhaki}{HTML}{BDB76B}
\definecolor{darkmagenta}{HTML}{8B008B}
\definecolor{darkolivegreen}{HTML}{556B2F}
\definecolor{darkorange}{HTML}{FF8C00}
\definecolor{darkorchid}{HTML}{9932CC}
\definecolor{darkred}{HTML}{8B0000}
\definecolor{darksalmon}{HTML}{E9967A}
\definecolor{darkseagreen}{HTML}{8FBC8F}
\definecolor{darkslateblue}{HTML}{483D8B}
\definecolor{darkslategray}{HTML}{2F4F4F}
\definecolor{darkslategrey}{HTML}{2F4F4F}
\definecolor{darkturquoise}{HTML}{00CED1}
\definecolor{darkviolet}{HTML}{9400D3}
\definecolor{deeppink}{HTML}{FF1493}
\definecolor{deepskyblue}{HTML}{00BFFF}
\definecolor{dimgray}{HTML}{696969}
\definecolor{dimgrey}{HTML}{696969}
\definecolor{dodgerblue}{HTML}{1E90FF}
\definecolor{firebrick}{HTML}{B22222}
\definecolor{floralwhite}{HTML}{FFFAF0}
\definecolor{forestgreen}{HTML}{228B22}
\definecolor{fuchsia}{HTML}{FF00FF}
\definecolor{gainsboro}{HTML}{DCDCDC}
\definecolor{ghostwhite}{HTML}{F8F8FF}
\definecolor{gold}{HTML}{FFD700}
\definecolor{goldenrod}{HTML}{DAA520}
\definecolor{gray}{HTML}{808080}
\definecolor{green}{HTML}{008000}
\definecolor{greenyellow}{HTML}{ADFF2F}
\definecolor{grey}{HTML}{808080}
\definecolor{honeydew}{HTML}{F0FFF0}
\definecolor{hotpink}{HTML}{FF69B4}
\definecolor{indianred}{HTML}{CD5C5C}
\definecolor{indigo}{HTML}{4B0082}
\definecolor{ivory}{HTML}{FFFFF0}
\definecolor{khaki}{HTML}{F0E68C}
\definecolor{lavender}{HTML}{E6E6FA}
\definecolor{lavenderblush}{HTML}{FFF0F5}
\definecolor{lawngreen}{HTML}{7CFC00}
\definecolor{lemonchiffon}{HTML}{FFFACD}
\definecolor{lightblue}{HTML}{ADD8E6}
\definecolor{lightcoral}{HTML}{F08080}
\definecolor{lightcyan}{HTML}{E0FFFF}
\definecolor{lightgoldenrodyellow}{HTML}{FAFAD2}
\definecolor{lightgray}{HTML}{D3D3D3}
\definecolor{lightgreen}{HTML}{90EE90}
\definecolor{lightgrey}{HTML}{D3D3D3}
\definecolor{lightpink}{HTML}{FFB6C1}
\definecolor{lightsalmon}{HTML}{FFA07A}
\definecolor{lightseagreen}{HTML}{20B2AA}
\definecolor{lightskyblue}{HTML}{87CEFA}
\definecolor{lightslategray}{HTML}{778899}
\definecolor{lightslategrey}{HTML}{778899}
\definecolor{lightsteelblue}{HTML}{B0C4DE}
\definecolor{lightyellow}{HTML}{FFFFE0}
\definecolor{lime}{HTML}{00FF00}
\definecolor{limegreen}{HTML}{32CD32}
\definecolor{linen}{HTML}{FAF0E6}
\definecolor{magenta}{HTML}{FF00FF}
\definecolor{maroon}{HTML}{800000}
\definecolor{mediumaquamarine}{HTML}{66CDAA}
\definecolor{mediumblue}{HTML}{0000CD}
\definecolor{mediumorchid}{HTML}{BA55D3}
\definecolor{mediumpurple}{HTML}{9370DB}
\definecolor{mediumseagreen}{HTML}{3CB371}
\definecolor{mediumslateblue}{HTML}{7B68EE}
\definecolor{mediumspringgreen}{HTML}{00FA9A}
\definecolor{mediumturquoise}{HTML}{48D1CC}
\definecolor{mediumvioletred}{HTML}{C71585}
\definecolor{midnightblue}{HTML}{191970}
\definecolor{mintcream}{HTML}{F5FFFA}
\definecolor{mistyrose}{HTML}{FFE4E1}
\definecolor{moccasin}{HTML}{FFE4B5}
\definecolor{navajowhite}{HTML}{FFDEAD}
\definecolor{navy}{HTML}{000080}
\definecolor{oldlace}{HTML}{FDF5E6}
\definecolor{olive}{HTML}{808000}
\definecolor{olivedrab}{HTML}{6B8E23}
\definecolor{orange}{HTML}{FFA500}
\definecolor{orangered}{HTML}{FF4500}
\definecolor{orchid}{HTML}{DA70D6}
\definecolor{palegoldenrod}{HTML}{EEE8AA}
\definecolor{palegreen}{HTML}{98FB98}
\definecolor{paleturquoise}{HTML}{AFEEEE}
\definecolor{palevioletred}{HTML}{DB7093}
\definecolor{papayawhip}{HTML}{FFEFD5}
\definecolor{peachpuff}{HTML}{FFDAB9}
\definecolor{peru}{HTML}{CD853F}
\definecolor{pink}{HTML}{FFC0CB}
\definecolor{plum}{HTML}{DDA0DD}
\definecolor{powderblue}{HTML}{B0E0E6}
\definecolor{purple}{HTML}{800080}
\definecolor{red}{HTML}{FF0000}
\definecolor{rosybrown}{HTML}{BC8F8F}
\definecolor{royalblue}{HTML}{4169E1}
\definecolor{saddlebrown}{HTML}{8B4513}
\definecolor{salmon}{HTML}{FA8072}
\definecolor{sandybrown}{HTML}{F4A460}
\definecolor{seagreen}{HTML}{2E8B57}
\definecolor{seashell}{HTML}{FFF5EE}
\definecolor{sienna}{HTML}{A0522D}
\definecolor{silver}{HTML}{C0C0C0}
\definecolor{skyblue}{HTML}{87CEEB}
\definecolor{slateblue}{HTML}{6A5ACD}
\definecolor{slategray}{HTML}{708090}
\definecolor{slategrey}{HTML}{708090}
\definecolor{snow}{HTML}{FFFAFA}
\definecolor{springgreen}{HTML}{00FF7F}
\definecolor{steelblue}{HTML}{4682B4}
\definecolor{tan}{HTML}{D2B48C}
\definecolor{teal}{HTML}{008080}
\definecolor{thistle}{HTML}{D8BFD8}
\definecolor{tomato}{HTML}{FF6347}
\definecolor{turquoise}{HTML}{40E0D0}
\definecolor{violet}{HTML}{EE82EE}
\definecolor{wheat}{HTML}{F5DEB3}
\definecolor{white}{HTML}{FFFFFF}
\definecolor{whitesmoke}{HTML}{F5F5F5}
\definecolor{yellow}{HTML}{FFFF00}
\definecolor{yellowgreen}{HTML}{9ACD32}
\usepackage[most]{tcolorbox}

\usepackage{ifthen}
\provideboolean{admonitiontwoside}
\makeatletter%
\if@twoside%
\setboolean{admonitiontwoside}{true}
\else%
\setboolean{admonitiontwoside}{false}
\fi%
\makeatother%

\newenvironment{env-d7082389-711f-4672-86cf-5d6e9e936c15}
{
    \savenotes\tcolorbox[blanker,breakable,left=5pt,borderline west={2pt}{-4pt}{firebrick}]
}
{
    \endtcolorbox\spewnotes
}
                

\newenvironment{env-b5098608-8443-4bf1-ab87-29613eab3647}
{
    \savenotes\tcolorbox[blanker,breakable,left=5pt,borderline west={2pt}{-4pt}{blue}]
}
{
    \endtcolorbox\spewnotes
}
                

\newenvironment{env-233126be-8f5a-4d88-835d-f2daf290d01d}
{
    \savenotes\tcolorbox[blanker,breakable,left=5pt,borderline west={2pt}{-4pt}{green}]
}
{
    \endtcolorbox\spewnotes
}
                

\newenvironment{env-bfb0cc18-da99-415c-876c-60f1f6c155b7}
{
    \savenotes\tcolorbox[blanker,breakable,left=5pt,borderline west={2pt}{-4pt}{aquamarine}]
}
{
    \endtcolorbox\spewnotes
}
                

\newenvironment{env-88e01d23-c127-44e7-b286-5ac92e547834}
{
    \savenotes\tcolorbox[blanker,breakable,left=5pt,borderline west={2pt}{-4pt}{orange}]
}
{
    \endtcolorbox\spewnotes
}
                

\newenvironment{env-716611b8-7a36-43f1-8759-3b3068bce296}
{
    \savenotes\tcolorbox[blanker,breakable,left=5pt,borderline west={2pt}{-4pt}{blue}]
}
{
    \endtcolorbox\spewnotes
}
                

\newenvironment{env-0064ec82-fa15-4465-a79a-9af9ba905d85}
{
    \savenotes\tcolorbox[blanker,breakable,left=5pt,borderline west={2pt}{-4pt}{yellow}]
}
{
    \endtcolorbox\spewnotes
}
                

\newenvironment{env-1b5ef2e8-8752-4ddc-8cbe-37100570f043}
{
    \savenotes\tcolorbox[blanker,breakable,left=5pt,borderline west={2pt}{-4pt}{darkred}]
}
{
    \endtcolorbox\spewnotes
}
                

\newenvironment{env-c188d876-1019-410b-be10-d5307bb965f9}
{
    \savenotes\tcolorbox[blanker,breakable,left=5pt,borderline west={2pt}{-4pt}{pink}]
}
{
    \endtcolorbox\spewnotes
}
                

\newenvironment{env-9ec6b8a2-eda3-481c-94c5-21f8c4c6d910}
{
    \savenotes\tcolorbox[blanker,breakable,left=5pt,borderline west={2pt}{-4pt}{cyan}]
}
{
    \endtcolorbox\spewnotes
}
                

\newenvironment{env-00c9cc36-00a1-40ca-b54b-ea2ff63ae4ec}
{
    \savenotes\tcolorbox[blanker,breakable,left=5pt,borderline west={2pt}{-4pt}{cyan}]
}
{
    \endtcolorbox\spewnotes
}
                

\newenvironment{env-7afd7ffe-65aa-42fc-807c-d0ed13c6a617}
{
    \savenotes\tcolorbox[blanker,breakable,left=5pt,borderline west={2pt}{-4pt}{purple}]
}
{
    \endtcolorbox\spewnotes
}
                

\newenvironment{env-bdf48894-dfb7-4cd7-a365-9ad5e6cf6e3d}
{
    \savenotes\tcolorbox[blanker,breakable,left=5pt,borderline west={2pt}{-4pt}{darksalmon}]
}
{
    \endtcolorbox\spewnotes
}
                

\newenvironment{env-a9f24d32-6e23-4552-96c2-ef34a9fd9c44}
{
    \savenotes\tcolorbox[blanker,breakable,left=5pt,borderline west={2pt}{-4pt}{gray}]
}
{
    \endtcolorbox\spewnotes
}
                
\ifLuaTeX
  \usepackage{selnolig}  % disable illegal ligatures
\fi
\IfFileExists{bookmark.sty}{\usepackage{bookmark}}{\usepackage{hyperref}}
\IfFileExists{xurl.sty}{\usepackage{xurl}}{} % add URL line breaks if available
\urlstyle{same} % disable monospaced font for URLs
\hypersetup{
  hidelinks,
  pdfcreator={LaTeX via pandoc}}

\author{}
\date{}

\begin{document}

{
\setcounter{tocdepth}{3}
\tableofcontents
}
\begin{Shaded}
\begin{Highlighting}[]
\NormalTok{min\_depth: 1}
\end{Highlighting}
\end{Shaded}

\hypertarget{final-preparations-trial-and-costs}{%
\section{Final Preparations, Trial and
Costs}\label{final-preparations-trial-and-costs}}

\hypertarget{final-preparations}{%
\subsection{Final Preparations}\label{final-preparations}}

\hypertarget{brief-counsel}{%
\subsubsection{Brief Counsel}\label{brief-counsel}}

\begin{itemize}
\tightlist
\item
  Often counsel briefed already
\item
  Deal with facts of issue and how to prove.
\item
  Sometimes there is a conference with counsel before the hearing.
\item
  Solicitor must negotiate the brief fee with counsel's clerk following
  the brief.

  \begin{itemize}
  \tightlist
  \item
    In fast-track, try to restrict to the maximum allowed for advocacy
    in fast track case
  \item
    In multi-track, covers only 1 day, so there could be a refresher
    fee.
  \end{itemize}
\end{itemize}

\hypertarget{witnesses}{%
\subsubsection{Witnesses}\label{witnesses}}

\begin{itemize}
\tightlist
\item
  All witnesses should be kept fully informed of the expected trial
  date.
\item
  Serve witnesses with a \textbf{witness summons}, even when they have
  been kept fully informed and involved.

  \begin{itemize}
  \tightlist
  \item
    This is a court document requiring the witness to attend court to
    give evidence, and produce evidence.
  \item
    Serve \(\geq 7\) days before court date. Then it is binding on the
    witness, and if they do not attend they could be fined/ imprisoned
    for contempt of court.
  \item
    If \(<7\) days until trial, must obtain permission of the court.
  \item
    Normally served by the court.
  \item
    Greater possibility of the trial judge being sympathetic and
    granting an adjournment of the trial if the witness does not show
    up.
  \item
    Police officers must always be served with a witness summons

    \begin{itemize}
    \tightlist
    \item
      Else they will not give evidence in a civil matter.
    \end{itemize}
  \end{itemize}
\item
  Witness can be paid:

  \begin{itemize}
  \tightlist
  \item
    Reasonable travel expenses
  \item
    Compensation for loss of time (PD 34A)
  \end{itemize}
\end{itemize}

\hypertarget{expert-witnesses}{%
\paragraph{Expert Witnesses}\label{expert-witnesses}}

A witness summons to require the attendance of an expert at trial should
be used only if required by the expert. If you don't tell the expert in
time when the trial date is, unlikely the court will agree to reschedule
({[}{[}Mitchell v Precis 548 Ltd {[}2019{]} EWHC 3314 (QB){]}{]}).

\hypertarget{trial-bundles}{%
\subsubsection{Trial Bundles}\label{trial-bundles}}

Claimant should file \(3 \leq T \leq 7\) days before trial.

\begin{Shaded}
\begin{Highlighting}[]
\NormalTok{title: PD 32 para 27.5}
\NormalTok{Unless the court orders otherwise, the trial bundle should include a copy of—}
\NormalTok{{-} (a) the claim form and all statements of case;}
\NormalTok{{-} (b) a case summary and/or chronology where appropriate;}
\NormalTok{{-} (c) requests for further information and responses to the requests;}
\NormalTok{{-} (d) all witness statements to be relied on as evidence;}
\NormalTok{{-} (e) any witness summaries;}
\NormalTok{{-} (f) any notices of intention to rely on hearsay evidence under rule 32.2;}
\NormalTok{{-} (g) any notices of intention to rely on evidence (such as a plan, photograph etc.) under rule 33.6 which is not—}
\NormalTok{    {-} (i) contained in a witness statement, affidavit or experts’ report;}
\NormalTok{    {-} (ii) being given orally at trial; and}
\NormalTok{    {-} (iii) hearsay evidence under rule 33.2;}
\NormalTok{{-} (h) any medical reports and responses to them;}
\NormalTok{{-} (i) any experts’ reports and responses to them;}
\NormalTok{{-} (j) any order giving directions as to the conduct of the trial; and}
\NormalTok{{-} (k) any other necessary documents.}
\end{Highlighting}
\end{Shaded}

Paragraph 27.12 of PD 32 provides that the contents of the trial bundle
should be agreed\\
where possible.

Where it is not possible to agree the contents of the bundle, a summary
of the points on which the parties are unable to agree should be
included. Unless the court otherwise directs, the documents in the trial
bundle should be copied double-sided (PD 32, para 27.15).

Supply identical bundles to all parties to the proceedings.

\hypertarget{case-summary}{%
\paragraph{Case Summary}\label{case-summary}}

Each party should prepare a case summary (often called a `skeleton
argument') to use at trial. This should:

\begin{enumerate}
\def\labelenumi{\arabic{enumi}.}
\tightlist
\item
  Review submissions of fact
\item
  Set out the propositions of law advanced with reference to main
  authorities relied on
\item
  Identify key documents the trial judge should read before the trial
  starts.
\end{enumerate}

Form: numbered paragraphs and pages.

\hypertarget{time-estimates-and-core-bundles}{%
\paragraph{Time Estimates and Core
Bundles}\label{time-estimates-and-core-bundles}}

Include with the trial bundle:

\begin{itemize}
\tightlist
\item
  Estimated length of reading tie
\item
  Agreed estimate of length of hearing
\item
  Core bundle of key documents, if the bundle is v long.
\end{itemize}

\hypertarget{trial-proceedings}{%
\subsection{Trial Proceedings}\label{trial-proceedings}}

\hypertarget{venue}{%
\subsubsection{Venue}\label{venue}}

\begin{longtable}[]{@{}
  >{\raggedright\arraybackslash}p{(\columnwidth - 2\tabcolsep) * \real{0.4615}}
  >{\raggedright\arraybackslash}p{(\columnwidth - 2\tabcolsep) * \real{0.5385}}@{}}
\toprule()
\begin{minipage}[b]{\linewidth}\raggedright
Court
\end{minipage} & \begin{minipage}[b]{\linewidth}\raggedright
Venue
\end{minipage} \\
\midrule()
\endhead
County Court & Hearing centre \\
High Court & Royal Courts of Justice in London, or District
Registries. \\
\bottomrule()
\end{longtable}

\hypertarget{timetable}{%
\subsubsection{Timetable}\label{timetable}}

\begin{itemize}
\tightlist
\item
  Usually pre-fixed for fast and multi-track

  \begin{itemize}
  \tightlist
  \item
    Timetable may limit time for, e.g., cross-examination.
  \end{itemize}
\item
  At the trial, the judge will confirm or vary any timetable previously
  given.
\item
  Fast track: trial should be finished in one day. Usually, if it
  overruns, continues the next day.
\item
  Multi-track: consecutive days.
\end{itemize}

\begin{Shaded}
\begin{Highlighting}[]
\NormalTok{{-} SRA Code para 1.4: solicitor must never mislead the court. }
\NormalTok{{-} If client admits they have committed perjury/ misled court in a material way}
\NormalTok{    {-} Refuse to act further unless the client tells the court the truth}
\NormalTok{    {-} Don\textquotesingle{}t inform court/ parties of the reason for ceasing to act (para 6.3 {-} client confidentiality)}
\NormalTok{{-} $\textbackslash{}exists$ information which the solicitor is obliged to disclose to court. }
\NormalTok{    {-} Relevant cases and statutory provisions {-}\textgreater{} within realistic limits. }
\NormalTok{    {-} But no duty to inform the court/ opponent of any facts/ witnesses which would assist the opponent. }
\NormalTok{{-} When acting as an advocate}
\NormalTok{    {-} Don\textquotesingle{}t insult/ say anything merely scandalous}
\NormalTok{    {-} Avoid naming in open court any 3rd party whose character would be called into question, unless necessary}
\NormalTok{    {-} Don\textquotesingle{}t call into question character of witness, unless witness is given an opportunity to defend themselves}
\NormalTok{    {-} Do not suggest a person is guilty of a crime/ fraud/ misconduct unless material to case \& supported by reassonable grounds. }
\end{Highlighting}
\end{Shaded}

Permission is needed to tweet in court, unless you are a journalist.

\hypertarget{order-of-proceedings}{%
\subsubsection{Order of Proceedings}\label{order-of-proceedings}}

\begin{enumerate}
\def\labelenumi{\arabic{enumi}.}
\tightlist
\item
  Preliminary issues

  \begin{itemize}
  \tightlist
  \item
    Permission to amend a statement
  \item
    Permission to adduce more evidence by way of examination-in-chief
    from a witness under r 32.5(3)
  \item
    Application to strike out part of an opponent's witness statement
  \item
    Variation to timetable made at an earlier hearing.
  \end{itemize}
\item
  Claimant

  \begin{itemize}
  \tightlist
  \item
    May make an opening speech setting out the background to the case
    and facts remaining in issue (v briefly).
  \end{itemize}
\item
  Evidence

  \begin{itemize}
  \tightlist
  \item
    Claimant and witness give evidence.

    \begin{itemize}
    \tightlist
    \item
      Witness statements stand as \textbf{evidence-in-chief}.
    \item
      Witness can amplify this only with judge permission.

      \begin{itemize}
      \tightlist
      \item
        Claimant should not be asked leading questions by their
        advocate.
      \end{itemize}
    \end{itemize}
  \item
    Cross-examination.

    \begin{itemize}
    \tightlist
    \item
      No bar on leading questions
    \item
      Pretty much everything is fair game.
    \item
      Cross-examination questions can be prepared in advance.
    \item
      Dig into inconsistencies.
    \item
      But no need to be confrontational.
    \item
      \textbf{Mandatory} that the cross-examining advocate must put
      their own party's case to the witness they are cross-examining.

      \begin{itemize}
      \tightlist
      \item
        Basically explicitly ask the other party all the questions which
        are contested.
      \item
        Highly unlikely that they will change their story.
      \item
        Failure to challenge the opponent's evidence implied acceptance
        of the evidence.
      \end{itemize}
    \end{itemize}
  \item
    Re-examination

    \begin{itemize}
    \tightlist
    \item
      The advocate can ask further questions of their own witness about
      matters arising out of cross-examination.
    \end{itemize}
  \item
    Problem witnesses

    \begin{itemize}
    \tightlist
    \item
      A witness may change their story during cross-examination.
    \item
      Not much you can do -- can't cross-examine your own witness.
    \item
      If the witness becomes hostile, possible for a party to apply for
      the witness to be declared hostile.

      \begin{itemize}
      \tightlist
      \item
        Then the party who called the witness can cross-examine the
        witness on the facts of the case and previous inconsistent
        statements
      \item
        But cannot attack general character/ cross-examine them about
        previous convictions.
      \end{itemize}
    \item
      s 4 Civil Evidence Act 1995: court can choose to attach weight to
      the previously inconsistent statement as evidence.
    \end{itemize}
  \item
    D gives evidence in the same way
  \item
    Children as witnesses

    \begin{itemize}
    \tightlist
    \item
      A child who understands the nature of an oath gives sworn evidence
      (generally, children over 14 assumed to be able to give sworn
      evidence).
    \item
      Else s 96(2) Children Act 1989: evidence can be heard unsworn if
      child understands their duty to speak the truth.
    \end{itemize}
  \end{itemize}
\item
  Closing speeches

  \begin{itemize}
  \tightlist
  \item
    First D's advocate, then C's advocate
  \end{itemize}
\item
  Judgment

  \begin{itemize}
  \tightlist
  \item
    Delivered immediately, or at a later date.
  \item
    Usually addresses

    \begin{itemize}
    \tightlist
    \item
      Liability
    \item
      Quantum
    \item
      Interest
    \item
      Costs
    \end{itemize}
  \item
    Judgment will name the parties unless an anonymous order is
    justified.
  \end{itemize}
\item
  Part 36

  \begin{itemize}
  \tightlist
  \item
    C should apply under r 36.17(4) and D under r 37.17(3), if
    appropriate.
  \end{itemize}
\end{enumerate}

\hypertarget{interest}{%
\subsubsection{Interest}\label{interest}}

\hypertarget{up-to-date-of-judgment}{%
\paragraph{Up to Date of Judgment}\label{up-to-date-of-judgment}}

Interest normally awarded, provided it had been claimed in the
particulars of claim. Could be:

\begin{itemize}
\tightlist
\item
  Contractual provision for interest
\item
  Entitlement to interest under Late Payment of Commercial Debts
  (Interest) Act 1998
\item
  Interest on damages usually awarded by the judge in their discretion,
  under SCA 1981, s 35A or CCA 1984, s 69
\item
  Interest is discretionary and may be reduced if a party has acted
  poorly/ precipitated delays in proceedings.
\end{itemize}

\hypertarget{after-judgment}{%
\paragraph{After Judgment}\label{after-judgment}}

\begin{itemize}
\tightlist
\item
  High Court

  \begin{itemize}
  \tightlist
  \item
    8\% pa under s 17 Judgments Act 1838 (unless contractual right to
    more)
  \end{itemize}
\item
  County Court

  \begin{itemize}
  \tightlist
  \item
    8\% pa under County Courts (Interest on Judgment Debts) Order 1991,
    provided the judgment was for at least \(£5,000\)
  \item
    Judgment \(<£5,000\): Late Payment of Commercial Debts (Interest)
    Act 1998 applies.
  \end{itemize}
\item
  Judgment rate of interest \textbf{not} discretionary. But the court
  does have discretion to delay when the judgment rate comes into effect
  (with a ``teaser'' rate for a few months -- {[}{[}Involnert Management
  Inc v Aprilgrange Ltd and others {[}2015{]} EWHC 2834 (Comm){]}{]})
\item
  Judgment interest cannot accrue on damages until they have been
  quantified, but interest is payable on costs from the date of
  judgment, even if they are yet to be assessed. So it is worth making a
  payment on account for costs asap.
\end{itemize}

\hypertarget{register-of-judgments-orders-and-fines}{%
\paragraph{Register of Judgments Orders and
Fines}\label{register-of-judgments-orders-and-fines}}

Judgments of the High Court and County Court are officially recorded on
the register for 6 years. May affect a judgment debtor's
creditworthiness. Judgment may be removed if set aside or paid in full
within a month.

\hypertarget{costs}{%
\subsection{Costs}\label{costs}}

\hypertarget{indemnity-principle}{%
\subsubsection{Indemnity Principle}\label{indemnity-principle}}

\begin{Shaded}
\begin{Highlighting}[]
\NormalTok{title: Indemnity principle}
\NormalTok{The winner is entitled to an indemnity in respect of the costs they have incurred, meaning they cannot seek more than their solicitor and client costs. Do not confuse with indemnity basis. }
\end{Highlighting}
\end{Shaded}

In reality, ``indemnity'' is misleading; the paying party will challenge
various items.

\hypertarget{general-provisions}{%
\subsubsection{General Provisions}\label{general-provisions}}

\begin{Shaded}
\begin{Highlighting}[]
\NormalTok{title: r 44.2 CPR}
\NormalTok{(1) The court has discretion as to –}
\NormalTok{{-} (a) whether costs are payable by one party to another;}
\NormalTok{{-} (b) the amount of those costs; and}
\NormalTok{{-} (c) when they are to be paid.}

\NormalTok{(2) If the court decides to make an order about costs –}
\NormalTok{{-} (a) the general rule is that the unsuccessful party will be ordered to pay the costs of the successful party; but}
\NormalTok{{-} (b) the court may make a different order.}
\end{Highlighting}
\end{Shaded}

``Costs'' include fees, charges, disbursements, expenses and
remuneration (r 44.1). This includes pre-action costs (r 44.2(6)(d)).

\hypertarget{interest-on-costs}{%
\subsubsection{Interest on Costs}\label{interest-on-costs}}

Interest is generally \textbf{not} payable on costs before judgment. But
in recent years, the court has increasingly awarded such costs to the
receiving party, who has had to put up money paying its solicitors.

The appropriate dates from which interest should run are the dates on
which the invoices for costs were actually paid by the party until the
date of judgment. The rate of interest should be the normal commercial
rate ({[}{[}Fiona Trust and Holding Corp v Privalov {[}2016{]} EWHC 2657
(Comm){]}{]}).

\hypertarget{factors}{%
\subsubsection{Factors}\label{factors}}

The court will consider all the circumstances, including:

\begin{itemize}
\tightlist
\item
  The conduct of all parties
\item
  Whether a party has succeeded on part of their case
\item
  Any admissible offer to settle made by a party that is drawn to the
  court's attention.

  \begin{itemize}
  \tightlist
  \item
    Considered when an open offer is made for more than is recovered.
  \end{itemize}
\end{itemize}

\hypertarget{conduct-of-parties}{%
\paragraph{Conduct of Parties}\label{conduct-of-parties}}

This includes

\begin{itemize}
\tightlist
\item
  Conduct before and during the proceedings.

  \begin{itemize}
  \tightlist
  \item
    Extent to which parties followed relevant pre-action protocol(s).
  \end{itemize}
\item
  Whether it was reasonable for a party to raise, pursue or contest a
  particular allegation.
\item
  The manner in which a party pursued/ defended their case or a
  particular allegation
\item
  Whether a claimant who won had exaggerated their claim.
\end{itemize}

\hypertarget{costs-budgets-and-cmos}{%
\paragraph{Costs Budgets and CMOs}\label{costs-budgets-and-cmos}}

If a Costs Management Order (CMO) has been made, the court will not
depart from the last approved budget unless there is a good reason for
doing so. Costs falling outside of the phases addressed in an agreed/
approved budget will be subject to a detailed assessment.

If no CMO was made, the court will have regard to any costs budgets when
assessing costs.

\hypertarget{disputes}{%
\subparagraph{Disputes}\label{disputes}}

If there is a difference of \(>20\%\) between the figure claimed by the
receiving party and the last costs budget filed, the receiving party
must file a statement setting out reasons for the difference (PD 44 para
3.2).

If the paying party:

\begin{itemize}
\item
  \begin{enumerate}
  \def\labelenumi{(\alph{enumi})}
  \tightlist
  \item
    claims to have reasonably relied on a budget filed by a receiving
    party; or
  \end{enumerate}
\item
  \begin{enumerate}
  \def\labelenumi{(\alph{enumi})}
  \setcounter{enumi}{1}
  \tightlist
  \item
    wishes to rely upon the costs shown in the budget in order to
    dispute the reasonableness or proportionality of the costs claimed,
  \end{enumerate}
\end{itemize}

the paying party must serve a statement setting out the case in this
regard in that party's points of dispute (PD 44, para 3.3).

If the court accepts that the paying party reasonably relied on the
budget, the court may restrict the recoverable costs to such sum as is
reasonable for the paying party to pay in the light of that reliance,
even if it is less than the amount of costs reasonably and
proportionately incurred by the receiving party (PD 44, para 3.6).

If any explanation provided by the receiving party is deemed
unsatisfactory, the court may regard the difference between the amount
claimed and the budget figure as evidence that the costs claimed are
unreasonable or disproportionate (PD 44, para 3.7).

\hypertarget{types-of-orders}{%
\paragraph{Types of Orders}\label{types-of-orders}}

Pretty flexible, could be a proportion of a party's costs, a stated
amount of party's costs, costs from a particular date, costs of a
particular step, etc.

\hypertarget{basis-of-assessment}{%
\subsubsection{Basis of Assessment}\label{basis-of-assessment}}

\begin{Shaded}
\begin{Highlighting}[]
\NormalTok{title: r 44.3(1)}
\NormalTok{(1) Where the court is to assess the amount of costs (whether by summary or detailed assessment) it will assess those costs –}
\NormalTok{{-} (a) on the standard basis; or}
\NormalTok{{-} (b) on the indemnity basis,}

\NormalTok{but the court will not in either case allow costs which have been unreasonably incurred or are unreasonable in amount.}

\NormalTok{(2) Where the amount of costs is to be assessed on the standard basis, the court will –}
\NormalTok{{-} (a) only allow costs which are **proportionate** to the matters in issue. Costs which are disproportionate in amount may be disallowed or reduced even if they were reasonably or necessarily incurred; and}
\NormalTok{{-} (b) resolve any doubt which it may have as to whether costs were reasonably and proportionately incurred or were reasonable and proportionate in amount in favour of the paying party.}

\NormalTok{(3) Where the amount of costs is to be assessed on the indemnity basis, the court will resolve any doubt which it may have as to whether costs were **reasonably incurred or were reasonable in amount** in favour of the receiving party.}

\NormalTok{(4) Where –}
\NormalTok{{-} (a) the court makes an order about costs without indicating the basis on which the costs are to be assessed; or}
\NormalTok{{-} (b) the court makes an order for costs to be assessed on a basis other than the standard basis or the indemnity basis,}

\NormalTok{the costs will be assessed on the standard basis.}

\NormalTok{(5) Costs incurred are proportionate if they bear a reasonable relationship to –}
\NormalTok{{-} (a) the sums in issue in the proceedings;}
\NormalTok{{-} (b) the value of any non{-}monetary relief in issue in the proceedings;}
\NormalTok{{-} (c) the complexity of the litigation;}
\NormalTok{{-} (d) any additional work generated by the conduct of the paying party,}
\NormalTok{{-} (e) any wider factors involved in the proceedings, such as reputation or public importance; and}
\NormalTok{{-} (f) any additional work undertaken or expense incurred due to the vulnerability of a party or any witness.}
\end{Highlighting}
\end{Shaded}

\hypertarget{factors-r-44.43}{%
\paragraph{Factors (r 44.4(3))}\label{factors-r-44.43}}

The court considers:

\begin{itemize}
\tightlist
\item
  conduct of all the parties;
\item
  value of claim;
\item
  importance of the matter to all the parties;
\item
  complexity/ novelty of the questions raised;
\item
  skill, effort, specialised knowledge and responsibility involved;
\item
  time;
\item
  place and circumstances under which work was done;
\item
  receiving party's last approved or agreed budget.
\end{itemize}

\hypertarget{standard-basis}{%
\paragraph{Standard Basis}\label{standard-basis}}

Costs on this basis must be proportionate to the matters in issue. Costs
that are disproportionate may be disallowed or reduced even if they were
reasonably or necessarily incurred. No costs that were unreasonably
incurred or are unreasonable in amount are allowable.

\hypertarget{indemnity-basis}{%
\paragraph{Indemnity Basis}\label{indemnity-basis}}

\begin{itemize}
\tightlist
\item
  Costs must be reasonably incurred and reasonable in amount.
\item
  The judge has a wide discretion to award indemnity costs ({[}{[}Three
  Rivers DC v Governor of the Bank of England {[}2006{]} EWHC 816
  (Comm){]}{]})
\item
  Dishonesty or moral blame does not need to be established to award
  indemnity costs ({[}{[}Reid Minty v Taylor {[}2002{]} WLR 2800{]}{]}).
\item
  The conduct of experts can justify an order for indemnity costs in
  respect of costs generated by them ({[}{[}Williams v Jervis {[}2009{]}
  EWHC 1837 (QB){]}{]}).
\item
  A failure to comply with Pre-Action Protocol requirements could result
  in indemnity costs being awarded.
\item
  A refusal to mediate or engage in mediation or some other alternative
  dispute resolution process could justify an award of indemnity costs.
\item
  If a claimant casts its claim disproportionately wide, the claimant
  forfeits its right to the benefit of the doubt on reasonableness
  ({[}{[}Digicel (St Lucia) Ltd v Cable and Wireless PLC {[}2010{]} EWHC
  888 (Ch){]}{]}).
\end{itemize}

\hypertarget{difference}{%
\paragraph{Difference}\label{difference}}

Where there is no doubt over the reasonableness of costs, the difference
between the two bases is that in assessing costs on the standard basis,
the court will allow only those costs that are proportionate to the
matters in issue.

\hypertarget{conduct-and-adr}{%
\paragraph{Conduct and ADR}\label{conduct-and-adr}}

Parties must seriously consider ADR proposals made by the other side.

\begin{itemize}
\tightlist
\item
  In {[}{[}Dunnett v Railtrack plc (in Railway Administration)
  {[}2002{]} EWCA Civ 303{]}{]}, the Court of Appeal deprived the
  successful party of its costs because it unreasonably refused to
  mediate before the appeal was heard.
\item
  Guidance given in {[}{[}Halsey v Milton Keynes General NHS Trust
  {[}2004{]} EWCA Civ 576{]}{]}

  \begin{itemize}
  \tightlist
  \item
    No presumption in favour of mediation.
  \item
    Whether a party has acted unreasonably in refusing ADR must be
    determined having regard to all the circumstances, including:

    \begin{enumerate}
    \def\labelenumi{\arabic{enumi}.}
    \tightlist
    \item
      the nature of the dispute;
    \item
      the merits of the case;
    \item
      the extent to which other settlement methods have been attempted;
    \item
      whether the costs of the ADR would be disproportionately high;
    \item
      whether any delay in setting up and attending the ADR would have
      been prejudicial; and
    \item
      whether the ADR had a reasonable prospect of success.
    \end{enumerate}
  \end{itemize}
\end{itemize}

Any refusal of a proposal for ADR by a party should be based on the
Halsey principles and clearly communicated to the proposing party. So
you can't just not respond; that's unreasonable conduct.

If a party agrees to mediate but frustrates the process by delaying and
dragging its feet for no good reason, such that the other party loses
confidence in the process which then does not go ahead, that conduct
will merit a costs sanction ({[}{[}Thakkar v Patel {[}2017{]} EWCA Civ
117{]}{]}).

\hypertarget{timing-of-mediation}{%
\subparagraph{Timing of Mediation}\label{timing-of-mediation}}

\begin{itemize}
\tightlist
\item
  Premature mediation wastes time and can often lead to a hardening of
  positions.
\item
  Delaying too long means costs racked up are the biggest obstacle to
  mediation being successful.
\item
  So try to time it such that the details of claim and response are
  fully known to both sides, but costs not yet crazy.
\end{itemize}

\begin{Shaded}
\begin{Highlighting}[]
\NormalTok{title: Is a party who agrees to mediation but then causes the mediation to fail by taking an unreasonable position to be treated the same as a party who unreasonably refuses to mediate?}
\NormalTok{Yes, held in [[Earl of Malmesbury v Strutt \& Parker [2008] EWHC 424]]. But remember that ADR procedures are confidential unless all parties agree to waive confidentiality.}
\end{Highlighting}
\end{Shaded}

The increasing importance the court attaches to the consideration of ADR
is evidenced by the standard directions that require a party who rejects
a proposal for ADR to file a witness statement detailing that party's
reasons for rejecting the proposal.

\begin{quote}
`No defence, however strong, by itself justifies a failure to engage in
any kind of alternative dispute resolution'\\
-{[}{[}DSN v Blackpool Football Club Ltd {[}2020{]} EWHC 670 (QB){]}{]}
\end{quote}

\hypertarget{time}{%
\paragraph{Time}\label{time}}

Regional hourly guideline figures for the time of solicitors of
different levels of seniority in different parts of the country are set
by the Master of the Rolls. Judges have lots of experience in assessing
costs.

\hypertarget{procedure-for-assessment}{%
\subsubsection{Procedure for
Assessment}\label{procedure-for-assessment}}

The court should generally make a summary assessment of costs for a
fast-track/ one-day case, at the conclusion of the trial. Exceptions:

\begin{itemize}
\tightlist
\item
  Paying party shows substantial grounds for disputing the sum claimed
  for costs
\item
  Insufficient time to summarily assess costs.
\end{itemize}

The parties should file and serve a breakdown of their costs (Costs Form
N260) at least 24 hours before an interim hearing.

\hypertarget{fast-track-costs-part-45}{%
\subsubsection{Fast-track Costs (Part
45)}\label{fast-track-costs-part-45}}

\begin{longtable}[]{@{}
  >{\raggedright\arraybackslash}p{(\columnwidth - 2\tabcolsep) * \real{0.4272}}
  >{\raggedright\arraybackslash}p{(\columnwidth - 2\tabcolsep) * \real{0.5728}}@{}}
\toprule()
\begin{minipage}[b]{\linewidth}\raggedright
Value of the claim
\end{minipage} & \begin{minipage}[b]{\linewidth}\raggedright
Amount of fast track trial costs which the court may award
\end{minipage} \\
\midrule()
\endhead
Up to £3,000 & £485 \\
More than £3,000 but not more than £10,000 & £690 \\
More than £10,000 but not more than £15,000 & £1,035 \\
More than £15,000 & £1,650 \\
\bottomrule()
\end{longtable}

An additional £345 may be awarded where it is necessary for a legal
representative to attend to assist the advocate.

\hypertarget{valuing-claim}{%
\paragraph{Valuing Claim}\label{valuing-claim}}

\begin{itemize}
\tightlist
\item
  If the claimant succeeds, then the value of the claim is based on the
  amount awarded, excluding any interest or reduction for contributory
  negligence.
\item
  If the defendant was successful, then the value of the claim is based
  on the amount claimed by the claimant.
\item
  But the above can be varied upon unreasonable behaviour by a party.
\end{itemize}

\hypertarget{summary-assessment}{%
\paragraph{Summary Assessment}\label{summary-assessment}}

In fast track cases, therefore, the party should always file and serve a
statement of costs \(\geq 48\) hours prior to the hearing.

\hypertarget{multi-track-costs}{%
\subsubsection{Multi-track Costs}\label{multi-track-costs}}

There are no specified trial costs. Although the judge does have the
power to make a summary assessment of costs, in multi-track cases the
court will usually order \textbf{detailed assessment}.

\hypertarget{commencement}{%
\paragraph{Commencement}\label{commencement}}

Detailed assessment proceedings are commenced by the receiving party
serving on the paying party:

\begin{enumerate}
\def\labelenumi{\arabic{enumi}.}
\tightlist
\item
  Notice of commencement in Form N252
\item
  A copy of the bill of costs
\item
  Copies of the fee notes of counsel and any other expert, in respect of
  fees claimed in the bill
\item
  Written evidence as to any other disbursement claimed exceeding £500
\item
  Statement giving the name and address for service of any person upon
  whom the receiving party intends to serve notice of commencement
\item
  If a costs management order has been made, a breakdown of costs
  claimed for each phase of proceedings
\item
  Precedent Q, where a case management order has been made.
\end{enumerate}

Must do this within 3 months of the date of judgment.

\hypertarget{bill-of-costs}{%
\paragraph{Bill of Costs}\label{bill-of-costs}}

Bills of costs for detailed assessment in most Part 7 claims must be in
a prescribed electronic spreadsheet format and compliant with CPR, PD
47, paras 5.A1 to 5.A4.

\hypertarget{categories-of-work}{%
\paragraph{Categories of Work}\label{categories-of-work}}

\begin{itemize}
\tightlist
\item
  Main body of the bill comprises a breakdown of the work performed,
  divided into different categories of work set out in para 5.12 PD 47.
\item
  Set out chronologically in numbered items
\item
  Most ``communications'' (phone calls, emails out, letters out) classed
  as routine and deemed to last 6 minutes for charging purposes.

  \begin{itemize}
  \tightlist
  \item
    Particularly complex/ lengthy communications not classed as routine
    and charged according to the time actually spent on them.
  \end{itemize}
\item
  Local travel expenses not claimable, but solicitor entitled to claim
  up to the hourly rate for time spent travelling and waiting.
\item
  Cost of postage, couriers etc. generally not claimable.
\end{itemize}

A claim may only be made for the reasonable costs of preparing and
checking the bill of costs (see PD 47, para 5.19) -- but this should not
be included in the bill of costs itself.

\hypertarget{late-commencement}{%
\paragraph{Late Commencement}\label{late-commencement}}

Permission not required to commence detailed assessment proceedings out
of time, but if the receiving party waits more than 3 months, the court
may disallow interest accruing on the costs. It is open to the paying
party to apply for an order that the receiving party lose their right to
costs unless detailed assessment proceedings are commenced by a certain
date.

\hypertarget{challenging-bill}{%
\paragraph{Challenging Bill}\label{challenging-bill}}

Party has 21 days from service of notice of commencement to serve points
of dispute on the receiving party (r 47.9). If they don't, the receiving
party can apply for a default costs certificate. This can only be set
aside with good reason.

On service of the points of dispute, the receiving party may serve a
reply within 21 days (r 47.13). The receiving party must file a request
for an assessment hearing within three months of the expiry of the
period for commencing detailed assessment proceedings.

There will then be a detailed assessment hearing, at which the court
will decide what costs are\\
to be paid. The receiving party must, within 14 days of the hearing,
file a completed bill\\
showing the amount of costs finally due.

Receiving party normally entitled to the costs of the detailed
assessment proceedings, though the court will take into account conduct,
reasonableness etc., and any written offer expressed to be without
prejudice save as to costs of the detailed assessment proceedings.

No time is specified for service of such an offer, but any offer made
more than 14 days after service of the notice of commencement (paying
party)/ service of the points of dispute (receiving party) will be given
less weight unless good reason is shown.

\hypertarget{provisional-assessment}{%
\paragraph{Provisional Assessment}\label{provisional-assessment}}

Procedure applies to all proceedings for detailed assessment in the High
Court or County Court where the amount of costs claimed is less than a
prescribed limit (currently £75,000).

\begin{itemize}
\tightlist
\item
  Apply for provisional assessment through Form N258 (filing request for
  assessment)
\item
  Court will try to complete provisional assessment within 6 weeks.
\item
  Parties have 14 days to agree sum due to receiving party on the basis
  of the court's provisional assessment

  \begin{itemize}
  \tightlist
  \item
    Can make written submissions to the court.
  \item
    Can request oral hearing within 21 days.

    \begin{itemize}
    \tightlist
    \item
      Unless they secure \(\geq 20\%\) adjustment in their favour, will
      have to pay for the oral assessment.
    \end{itemize}
  \end{itemize}
\end{itemize}

\hypertarget{assessment}{%
\paragraph{Assessment}\label{assessment}}

Areas of costs typically targeted by a paying party:

\begin{enumerate}
\def\labelenumi{\arabic{enumi}.}
\tightlist
\item
  Hourly charging rate
\item
  Status of fee earner who did the work
\item
  Length of time taken to complete work
\item
  No entitlement under CPR (often argued that an item is not allowed
  under PD 47 para 5.22)
\end{enumerate}

\hypertarget{damages-based-agreements}{%
\paragraph{Damages-based Agreements}\label{damages-based-agreements}}

If a party has entered into a damages-based agreement, the costs
recoverable cannot exceed\\
the amount payable by that party under the agreement (r 44.18).

\hypertarget{capped-costs-pilot-scheme}{%
\paragraph{Capped Costs Pilot Scheme}\label{capped-costs-pilot-scheme}}

In 2019, a Capped Costs Pilot Scheme started in selected Business and
Property Courts. The specific rules and procedure are set out in PD 51W.
Designed for claims of up to £250,000 and trial \(\leq 2\) days. Scheme
is voluntary.

\hypertarget{appeals}{%
\paragraph{Appeals}\label{appeals}}

Where the assessment is by a judge, the appeals process is governed by
Part 52.

MERMAID1

If permission is not sought at the original assessment, it must be
sought from the appeal court within 21 days. The appeal takes the form
of a review of original decision, not a re-hearing.

Alternatively,

MERMAID2

\hypertarget{agreeing-costs}{%
\paragraph{Agreeing Costs}\label{agreeing-costs}}

Very often, the parties will attempt to agree a figure for costs, and
proceed to a detailed assessment only if they are unable to reach
agreement.

\hypertarget{offers-to-settle-detailed-assessment}{%
\paragraph{Offers to Settle Detailed
Assessment}\label{offers-to-settle-detailed-assessment}}

PD 47, para 8.3: the paying party must state in an open letter
accompanying the points of dispute what sum, if any, that party offers
to pay in settlement of the total costs claimed. The paying party can
also make an offer under Part 36.

\hypertarget{interim-orders}{%
\paragraph{Interim Orders}\label{interim-orders}}

Rule 44.3(8): allows the court at trial or on a later application to
order an interim payment of part of these costs (since detailed
assessment usually takes a while). This should usually be ordered
({[}{[}Mars UK Ltd v Teknowledge Ltd (No 2) {[}1999{]} {]}{]}).

\hypertarget{costs-only-proceedings}{%
\subsubsection{Costs Only Proceedings}\label{costs-only-proceedings}}

r 46.14: either party may start proceedings by issuing a claim form in
accordance with Part 8. This should include the agreement the parties
have reached as to liability and quantum.

\hypertarget{practical}{%
\subsubsection{Practical}\label{practical}}

\begin{Shaded}
\begin{Highlighting}[]
\NormalTok{When scrutinising an opponent’s bill of costs, consider the following:}
\NormalTok{1) Check the scope of the costs order}
\NormalTok{2) What is the basis of assessment – standard or indemnity? Where standard, consider overall proportionality first and identify the correct tests.}
\NormalTok{3) Interim costs orders?}
\NormalTok{4) Check the costs budgets/CMOs. Is there a 20\% or more difference?}
\NormalTok{5) Compare solicitors’ hourly rates and counsels’ fees with the summary assessment rates.}
\NormalTok{6) Check compliance with PD 47 as to content. Are all items claimed allowable?}
\end{Highlighting}
\end{Shaded}

\hypertarget{enforcing-money-judgments}{%
\section{Enforcing Money Judgments}\label{enforcing-money-judgments}}

\hypertarget{introduction}{%
\subsection{Introduction}\label{introduction}}

Once a party has obtained a judgment against their opponent, the
opponent will usually pay the amount ordered without any further
enforcement action necessary. If they do not, the receiving party can
take enforcement action.

If the opponent is not insured, enforcement should be considered before
proceedings are commenced, since it is not worth obtaining a judgment
against a party who does not have means to pay.

A client may be able to claim under the terms of their own household
insurance for an unsatisfied judgment in respect of personal injury,
damage to property or a fatal accident claim.

\hypertarget{partnership-property}{%
\subsubsection{Partnership Property}\label{partnership-property}}

Practice Direction 70, para 6A.1 provides that a judgment made against a
partnership may be enforced against any property of the partnership
within the jurisdiction.

A judgment against a partnership can be enforced against any partner if
the partner:

\begin{itemize}
\tightlist
\item
  Acknowledged service of the claim form as partner
\item
  Failed to acknowledge service of the claim form, having being served
  as partner
\item
  Admitted in their statement that they were a partner at the material
  time
\item
  Found by the court to have been a partner at the material time.
\end{itemize}

Proceedings can be taken against partnership property for a partner's
separate judgment\\
debt (s 23 Partnership Act 1890).

\hypertarget{interest-on-judgment-debts}{%
\subsection{Interest on Judgment
Debts}\label{interest-on-judgment-debts}}

\hypertarget{high-court-judgments}{%
\subsubsection{High Court Judgments}\label{high-court-judgments}}

Interest accrues on all High Court judgments from the date judgment is
pronounced at 8\% pa (Judgments Act 1838).

\begin{itemize}
\tightlist
\item
  Where judgment is entered for damages to be assessed, interest begins
  to run from the date when damages are finally assessed or agreed
\item
  Interest on an order for the payment of costs runs from the date of
  the judgment, not from the date of the final costs certificate
  following a detailed assessment (so make a payment on account).
\end{itemize}

\hypertarget{county-court-judgments}{%
\subsubsection{County Court Judgments}\label{county-court-judgments}}

\begin{itemize}
\tightlist
\item
  s 74 of the CCA 1984, interest accrues on County Court judgments of
  £5,000 or more, at 8\% pa.
\item
  Where payment is deferred/ by instalments, interest will not accrue
  until that date/ until an instalment falls due.
\item
  Where enforcement proceedings are taken, the judgment debt ceases to
  carry interest unless the enforcement proceedings fail to produce any
  payment, in which case interest will continue to accrue on the
  judgment debt as if the enforcement proceedings had never been issued.
\item
  If enforcement proceedings are taken and anything at all is recovered,
  the balance of the debt will become interest free.
\end{itemize}

\hypertarget{tracing-other-party}{%
\subsection{Tracing Other Party}\label{tracing-other-party}}

Consider employing an enquiry agent to trace the other party, with a
limit on the costs that may be incurred by them. Enquiries likely to be
made at the outset of the case.

\hypertarget{investigating-debtors-means}{%
\subsection{Investigating Debtor's
Means}\label{investigating-debtors-means}}

To investigate the judgment debtor's assets, either:

\begin{enumerate}
\def\labelenumi{\arabic{enumi}.}
\tightlist
\item
  Instruct an enquiry agent, or

  \begin{itemize}
  \tightlist
  \item
    Likely to be quicker but more expensive than applying to court.
  \end{itemize}
\item
  Apply to the court for an order to obtain information from the
  judgment debtor.

  \begin{itemize}
  \tightlist
  \item
    Part 71
  \item
    This is a court order requiring the judgment debtor to attend before
    an officer of the court to be examined on oath as to their means.
  \item
    To make an application:

    \begin{itemize}
    \tightlist
    \item
      Judgment creditor filed form N316 (for individual debtor) or N316A
      (officer of company/ corporation is to be questioned).
    \item
      PD 71 para 1 sets out matters to be contained in the application
      notice (name and address of debtor, judgment sought to be
      enforced, amount owed).
    \item
      r 71.3 \& 71.4 procedure for service of the order and for payment
      of the judgment debtor's travel expenses.
    \end{itemize}
  \item
    Hearing

    \begin{itemize}
    \tightlist
    \item
      County Court hearing centre near to debtor
    \item
      Conducted by officer of court/ judge
    \item
      Standard questions asked if officer of court; the creditor can
      also add some in their application notice/ ask in person.
    \item
      Fixed costs may be awarded if solicitor attends hearing (r 45.8)
    \item
      Judge will not use standard questions, meeting will be tape
      recorded, judge may make a summary assessment (r 44.6)
    \item
      If debtor fails to attend/ take an oath/ answer questions, a
      committal order may be made against them.
    \end{itemize}
  \end{itemize}
\end{enumerate}

Recall that pre-action it is usual to make bankruptcy/ company searches.

A person can be compelled for examination as a judgment debtor if the
only outstanding parts of the judgment against them are costs orders for
sums which have yet to be agreed or determined by the assessment
({[}{[}Nagel (a firm) v Pluczenik Diamond Company NV {[}2019{]} EWHC
3126 (QB){]}{]}).

\hypertarget{enforcement-methods}{%
\subsection{Enforcement Methods}\label{enforcement-methods}}

4 common methods:

\begin{enumerate}
\def\labelenumi{\arabic{enumi}.}
\tightlist
\item
  Seizure and sale of debtor's goods
\item
  Charging order (charge over debtor's land/ securities)
\item
  Third party debt order (requiring 3rd party owing debtor money to pay
  it directly to the creditor)
\item
  Attachment of earnings order (requiring debtor's employer to make
  payments direct to the creditor).
\end{enumerate}

\hypertarget{taking-control-of-goods}{%
\subsubsection{Taking Control of Goods}\label{taking-control-of-goods}}

\begin{itemize}
\tightlist
\item
  Part 83: allows an enforcement agent/ officer to seize and sell the
  debtor's personal goods to pay the judgment debt and costs.
\item
  Sold at public auction.
\item
  Cannot force entry into the debtor's home, though can break into
  business premises.
\item
  Choice of court

  \begin{itemize}
  \tightlist
  \item
    A party who obtained judgment in the High Court may issue a writ of
    control in that court.
  \item
    Where a party has obtained judgment in the County Court, and the
    amount to be enforced is £5,000 or more, it must be enforced in the
    High Court unless the proceedings originated under the Consumer
    Credit Act 1974.
  \item
    Where the sum to be enforced is less than £600, it must be enforced
    in a County Court.
  \item
    So for \(£600 \leq P < £5,000\), judgment creditor can elect the
    court. If choosing High Court, the County Court judgment must first
    be transferred to the High Court, but this has the advantage that
    interest accrues on the judgment debt.
  \item
    To enforce in the County Court, the party must apply for a warrant
    of control.
  \end{itemize}
\item
  Exempt items

  \begin{itemize}
  \tightlist
  \item
    Goods on hire or hire-purchase
  \item
    Tools, books, phones, computers, vehicles and other items necessary
    for the debtor to use personally in their job/ trade/ profession/
    study, subject to £1,350 maximum aggregate value
  \item
    Clothing, bedding, furniture etc. reasonably required for satisfying
    basic domestic needs of the debtor and their family.
  \item
    Exemptions only apply to judgment debts against individuals.
  \item
    Usually motor vehicles can be seized, as can household items such as
    stereos, TVs etc.
  \item
    Control may be taken by goods owned by the judgment debtor jointly
    with another person (common law).
  \end{itemize}
\item
  Controlled goods agreement

  \begin{itemize}
  \tightlist
  \item
    Usually the debtor enters into an agreement, where the enforcement
    officer agrees not to remove the goods at once, and the debtor
    agrees not to dispose of them or permit them to be moved. Gives the
    debtor a further opportunity to pay the sum due.
  \end{itemize}
\end{itemize}

\hypertarget{charging-order-on-land}{%
\subsubsection{Charging Order on Land}\label{charging-order-on-land}}

A judgment creditor may apply to the court for an order charging the
judgment debtor's land with the amount due under a judgment. A charging
order can also be made in respect of land the debtor owns jointly with
another person (charge upon the debtor's beneficial interest).

\begin{itemize}
\tightlist
\item
  Restrictions

  \begin{itemize}
  \tightlist
  \item
    Where there is a plan in place for the debtor to pay a sum by
    instalments, a charging order can be made even though there has been
    no default in payments.
  \item
    The court must take the lack of default into account in deciding
    whether or not to make the order.
  \end{itemize}
\item
  Registration

  \begin{itemize}
  \tightlist
  \item
    When a charging order has been made, should be registered if
    possible (at the Land Registry).
  \item
    Where the land is jointly owned, the debtor technically has an
    interest only in the proceeds of sale under the trust for sale,
    rather than an interest in the land itself. This should be protected
    by a restriction on the register.
  \end{itemize}
\item
  Notice

  \begin{itemize}
  \tightlist
  \item
    Whether or not the charging order is registered, written notice of
    it should be given to any prior chargee(s) to prevent any tacking of
    later advances.
  \end{itemize}
\item
  Order for sale

  \begin{itemize}
  \tightlist
  \item
    To obtain the money, the creditor can apply to the court for an
    order for sale of the land charged.
  \item
    The judgment will then be satisfied out of the proceeds of sale.
  \end{itemize}
\end{itemize}

\hypertarget{charging-order-of-securities}{%
\subsubsection{Charging Order of
Securities}\label{charging-order-of-securities}}

\begin{itemize}
\tightlist
\item
  A judgment creditor may also obtain a charging order on a judgment
  debtor's beneficial interest in certain specified securities.
\item
  The court may order that the charge extends to any dividend that is
  payable.
\item
  Apply with Form N380
\end{itemize}

\hypertarget{third-party-debt-orders}{%
\subsubsection{Third Party Debt Orders}\label{third-party-debt-orders}}

\begin{itemize}
\tightlist
\item
  Where a third party owes money to the judgment debtor, the court can
  make an order requiring that third party to pay the judgment creditor
  the whole of that debt, or such part of it as is sufficient to satisfy
  the judgment debt and costs.
\item
  The debt must belong to the judgment debtor solely and beneficially.
  An order cannot be made against joint debts unless all the people owed
  the debt are joint judgment debtors.
\item
  The third party must be within the jurisdiction.
\item
  Choice of court

  \begin{itemize}
  \tightlist
  \item
    Generally, County Court hearing centre.
  \end{itemize}
\item
  Costs

  \begin{itemize}
  \tightlist
  \item
    Costs are fixed (Part 45) and may be retained out of money recovered
    from the 3rd party.
  \end{itemize}
\end{itemize}

\hypertarget{attachment-of-earnings}{%
\subsubsection{Attachment of Earnings}\label{attachment-of-earnings}}

An attachment of earnings order is an order which compels the judgment
debtor's employer to make regular deductions from the debtor's earnings
and pay them into court. The High Court has no power to make an
attachment of earnings order. If the judgment has been obtained in the
High Court, the proceedings will have to be transferred to the County
Court before this method of enforcement can be used.

\hypertarget{conditions}{%
\paragraph{Conditions}\label{conditions}}

\begin{itemize}
\tightlist
\item
  The amount remaining due under the judgment must be at least £50
\item
  The debtor must be employed (not self-employed)
\item
  Debtor must be an individual.
\end{itemize}

\hypertarget{insolvency}{%
\subsubsection{Insolvency}\label{insolvency}}

Where the judgment debt is for £5,000 or more, the judgment creditor may
decide to petition for bankruptcy of the judgment debtor. However, they
will not be able to do this if they have already registered a charging
order because they are then in the position of a secured creditor, and a
secured creditor cannot petition for bankruptcy unless they give up
their security.

If a bankruptcy order is made, all the debtor's property vests in the
trustee in bankruptcy.

\hypertarget{winding-up}{%
\subsubsection{Winding up}\label{winding-up}}

If the judgment debtor is a company, the judgment creditor may consider
winding up the company if the debt is £750 or more.

\end{document}
