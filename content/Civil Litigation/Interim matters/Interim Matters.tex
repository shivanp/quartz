% Options for packages loaded elsewhere
\PassOptionsToPackage{unicode}{hyperref}
\PassOptionsToPackage{hyphens}{url}
%
\documentclass[
]{article}
\usepackage{amsmath,amssymb}
\usepackage{lmodern}
\usepackage{iftex}
\ifPDFTeX
  \usepackage[T1]{fontenc}
  \usepackage[utf8]{inputenc}
  \usepackage{textcomp} % provide euro and other symbols
\else % if luatex or xetex
  \usepackage{unicode-math}
  \defaultfontfeatures{Scale=MatchLowercase}
  \defaultfontfeatures[\rmfamily]{Ligatures=TeX,Scale=1}
\fi
% Use upquote if available, for straight quotes in verbatim environments
\IfFileExists{upquote.sty}{\usepackage{upquote}}{}
\IfFileExists{microtype.sty}{% use microtype if available
  \usepackage[]{microtype}
  \UseMicrotypeSet[protrusion]{basicmath} % disable protrusion for tt fonts
}{}
\makeatletter
\@ifundefined{KOMAClassName}{% if non-KOMA class
  \IfFileExists{parskip.sty}{%
    \usepackage{parskip}
  }{% else
    \setlength{\parindent}{0pt}
    \setlength{\parskip}{6pt plus 2pt minus 1pt}}
}{% if KOMA class
  \KOMAoptions{parskip=half}}
\makeatother
\usepackage{xcolor}
\usepackage[margin=1in]{geometry}
\usepackage{color}
\usepackage{fancyvrb}
\newcommand{\VerbBar}{|}
\newcommand{\VERB}{\Verb[commandchars=\\\{\}]}
\DefineVerbatimEnvironment{Highlighting}{Verbatim}{commandchars=\\\{\}}
% Add ',fontsize=\small' for more characters per line
\newenvironment{Shaded}{}{}
\newcommand{\AlertTok}[1]{\textcolor[rgb]{1.00,0.00,0.00}{\textbf{#1}}}
\newcommand{\AnnotationTok}[1]{\textcolor[rgb]{0.38,0.63,0.69}{\textbf{\textit{#1}}}}
\newcommand{\AttributeTok}[1]{\textcolor[rgb]{0.49,0.56,0.16}{#1}}
\newcommand{\BaseNTok}[1]{\textcolor[rgb]{0.25,0.63,0.44}{#1}}
\newcommand{\BuiltInTok}[1]{#1}
\newcommand{\CharTok}[1]{\textcolor[rgb]{0.25,0.44,0.63}{#1}}
\newcommand{\CommentTok}[1]{\textcolor[rgb]{0.38,0.63,0.69}{\textit{#1}}}
\newcommand{\CommentVarTok}[1]{\textcolor[rgb]{0.38,0.63,0.69}{\textbf{\textit{#1}}}}
\newcommand{\ConstantTok}[1]{\textcolor[rgb]{0.53,0.00,0.00}{#1}}
\newcommand{\ControlFlowTok}[1]{\textcolor[rgb]{0.00,0.44,0.13}{\textbf{#1}}}
\newcommand{\DataTypeTok}[1]{\textcolor[rgb]{0.56,0.13,0.00}{#1}}
\newcommand{\DecValTok}[1]{\textcolor[rgb]{0.25,0.63,0.44}{#1}}
\newcommand{\DocumentationTok}[1]{\textcolor[rgb]{0.73,0.13,0.13}{\textit{#1}}}
\newcommand{\ErrorTok}[1]{\textcolor[rgb]{1.00,0.00,0.00}{\textbf{#1}}}
\newcommand{\ExtensionTok}[1]{#1}
\newcommand{\FloatTok}[1]{\textcolor[rgb]{0.25,0.63,0.44}{#1}}
\newcommand{\FunctionTok}[1]{\textcolor[rgb]{0.02,0.16,0.49}{#1}}
\newcommand{\ImportTok}[1]{#1}
\newcommand{\InformationTok}[1]{\textcolor[rgb]{0.38,0.63,0.69}{\textbf{\textit{#1}}}}
\newcommand{\KeywordTok}[1]{\textcolor[rgb]{0.00,0.44,0.13}{\textbf{#1}}}
\newcommand{\NormalTok}[1]{#1}
\newcommand{\OperatorTok}[1]{\textcolor[rgb]{0.40,0.40,0.40}{#1}}
\newcommand{\OtherTok}[1]{\textcolor[rgb]{0.00,0.44,0.13}{#1}}
\newcommand{\PreprocessorTok}[1]{\textcolor[rgb]{0.74,0.48,0.00}{#1}}
\newcommand{\RegionMarkerTok}[1]{#1}
\newcommand{\SpecialCharTok}[1]{\textcolor[rgb]{0.25,0.44,0.63}{#1}}
\newcommand{\SpecialStringTok}[1]{\textcolor[rgb]{0.73,0.40,0.53}{#1}}
\newcommand{\StringTok}[1]{\textcolor[rgb]{0.25,0.44,0.63}{#1}}
\newcommand{\VariableTok}[1]{\textcolor[rgb]{0.10,0.09,0.49}{#1}}
\newcommand{\VerbatimStringTok}[1]{\textcolor[rgb]{0.25,0.44,0.63}{#1}}
\newcommand{\WarningTok}[1]{\textcolor[rgb]{0.38,0.63,0.69}{\textbf{\textit{#1}}}}
\usepackage{longtable,booktabs,array}
\usepackage{calc} % for calculating minipage widths
% Correct order of tables after \paragraph or \subparagraph
\usepackage{etoolbox}
\makeatletter
\patchcmd\longtable{\par}{\if@noskipsec\mbox{}\fi\par}{}{}
\makeatother
% Allow footnotes in longtable head/foot
\IfFileExists{footnotehyper.sty}{\usepackage{footnotehyper}}{\usepackage{footnote}}
\makesavenoteenv{longtable}
\setlength{\emergencystretch}{3em} % prevent overfull lines
\providecommand{\tightlist}{%
  \setlength{\itemsep}{0pt}\setlength{\parskip}{0pt}}
\setcounter{secnumdepth}{-\maxdimen} % remove section numbering
\usepackage{xcolor}
\definecolor{aliceblue}{HTML}{F0F8FF}
\definecolor{antiquewhite}{HTML}{FAEBD7}
\definecolor{aqua}{HTML}{00FFFF}
\definecolor{aquamarine}{HTML}{7FFFD4}
\definecolor{azure}{HTML}{F0FFFF}
\definecolor{beige}{HTML}{F5F5DC}
\definecolor{bisque}{HTML}{FFE4C4}
\definecolor{black}{HTML}{000000}
\definecolor{blanchedalmond}{HTML}{FFEBCD}
\definecolor{blue}{HTML}{0000FF}
\definecolor{blueviolet}{HTML}{8A2BE2}
\definecolor{brown}{HTML}{A52A2A}
\definecolor{burlywood}{HTML}{DEB887}
\definecolor{cadetblue}{HTML}{5F9EA0}
\definecolor{chartreuse}{HTML}{7FFF00}
\definecolor{chocolate}{HTML}{D2691E}
\definecolor{coral}{HTML}{FF7F50}
\definecolor{cornflowerblue}{HTML}{6495ED}
\definecolor{cornsilk}{HTML}{FFF8DC}
\definecolor{crimson}{HTML}{DC143C}
\definecolor{cyan}{HTML}{00FFFF}
\definecolor{darkblue}{HTML}{00008B}
\definecolor{darkcyan}{HTML}{008B8B}
\definecolor{darkgoldenrod}{HTML}{B8860B}
\definecolor{darkgray}{HTML}{A9A9A9}
\definecolor{darkgreen}{HTML}{006400}
\definecolor{darkgrey}{HTML}{A9A9A9}
\definecolor{darkkhaki}{HTML}{BDB76B}
\definecolor{darkmagenta}{HTML}{8B008B}
\definecolor{darkolivegreen}{HTML}{556B2F}
\definecolor{darkorange}{HTML}{FF8C00}
\definecolor{darkorchid}{HTML}{9932CC}
\definecolor{darkred}{HTML}{8B0000}
\definecolor{darksalmon}{HTML}{E9967A}
\definecolor{darkseagreen}{HTML}{8FBC8F}
\definecolor{darkslateblue}{HTML}{483D8B}
\definecolor{darkslategray}{HTML}{2F4F4F}
\definecolor{darkslategrey}{HTML}{2F4F4F}
\definecolor{darkturquoise}{HTML}{00CED1}
\definecolor{darkviolet}{HTML}{9400D3}
\definecolor{deeppink}{HTML}{FF1493}
\definecolor{deepskyblue}{HTML}{00BFFF}
\definecolor{dimgray}{HTML}{696969}
\definecolor{dimgrey}{HTML}{696969}
\definecolor{dodgerblue}{HTML}{1E90FF}
\definecolor{firebrick}{HTML}{B22222}
\definecolor{floralwhite}{HTML}{FFFAF0}
\definecolor{forestgreen}{HTML}{228B22}
\definecolor{fuchsia}{HTML}{FF00FF}
\definecolor{gainsboro}{HTML}{DCDCDC}
\definecolor{ghostwhite}{HTML}{F8F8FF}
\definecolor{gold}{HTML}{FFD700}
\definecolor{goldenrod}{HTML}{DAA520}
\definecolor{gray}{HTML}{808080}
\definecolor{green}{HTML}{008000}
\definecolor{greenyellow}{HTML}{ADFF2F}
\definecolor{grey}{HTML}{808080}
\definecolor{honeydew}{HTML}{F0FFF0}
\definecolor{hotpink}{HTML}{FF69B4}
\definecolor{indianred}{HTML}{CD5C5C}
\definecolor{indigo}{HTML}{4B0082}
\definecolor{ivory}{HTML}{FFFFF0}
\definecolor{khaki}{HTML}{F0E68C}
\definecolor{lavender}{HTML}{E6E6FA}
\definecolor{lavenderblush}{HTML}{FFF0F5}
\definecolor{lawngreen}{HTML}{7CFC00}
\definecolor{lemonchiffon}{HTML}{FFFACD}
\definecolor{lightblue}{HTML}{ADD8E6}
\definecolor{lightcoral}{HTML}{F08080}
\definecolor{lightcyan}{HTML}{E0FFFF}
\definecolor{lightgoldenrodyellow}{HTML}{FAFAD2}
\definecolor{lightgray}{HTML}{D3D3D3}
\definecolor{lightgreen}{HTML}{90EE90}
\definecolor{lightgrey}{HTML}{D3D3D3}
\definecolor{lightpink}{HTML}{FFB6C1}
\definecolor{lightsalmon}{HTML}{FFA07A}
\definecolor{lightseagreen}{HTML}{20B2AA}
\definecolor{lightskyblue}{HTML}{87CEFA}
\definecolor{lightslategray}{HTML}{778899}
\definecolor{lightslategrey}{HTML}{778899}
\definecolor{lightsteelblue}{HTML}{B0C4DE}
\definecolor{lightyellow}{HTML}{FFFFE0}
\definecolor{lime}{HTML}{00FF00}
\definecolor{limegreen}{HTML}{32CD32}
\definecolor{linen}{HTML}{FAF0E6}
\definecolor{magenta}{HTML}{FF00FF}
\definecolor{maroon}{HTML}{800000}
\definecolor{mediumaquamarine}{HTML}{66CDAA}
\definecolor{mediumblue}{HTML}{0000CD}
\definecolor{mediumorchid}{HTML}{BA55D3}
\definecolor{mediumpurple}{HTML}{9370DB}
\definecolor{mediumseagreen}{HTML}{3CB371}
\definecolor{mediumslateblue}{HTML}{7B68EE}
\definecolor{mediumspringgreen}{HTML}{00FA9A}
\definecolor{mediumturquoise}{HTML}{48D1CC}
\definecolor{mediumvioletred}{HTML}{C71585}
\definecolor{midnightblue}{HTML}{191970}
\definecolor{mintcream}{HTML}{F5FFFA}
\definecolor{mistyrose}{HTML}{FFE4E1}
\definecolor{moccasin}{HTML}{FFE4B5}
\definecolor{navajowhite}{HTML}{FFDEAD}
\definecolor{navy}{HTML}{000080}
\definecolor{oldlace}{HTML}{FDF5E6}
\definecolor{olive}{HTML}{808000}
\definecolor{olivedrab}{HTML}{6B8E23}
\definecolor{orange}{HTML}{FFA500}
\definecolor{orangered}{HTML}{FF4500}
\definecolor{orchid}{HTML}{DA70D6}
\definecolor{palegoldenrod}{HTML}{EEE8AA}
\definecolor{palegreen}{HTML}{98FB98}
\definecolor{paleturquoise}{HTML}{AFEEEE}
\definecolor{palevioletred}{HTML}{DB7093}
\definecolor{papayawhip}{HTML}{FFEFD5}
\definecolor{peachpuff}{HTML}{FFDAB9}
\definecolor{peru}{HTML}{CD853F}
\definecolor{pink}{HTML}{FFC0CB}
\definecolor{plum}{HTML}{DDA0DD}
\definecolor{powderblue}{HTML}{B0E0E6}
\definecolor{purple}{HTML}{800080}
\definecolor{red}{HTML}{FF0000}
\definecolor{rosybrown}{HTML}{BC8F8F}
\definecolor{royalblue}{HTML}{4169E1}
\definecolor{saddlebrown}{HTML}{8B4513}
\definecolor{salmon}{HTML}{FA8072}
\definecolor{sandybrown}{HTML}{F4A460}
\definecolor{seagreen}{HTML}{2E8B57}
\definecolor{seashell}{HTML}{FFF5EE}
\definecolor{sienna}{HTML}{A0522D}
\definecolor{silver}{HTML}{C0C0C0}
\definecolor{skyblue}{HTML}{87CEEB}
\definecolor{slateblue}{HTML}{6A5ACD}
\definecolor{slategray}{HTML}{708090}
\definecolor{slategrey}{HTML}{708090}
\definecolor{snow}{HTML}{FFFAFA}
\definecolor{springgreen}{HTML}{00FF7F}
\definecolor{steelblue}{HTML}{4682B4}
\definecolor{tan}{HTML}{D2B48C}
\definecolor{teal}{HTML}{008080}
\definecolor{thistle}{HTML}{D8BFD8}
\definecolor{tomato}{HTML}{FF6347}
\definecolor{turquoise}{HTML}{40E0D0}
\definecolor{violet}{HTML}{EE82EE}
\definecolor{wheat}{HTML}{F5DEB3}
\definecolor{white}{HTML}{FFFFFF}
\definecolor{whitesmoke}{HTML}{F5F5F5}
\definecolor{yellow}{HTML}{FFFF00}
\definecolor{yellowgreen}{HTML}{9ACD32}
\usepackage[most]{tcolorbox}

\usepackage{ifthen}
\provideboolean{admonitiontwoside}
\makeatletter%
\if@twoside%
\setboolean{admonitiontwoside}{true}
\else%
\setboolean{admonitiontwoside}{false}
\fi%
\makeatother%

\newenvironment{env-0576f49b-f093-4961-b78b-d63ea3e3ec78}
{
    \savenotes\tcolorbox[blanker,breakable,left=5pt,borderline west={2pt}{-4pt}{firebrick}]
}
{
    \endtcolorbox\spewnotes
}
                

\newenvironment{env-0505cb41-9007-48c6-b494-2e1b2ff19f7f}
{
    \savenotes\tcolorbox[blanker,breakable,left=5pt,borderline west={2pt}{-4pt}{blue}]
}
{
    \endtcolorbox\spewnotes
}
                

\newenvironment{env-771e922a-337a-48da-a300-f790c3037606}
{
    \savenotes\tcolorbox[blanker,breakable,left=5pt,borderline west={2pt}{-4pt}{green}]
}
{
    \endtcolorbox\spewnotes
}
                

\newenvironment{env-b3fb6151-a170-4374-b44b-9e6e15984be8}
{
    \savenotes\tcolorbox[blanker,breakable,left=5pt,borderline west={2pt}{-4pt}{aquamarine}]
}
{
    \endtcolorbox\spewnotes
}
                

\newenvironment{env-94be0e31-27aa-4465-8a10-d36eab71a4ae}
{
    \savenotes\tcolorbox[blanker,breakable,left=5pt,borderline west={2pt}{-4pt}{orange}]
}
{
    \endtcolorbox\spewnotes
}
                

\newenvironment{env-b84f810a-fa40-423b-90c8-5dedd19af475}
{
    \savenotes\tcolorbox[blanker,breakable,left=5pt,borderline west={2pt}{-4pt}{blue}]
}
{
    \endtcolorbox\spewnotes
}
                

\newenvironment{env-909da91a-4dc1-4cbb-a646-1fc89a632a23}
{
    \savenotes\tcolorbox[blanker,breakable,left=5pt,borderline west={2pt}{-4pt}{yellow}]
}
{
    \endtcolorbox\spewnotes
}
                

\newenvironment{env-0848aace-ca3d-48e0-af52-2b0ab5591cd9}
{
    \savenotes\tcolorbox[blanker,breakable,left=5pt,borderline west={2pt}{-4pt}{darkred}]
}
{
    \endtcolorbox\spewnotes
}
                

\newenvironment{env-5004d97c-cd6c-476a-9c19-78c2461bc496}
{
    \savenotes\tcolorbox[blanker,breakable,left=5pt,borderline west={2pt}{-4pt}{pink}]
}
{
    \endtcolorbox\spewnotes
}
                

\newenvironment{env-5beae96b-d64b-4c70-91a2-faf0839de56e}
{
    \savenotes\tcolorbox[blanker,breakable,left=5pt,borderline west={2pt}{-4pt}{cyan}]
}
{
    \endtcolorbox\spewnotes
}
                

\newenvironment{env-62e5aa95-2800-4cb4-8392-99d6c5bc1f21}
{
    \savenotes\tcolorbox[blanker,breakable,left=5pt,borderline west={2pt}{-4pt}{cyan}]
}
{
    \endtcolorbox\spewnotes
}
                

\newenvironment{env-cddba61e-d72d-48e1-abd7-9e45a325c713}
{
    \savenotes\tcolorbox[blanker,breakable,left=5pt,borderline west={2pt}{-4pt}{purple}]
}
{
    \endtcolorbox\spewnotes
}
                

\newenvironment{env-1570d78f-cf36-4846-ba4a-6672f79ee5c8}
{
    \savenotes\tcolorbox[blanker,breakable,left=5pt,borderline west={2pt}{-4pt}{darksalmon}]
}
{
    \endtcolorbox\spewnotes
}
                

\newenvironment{env-f62d5e0b-7b81-43c8-9cfe-3d798dbacc47}
{
    \savenotes\tcolorbox[blanker,breakable,left=5pt,borderline west={2pt}{-4pt}{gray}]
}
{
    \endtcolorbox\spewnotes
}
                
\ifLuaTeX
  \usepackage{selnolig}  % disable illegal ligatures
\fi
\IfFileExists{bookmark.sty}{\usepackage{bookmark}}{\usepackage{hyperref}}
\IfFileExists{xurl.sty}{\usepackage{xurl}}{} % add URL line breaks if available
\urlstyle{same} % disable monospaced font for URLs
\hypersetup{
  hidelinks,
  pdfcreator={LaTeX via pandoc}}

\author{}
\date{}

\begin{document}

{
\setcounter{tocdepth}{3}
\tableofcontents
}
\begin{Shaded}
\begin{Highlighting}[]
\NormalTok{min\_depth: 1}
\end{Highlighting}
\end{Shaded}

\hypertarget{settlement}{%
\section{Settlement}\label{settlement}}

\hypertarget{pre-action-settlements}{%
\subsection{Pre-action Settlements}\label{pre-action-settlements}}

\hypertarget{costs-and-interest}{%
\subsubsection{Costs and Interest}\label{costs-and-interest}}

When a settlement is reached prior to the issue of proceedings, the
prospective claimant will not be entitled to recover their legal costs
unless this has been agreed. Neither will they be entitled to interest
on any sum agreed under s 69 of the CCA 1984, or s 35A of the SCA 1981.

There may, however, be an entitlement to interest under contract or the
Late Payment of Commercial Debts (Interest) Act 1998, and such should be
taken into account during any negotiations and as part of any
settlement.

\hypertarget{recording-pre-action-settlement}{%
\subsubsection{Recording Pre-action
Settlement}\label{recording-pre-action-settlement}}

Record settlement terms in writing for safety. An exchange of
correspondence is usually fine, but for complicated stuff sign a formal
settlement agreement.

Take care to ensure that all necessary terms are included in any offer
to settle ({[}{[}Evans v Trebuchet Design Ltd {[}2020{]} EWHC 3037
(IPEC){]}{]}). Remember non-disclosure clauses, if applicable.

\hypertarget{settlements-after-proceedings}{%
\subsection{Settlements After
Proceedings}\label{settlements-after-proceedings}}

Preferable for the settlement to be recorded in a court order or
judgment, to make enforcement easier. The date by which debt/ damages
must be paid should be specified in the judgment/ order. Unless the
settlement provides otherwise, interest is not payable on costs until
judgment is entered.

\hypertarget{consent-orders-or-judgments}{%
\subsubsection{Consent Orders or
Judgments}\label{consent-orders-or-judgments}}

Where none of the parties is a litigant in person, it will often be
possible to avoid an application to the court by drawing up a consent
order or judgment for sealing by a court officer under r 40.6.

The court has theoretical power not to approve a proposed order, but
will only be referred to a judge if it appears incorrect/ unclear.

\begin{Shaded}
\begin{Highlighting}[]
\NormalTok{title: Formalities for a consent order: r 40.6(7)}
\NormalTok{1. the order agreed by the parties must be drawn up in the terms agreed;}
\NormalTok{2. it must be expressed as being ‘By Consent’;}
\NormalTok{3. it must be signed by the legal representative acting for each of the parties to whom the order relates.}
\end{Highlighting}
\end{Shaded}

\begin{Shaded}
\begin{Highlighting}[]
\NormalTok{1. The terms of a consent order will be open to public inspection. }
\NormalTok{2. The terms agreed must be within the powers of a court to order. }
\end{Highlighting}
\end{Shaded}

If the parties want any terms to be confidential and/or beyond the
powers of a court to order, they should use a Tomlin order.

\hypertarget{tomlin-orders}{%
\subsubsection{Tomlin Orders}\label{tomlin-orders}}

A Tomlin order \textbf{stays the claim on agreed terms} that are set out
in a schedule to the order, an agreement annexed to the order or a
separate document.

\begin{Shaded}
\begin{Highlighting}[]
\NormalTok{The key to drafting a Tomlin order correctly is to appreciate that certain terms must appear in the order itself, whilst others can be put in the schedule, agreement or separate document.}
\end{Highlighting}
\end{Shaded}

which comes from:

\begin{Shaded}
\begin{Highlighting}[]
\NormalTok{title: PD 40B, para 3.5}
\NormalTok{Where the parties draw up a consent order in the form of a stay of proceedings on agreed terms, disposing of the proceedings, and where the terms are recorded in a schedule to the order, any direction for:}
\NormalTok{1. payment of money out of court, or}
\NormalTok{2. payment and assessment of costs}

\NormalTok{should be contained in the body of the order and not in the schedule.}
\end{Highlighting}
\end{Shaded}

Where one party is to pay another party's costs and/or the parties want
the amount of those costs assessed by the court (known as a
\emph{detailed assessment}), that direction must go in the order.

When assessing payment of costs, ask

\begin{itemize}
\tightlist
\item
  Why pays?
\item
  For what?
\item
  How much?
\item
  By when?
\end{itemize}

Moreover, because there is a possibility that a party may not perform
its part of the agreement, the order should include a provision that any
party is at liberty to apply for the stay to be lifted so that the court
can enforce the settlement. This means that it is not necessary to start
new proceedings to enforce the terms ({[}{[}Trebisol Sud Ouest SAS v
Berkley Finance Ltd {[}2021{]} EWHC 2494 (QB){]}{]}).

\hypertarget{degrees-of-confidentiality}{%
\paragraph{Degrees of
Confidentiality}\label{degrees-of-confidentiality}}

Whether the terms of the settlement are recorded in a schedule to the
Tomlin order, an agreement annexed to the document or a separate
document depends largely on the degree of confidentiality required.

\begin{Shaded}
\begin{Highlighting}[]
\NormalTok{Under r 5.4B and r 5.4C, parties and non{-}parties can apply for permission to obtain copies of documents filed with the court.}
\end{Highlighting}
\end{Shaded}

Therefore, the only way to ensure complete confidentiality is to
\textbf{record the settlement in a separate document which is referred
to in the schedule and clearly identifiable but not filed with the
court}. In {[}{[}Zenith Logistics Services (UK) Ltd \& Others v Coury
{[}2020{]} EWHC 774 (QB){]}{]}, the court confirmed that the fact the
terms were referred to but not set out in the schedule was entirely
unobjectionable. The reason was to preserve confidentiality so that the
terms were not accessible to third parties. And there was no breach of
the `open justice' principle.

\hypertarget{summary}{%
\paragraph{Summary}\label{summary}}

Points to consider when drafting order/ schedule:

\begin{longtable}[]{@{}
  >{\raggedright\arraybackslash}p{(\columnwidth - 2\tabcolsep) * \real{0.3119}}
  >{\raggedright\arraybackslash}p{(\columnwidth - 2\tabcolsep) * \real{0.6881}}@{}}
\toprule()
\begin{minipage}[b]{\linewidth}\raggedright
Order
\end{minipage} & \begin{minipage}[b]{\linewidth}\raggedright
Schedule/Annexed Agreement/Separate Document
\end{minipage} \\
\midrule()
\endhead
`By Consent' & Include any agreed term that court could not order \\
Stay of proceedings & Any payment of money should include provision for
interest on late payment \\
Liberty to apply & \\
Payment of money out of court & \\
Payment of costs & \\
Detailed assessment of costs & \\
Signed by the parties' solicitors & \\
\bottomrule()
\end{longtable}

\hypertarget{part-36}{%
\subsection{Part 36}\label{part-36}}

{[}{[}part36-diagram.png{]}{]}

Before litigation starts, parties can make `without prejudice' offers to
settle. If proceedings issued, continually review the case.

\begin{itemize}
\tightlist
\item
  r 36.7(1): a Part 36 offer may be made both before and during
  proceedings.
\item
  r 44.2: the court will take into account when deciding the issue of
  costs any admissible offer to settle that has been made.
\item
  r 36.16(1): a Part 36 offer will be treated as `without prejudice
  except as to costs'.
\end{itemize}

Part 36 offer has been made must not normally be communicated to the
trial judge (or to any judge allocated in advance to conduct the trial)
until the case has been decided.

\begin{Shaded}
\begin{Highlighting}[]
\NormalTok{If a defendant makes a Part 36 offer and the claimant fails to obtain a judgment more advantageous than that offer, the claimant will usually suffer severe financial penalties}
\end{Highlighting}
\end{Shaded}

\hypertarget{form-and-content}{%
\subsubsection{Form and Content}\label{form-and-content}}

r 36.5(1):

\begin{enumerate}
\def\labelenumi{\arabic{enumi}.}
\tightlist
\item
  be in writing;
\item
  make clear that it is made pursuant to Part 36;
\item
  specify a period of not less than 21 days during which, if the offeree
  accepts the offer, the defendant will pay the claimant's costs under r
  36.13 (known as `the relevant period');
\item
  state whether it relates to the whole of the claim or to part of it;
  and
\item
  state whether it takes into account any counterclaim.
\end{enumerate}

But it is not part of this mandatory requirement, once the period has
been specified, to state expressly that this is the period `within which
the defendant will be liable for the claimant's costs in accordance with
rule 36.13 if the offer is accepted'. If an offeree receives a Part 36
offer and they wish to challenge the validity of the offer, they should
do so without delay.

Part 36 offers are made by a claimant and/or a defendant in a monetary
claim to settle the claim on payment of a lump sum. r 36.6, a Part 36
offer by a defendant to pay a sum of money in settlement of a claim must
be an offer to pay a single sum of money. Under r 36.5(5), 'a Part 36
offer to accept a sum of money may make provision for accrual of
interest on such sum after the date specified in paragraph (4).

\hypertarget{offer}{%
\paragraph{\textgreater1 Offer}\label{offer}}

Can make multiple offers. If none is accepted, the financial
consequences of r 36.17 will be imposed on the defendant should the
claimant secure a judgment that is at least as advantageous to them as
the proposals contained in one of their Part 36 offers.

A Part 36 offer is not like an offer in the ordinary law of contract
where an offer which is rejected, either expressly or by the making of a
counter-offer, cannot subsequently be accepted. That is not true of a
Part 36 offer, which may be accepted even after the offeree has put
forward a different proposal.

\hypertarget{offers-made-close-to-trial}{%
\subsubsection{Offers Made Close to
Trial}\label{offers-made-close-to-trial}}

As the relevant period is a minimum of 21 days, can an offer be made
less than 21 days before the start of a trial?

\begin{itemize}
\tightlist
\item
  r 36.3(g)(ii): the relevant period is the period up to the end of the
  trial.
\item
  r 36.5(4)(b): such an offer is deemed inclusive of interest up to a
  date 21 days after the offer was made.
\item
  If the trial starts then, by r 36.11(3)(d), the offeree will need the
  court's permission to accept the Part 36 offer.
\end{itemize}

\hypertarget{offer-made-when-served}{%
\subsubsection{Offer Made When Served}\label{offer-made-when-served}}

r 36.7(2): Part 36 offer is made when served on the offeree. If the
offeree is legally represented, the offer must be served on solicitors.
The usual rules of deemed service apply.

\hypertarget{withdrawal-or-changing-terms}{%
\subsubsection{Withdrawal or Changing
Terms}\label{withdrawal-or-changing-terms}}

r 36.10(3): possible to withdraw/ change terms only if court agrees. The
offeror must serve written notice of the withdrawal or change of terms
on the offeree. This can be done by letter.

Suppose a defendant wishes to decrease the amount they previously
offered. They should serve a notice of variation of the original offer
under r 36.9(4). This preserves the relevant period of the original
offer for the purpose of costs where the offeree fails to beat the less
advantageous offer.

\begin{Shaded}
\begin{Highlighting}[]
\NormalTok{By r 36.17(7), where a Part 36 offer is withdrawn or its terms made less advantageous to the offeree, and the offeree has beaten the less advantageous offer, the financial consequences of r 36.17 will not apply to it.}
\end{Highlighting}
\end{Shaded}

\hypertarget{clarification}{%
\subsubsection{Clarification}\label{clarification}}

Where monetary and/or non-monetary remedies are sought, it is possible
that an offeree may not be clear about the terms proposed. By r 36.8, an
offeree may, within seven days of receiving a Part 36 offer, request
clarification from the offeror. Best practice to make request in
writing. 21 days minimum is given to an offeree to accept a Part 36
offer. After this, offeree may accept offer but suffer a costs penalty.

\hypertarget{acceptance}{%
\subsubsection{Acceptance}\label{acceptance}}

\begin{itemize}
\tightlist
\item
  A Part 36 offer may be accepted at any time, whether or not the
  offeree has subsequently made a different offer, unless the offeror
  has served notice of withdrawal of that offer on the offeree.
\item
  r 36.11(1): an offeree can accept a Part 36 offer by serving written
  notice of the acceptance on the offeror.
\item
  No prescribed form.
\item
  If proceedings have started, must also file with the court (PD 36 para
  3.1).
\item
  r 21.10: acceptance on behalf of a child/ protected party is not valid
  unless the court has approved the settlement.
\end{itemize}

\hypertarget{consequence}{%
\paragraph{Consequence}\label{consequence}}

Where a Part 36 offer is accepted, \textbf{the claim is stayed}. This
does not affect the power of the court to enforce the terms of a Part 36
offer and deal with any question of costs.

\begin{Shaded}
\begin{Highlighting}[]
\NormalTok{title: Is an agreement reached by the acceptance of a Part 36 offer void if that Part 36 offer was a clear and obvious mistake that was appreciated by the Part 36 offeree at the point of acceptance? }
\NormalTok{Yes, and to that extent the doctrine of common law mistake can apply to a Part 36 offer ([[O’Grady v B15 Group Ltd [2022] EWHC 67 (QB)]]).}
\end{Highlighting}
\end{Shaded}

Where a claimant accepts a defendant's Part 36 offer that is, or
includes, an offer to pay a\\
single sum of money, that sum must be paid to the claimant within
\textbf{14 days} of the date of\\
acceptance. If the accepted sum is not paid within 14 days, the claimant
may enter judgment for the unpaid sum.

\hypertarget{costs-consequences}{%
\paragraph{Costs Consequences}\label{costs-consequences}}

Where a defendant's Part 36 offer is accepted by the claimant within the
relevant period, the claimant is entitled to their costs of the
proceedings up to the date on which notice of acceptance is served on
the defendant.

{[}{[}part36-acceptance-consequences.png{]}{]}

Where a claimant's Part 36 offer is accepted by the defendant within the
relevant period, the claimant is entitled to their costs of the
proceedings up to the date on which notice of acceptance is served on
the claimant.

{[}{[}part36-2.png{]}{]}

\hypertarget{late-acceptance}{%
\paragraph{Late Acceptance}\label{late-acceptance}}

If a claimant accepts D's offer `late' (after the relevant period
specified in the offer has expired) and the parties cannot agree on
costs, the court will make an order.

r 36.13(5): unless the court considers it unjust to do so:

\begin{enumerate}
\def\labelenumi{\arabic{enumi}.}
\tightlist
\item
  the defendant will be ordered to pay the claimant's costs of the
  proceedings up to the date on which the relevant period expired; and
\item
  the claimant will be ordered to pay the defendant's costs for the
  period from the date of expiry of the relevant period to the date of
  acceptance.
\end{enumerate}

{[}{[}part36-3.png{]}{]}

For these purposes, the claimant's costs include any costs incurred in
dealing with a defendant's counterclaim if the Part 36 offer states that
it takes that into account.

The presumption is that the price of late acceptance for a claimant is
to pay the defendant's costs on the standard basis from when the
relevant period expired until the date of acceptance. The court must
guard against making an exception from the norm on the grounds that the
regime itself is harsh or unjust, but must find something about the
particular circumstances of the case which takes it out of the norm.
Remember, the purpose of Part 36 is to promote compromise and avoid
unnecessary expenditure of costs and court time.

\begin{itemize}
\tightlist
\item
  The burden is on the claimant to show injustice.
\item
  Only in exceptional circumstances will the court not require the
  claimant to make the payment of costs under r 26.13(5)(b).
\item
  Particular care should be taken when making a Part 36 offer to a
  litigant in person, to draw special attention to the provision.
\item
  A claimant can also be ordered to pay D's costs incurred (indemnity
  basis) between expiry of the relevant period and acceptance.
\end{itemize}

\hypertarget{part-36-consequences-at-trial}{%
\subsubsection{Part 36 Consequences at
Trial}\label{part-36-consequences-at-trial}}

\begin{itemize}
\tightlist
\item
  If claimant at trial/ summary judgment gets a more advantageous offer,
  D pays amount of judgment and usually V's costs on a standard basis.
\item
  If V fails to obtain a better judgment than that Part 36, the court
  will usually make a split costs order r 36.17(3).
\end{itemize}

\hypertarget{like-for-like-comparison}{%
\subsubsection{`Like for Like'
Comparison}\label{like-for-like-comparison}}

To make a like-for-like comparison between the Part 36 offer and
(judgment + interest + costs):

\begin{itemize}
\tightlist
\item
  Calculate interest that would have accrued on the sum awarded by the
  judge from the date it becomes payable \textless{} period \(\leq\) end
  of relevant period (normally day 21)
\item
  Add this to amount of judgment.
\end{itemize}

\hypertarget{split-costs-order}{%
\subsubsection{Split Costs Order}\label{split-costs-order}}

Court must, unless unjust, order that D entitled to costs from expiry of
relevant period + interests on costs (r 36.17(3)).

The usual costs order is that the losing party should pay the winning
party's costs, and interest on costs is normally payable from the day of
judgment.

Effects of r 36.17(3) is:

\begin{enumerate}
\def\labelenumi{\arabic{enumi}.}
\tightlist
\item
  D pays V's costs from when those costs were first incurred until the
  relevant period expired (21 days, usually)
\item
  V pays D's costs from expiry of relevant period (day 22) until
  judgment. Costs should be agreed by the parties, or otherwise will be
  assessed on the standard basis. Also pays \textbf{interest} on these
  costs from when each item was incurred.
\end{enumerate}

\begin{Shaded}
\begin{Highlighting}[]
\NormalTok{title: Why is the standard basis rather than the indemnity basis used as the starting point for r 36.16(3)(a)?}
\NormalTok{Indemnity costs only appropriate when conduct of a paying party is unreasonable to a high degree. So there must be something taking the case outside the norm. Ask: was there a point when the reasonable claimant would have concluded that the offer represented a better outcome than the likely outcome at trial. }
\end{Highlighting}
\end{Shaded}

So the pursuit of a weak claim will not normally, on its own, justify an
order for indemnity costs. But pursuit of a hopeless claim may lead to
an order. Interest is not payable on costs until after judgment. This is
usually at the normal commercial rate.

{[}{[}part36-4.png{]}{]}

\hypertarget{claimant-who-loses-at-trial}{%
\paragraph{Claimant Who Loses at
Trial}\label{claimant-who-loses-at-trial}}

What if the claimant fails to establish liability and so is awarded no
sum of money? The normal court order would be for the claimant to pay
D's costs from when these were first incurred by D to judgment. But
because of the Part 36 order, D should argue that the claimant should
pay interest on D's costs, from when the relevant period for acceptance
of the offer expired.

{[}{[}part36-5.png{]}{]}

\hypertarget{tactical-considerations}{%
\subsubsection{Tactical Considerations}\label{tactical-considerations}}

A Part 36 offer can provide a useful mechanism for pressurising the
claimant to accept a reasonable settlement. The earlier a Part 36 offer
is made, the greater the potential costs protection for D.

\hypertarget{part-36-consequences-at-trial-of-claimant-offer}{%
\subsubsection{Part 36 Consequences at Trial of Claimant
Offer}\label{part-36-consequences-at-trial-of-claimant-offer}}

If judgment against D is at least as advantageous as V's Part 36, court
must order that V is entitled to:

\begin{enumerate}
\def\labelenumi{\arabic{enumi}.}
\tightlist
\item
  Interest for some/ all of the period starting \textbf{on} the date the
  relevant period expired (\(\leq 10\% +\) base rate)

  \begin{itemize}
  \tightlist
  \item
    Common-sense reading = day 22
  \item
    There is a cap on the amount of interest that can be awarded on
    damages by the trial judge and under r 36.17(4), which is 10\% over
    base rate.
  \item
    Enhanced interest: Award of interest on damages under r 36.17(4)
  \item
    The level of interest awarded must be proportionate to the
    circumstances of the case.
  \end{itemize}
\item
  Costs on the indemnity basis, \textbf{from} date on which the relevant
  period expired

  \begin{itemize}
  \tightlist
  \item
    Starts running day 22
  \item
    Principles from {[}{[}OMV Petrom SA v Glencore International AG
    {[}2017{]} EWCA Civ 195{]}{]}

    \begin{itemize}
    \tightlist
    \item
      10\% is the starting point.
    \item
      Aim is to provide penalties and rewards to encourage reasonable
      settlements being made.
    \item
      Flexible to account for different factors.
    \end{itemize}
  \end{itemize}
\item
  Interest on costs (\(\leq 10\% +\) base rate)
\item
  Additional amount \(\leq £75,000\) calculated as a prescribed
  percentage.

  \begin{itemize}
  \tightlist
  \item
    Note interest should be included only where it arises under
    contract/ court's discretionary power.
  \item
    Interest not payable on additional sum.
  \end{itemize}
\end{enumerate}

\begin{longtable}[]{@{}
  >{\raggedright\arraybackslash}p{(\columnwidth - 2\tabcolsep) * \real{0.3229}}
  >{\raggedright\arraybackslash}p{(\columnwidth - 2\tabcolsep) * \real{0.6771}}@{}}
\toprule()
\begin{minipage}[b]{\linewidth}\raggedright
Amount awarded by the court
\end{minipage} & \begin{minipage}[b]{\linewidth}\raggedright
Prescribed percentage
\end{minipage} \\
\midrule()
\endhead
\(\leq £500,000\) & 10\% of the amount awarded; \\
\(£500,000 < X \leq £1,000,000\) & 10\% of the first £500,000 and 5\% of
any amount above that figure \\
\(X > £100,000\) & \(£75,000\) \\
\bottomrule()
\end{longtable}

\hypertarget{tactical-considerations-1}{%
\paragraph{Tactical Considerations}\label{tactical-considerations-1}}

D refuses V's Part 36 and V obtains a judgment the same or better than
Part 36 ⇾ D is likely to pay a heavy price. Whereas the other way
around, V unlikely to pay a heavy price. So Part 36 has the effect of
pressurising defendants.

\begin{Shaded}
\begin{Highlighting}[]
\NormalTok{Since the penalties in r 36.17(4) are imposed as from the expiry of the relevant period, the earlier the claimant makes their offer, the greater the pressure on the defendant.}
\end{Highlighting}
\end{Shaded}

Note that the court retains ultimate discretion; it can choose to award
some of the s 36.17(4) penalties, but not others.

\hypertarget{advantageous}{%
\subsubsection{``Advantageous''}\label{advantageous}}

Means better in monetary terms, by no matter how small an amount.

\hypertarget{unjust-orders}{%
\subsubsection{Unjust Orders}\label{unjust-orders}}

Recall, a well-judged offer that is not accepted will lead to an order
under r 36.17(3) or (4), unless it would be unjust for the court to make
that order.

\begin{Shaded}
\begin{Highlighting}[]
\NormalTok{title: r 36.17(5)}
\NormalTok{In considering whether it would be unjust to make the orders referred to in paragraphs (3) and (4), the court must take into account all the circumstances of the case including—}
\NormalTok{{-} (a) the terms of any Part 36 offer;}
\NormalTok{{-} (b) the stage in the proceedings when any Part 36 offer was made, including in particular how long before the trial started the offer was made;}
\NormalTok{{-} (c) the information available to the parties at the time when the Part 36 offer was made;}
\NormalTok{{-} (d) the conduct of the parties with regard to the giving of or refusal to give information for the purposes of enabling the offer to be made or evaluated; and}
\NormalTok{{-} (e) whether the offer was a genuine attempt to settle the proceedings.}
\end{Highlighting}
\end{Shaded}

Note these are just examples which do not constrain the creativity of
the court.

\begin{Shaded}
\begin{Highlighting}[]
\NormalTok{In determining whether or not it is unjust to make orders rr 36.17(3) \& (4), the court cannot consider by how much the judgment against D was more advantageous than the Part 36 offer ([[JLE v Warrington \& Halton Hospitals NHS Foundation Trust [2019] EWHC 1582 (QB)]])}
\end{Highlighting}
\end{Shaded}

The test of unjustness should be separately applied to each of the four
consequences.

Tactics: make a Part 36 offer for marginally less than the full sum
claimed (0.3\% in {[}{[}Rawbank SA v Travelex Banknotes Ltd {[}2020{]}
EWHC 1619 (Ch){]}{]}). Then if D rejects and you win the full amount,
get bare costs.

\hypertarget{r-36.173-cfa}{%
\subsubsection{R 36.17(3) \& CFA}\label{r-36.173-cfa}}

\begin{longtable}[]{@{}
  >{\raggedright\arraybackslash}p{(\columnwidth - 2\tabcolsep) * \real{0.5591}}
  >{\raggedright\arraybackslash}p{(\columnwidth - 2\tabcolsep) * \real{0.4409}}@{}}
\toprule()
\begin{minipage}[b]{\linewidth}\raggedright
Scenario
\end{minipage} & \begin{minipage}[b]{\linewidth}\raggedright
Consequence
\end{minipage} \\
\midrule()
\endhead
V rejects the offer on the basis of advice from the solicitors/ counsel
& Success fee not payable on charges incurred after day 22 \\
V rejects offer against advice of solicitor/ counsel & Success fee
payable. \\
\bottomrule()
\end{longtable}

\hypertarget{secrecy}{%
\subsubsection{Secrecy}\label{secrecy}}

Part 36 offer should not be revealed to the trial judge until all
questions of liability and quantum have been decided (except in super
rare cases). Part 36 offers are ``without prejudice save as to costs'' -
this does not need to be said explicitly.

\hypertarget{part-36-counterclaim}{%
\subsubsection{Part 36 \& Counterclaim}\label{part-36-counterclaim}}

Rules still apply for counterclaim.

\hypertarget{indemnity-vs-standard-basis}{%
\subsubsection{Indemnity Vs Standard
Basis}\label{indemnity-vs-standard-basis}}

If costs are calculated on the standard basis, any ambiguity is resolved
in favour of the paying party. But on the indemnity basis, ambiguities
resolved in favour of the receiving party.

\hypertarget{applications-to-court}{%
\section{Applications to Court}\label{applications-to-court}}

\hypertarget{general}{%
\subsection{General}\label{general}}

General rules set out in Part 23 CPR 1998:

\begin{itemize}
\tightlist
\item
  Application to set aside default judgment (part 13)
\item
  Application for summary judgment (part 24)
\item
  Application for an interim injunction (part 25.1)
\item
  Application for an interim payment on account of damages (part 25.6)
\item
  Security for costs order (r 25.12)
\end{itemize}

\hypertarget{n244}{%
\subsubsection{N244}\label{n244}}

\textbf{An application to the court is made by an application notice}:
Form N244 should be used

\emph{The person making the application is the applicant, and the party
whom the order is sought is the respondent.}

\begin{longtable}[]{@{}
  >{\raggedright\arraybackslash}p{(\columnwidth - 4\tabcolsep) * \real{0.0492}}
  >{\raggedright\arraybackslash}p{(\columnwidth - 4\tabcolsep) * \real{0.0284}}
  >{\raggedright\arraybackslash}p{(\columnwidth - 4\tabcolsep) * \real{0.9223}}@{}}
\toprule()
\begin{minipage}[b]{\linewidth}\raggedright
Aspect
\end{minipage} & \begin{minipage}[b]{\linewidth}\raggedright
Rule
\end{minipage} & \begin{minipage}[b]{\linewidth}\raggedright
Details
\end{minipage} \\
\midrule()
\endhead
Where & r 23.2 & The application must be made to the court where the
claim has been started/ court where the claim has been sent/ court where
trial is to take place. Any application before a claim form should be
made to the court where proceedings will be started (Exception:
{[}{[}County Court{]}{]} applications) \\
Content & r 23.6 & Application notice should state what order the
applicant is seeking and, briefly, why the applicant is seeking the
order. Statement of truth if the applicant wishes to rely on matters set
out in the application notice as evidence at the hearing. \\
Draft order & PD 23A & Except in the most simple application, the
applicant should attach a draft of the order sought. \\
Evidence & PD 23A para 9.1 & Application notice should state what order
the applicant is seeking and, briefly, why the applicant is seeking the
order. Statement of truth if the applicant wishes to rely on matters set
out in the application notice as evidence at the hearing. Affidavit
costs may be disallowed. Any evidence relied upon must be filed at the
court as well as served on the parties with the application notice. \\
Service & r 23.7(1)(b) & The application notice must be served at least
three clear days before the court is to deal with the application. The
court may allow a shorter period of notice if this is appropriate in the
circumstances. \\
Consent orders & & If the parties have reached agreement on the order
they wish the court to make, they can apply for an order to be made by
consent without the need for attendance by the parties. \\
Application without notice & PD 23A para 3 & Most applications must be
made on notice. Exceptions include exceptional urgency, best furthering
the overriding interest, consent of all parties, permission of the
court. When an order is made on an application without notice to the
respondent, a copy of the order must be served on the respondent,
together with a copy of the application notice and the supporting
evidence. The respondent may then apply to set aside or vary the order
within seven days of service of the order on them. \\
\bottomrule()
\end{longtable}

\hypertarget{telephone-hearings-video-conferencing}{%
\subsubsection{Telephone hearings/ Video
Conferencing}\label{telephone-hearings-video-conferencing}}

Many district registries \textbf{now have facilities to deal with
interim applications by telephone conferencing}.

\begin{Shaded}
\begin{Highlighting}[]
\NormalTok{title: General rule}
\NormalTok{At a telephone enabled court all allocation hearings, listing hearings, interim applications, case management conferences and pre{-}trial reviews with a time estimate of less than an hour will be conducted by telephone}
\end{Highlighting}
\end{Shaded}

Exceptions:

\begin{itemize}
\tightlist
\item
  applications made with no notice to other party
\item
  where all parties are unrepresented
\item
  where more than 4 parties wish to make representations at hearing
\item
  if application is being heard by telephone then no party/party's legal
  representative can attend the judge in person

  \begin{itemize}
  \tightlist
  \item
    unless every other party has agreed
  \end{itemize}
\end{itemize}

PD 23A, paras 6.9 and 6.10: mechanics of telephone conference hearing

\begin{Shaded}
\begin{Highlighting}[]
\NormalTok{Applicant’s legal representative must file and serve case summary and draft order no later than 4pm 2 days before the telephone hearing if the claim has been allocated to multi{-}track or if the court directs.}
\end{Highlighting}
\end{Shaded}

\hypertarget{preparing-supporting-evidence}{%
\subsubsection{Preparing Supporting
Evidence}\label{preparing-supporting-evidence}}

General questions about witness statement:

\begin{enumerate}
\def\labelenumi{\arabic{enumi}.}
\tightlist
\item
  Why should make statement?
\item
  What should be included?
\item
  How should the statement appear?
\item
  What about hearsay?
\end{enumerate}

Don't include points of law.

\hypertarget{public-private-hearings}{%
\subsubsection{Public/ Private Hearings}\label{public-private-hearings}}

See r 39.2 CPR for guidance.

\begin{itemize}
\tightlist
\item
  Generally, hearings are public, except if the court decides it should
  be private.
\item
  Court will bear in mind freedom of expression considerations.
\item
  When deciding on private, court considers factors including:

  \begin{itemize}
  \tightlist
  \item
    If publicity would defeat the object
  \item
    National security
  \item
    Confidential information (e.g., personal financial information)
  \item
    Child/ protected party
  \end{itemize}
\end{itemize}

\hypertarget{interim-costs}{%
\subsection{Interim Costs}\label{interim-costs}}

\hypertarget{costs-of-applying}{%
\subsubsection{Costs of Applying}\label{costs-of-applying}}

Any interim court application costs money. The judge may decide that one
party should pay the other's costs--`pay as you go litigation'.

The normal order is that an applicant pays the respondent's costs of
application, including additional costs. Expressed as `costs of and
caused by'.

\hypertarget{possible-orders}{%
\subsubsection{Possible Orders}\label{possible-orders}}

For named party A, unnamed party B:

\begin{longtable}[]{@{}
  >{\raggedright\arraybackslash}p{(\columnwidth - 2\tabcolsep) * \real{0.1509}}
  >{\raggedright\arraybackslash}p{(\columnwidth - 2\tabcolsep) * \real{0.8491}}@{}}
\toprule()
\begin{minipage}[b]{\linewidth}\raggedright
Term
\end{minipage} & \begin{minipage}[b]{\linewidth}\raggedright
Effect
\end{minipage} \\
\midrule()
\endhead
A's costs/ A's costs in any event & A entitled to costs wrt part of
proceedings to which the order relates. Summarily assessed and ordered
to pay within 14 days. \\
Costs in the case/ costs in the application & No party is named, no
party is able to recover costs of interim hearing. Outcome will
determine costs. \\
A's costs in case & If A awarded costs at the end of proceedings, they
are entitled to their costs of the part of proceedings to which the
order relates. B is never entitled to recover the interim costs. A
recovers interim costs only if awarded costs at the end. \\
A's costs thrown away & Where a judgment/ order against B set aside, B
entitled to costs incurred and A cannot recover. \\
A's costs of and caused by & Court makes order on an application, A gets
costs and any consequential costs \\
No order as to costs & Each party bears their own costs, whichever way
the judgment goes. \\
\bottomrule()
\end{longtable}

See
\href{https://uk.westlaw.com/Document/Idfa75d80e25211e398db8b09b4f043e0/View/FullText.html?navigationPath=Search\%2Fv1\%2Fresults\%2Fnavigation\%2Fi0ad7401100000182351e1646a78eccc7\%3Fppcid\%3Dc0a980e2f5064b068470b2d9a1d8da35\%26Nav\%3DKNOWHOW_UK\%26fragmentIdentifier\%3DIdfa75d80e25211e398db8b09b4f043e0\%26parentRank\%3D0\%26startIndex\%3D1\%26contextData\%3D\%2528sc.Search\%2529\%26transitionType\%3DSearchItem\&listSource=Search\&listPageSource=44bcbc1c4ad1b703331d2ac9478bb7c7\&list=KNOWHOW_UK\&rank=4\&sessionScopeId=41ee73cb61e521307cbc865612f79b4a57e9843873b7cea25b7a1d4abfd584fc\&ppcid=c0a980e2f5064b068470b2d9a1d8da35\&originationContext=Search\%20Result\&transitionType=SearchItem\&contextData=(sc.Search)\&comp=pluk}{Costs
orders commonly made: checklist, Practical Law}.

If there is no mention of costs, the general rule is that none are
payable in respect of that application.

\hypertarget{summary-assessment}{%
\subsubsection{Summary Assessment}\label{summary-assessment}}

If costs are ordered in favour of one of the parties, the court will
make a summary assessment of costs on the spot, payable within 14 days
unless otherwise specified.

Parties are required, \(\geq 24\) hours prior to hearing, to file and
serve a statement of costs, providing a breakdown of costs.

Model form of the statement of costs (Form N260).

\hypertarget{fixed-costs}{%
\subsection{Fixed Costs}\label{fixed-costs}}

The court may award fixed costs rather than making an order. Part 45
sets out the occasions on which fixed costs may be granted and specifies
the amount awarded to the receiving party.

\hypertarget{cost-capping-order-r-3.19}{%
\subsubsection{Cost-capping Order (r
3.19)}\label{cost-capping-order-r-3.19}}

If one or more parties does not follow relevant pre-action protocol,
unnecessary costs may be incurred. Court may make an order capping the
amount of costs recoverable by a party incurred after that date.

\begin{longtable}[]{@{}
  >{\raggedright\arraybackslash}p{(\columnwidth - 2\tabcolsep) * \real{0.3714}}
  >{\raggedright\arraybackslash}p{(\columnwidth - 2\tabcolsep) * \real{0.6286}}@{}}
\toprule()
\begin{minipage}[b]{\linewidth}\raggedright
Costs capping
\end{minipage} & \begin{minipage}[b]{\linewidth}\raggedright
Costs management order
\end{minipage} \\
\midrule()
\endhead
Caps costs, made in exceptional circumstances. & Does not limit the
costs recoverable unless varied. \\
\bottomrule()
\end{longtable}

Most reported applications of costs capping have failed. There is
specific provision for costs\\
capping orders in judicial review proceedings. Governed by s 88 Criminal
Justice and Courts Act 2015. Provisions apply where the application for
leave to bring judicial review proceedings has been granted. Test set
out in s 88(6).

\begin{Shaded}
\begin{Highlighting}[]
\NormalTok{title: s 88(6) Criminal Justice and Courts Act 2015}
\NormalTok{The court may make a costs capping order only if it is satisfied that—}
\NormalTok{{-} (a) the proceedings are public interest proceedings,}
\NormalTok{{-} (b) in the absence of the order, the applicant for judicial review would withdraw the application for judicial review or cease to participate in the proceedings, and}
\NormalTok{{-} (c) it would be reasonable for the applicant for judicial review to do so.}
\end{Highlighting}
\end{Shaded}

\hypertarget{making-an-application}{%
\subsubsection{Making an Application}\label{making-an-application}}

Normally it is D seeking to cap what V recovers in costs. Make claim as
soon as possible.

\begin{itemize}
\tightlist
\item
  Evidence should set out whether the costs-capping order applies to
  whole litigation/ some part.
\item
  Accompany with budget, setting out costs and disbursements.
\item
  Court will direct that other parties file a similar budget.
\end{itemize}

\hypertarget{grounds-for-making-order}{%
\subsubsection{Grounds for Making
Order}\label{grounds-for-making-order}}

\begin{Shaded}
\begin{Highlighting}[]
\NormalTok{title: r 3.19(5)}
\NormalTok{The court may at any stage of proceedings make a costs capping order against all or any of the parties, if –}

\NormalTok{(a) it is in the interests of justice to do so;}

\NormalTok{(b) there is a substantial risk that without such an order costs will be disproportionately incurred; and}

\NormalTok{(c) it is not satisfied that the risk in subparagraph (b) can be adequately controlled by –}
\NormalTok{{-} (i) case management directions or orders made under this Part; and}
\NormalTok{{-} (ii) detailed assessment of costs.}
\end{Highlighting}
\end{Shaded}

\begin{Shaded}
\begin{Highlighting}[]
\NormalTok{title: r 3.19(6)}
\NormalTok{In considering whether to exercise its discretion under this rule, the court will consider all the circumstances of the case, including –}
\NormalTok{{-} (a) whether there is a substantial imbalance between the financial position of the parties;}
\NormalTok{{-} (b) whether the costs of determining the amount of the cap are likely to be proportionate to the overall costs of the litigation;}
\NormalTok{{-} (c) the stage which the proceedings have reached; and}
\NormalTok{{-} (d) the costs which have been incurred to date and the future costs.}
\end{Highlighting}
\end{Shaded}

\hypertarget{effect-of-order}{%
\subsubsection{Effect of Order}\label{effect-of-order}}

A costs-capping order, once made, limits the costs recoverable by the
party subject to the order unless that party successfully applies to
vary it. Variation will only be made if there is a material and
substantial change in circumstances/ some other compelling reason.

\hypertarget{conditional-fee-agreements}{%
\subsubsection{Conditional Fee
Agreements}\label{conditional-fee-agreements}}

Does not prevent costs of interim application from being summarily
assessed if the court has awarded costs in favour of a party.

\hypertarget{receiving-party-cfa-funded}{%
\paragraph{Receiving Party
CFA-funded}\label{receiving-party-cfa-funded}}

The court \textbf{cannot} order payment unless satisfied that the
receiving party is immediately liable to their solicitor for the costs
of the application under the CFA terms. Summary assessment can only deal
with base costs.

\hypertarget{paying-party-cfa-funded}{%
\paragraph{Paying Party CFA-funded}\label{paying-party-cfa-funded}}

Party may not be in a position to pay interim costs if ordered to do so
Court may decide to defer payment of the interim costs until the end of
proceedings.

\hypertarget{appeals-against-interim-orders}{%
\subsection{Appeals Against Interim
Orders}\label{appeals-against-interim-orders}}

Set out in Part 52 CPR.

\hypertarget{courts}{%
\subsubsection{Courts}\label{courts}}

County Court:

MERMAID1

High Court:

MERMAID2

\hypertarget{factors}{%
\subsubsection{Factors}\label{factors}}

Permission will be granted if:

\begin{itemize}
\tightlist
\item
  Appeal has a real prospect of success, or
\item
  Some other compelling reason for the appeal to be heard (r 52.3(6)).
\end{itemize}

Seek permission at original hearing/ from Appeal Court within 21 days of
decision (either at the first instance, or if permission at the original
hearing is refused).

\hypertarget{appealing}{%
\subsubsection{Appealing}\label{appealing}}

Appeal hearing limited to a review of original decision. No new evidence
admitted unless court orders otherwise (r 52.11). The appeal will be
allowed if the original decision was wrong/ unjust.

\hypertarget{set-aside-default-judgment-part-13}{%
\subsection{Set Aside Default Judgment (part
13)}\label{set-aside-default-judgment-part-13}}

\hypertarget{mandatory-grounds}{%
\subsubsection{Mandatory Grounds}\label{mandatory-grounds}}

Under r 13.2, the court is obliged to set aside a default judgment that
was wrongly entered\\
before the defendant's deadline for filing an acknowledgement of service
or a defence (whichever is applicable) expired. The court is also
obliged to set aside a default judgment\\
entered after the claim was paid in full.

\hypertarget{discretionary-grounds}{%
\subsubsection{Discretionary Grounds}\label{discretionary-grounds}}

\begin{Shaded}
\begin{Highlighting}[]
\NormalTok{title: r 13.3(1)}
\NormalTok{(1) In any other case, the court may set aside or vary a judgment entered under Part 12 if—}
\NormalTok{{-} (a) the defendant has a real prospect of successfully defending the claim; or}
\NormalTok{{-} (b) it appears to the court that there is some other good reason why—}
\NormalTok{    {-} (i) the judgment should be set aside or varied; or}
\NormalTok{    {-} (ii) the defendant should be allowed to defend the claim.}

\NormalTok{(2) In considering whether to set aside or vary a judgment entered under Part 12, the matters to which the court must have regard include whether the person seeking to set aside the judgment made an application to do so promptly.}
\end{Highlighting}
\end{Shaded}

\begin{itemize}
\tightlist
\item
  The court will take account of the promptness of the defendant's
  application to set aside
\item
  The overriding objective expressly recognises the importance of
  ensuring that cases are dealt with expeditiously and fairly.
\end{itemize}

The application to the court must be on notice and must be supported by
evidence. Although\\
the rule states that only one of the grounds needs to be satisfied, in
practice the defendant will\\
usually have to show a defence with a real prospect of success at trial
in order to persuade the\\
court to exercise its discretion.

If the court accepts that there is a good reason for setting aside, as
neither party can be said to be at fault in those circumstances, it is
likely that the court will set aside the default judgment and make an
order for costs in the case.

\hypertarget{possible-orders-1}{%
\subsubsection{Possible Orders}\label{possible-orders-1}}

The court may:

\begin{itemize}
\tightlist
\item
  Set aside default judgment
\item
  Refuse application, or
\item
  Make a conditional order.

  \begin{itemize}
  \tightlist
  \item
    The normal condition is that the defendant pays into court the
    amount of the claim or such amount as they can reasonably afford.
  \end{itemize}
\end{itemize}

\hypertarget{costs}{%
\subsubsection{Costs}\label{costs}}

\begin{longtable}[]{@{}
  >{\raggedright\arraybackslash}p{(\columnwidth - 2\tabcolsep) * \real{0.3037}}
  >{\raggedright\arraybackslash}p{(\columnwidth - 2\tabcolsep) * \real{0.6963}}@{}}
\toprule()
\begin{minipage}[b]{\linewidth}\raggedright
Grounds
\end{minipage} & \begin{minipage}[b]{\linewidth}\raggedright
Who pays costs
\end{minipage} \\
\midrule()
\endhead
Mandatory grounds & Claimant at fault for entering judgment at the wrong
time, so usually ordered to pay D's costs \\
Discretionary grounds & Neither side at fault, more likely D pays
depending on particulars \\
Conditional order due to late application & D pays for being at
fault. \\
\bottomrule()
\end{longtable}

{[}{[}costs-application.png{]}{]}

\hypertarget{summary-judgment-part-24}{%
\subsection{Summary Judgment (Part 24)}\label{summary-judgment-part-24}}

Aim: to enable C/D to obtain judgment at an early stage without the time
and expense of a full trial.

\begin{Shaded}
\begin{Highlighting}[]
\NormalTok{Summary judgment is not the same as default judgment.}
\end{Highlighting}
\end{Shaded}

\hypertarget{grounds}{%
\subsubsection{Grounds}\label{grounds}}

\begin{Shaded}
\begin{Highlighting}[]
\NormalTok{title: CPR 24.2}
\NormalTok{The court may give summary judgment against a claimant or defendant on the whole of a claim or on a particular issue if –}
\NormalTok{{-} (a) it considers that –}
\NormalTok{    {-} (i) that claimant has **no real prospect of succeeding** on the claim or issue; or}
\NormalTok{    {-} (ii) that defendant has **no real prospect of successfully defending** the claim or issue; and}
\NormalTok{{-} (b) there is **no other compelling reason** why the case or issue should be disposed of at a trial.}
\end{Highlighting}
\end{Shaded}

(Rule 3.4 makes provision for the court to strike out a statement of
case or part of a statement of case if it appears that it discloses no
reasonable grounds for bringing or defending a claim).

Either party can apply for summary judgment, or the court can schedule a
hearing on its own initiative.

\hypertarget{no-real-prospect-of-success}{%
\paragraph{No Real Prospect of
Success}\label{no-real-prospect-of-success}}

The court must scrutinise how the parties have stated their respective
cases and examine the essence of the way in which the case or issue is
put at the hearing, but no more than that ({[}{[}Barks v Instant Access
Properties Ltd (In Liquidation) {[}2013{]} EWHC 114{]}{]}).

To defeat an application for summary judgment, the respondent must have
a case which has a ``real, as opposed to a fanciful, chance of winning''
({[}{[}Swain v Hillman {[}2001{]} 1 All ER 91{]}{]}).

\begin{Shaded}
\begin{Highlighting}[]
\NormalTok{title: PD 24 para 1.3}
\NormalTok{An application for summary judgment under rule 24.2 may be based on:}
\NormalTok{1. a point of law (including a question of construction of a document),}
\NormalTok{2. the evidence which can reasonably be expected to be available at trial or the lack of it, or}
\NormalTok{3. a combination of these.}
\end{Highlighting}
\end{Shaded}

\hypertarget{no-other-compelling-reason}{%
\paragraph{No Other Compelling
Reason}\label{no-other-compelling-reason}}

Compelling reasons include:

\begin{itemize}
\tightlist
\item
  Need to investigate ({[}{[}Celador Productions Limited v Melville
  {[}2004{]} EWHC 2362 (Ch){]}{]})
\item
  Where one party holds all of the factual cards ({[}{[}Harrison v
  Bottenheim (1878) 26 WR 362{]}{]})
\item
  Questionable conduct by the applicant ({[}{[}Miles v Bull {[}1969{]} 1
  QB 258{]}{]})
\item
  The case is particularly complex ({[}{[}Three Rivers DC v Bank of
  England (No 3) {[}2001{]} UKHL 16{]}{]})
\item
  The case is on a novel point of law for which there is little
  authority ({[}{[}Swain v Hillman {[}2001{]} 1 All ER 91{]}{]})
\end{itemize}

\hypertarget{procedure}{%
\subsubsection{Procedure}\label{procedure}}

\begin{itemize}
\tightlist
\item
  C \textbf{may not} apply for summary judgment \textbf{until D has
  filed an acknowledgement of service or a defence}, unless the court
  gives permission.

  \begin{itemize}
  \tightlist
  \item
    If D fails to file acknowledgement of service/ defence, C can enter
    default judgment without having to make an application for summary
    judgment.
  \item
    If C files for summary judgment before D files a defence, D need not
    file a defence until the application has been heard.
  \end{itemize}
\item
  D \textbf{may} apply for summary judgment \textbf{at any time}.
\item
  Application should be made without delay, usually before filing of
  direction questionnaires.
\item
  Application notice and evidence should be served on the opponent
  \(\geq 14\) days' before the hearing, unless varied by a practice
  direction or the court (CPR 24.4(3)).

  \begin{itemize}
  \tightlist
  \item
    The application notice should

    \begin{itemize}
    \tightlist
    \item
      State that it is an application for summary judgment
    \item
      Identify concisely any point of law/ provision on which the
      applicant relies (PD 24 para 2(3))
    \item
      State that it is made because the applicant believes on the
      evidence that the respondent has no real prospect of succeeding/
      successfully defending (PD 24 para 2(3))
    \item
      Inform the other party of their right to file and serve written
      evidence in reply.
    \end{itemize}
  \end{itemize}
\item
  Respondent should file and serve evidence \(\geq 7\) days before the
  hearing (CPR 24.5(1))
\item
  If the applicant wishes to respond to the respondent's evidence, he
  should do so \(\geq 3\) days before the hearing (CPR 24.5(2)).
\end{itemize}

\hypertarget{orders-a-court-may-make}{%
\subsubsection{Orders a Court May Make}\label{orders-a-court-may-make}}

\begin{Shaded}
\begin{Highlighting}[]
\NormalTok{title: PD 24 para 5.1}
\NormalTok{The orders the court may make on an application under Part 24 include:}
\NormalTok{1. judgment on the claim,}
\NormalTok{2. the striking out or dismissal of the claim,}
\NormalTok{3. the dismissal of the application,}
\NormalTok{4. a conditional order.}
\end{Highlighting}
\end{Shaded}

\begin{Shaded}
\begin{Highlighting}[]
\NormalTok{The claims only go one way. So if C is the applicant, the only possible orders which the court can make are:}
\NormalTok{1. Granting summary judgment}
\NormalTok{2. Dismissing the application and allowing the case to proceed to trial.}

\NormalTok{If D, in addition to opposing the application for summary judgment, wants to apply for an order striking out the (OG) claim, D must make their own application for summary judgment. }
\end{Highlighting}
\end{Shaded}

If there are credible witnesses both saying opposite things and there is
no conclusive circumstantial evidence, the application will be dismissed
({[}{[}Hussain v Woods and Another {[}2001{]} Lloyd's Rep PN 134{]}{]}).

\begin{Shaded}
\begin{Highlighting}[]
\NormalTok{title: PD 24 para 5.2}
\NormalTok{A conditional order is an order which requires a party:}
\NormalTok{1. to pay a sum of money into court, or}
\NormalTok{2. to take a specified step in relation to his claim or defence, as the case may be, and provides that that party’s claim will be dismissed or his statement of case will be struck out if he does not comply.}
\end{Highlighting}
\end{Shaded}

A conditional order will be granted where it is likely but improbable
that a claim/ defence will succeed -- where a statement of case is
``shadowy and unsatisfactory'' ({[}{[}Bates v Microstar Ltd{]}{]}) or
where D has a weak claim and has behaved dishonestly ({[}{[}Kooh Veisin
Trading Co v Parsai{]}{]}).

If the claim might possibly succeed but is unlikely to do so, C will
have to pay 75-80\% of costs D could reasonably expect to recover if the
claim was fully contested ({[}{[}Sweetman v Shepherd{]}{]}).

\hypertarget{costs-1}{%
\subsubsection{Costs}\label{costs-1}}

\begin{itemize}
\tightlist
\item
  Where C is successful in obtaining summary judgment

  \begin{itemize}
  \tightlist
  \item
    The court will award fixed costs in straightforward cases.

    \begin{itemize}
    \tightlist
    \item
      £175 if judgment \(£25 < P \leq £5,000\)
    \item
      £210 if judgment \(P > £5,000\)
    \end{itemize}
  \item
    Open to C to ask for costs to be summarily assessed.

    \begin{itemize}
    \tightlist
    \item
      Court will award C costs of making application (claimant's costs)
    \item
      Fix a date to assess quantum and deal with the costs of the claim
      (disposal hearing)
    \end{itemize}
  \end{itemize}
\item
  Where D secures summary judgment (claim struck out)

  \begin{itemize}
  \tightlist
  \item
    D awarded costs of claim (including pre-action), subject to summary
    assessment.
  \end{itemize}
\item
  Conditional order

  \begin{itemize}
  \tightlist
  \item
    Costs in the case.
  \end{itemize}
\end{itemize}

{[}{[}summary-judgment-costs.png{]}{]}

\hypertarget{interim-payments-orders}{%
\subsection{Interim Payments Orders}\label{interim-payments-orders}}

\begin{Shaded}
\begin{Highlighting}[]
\NormalTok{An interim payment is an advance payment on account of any damages, debt or other sum (excluding costs) which a defendant may be held liable to pay.}
\end{Highlighting}
\end{Shaded}

This gives a claimant with a strong case some money before the case has
concluded, to avoid financial hardship. Before applying, the claimant
should try to negotiate with D/ D's insurance to obtain a voluntary
interim payment.

Cannot be made before the time for acknowledging service has expired.

\hypertarget{procedure-1}{%
\subsubsection{Procedure}\label{procedure-1}}

Application must be supported by evidence and served \(\geq 14\) days
before hearing date. Evidence required includes (PD 25B para 2.1):

\begin{itemize}
\tightlist
\item
  Amount of interim payment sought
\item
  Items or matters wrt which interim payment sought
\item
  Likely amount of final judgment
\item
  Reasons for application.
\end{itemize}

\textbf{Respondent's} evidence must be served \(\geq 7\) days before
hearing date. \textbf{Applicant's} evidence in response should be served
\(\geq 3\) days before hearing.

\hypertarget{grounds-1}{%
\subsubsection{Grounds}\label{grounds-1}}

\begin{Shaded}
\begin{Highlighting}[]
\NormalTok{title: r 25.7(1) CPR 1998}
\NormalTok{The court may only make an order for an interim payment where any of the following conditions are satisfied –}

\NormalTok{(a) the defendant against whom the order is sought has admitted liability to pay damages or some other sum of money to the claimant;}

\NormalTok{(b) the claimant has obtained judgment against that defendant for damages to be assessed or for a sum of money (other than costs) to be assessed;}

\NormalTok{(c) it is satisfied that, if the claim went to trial, the claimant would obtain judgment for a substantial amount of money (other than costs) against the defendant from whom he is seeking an order for an interim payment whether or not that defendant is the only defendant or one of a number of defendants to the claim;}

\NormalTok{(d) the following conditions are satisfied –}
\NormalTok{{-} (i) the claimant is seeking an order for possession of land (whether or not any other order is also sought); and}
\NormalTok{{-} (ii) the court is satisfied that, if the case went to trial, the defendant would be held liable (even if the claim for possession fails) to pay the claimant a sum of money for the defendant’s occupation and use of the land while the claim for possession was pending; or}

\NormalTok{(e) in a claim in which there are two or more defendants and the order is sought against any one or more of those defendants, the following conditions are satisfied –}

\NormalTok{{-} (i) the court is satisfied that, if the claim went to trial, the claimant would obtain judgment for a substantial amount of money (other than costs) against at least one of the defendants (but the court cannot determine which); and}

\NormalTok{{-} (ii) all the defendants are either –}

\NormalTok{    {-} (a) a defendant that is insured in respect of the claim;}
    
\NormalTok{    {-} (b) a defendant whose liability will be met by an insurer under section 151 of the Road Traffic Act 1988 or an insurer acting under the Motor Insurers Bureau Agreement, or the Motor Insurers Bureau where it is acting itself; or}
    
\NormalTok{    {-} (c) a defendant that is a public body.}
\end{Highlighting}
\end{Shaded}

No obligation on applicant to show their need for interim payment,
though this would help persuade the court if the delay in assessing
damages is unlikely to be substantial.

\hypertarget{financial-position}{%
\paragraph{Financial Position}\label{financial-position}}

If the respondent wants the court to take into account their financial
situation, they should disclose this in their evidence in reply to the
application.

\hypertarget{standard-of-proof}{%
\paragraph{Standard of Proof}\label{standard-of-proof}}

The applicant must prove the grounds of application relied upon up to
the civil standard on the balance of probabilities, though in practice
this is interpreted as the court being convinced that the applicant
\textbf{will} succeed, not just it being likely ({[}{[}British and
Commonwealth Holdings plc v Quadrex Holdings Inc {[}1989{]} 3 WLR
723{]}{]}).

\hypertarget{summary-judgment}{%
\paragraph{Summary Judgment}\label{summary-judgment}}

Applications for interim payments and summary judgments often combined.

If the applicant can establish an entitlement to an interim payment, the
court then has a\\
discretion as to:

\begin{enumerate}
\def\labelenumi{\arabic{enumi}.}
\tightlist
\item
  Whether to make an application, and

  \begin{itemize}
  \tightlist
  \item
    May be inappropriate if issues are complicated.
  \item
    Inappropriate if significant causation issues remain.
  \end{itemize}
\item
  If so, the amount.

  \begin{itemize}
  \tightlist
  \item
    The court must not make an interim payment of more than a
    `reasonable proportion of the likely amount of the final judgment'
  \item
    Take into account {[}{[}Contributory negligence{]}{]} and any
    counter-claim.
  \end{itemize}
\end{enumerate}

\hypertarget{consequences-of-interim-payment-order}{%
\subsubsection{Consequences of Interim Payment
Order}\label{consequences-of-interim-payment-order}}

If D has made an interim payment that exceeds their total liability
under the final judgment, the court will invariably order repayment of
the excess ({[}{[}Wakefield v NJS {[}2021{]} EWHC 3452 (QB){]}{]}) and
may award interest on the overpaid amount from the date of the interim
payment.

\hypertarget{security-for-costs}{%
\subsection{Security for Costs}\label{security-for-costs}}

\begin{Shaded}
\begin{Highlighting}[]
\NormalTok{title: r 25.12 {-} Security for Costs}

\NormalTok{(1) A defendant to any claim may apply under this Section of this Part for security for his costs of the proceedings.}

\NormalTok{(Part 3 provides for the court to order payment of sums into court in other circumstances. Rule 20.3 provides for this Section of this Part to apply to Part 20 claims)}

\NormalTok{(2) An application for security for costs must be supported by written evidence.}

\NormalTok{(3) Where the court makes an order for security for costs, it will –}
\NormalTok{{-} (a) determine the amount of security; and}
\NormalTok{{-} (b) direct –}
\NormalTok{    {-} (i) the manner in which; and}
\NormalTok{    {-} (ii) the time within which}

\NormalTok{the security must be given.}
\end{Highlighting}
\end{Shaded}

\begin{Shaded}
\begin{Highlighting}[]
\NormalTok{title: r 25.13 {-} Conditions to be satisfied}
\NormalTok{(1) The court may make an order for security for costs under rule 25.12 if –}
\NormalTok{{-} (a) it is satisfied, having regard to all the circumstances of the case, that it is just to make such an order; and}
\NormalTok{{-} (b)}
\NormalTok{    {-} (i) one or more of the conditions in paragraph (2) applies, or}
\NormalTok{    {-} (ii) an enactment permits the court to require security for costs.}

\NormalTok{(2) The conditions are –}
\NormalTok{{-} (a) the claimant is –}
\NormalTok{    {-} (i) resident out of the jurisdiction; but}
\NormalTok{    {-} (ii) not resident in a State bound by the 2005 Hague Convention, as defined in section 1(3) of the Civil Jurisdiction and Judgments Act 1982 7;}
\NormalTok{{-} (c) the claimant is a company or other body (whether incorporated inside or outside Great Britain) and there is reason to believe that it will be unable to pay the defendant’s costs if ordered to do so;}
\NormalTok{{-} (d) the claimant has changed his address since the claim was commenced with a view to evading the consequences of the litigation;}
\NormalTok{{-} (e) the claimant failed to give his address in the claim form, or gave an incorrect address in that form;}
\NormalTok{{-} (f) the claimant is acting as a nominal claimant, other than as a representative claimant under Part 19, and there is reason to believe that he will be unable to pay the defendant’s costs if ordered to do so;}
\NormalTok{{-} (g) the claimant has taken steps in relation to his assets that would make it difficult to enforce an order for costs against him.}

\NormalTok{(Rule 3.4 allows the court to strike out a statement of case and Part 24 for it to give summary judgment)}
\end{Highlighting}
\end{Shaded}

\hypertarget{purpose}{%
\subsubsection{Purpose}\label{purpose}}

Allows for a redistribution of costs, which can rack up very quickly.
Better than being in a position where D has spent all their money on
legal costs and does not have the funds to pay the final ordered
damages.

r 25.12(1): D may apply for security for their costs of the proceedings
(note this includes a claimant defending a counterclaim).

\hypertarget{discretionary-power}{%
\subsubsection{Discretionary Power}\label{discretionary-power}}

r 25.13(1): the court may make an order for security for costs if:

\begin{enumerate}
\def\labelenumi{\arabic{enumi}.}
\tightlist
\item
  Considering all the circumstances, making such an order is just, and
\item
  At least one of Part 25 conditions are met (set out in rr 25.13 \&
  25.14). These include:

  \begin{enumerate}
  \def\labelenumii{\arabic{enumii}.}
  \tightlist
  \item
    Claimant resident outside of a 2005 Hague Convention State

    \begin{itemize}
    \tightlist
    \item
      Resident means where an individual claimant usually lives

      \begin{itemize}
      \tightlist
      \item
        Construed according to its ordinary and natural meaning
      \item
        Includes, e.g., full-time education ({[}{[}R v Barnet LBC Ex
        p.~Shah (Nilish) {[}1983{]} 2 A.C. 309{]}{]})
      \end{itemize}
    \item
      For a company, where the company's central management is located.
    \item
      Where some claimants are resident out of the jurisdiction, but
      others are not, the court has jurisdiction to order security for
      costs based on all the circumstances of the case.
    \end{itemize}
  \item
    Claimant in an impecunious company unlikely to be able to pay D's
    costs

    \begin{itemize}
    \tightlist
    \item
      D will produce evidence of poor credit ratings etc.
    \item
      Court can make presumptions against a company which does not
      disclose its assets where there is no publicly available data.
    \item
      D does not have to show on a balance of probabilities that
      claimant company will be unable to pay, but does need to give the
      court reason to believe company will be unable to pay
      ({[}{[}Jirehouse Capital v Beller {[}2008{]} EWCA Civ 908;
      {[}2009{]} 1 W.L.R. 751 {]}{]})
    \item
      Note, this is `will be unable', rather than `might be unable'
      ({[}{[}Re Unisoft Group (No.2) {[}1993{]} B.C.L.C. 532{]}{]})
    \item
      A liquidator or receiver bringing proceedings in the name of an
      insolvent company is not under any duty to ensure that the company
      has sufficient funds to pay any costs awarded to the defendant.
    \end{itemize}
  \item
    Claimant has taken steps to make enforcement difficult

    \begin{itemize}
    \tightlist
    \item
      See {[}{[}Ackerman v Ackerman {[}2012{]} 3 Costs LO 303{]}{]}
    \item
      Objective test: what steps have the claimant taken in relation to
      their assets
    \item
      No time limit for when steps were taken; could be before the
      commencement of proceedings.
    \item
      But motive/ intention/ time of taking steps all relevant factors.
    \end{itemize}
  \item
    Statutory provisions

    \begin{itemize}
    \tightlist
    \item
      s 70(6) of the Arbitration Act 1996 enables the court to grant
      security in relation to various applications to challenge an
      arbitration award or appeal on a point of law.
    \end{itemize}
  \end{enumerate}
\end{enumerate}

\hypertarget{relevant-factors}{%
\subsubsection{Relevant Factors}\label{relevant-factors}}

\begin{enumerate}
\def\labelenumi{\arabic{enumi}.}
\tightlist
\item
  Strength of claim and defence.

  \begin{itemize}
  \tightlist
  \item
    Any open admission or offer of settlement and any Part 36 offer made
    by the defendant will tend to undermine the application.
  \item
    The true strength of each party's case may not be easy to assess,
    and a detailed consideration of the merits is discouraged.
  \item
    Merits of the case only relevant where it can be demonstrated that
    the party relying on merits could show a very high probability that
    the party would succeed ({[}{[}Porzelack KG v Porzelack (UK) Ltd
    {[}1987{]} 1 W.L.R. 420{]}{]})
  \end{itemize}
\item
  Claimant's ability to provide security.

  \begin{itemize}
  \tightlist
  \item
    The court must strike a balance between the right of the claimant to
    pursue a genuine claim and the unfairness to the defendant of
    allowing the claimant to do so without providing security.
  \item
    If the effect of an order for security would be to prevent the
    respondent to application from continuing its claim, then security
    should not be ordered.

    \begin{itemize}
    \tightlist
    \item
      Making an order which the claimant cannot meet could breach their
      right to a fair trial (Art 6(1) ECHR).
    \item
      Burden of proving that an order for security will \textbf{stifle}
      the claim lies on the claimant on a balance of probabilities
      ({[}{[}Al-Koronky v Time Life Entertainment Group Ltd {[}2005{]}
      EWHC 1688 (QB){]}{]})
    \end{itemize}
  \item
    Cost of providing securities of different types is relevant.
  \item
    D unlikely to be able to persuade the court to order security where
    the claimant is funding litigation by a CFA.
  \end{itemize}
\item
  Causes of impecuniosity

  \begin{itemize}
  \tightlist
  \item
    V may be able to persuade the court that shortage of money is caused
    by D's behaviour.
  \end{itemize}
\item
  Property within jurisdiction

  \begin{itemize}
  \tightlist
  \item
    If C has substantial property of a permanent or fixed nature within
    the jurisdiction, unlikely that a security will be granted, since
    the property should be adequate security.
  \item
    The court will not infer the existence of a real risk that assets
    within this country will be dissipated or shipped abroad to avoid
    their being available to satisfy a judgment for costs unless there
    is reason to question the probity of the claimant.
  \end{itemize}
\item
  Timing of application

  \begin{itemize}
  \tightlist
  \item
    As soon as practicable.
  \item
    Allowing a very late claim for security of costs could be
    disproportionate under Art 6 ECHR.
  \end{itemize}
\item
  Counter-claim

  \begin{itemize}
  \tightlist
  \item
    The court may refuse to order security for costs in respect of a
    claim where the same issues arise on a counterclaim in the same
    proceedings
    (\href{https://uk.westlaw.com/Link/Document/FullText?findType=Y\&serNum=1993253243\&pubNum=4667\&originatingDoc=I32D50AF055AF11E797D3B1B628A5D84C\&refType=UC\&originationContext=document\&transitionType=CommentaryUKLink\&ppcid=8dad670bb95640e991013ba44df97305\&contextData=(sc.Category)}{\emph{BJ
    Crabtree (Insulation) Ltd v GPT Communications Systems (1990) 59
    B.L.R. 43, CA}}).
  \item
    Undesirability of one-sided litigation.
  \item
    Where a claim and counterclaim both have independent vitality and
    each side can establish grounds for security against the other, the
    court will normally make orders for security against both sides or
    neither side.
  \end{itemize}
\end{enumerate}

\hypertarget{procedure-2}{%
\subsubsection{Procedure}\label{procedure-2}}

\begin{itemize}
\tightlist
\item
  Make application for security as soon as facts justifying application
  are apparent
\item
  For other applications, applicant encouraged to indicate they intend
  to apply on directions questionnaire.
\item
  D should write to claimant and ask for voluntary security, before
  going ahead with a court order.
\item
  Application must be supported by written evidence.
\item
  D usually applies in relation to costs already incurred and estimated
  future costs.
\end{itemize}

\hypertarget{form-and-effect-of-order}{%
\subsubsection{Form and Effect of
Order}\label{form-and-effect-of-order}}

Order will specify:

\begin{itemize}
\tightlist
\item
  Amount of security
\item
  Date by which claimant must provide it
\item
  Form it must take.

  \begin{itemize}
  \tightlist
  \item
    Most commonly, C required to make a payment into court.
  \item
    Security held by D's solicitor
  \item
    Banker's draft or guarantee.
  \item
    Charge over C's property ({[}{[}Xhosa Office Rentals Ltd v Multi
    High Tec PCB Ltd{]}{]})
  \item
    Suitable undertaking to the court.
  \end{itemize}
\end{itemize}

\begin{Shaded}
\begin{Highlighting}[]
\NormalTok{title: What if C fails to provide the security by the date ordered?}
\NormalTok{No CPR rule, but court will usually order that the claim is struck out and D entitled to apply for judgment to be entered and costs assessed. }
\end{Highlighting}
\end{Shaded}

\hypertarget{security-under-r-3.15}{%
\subsubsection{Security Under R 3.1(5)}\label{security-under-r-3.15}}

The court has the \textbf{discretionary power} under r 3.1(5) to order a
party to \textbf{make a payment into court} if that party has, without
good reason, failed to comply with a rule, Practice Direction or
pre-action protocol.

The court will have regard to:

\begin{itemize}
\tightlist
\item
  Amount in dispute
\item
  Costs the parties have incurred or may incur in future (r 3.1(6))
\item
  Ability of the party to apply for relief under r 3.
\end{itemize}

Such orders are rare, and tends to be in cases of repeated breaches/ bad
faith.

\hypertarget{case-management}{%
\section{Case Management}\label{case-management}}

\hypertarget{courts-powers}{%
\subsection{Court's Powers}\label{courts-powers}}

The court has a duty to manage cases actively:

\begin{Shaded}
\begin{Highlighting}[]
\NormalTok{title: r 1.4 CPR 1998}
\NormalTok{(1) The court must further the overriding objective by actively managing cases.}

\NormalTok{(2) Active case management includes —}
\NormalTok{{-} (a) encouraging the parties to co{-}operate with each other in the conduct of the proceedings;}
\NormalTok{{-} (b) identifying the issues at an early stage;}
\NormalTok{{-} (c) deciding promptly which issues need full investigation and trial and accordingly disposing summarily of the others;}
\NormalTok{{-} (d) deciding the order in which issues are to be resolved;}
\NormalTok{{-} (e) encouraging the parties to use an alternative dispute resolution(GL) procedure if the court considers that appropriate and facilitating the use of such procedure;}
\NormalTok{{-} (f) helping the parties to settle the whole or part of the case;}
\NormalTok{{-} (g) fixing timetables or otherwise controlling the progress of the case;}
\NormalTok{{-} (h) considering whether the likely benefits of taking a particular step justify the cost of taking it;}
\NormalTok{{-} (i) dealing with as many aspects of the case as it can on the same occasion;}
\NormalTok{{-} (j) dealing with the case without the parties needing to attend at court;}
\NormalTok{{-} (k) making use of technology; and}
\NormalTok{{-} (l) giving directions to ensure that the trial of a case proceeds quickly and efficiently.}
\end{Highlighting}
\end{Shaded}

r 3.1(2): lists many of the court's powers. The court can make any order
subject to conditions and can specify the consequence of non-compliance.

Under r 3.1(5) the court may make an order that a party pay a sum of
money into court pending the outcome of proceedings if that party has
failed to comply with a rule/ PD/ pre-action protocol without good
reason. The court must have regard to both the amount in dispute and the
costs the parties have/ will incur.

\hypertarget{striking-out-statement-of-case}{%
\subsubsection{Striking Out Statement of
Case}\label{striking-out-statement-of-case}}

\begin{Shaded}
\begin{Highlighting}[]
\NormalTok{title: r 3.4(2) CPR 1998}
\NormalTok{The court may strike out a statement of case if it appears to the court –}
\NormalTok{{-} (a) that the statement of case discloses no reasonable grounds for bringing or defending the claim;}
\NormalTok{{-} (b) that the statement of case is an abuse of the court’s process or is otherwise likely to obstruct the just disposal of the proceedings; or}
\NormalTok{{-} (c) that there has been a failure to comply with a rule, practice direction or court order.}
\end{Highlighting}
\end{Shaded}

This is frequently used when a statement of case is vague/ incoherent,
or where the facts disclosed do not disclose any legally recognisable
claim. If a statement of case is salvageable through amendment, the
court may give a part the opportunity to do this.

The court may make such an order on its own volition/ following an
application to court by the other party. If D did not comply with the
order, then the claimant would be able to obtain judgment simply by
filing a request for judgment (stating that the right to enter judgment
has arisen because the court's order has not been obeyed). D could then
apply to court under r 3.6 for the judgment to be set aside, not after
14 days after judgment served. r 3.9 sets out relief from sanctions.

\hypertarget{non-compliance-with-rule-pd-court-order}{%
\subsubsection{Non-compliance with rule/ PD/ Court
Order}\label{non-compliance-with-rule-pd-court-order}}

Valid grounds for striking out a statement of case. But other sanctions
also available. So which sanction should be used when?

\begin{itemize}
\tightlist
\item
  Starting point: there are less drastic but equally effective ways of
  dealing with a failure to comply with CPR/ court orders ({[}{[}Biguzzi
  v Rank Leisure plc {[}1999{]} 1 WLR 1926{]}{]}, which in many cases
  would produce a more just result.
\item
  All circumstances are considered, particularly factors relevant to r
  3.9.
\item
  Overriding objective and duty to ensure fairness, a central
  consideration in the exercise of the court's discretion.
\item
  Striking out a case because of procedural irregularities may violate
  Article 6(1) ECHR right to a fair trial.
\item
  So striking out used where default is so bad it amounts to abuse of
  court/ where it is no longer possible to have a fair trial.
\end{itemize}

\hypertarget{other-sanctions}{%
\subsubsection{Other Sanctions}\label{other-sanctions}}

\begin{longtable}[]{@{}
  >{\raggedright\arraybackslash}p{(\columnwidth - 2\tabcolsep) * \real{0.1167}}
  >{\raggedright\arraybackslash}p{(\columnwidth - 2\tabcolsep) * \real{0.8833}}@{}}
\toprule()
\begin{minipage}[b]{\linewidth}\raggedright
Sanction
\end{minipage} & \begin{minipage}[b]{\linewidth}\raggedright
Details
\end{minipage} \\
\midrule()
\endhead
Costs & Require the party in default to pay the other party's costs
occasioned by the delay on an indemnity basis. \\
Interest & Court may make orders affecting the interest payable on any
damages subsequently awarded to C. \\
Limited issues & Limiting the issues which are allowed to proceed to
trial. \\
\bottomrule()
\end{longtable}

\hypertarget{unless-order}{%
\subsubsection{Unless Order}\label{unless-order}}

\begin{Shaded}
\begin{Highlighting}[]
\NormalTok{title: A party has not taken a step in proceedings according to a court order, what should the other party do?}
\NormalTok{1. Chase up the defaulting party in correspondence. }
\NormalTok{2. Apply to the court for an unless order. }
\NormalTok{    {-} Do this promptly}
\NormalTok{    {-} First warn the other party of intention to do so.}
\end{Highlighting}
\end{Shaded}

PD 28 para 5.1:

\begin{quote}
{[}w{]}here a party has failed to comply with a direction given by the
court, any other party may apply for an order to enforce compliance or
for a sanction to be imposed or both of these.
\end{quote}

If the other party cannot make the deadline set by an unless order,
should apply to the court to have it extended, since the strike out
takes effect without any other order.

\hypertarget{relief-from-sanctions}{%
\subsection{Relief From Sanctions}\label{relief-from-sanctions}}

\hypertarget{timing}{%
\subsubsection{Timing}\label{timing}}

If a Rule/ PD/ court order requires a party to do something in a
specific time, this time can be extended by up to 28 days by agreement
between the parties, unless explicitly prohibited by the court/ risks
breaking hearing date (r 3.8(4)).

\hypertarget{relevant-factors-1}{%
\subsubsection{Relevant Factors}\label{relevant-factors-1}}

By r 3.9(1), where a party applies for relief from any sanction for
failure to comply with any\\
rule, practice direction or court order, the court will consider all the
circumstances of the\\
case, so as to enable it to deal justly with the application, including
the need:

\begin{enumerate}
\def\labelenumi{\arabic{enumi}.}
\tightlist
\item
  for litigation to be conducted efficiently and at proportionate cost;
  and
\item
  to enforce compliance with rules, practice directions and orders.
\end{enumerate}

\hypertarget{denton}{%
\subsubsection{Denton}\label{denton}}

\begin{Shaded}
\begin{Highlighting}[]
\NormalTok{title: Denton test}
\NormalTok{The test was established in [[Denton v TH White Ltd [2014] EWCA Civ 906]] as follows:}
\NormalTok{1. Identify and assess the seriousness or significance of the relevant failure. If a breach was not serious or significant, relief would usually be granted and there would be no need to spend much time on the second and third stages.}
\NormalTok{2. Consider why the failure or default occurred.}
\NormalTok{3. Evaluate ‘all the circumstances of the case, so as to enable [the court] to deal justly with the application’.}
\end{Highlighting}
\end{Shaded}

\begin{itemize}
\tightlist
\item
  The court normally considers as one of the relevant circumstances of
  the case the defaulter's previous conduct in the litigation, including
  any non-compliance with court orders.
\item
  A party might be penalised if they sought to take advantage of a
  mistake by an opponent where the failure was neither serious nor
  significant. Opposing an application for relief would be viewed as a
  breach of r 1.3, which requires parties to help the court to further
  the \textbf{overriding objective}.
\item
  Breaches which affect the effective timetabling of cases are likely to
  be considered significant.
\item
  Lack of promptness in applying for relief may often be a critical
  factor.
\end{itemize}

\hypertarget{directions-questionnaire}{%
\subsection{Directions Questionnaire}\label{directions-questionnaire}}

Allocation is dealt with by Part 26 CPR 1998. Timeline:

\begin{itemize}
\tightlist
\item
  Defence served
\item
  Court serves on each party a notice of the proposed allocation (Form
  149A/ N149B/ N149C for small claims/ fast-track/ multi-track).
\item
  Parties file and serve answers to `directions questionnaire' (form
  N181, or N180 if small claims track provisionally allocated).

  \begin{itemize}
  \tightlist
  \item
    Directions questionnaire available online, legal representatives
    should download and fill this in unprompted.
  \end{itemize}
\item
  If multi-track provisionally allocated, also send:

  \begin{itemize}
  \tightlist
  \item
    Case summary (if case management conference)
  \item
    Disclosure report
  \item
    Costs budget
  \item
    Budget discussion report.
  \end{itemize}
\end{itemize}

\hypertarget{completing-questionnaire}{%
\subsubsection{Completing
Questionnaire}\label{completing-questionnaire}}

Form N181 must be completed by each party and filed by the set date.
Parties should consult one another and cooperate in completing the
questionnaire.

r 26.3(6A): the date for filing \textbf{cannot} be varied by agreement
between the parties. Divided into 10 sections.

\begin{longtable}[]{@{}
  >{\raggedright\arraybackslash}p{(\columnwidth - 2\tabcolsep) * \real{0.1018}}
  >{\raggedright\arraybackslash}p{(\columnwidth - 2\tabcolsep) * \real{0.8982}}@{}}
\toprule()
\begin{minipage}[b]{\linewidth}\raggedright
Directions questionnaire section
\end{minipage} & \begin{minipage}[b]{\linewidth}\raggedright
Details
\end{minipage} \\
\midrule()
\endhead
Part A Settlement & Parties reminded that they should try to settle a
case, and that answers will impact costs decision. Parties must indicate
if they want to try to settle (one month stay of proceedings). Or court
can order a stay on its own initiative. Parties can request the court to
arrange a mediation appointment. \\
Part B Court & Asks parties whether there is any reason to be heard in a
particular court, along with reasons. Special provisions if claim issued
at Central Office of Royal Courts of Justice (RCJ). \\
Part C Pre-action protocols & Have parties complied with pre-action
protocols, if not then why. \\
Part D Case management information & Asks parties if they have made an
application to the court. Relevant to allocating a track: the largest of
the claims/ counterclaims will determine the track which the case is
placed on. More questions if multi-track allocation (e.g., about
disclosure of electronic documents). \\
Part E Experts & Parties should indicate if they wis hto rely on expert
evidence at trial. \\
Part F Witnesses & Parties name witnesses of fact they intend to call at
trial, and identify facts each witness will address. But not obliged to
`name names', can just say number of witnesses. Court can limit
numbers. \\
Part G Trial & Parties give a realistic estimate of how long the trial
will last. \\
Part H Costs & If multi-track, legal advisers should file and serve a
costs budged. \\
Part I Other info & State purpose of any application the party intends
to make in the near future. \\
Part J Directions & Parties should attempt to agree proposed directions.
Specimen directions for multi-track found on MOJ website. All proposed
directions for fast-track must be based on CPR Part 28. \\
\bottomrule()
\end{longtable}

\hypertarget{transfer-of-money-claims}{%
\subsubsection{Transfer of Money
Claims}\label{transfer-of-money-claims}}

r 26.2: automatic transfer of High Court proceedings where:

\begin{itemize}
\tightlist
\item
  The claim is for a specified amount of money,
\item
  D an individual
\item
  Claim commenced in a court other than D's home court.
\end{itemize}

When a defence is filed, the claim will be sent to D's home court.

r 26.2A: transfer of money claims within the County Court--similar
provisions.

If proceedings are transferred, the court in which the proceedings
commenced will serve the notice of proposed allocation before the
proceedings are transferred and will not transfer proceedings until all
parties have complied with the notice/ time for doing so expired.

\hypertarget{failure-to-file-directions-questionnaire}{%
\subsubsection{Failure to File Directions
Questionnaire}\label{failure-to-file-directions-questionnaire}}

\hypertarget{county-court-money-claim}{%
\paragraph{{[}{[}County Court{]}{]} Money
Claim}\label{county-court-money-claim}}

If r 26.2A applies, and any party does not comply with notice of
proposed allocation, the court will serve a further notice, requiring
compliance within 7 days. If they still don't comply, the party's
statement of case is struck out automatically without a further court
order.

\hypertarget{other-claims}{%
\paragraph{Other Claims}\label{other-claims}}

If a party fails to comply with the notice of proposed allocation, the
court makes an order it considers appropriate. This may include: (i) an
order for directions; (ii) an order striking out the claim; (iii) an
order striking out the defence and entering judgment; or (iv) listing
the case for a case management conference.

\hypertarget{exchanging-questionnaires}{%
\paragraph{Exchanging Questionnaires}\label{exchanging-questionnaires}}

Parties must exchange questionnaires--check for any new info.

\hypertarget{allocation-of-track}{%
\subsection{Allocation of Track}\label{allocation-of-track}}

Most important factor is the value of the claim. But the court also
takes into account other factors (r 26.8(1)).

MERMAID3

\begin{Shaded}
\begin{Highlighting}[]
\NormalTok{title: r 16.8(1) CPR 1998}
\NormalTok{When deciding the track for a claim, the matters to which the court shall have regard include –}
\NormalTok{{-} (a) the financial value, if any, of the claim;}
\NormalTok{{-} (b) the nature of the remedy sought;}
\NormalTok{{-} (c) the likely complexity of the facts, law or evidence;}
\NormalTok{{-} (d) the number of parties or likely parties;}
\NormalTok{{-} (e) the value of any counterclaim or other Part 20 claim and the complexity of any matters relating to it;}
\NormalTok{{-} (f) the amount of oral evidence which may be required;}
\NormalTok{{-} (g) the importance of the claim to persons who are not parties to the proceedings;}
\NormalTok{{-} (h) the views expressed by the parties; and}
\NormalTok{{-} (i) the circumstances of the parties.}
\end{Highlighting}
\end{Shaded}

\begin{Shaded}
\begin{Highlighting}[]
\NormalTok{title: r 26.8(2)}
\NormalTok{It is for the court to assess the financial value of a claim and in doing so it will disregard –}
\NormalTok{{-} (a) any amount not in dispute;}
\NormalTok{{-} (b) any claim for interest;}
\NormalTok{{-} (c) costs; and}
\NormalTok{{-} (d) any contributory negligence.}
\end{Highlighting}
\end{Shaded}

The fast track is the normal track for claims with a value exceeding
£10,000, but not £25,000, only if the trial is likely to last for no
longer than one day; oral expert evidence at trial will be limited to no
more than one expert per party in relation to any expert field and there
will be expert evidence in no more than two expert fields.

PD 26, para 8.1(2): the court may allocate a claim to the small claims
track even if it exceeds £10,000.

If a party is dissatisfied, may appeal/ apply to the court for
reallocation of the claim (PD 26 para 11).

\hypertarget{part-27-small-claims-track}{%
\subsubsection{Part 27: Small Claims
Track}\label{part-27-small-claims-track}}

Aims are to process claims quickly at minimal cost.

\begin{Shaded}
\begin{Highlighting}[]
\NormalTok{title: PD 26 para 8(1)}
\NormalTok{{-} (a) The small claims track is intended to provide a proportionate procedure by which most straightforward claims with a financial value of not more than £10,000 can be decided, without the need for substantial pre{-}hearing preparation and the formalities of a traditional trial, and without incurring large legal costs.}
\NormalTok{{-} (b) The procedures laid down in Part 27 for the preparation of the case and the conduct of the hearing are designed to make it possible for a litigant to conduct their own case without legal representation if they wish.}
\NormalTok{{-} (c) Cases generally suitable for the small claims track will include consumer disputes, accident claims, disputes about the ownership of goods, and most disputes between a landlord and tenant other than those for possession.}
\NormalTok{{-} (d) A case involving a disputed allegation of dishonesty will not usually be suitable for the small claims track.}
\end{Highlighting}
\end{Shaded}

All claims other than road traffic accidents, personal injury or housing
disrepair will be\\
referred to the Small Claims Mediation Service if all parties consent to
referral in their\\
directions' questionnaire. If the claim is settled, proceeds
automatically stayed (r 26.4A(5)).

In most cases, the court will order standard direction s and fix a date
for the final hearing. Rare for there to be a preliminary hearing. By r
27.2(3), the court may of its own initiative order a party to provide
further information. Under r 27.14, the costs which can be recovered by
a successful party are extremely limited.

Standard directions which a court gives set out in PD 27 appendices. The
hearing will be informal, and if all parties agree, the court can deal
with the claim without a hearing (i.e., just based on the statements of
case).

r 27.14: generally, the only costs recoverable are the fixed costs
attributable to issuing the claim, any court fees paid and sums to
represent travelling expenses and loss of earnings or leave.

\hypertarget{part-28-fast-track}{%
\subsubsection{Part 28: Fast Track}\label{part-28-fast-track}}

Generally, the court will allocate a case to this track without a
hearing and order standard directions.

General timetable:

\begin{longtable}[]{@{}
  >{\raggedright\arraybackslash}p{(\columnwidth - 2\tabcolsep) * \real{0.6552}}
  >{\raggedright\arraybackslash}p{(\columnwidth - 2\tabcolsep) * \real{0.3448}}@{}}
\toprule()
\begin{minipage}[b]{\linewidth}\raggedright
Event
\end{minipage} & \begin{minipage}[b]{\linewidth}\raggedright
Period from date of allocation
\end{minipage} \\
\midrule()
\endhead
Disclosure & 4 weeks \\
Exchange of witness statements & 10 weeks \\
Exchange of experts' reports & 14 weeks \\
Court sends pre-trial checklist, listing questionnaires & 20 weeks \\
Parties file pre-trial checklists, listing questionnaires & 22 weeks \\
Hearing & 30 weeks \\
\bottomrule()
\end{longtable}

\hypertarget{varying-directions}{%
\paragraph{Varying Directions}\label{varying-directions}}

\begin{itemize}
\tightlist
\item
  To vary certain directions, an application must be made to the court
  (r 28.4):

  \begin{itemize}
  \tightlist
  \item
    The return of a pre-trial checklist
  \item
    Trial
  \item
    Trial period.
  \end{itemize}
\item
  Also, parties cannot agree to vary any matter which would alter these
  dates.
\item
  Apply ASAP (PD 28 para 4.2(1))--and at least within 14 days (para
  4.2(2)).
\end{itemize}

\hypertarget{variation-by-consent}{%
\paragraph{Variation by Consent}\label{variation-by-consent}}

If agreement relates to an act which does not need the court's consent,
parties need not file written agreement to vary. Else, apply to court
for an order by consent. File a draft of the order sought and agreed
statement of the reasons why.

\hypertarget{failure-to-comply}{%
\paragraph{Failure to Comply}\label{failure-to-comply}}

PD 28 para 5: if a party fails to comply with a direction, another party
may apply for an order enforcing compliance/ for sanctions.

This will only exceptionally lead to a postponement of the trial date
(para 5.4(1)). Court may order that the trial continue for issues which
can be resolved.

\begin{Shaded}
\begin{Highlighting}[]
\NormalTok{The trial date is sacrosanct. }
\end{Highlighting}
\end{Shaded}

\hypertarget{exchange-of-witness-statements-reports}{%
\paragraph{Exchange of Witness Statements \&
Reports}\label{exchange-of-witness-statements-reports}}

Usually exchanged simultaneously. The court will not make a direction
giving permission for an expert to give oral evidence unless it believes
that it is necessary in the interests of justice to do so (PD 28, para
7.2(4)(b) and r 35.5(2)). Usual provision is expert evidence by written
reports. Often a single joint expert is appointed.

\hypertarget{pre-trial-checklist-pd-28-para-6}{%
\paragraph{Pre-trial Checklist (PD 28 Para
6)}\label{pre-trial-checklist-pd-28-para-6}}

\begin{Shaded}
\begin{Highlighting}[]
\NormalTok{title: Purpose}
\NormalTok{To check that directions have been complied with so the court can fix a date for trial.}
\end{Highlighting}
\end{Shaded}

The court will specify when to return the pre-trial checklist (
\(\geq 8\) weeks before trial). Form N170 used. Parties encouraged to
exchange before filing.

\hypertarget{listing-directions-pd-28-para-7}{%
\paragraph{Listing Directions (PD 28 Para
7)}\label{listing-directions-pd-28-para-7}}

The court will confirm or fix the date, length and place of the trial.
Normally \(\geq 3\) weeks notice of trial. Parties should try to agree
directions, including

\begin{itemize}
\item
  \begin{enumerate}
  \def\labelenumi{(\alph{enumi})}
  \tightlist
  \item
    evidence;
  \end{enumerate}
\item
  \begin{enumerate}
  \def\labelenumi{(\alph{enumi})}
  \setcounter{enumi}{1}
  \tightlist
  \item
    a trial timetable and time estimate;
  \end{enumerate}
\item
  \begin{enumerate}
  \def\labelenumi{(\alph{enumi})}
  \setcounter{enumi}{2}
  \tightlist
  \item
    preparation of a trial bundle.
  \end{enumerate}

  \begin{itemize}
  \tightlist
  \item
    Should be lodged with court \(3\leq \text{days} \leq 7\) before
    trial.
  \item
    Should include 250 word case summary, agreed by parties if possible.
  \end{itemize}
\end{itemize}

The court may fix a listing hearing on three days' notice if either:

\begin{itemize}
\item
  \begin{enumerate}
  \def\labelenumi{(\alph{enumi})}
  \tightlist
  \item
    a party has failed to file the pre-trial checklist; or
  \end{enumerate}
\item
  \begin{enumerate}
  \def\labelenumi{(\alph{enumi})}
  \setcounter{enumi}{1}
  \tightlist
  \item
    a party has filed an incomplete pre-trial checklist; or
  \end{enumerate}
\item
  \begin{enumerate}
  \def\labelenumi{(\alph{enumi})}
  \setcounter{enumi}{2}
  \tightlist
  \item
    a hearing is needed to decide what directions for trial are
    appropriate.
  \end{enumerate}
\end{itemize}

\hypertarget{part-29-multi-track}{%
\subsubsection{Part 29: Multi-track}\label{part-29-multi-track}}

\hypertarget{directions}{%
\paragraph{Directions}\label{directions}}

Court will either:

\begin{itemize}
\tightlist
\item
  Give directions and set a timetable

  \begin{itemize}
  \tightlist
  \item
    PD 29 para 410
  \item
    General approach is to give directions and maximise reconciliation.
  \end{itemize}
\item
  Fix a case management conference/ pre-trial review, and give
  directions it sees fit.
\end{itemize}

No deadline for the trial date.

The starting point for drafting case management directions will be
relevant model standard directions. It is usual for the last direction
to require the parties to `inform the court immediately if the claim is
settled, whether or not it is then possible to file a draft consent
order to give effect to their agreement'.

\hypertarget{case-management-conference}{%
\paragraph{Case Management
Conference}\label{case-management-conference}}

\begin{Shaded}
\begin{Highlighting}[]
\NormalTok{title: PD 29 para 5.1}
\NormalTok{The court will at any case management conference:}

\NormalTok{(1) review the steps which the parties have taken in the preparation of the case, and in particular their compliance with any directions that the court may have given,}

\NormalTok{(2) decide and give directions about the steps which are to be taken to secure the progress of the claim in accordance with the overriding objective, and}

\NormalTok{(3) ensure as far as it can that all agreements that can be reached between the parties about the matters in issue and the conduct of the claim are made and recorded.}
\end{Highlighting}
\end{Shaded}

Court will consider (PD 29 para 5.3):

\begin{itemize}
\tightlist
\item
  Whether each party has clearly stated their case
\item
  Whether any amendments to any documents needed
\item
  Necessary disclosure of documents
\item
  Expert evidence reasonably required
\item
  Factual evidence to disclose
\item
  Arrangements about further info/ questioning experts
\item
  Whether to order a split trial on one or more issue.
\end{itemize}

\hypertarget{who-should-attend}{%
\subparagraph{Who Should Attend}\label{who-should-attend}}

\begin{itemize}
\tightlist
\item
  r 29.3(2): representative familiar with the case must attend the case
  management conference.
\item
  PD 29, para 5.2(2): the representative should be someone who is
  personally involved in the conduct of the case, and who has both the
  authority and information available to deal with any matter that may
  reasonably be expected to be dealt with.
\item
  PD 29 para 5.2(3): where the inadequacy of the person attending, or of
  their instructions, leads to the adjournment of a hearing, the court
  will expect to make a wasted costs order.
\item
  May also be SRA implications: principle 7 (client's best interest),
  principle 1 (proper administration of justice) .
\end{itemize}

\hypertarget{expert-evidence}{%
\subparagraph{Expert Evidence}\label{expert-evidence}}

\begin{Shaded}
\begin{Highlighting}[]
\NormalTok{title: PD 29 para 5.5}
\NormalTok{1. The court will not at this stage give permission to use expert evidence unless it can identify each expert by name or field in its order and say whether his evidence is to be given orally or by the use of his report.}
\NormalTok{2. A party who obtains expert evidence before obtaining a direction about it does so at his own risk as to costs, except where he obtained the evidence in compliance with a pre{-}action protocol.}
\end{Highlighting}
\end{Shaded}

\hypertarget{preparing}{%
\subparagraph{Preparing}\label{preparing}}

\begin{Shaded}
\begin{Highlighting}[]
\NormalTok{title: PD 29 para 5.6}
\NormalTok{To assist the court, the parties and their legal advisers should:}
\NormalTok{1. ensure that all documents that the court is likely to ask to see (including witness statements and experts’ reports) are brought to the hearing,}
\NormalTok{2. consider whether the parties should attend,}
\NormalTok{3. consider whether a case summary will be useful, and}
\NormalTok{4. consider what orders each wishes to be made and give notice of them to the other parties.}
\end{Highlighting}
\end{Shaded}

\hypertarget{case-summary}{%
\subparagraph{Case Summary}\label{case-summary}}

\begin{Shaded}
\begin{Highlighting}[]
\NormalTok{title: PD 29 para 5.7}
\NormalTok{(1) A case summary:}
\NormalTok{{-} (a) should be designed to assist the court to understand and deal with the questions before it,}
\NormalTok{{-} (b) should set out a brief chronology of the claim, the issues of fact which are agreed or in dispute and the evidence needed to decide them,}
\NormalTok{{-} (c) should not normally exceed 500 words in length, and}
\NormalTok{{-} (d) should be prepared by the claimant and agreed with the other parties if possible.}
\end{Highlighting}
\end{Shaded}

The case summary is meant to assist the judge to identify issues in
dispute. In the case summary, parties can name witnesses, or just say
how many will likely be used and fields of expertise.

\hypertarget{variation-of-directions}{%
\paragraph{Variation of Directions}\label{variation-of-directions}}

A party must apply to the court if they wish to vary the date which the
court has fixed for:

\begin{itemize}
\item
  \begin{enumerate}
  \def\labelenumi{(\alph{enumi})}
  \tightlist
  \item
    a case management conference;
  \end{enumerate}
\item
  \begin{enumerate}
  \def\labelenumi{(\alph{enumi})}
  \setcounter{enumi}{1}
  \tightlist
  \item
    a pre-trial review;
  \end{enumerate}
\item
  \begin{enumerate}
  \def\labelenumi{(\alph{enumi})}
  \setcounter{enumi}{2}
  \tightlist
  \item
    the return of a pre-trial checklist under r 29.6;
  \end{enumerate}
\item
  \begin{enumerate}
  \def\labelenumi{(\alph{enumi})}
  \setcounter{enumi}{3}
  \tightlist
  \item
    the trial; or
  \end{enumerate}
\item
  \begin{enumerate}
  \def\labelenumi{(\alph{enumi})}
  \setcounter{enumi}{4}
  \tightlist
  \item
    the trial period.
  \end{enumerate}
\end{itemize}

A party wishing to vary a direction must apply asap. There is an appeals
process - the court will give all parties 3 days' notice of any
associated hearing.

\hypertarget{non-compliance-with-directions-pd-29-para-7}{%
\paragraph{Non-compliance with Directions (PD 29 Para
7)}\label{non-compliance-with-directions-pd-29-para-7}}

The other party can apply for an order for compliance/ sanctions.

\hypertarget{pre-trial-checklist-pd-29-para-8}{%
\paragraph{Pre-trial Checklist (PD 29 Para
8)}\label{pre-trial-checklist-pd-29-para-8}}

Again, deadline \(\geq 8\) weeks before trial, and parties given
\(\geq 14\) days to complete pre-trial checklist. Parties encouraged to
exchange copies.

On receipt of pre-trial checklists, the court may decide to hold a
pre-trial review (with \(\geq 7\) days notice). Standard practice in
complex cases.

As soon as practicable after pre-trial checklists/ listing hearing/
pre-trial review, court will set a timetable and notify parties.

\hypertarget{costs-management}{%
\subsection{Costs Management}\label{costs-management}}

\hypertarget{costs-budgets}{%
\subsubsection{Costs Budgets}\label{costs-budgets}}

Parties provide budgets in a prescribed form of their own future costs.
These are updated regularly and submitted for agreement to the party,
and then to the court for adjudication in case of dispute.

Courts make a \textbf{costs management order} to ensure \textbf{future}
costs are reasonable and do not become disproportionate. Not possible to
incur costs and then put them in a budget.

\hypertarget{scope}{%
\paragraph{Scope}\label{scope}}

\begin{itemize}
\tightlist
\item
  r 3.12(1) limits costs management to all multi-track cases where the
  amount of money claimed is less than £10 million.
\item
  A litigant in person does not have to prepare a budget or file a
  budget discussion report.
\end{itemize}

\hypertarget{timing-1}{%
\paragraph{Timing}\label{timing-1}}

The court will set a date for filling costs budget.

\begin{longtable}[]{@{}
  >{\raggedright\arraybackslash}p{(\columnwidth - 2\tabcolsep) * \real{0.3214}}
  >{\raggedright\arraybackslash}p{(\columnwidth - 2\tabcolsep) * \real{0.6786}}@{}}
\toprule()
\begin{minipage}[b]{\linewidth}\raggedright
Size of claim
\end{minipage} & \begin{minipage}[b]{\linewidth}\raggedright
Deadline for filing costs budget
\end{minipage} \\
\midrule()
\endhead
Claims \(< £50,000\) & When directions questionnaire is filed \\
Other claims & 21 days before first case management conference. \\
\bottomrule()
\end{longtable}

\hypertarget{contents}{%
\paragraph{Contents}\label{contents}}

\begin{itemize}
\tightlist
\item
  Details of costs already incurred
\item
  Estimate of future costs
\item
  Assumptions on which future costs are based
\item
  Any ADR or settlement discussions.
\end{itemize}

The above are known as `phases'. Budget must be verified by a statement
of truth by a senior lawyer. The Statement of Truth to be completed for
a~budget should read: ``This budget is a fair and accurate statement of
incurred and estimated costs which it would be reasonable and
proportionate for my client to incur in this litigation.'' See Appendix
A(9) for Precedent H.

\hypertarget{budget-discussion-report}{%
\paragraph{Budget Discussion Report}\label{budget-discussion-report}}

Parties should complete a budget discussion report (Precedent R), filed
\(\geq 7\) days before first case management conference. Sets out:

\begin{itemize}
\tightlist
\item
  Figures agreed for each phase
\item
  Figures not agreed
\item
  Summary of grounds of dispute.
\end{itemize}

The court will review this if there are entire phases not agreed upon.

\hypertarget{subsequent-changes-in-figures}{%
\paragraph{Subsequent Changes in
Figures}\label{subsequent-changes-in-figures}}

Difficult to persuade a court to rectify a costs budget. Budget must be
revised if there is a ``significant development''. Significance must be
decided upon in light of all the circumstances.

\begin{itemize}
\tightlist
\item
  Starting point: last approved budget.
\item
  The court can challenge.
\item
  Amended budget must use Precedent T.
\end{itemize}

\hypertarget{failure-to-file-budget}{%
\paragraph{Failure to File Budget}\label{failure-to-file-budget}}

r 3.14: unless the court orders otherwise, any party who fails to file a
budget despite\\
being required to do so, will be treated as having filed a budget
comprising only the applicable court fees: in effect, only the court
fees incurred by the defaulting party may potentially be recoverable
under a costs order in the future. This can subsequently be eased.

\hypertarget{costs-of-preparing-budget}{%
\paragraph{Costs of Preparing Budget}\label{costs-of-preparing-budget}}

Recoverable costs cannot exceed £1,000/ 1\% of approved budget. All
budgeting and costs management costs cannot exceed 2\% of approved
budget.

\hypertarget{judicial-approach}{%
\subsubsection{Judicial Approach}\label{judicial-approach}}

Parties encouraged to agree budgets when possible. ``Reasonable and
proportionate'' construed loosely - lawyers are hella expensive. The
court may not approve costs incurred before the date of any budget.

\hypertarget{practical-points}{%
\subsubsection{Practical Points}\label{practical-points}}

\begin{itemize}
\tightlist
\item
  After a budget has been approved, party must re-file and re-serve the
  budget.
\item
  r 3.18: A costs management order will affect a party who secures a
  standard basis costs order in the litigation, since standard basis
  costs are assessed with reference to the submitted budget. Cannot be
  departed from unless there is good reason. Costs outside the phases
  will be subject to detailed assessment.
\item
  If no costs management order has been made, and there is a difference
  of \(\geq 20\%\) between the costs claimed by a receiving party /costs
  shown in a budget, then the receiving party must provide reasons for
  the difference. If the court thinks this is unreasonable, they can
  choose the recoverable costs.
\end{itemize}

\end{document}
