% Options for packages loaded elsewhere
\PassOptionsToPackage{unicode}{hyperref}
\PassOptionsToPackage{hyphens}{url}
%
\documentclass[
]{article}
\usepackage{amsmath,amssymb}
\usepackage{lmodern}
\usepackage{iftex}
\ifPDFTeX
  \usepackage[T1]{fontenc}
  \usepackage[utf8]{inputenc}
  \usepackage{textcomp} % provide euro and other symbols
\else % if luatex or xetex
  \usepackage{unicode-math}
  \defaultfontfeatures{Scale=MatchLowercase}
  \defaultfontfeatures[\rmfamily]{Ligatures=TeX,Scale=1}
\fi
% Use upquote if available, for straight quotes in verbatim environments
\IfFileExists{upquote.sty}{\usepackage{upquote}}{}
\IfFileExists{microtype.sty}{% use microtype if available
  \usepackage[]{microtype}
  \UseMicrotypeSet[protrusion]{basicmath} % disable protrusion for tt fonts
}{}
\makeatletter
\@ifundefined{KOMAClassName}{% if non-KOMA class
  \IfFileExists{parskip.sty}{%
    \usepackage{parskip}
  }{% else
    \setlength{\parindent}{0pt}
    \setlength{\parskip}{6pt plus 2pt minus 1pt}}
}{% if KOMA class
  \KOMAoptions{parskip=half}}
\makeatother
\usepackage{xcolor}
\usepackage[margin=1in]{geometry}
\usepackage{color}
\usepackage{fancyvrb}
\newcommand{\VerbBar}{|}
\newcommand{\VERB}{\Verb[commandchars=\\\{\}]}
\DefineVerbatimEnvironment{Highlighting}{Verbatim}{commandchars=\\\{\}}
% Add ',fontsize=\small' for more characters per line
\newenvironment{Shaded}{}{}
\newcommand{\AlertTok}[1]{\textcolor[rgb]{1.00,0.00,0.00}{\textbf{#1}}}
\newcommand{\AnnotationTok}[1]{\textcolor[rgb]{0.38,0.63,0.69}{\textbf{\textit{#1}}}}
\newcommand{\AttributeTok}[1]{\textcolor[rgb]{0.49,0.56,0.16}{#1}}
\newcommand{\BaseNTok}[1]{\textcolor[rgb]{0.25,0.63,0.44}{#1}}
\newcommand{\BuiltInTok}[1]{#1}
\newcommand{\CharTok}[1]{\textcolor[rgb]{0.25,0.44,0.63}{#1}}
\newcommand{\CommentTok}[1]{\textcolor[rgb]{0.38,0.63,0.69}{\textit{#1}}}
\newcommand{\CommentVarTok}[1]{\textcolor[rgb]{0.38,0.63,0.69}{\textbf{\textit{#1}}}}
\newcommand{\ConstantTok}[1]{\textcolor[rgb]{0.53,0.00,0.00}{#1}}
\newcommand{\ControlFlowTok}[1]{\textcolor[rgb]{0.00,0.44,0.13}{\textbf{#1}}}
\newcommand{\DataTypeTok}[1]{\textcolor[rgb]{0.56,0.13,0.00}{#1}}
\newcommand{\DecValTok}[1]{\textcolor[rgb]{0.25,0.63,0.44}{#1}}
\newcommand{\DocumentationTok}[1]{\textcolor[rgb]{0.73,0.13,0.13}{\textit{#1}}}
\newcommand{\ErrorTok}[1]{\textcolor[rgb]{1.00,0.00,0.00}{\textbf{#1}}}
\newcommand{\ExtensionTok}[1]{#1}
\newcommand{\FloatTok}[1]{\textcolor[rgb]{0.25,0.63,0.44}{#1}}
\newcommand{\FunctionTok}[1]{\textcolor[rgb]{0.02,0.16,0.49}{#1}}
\newcommand{\ImportTok}[1]{#1}
\newcommand{\InformationTok}[1]{\textcolor[rgb]{0.38,0.63,0.69}{\textbf{\textit{#1}}}}
\newcommand{\KeywordTok}[1]{\textcolor[rgb]{0.00,0.44,0.13}{\textbf{#1}}}
\newcommand{\NormalTok}[1]{#1}
\newcommand{\OperatorTok}[1]{\textcolor[rgb]{0.40,0.40,0.40}{#1}}
\newcommand{\OtherTok}[1]{\textcolor[rgb]{0.00,0.44,0.13}{#1}}
\newcommand{\PreprocessorTok}[1]{\textcolor[rgb]{0.74,0.48,0.00}{#1}}
\newcommand{\RegionMarkerTok}[1]{#1}
\newcommand{\SpecialCharTok}[1]{\textcolor[rgb]{0.25,0.44,0.63}{#1}}
\newcommand{\SpecialStringTok}[1]{\textcolor[rgb]{0.73,0.40,0.53}{#1}}
\newcommand{\StringTok}[1]{\textcolor[rgb]{0.25,0.44,0.63}{#1}}
\newcommand{\VariableTok}[1]{\textcolor[rgb]{0.10,0.09,0.49}{#1}}
\newcommand{\VerbatimStringTok}[1]{\textcolor[rgb]{0.25,0.44,0.63}{#1}}
\newcommand{\WarningTok}[1]{\textcolor[rgb]{0.38,0.63,0.69}{\textbf{\textit{#1}}}}
\usepackage{longtable,booktabs,array}
\usepackage{calc} % for calculating minipage widths
% Correct order of tables after \paragraph or \subparagraph
\usepackage{etoolbox}
\makeatletter
\patchcmd\longtable{\par}{\if@noskipsec\mbox{}\fi\par}{}{}
\makeatother
% Allow footnotes in longtable head/foot
\IfFileExists{footnotehyper.sty}{\usepackage{footnotehyper}}{\usepackage{footnote}}
\makesavenoteenv{longtable}
\setlength{\emergencystretch}{3em} % prevent overfull lines
\providecommand{\tightlist}{%
  \setlength{\itemsep}{0pt}\setlength{\parskip}{0pt}}
\setcounter{secnumdepth}{-\maxdimen} % remove section numbering
\usepackage{xcolor}
\definecolor{aliceblue}{HTML}{F0F8FF}
\definecolor{antiquewhite}{HTML}{FAEBD7}
\definecolor{aqua}{HTML}{00FFFF}
\definecolor{aquamarine}{HTML}{7FFFD4}
\definecolor{azure}{HTML}{F0FFFF}
\definecolor{beige}{HTML}{F5F5DC}
\definecolor{bisque}{HTML}{FFE4C4}
\definecolor{black}{HTML}{000000}
\definecolor{blanchedalmond}{HTML}{FFEBCD}
\definecolor{blue}{HTML}{0000FF}
\definecolor{blueviolet}{HTML}{8A2BE2}
\definecolor{brown}{HTML}{A52A2A}
\definecolor{burlywood}{HTML}{DEB887}
\definecolor{cadetblue}{HTML}{5F9EA0}
\definecolor{chartreuse}{HTML}{7FFF00}
\definecolor{chocolate}{HTML}{D2691E}
\definecolor{coral}{HTML}{FF7F50}
\definecolor{cornflowerblue}{HTML}{6495ED}
\definecolor{cornsilk}{HTML}{FFF8DC}
\definecolor{crimson}{HTML}{DC143C}
\definecolor{cyan}{HTML}{00FFFF}
\definecolor{darkblue}{HTML}{00008B}
\definecolor{darkcyan}{HTML}{008B8B}
\definecolor{darkgoldenrod}{HTML}{B8860B}
\definecolor{darkgray}{HTML}{A9A9A9}
\definecolor{darkgreen}{HTML}{006400}
\definecolor{darkgrey}{HTML}{A9A9A9}
\definecolor{darkkhaki}{HTML}{BDB76B}
\definecolor{darkmagenta}{HTML}{8B008B}
\definecolor{darkolivegreen}{HTML}{556B2F}
\definecolor{darkorange}{HTML}{FF8C00}
\definecolor{darkorchid}{HTML}{9932CC}
\definecolor{darkred}{HTML}{8B0000}
\definecolor{darksalmon}{HTML}{E9967A}
\definecolor{darkseagreen}{HTML}{8FBC8F}
\definecolor{darkslateblue}{HTML}{483D8B}
\definecolor{darkslategray}{HTML}{2F4F4F}
\definecolor{darkslategrey}{HTML}{2F4F4F}
\definecolor{darkturquoise}{HTML}{00CED1}
\definecolor{darkviolet}{HTML}{9400D3}
\definecolor{deeppink}{HTML}{FF1493}
\definecolor{deepskyblue}{HTML}{00BFFF}
\definecolor{dimgray}{HTML}{696969}
\definecolor{dimgrey}{HTML}{696969}
\definecolor{dodgerblue}{HTML}{1E90FF}
\definecolor{firebrick}{HTML}{B22222}
\definecolor{floralwhite}{HTML}{FFFAF0}
\definecolor{forestgreen}{HTML}{228B22}
\definecolor{fuchsia}{HTML}{FF00FF}
\definecolor{gainsboro}{HTML}{DCDCDC}
\definecolor{ghostwhite}{HTML}{F8F8FF}
\definecolor{gold}{HTML}{FFD700}
\definecolor{goldenrod}{HTML}{DAA520}
\definecolor{gray}{HTML}{808080}
\definecolor{green}{HTML}{008000}
\definecolor{greenyellow}{HTML}{ADFF2F}
\definecolor{grey}{HTML}{808080}
\definecolor{honeydew}{HTML}{F0FFF0}
\definecolor{hotpink}{HTML}{FF69B4}
\definecolor{indianred}{HTML}{CD5C5C}
\definecolor{indigo}{HTML}{4B0082}
\definecolor{ivory}{HTML}{FFFFF0}
\definecolor{khaki}{HTML}{F0E68C}
\definecolor{lavender}{HTML}{E6E6FA}
\definecolor{lavenderblush}{HTML}{FFF0F5}
\definecolor{lawngreen}{HTML}{7CFC00}
\definecolor{lemonchiffon}{HTML}{FFFACD}
\definecolor{lightblue}{HTML}{ADD8E6}
\definecolor{lightcoral}{HTML}{F08080}
\definecolor{lightcyan}{HTML}{E0FFFF}
\definecolor{lightgoldenrodyellow}{HTML}{FAFAD2}
\definecolor{lightgray}{HTML}{D3D3D3}
\definecolor{lightgreen}{HTML}{90EE90}
\definecolor{lightgrey}{HTML}{D3D3D3}
\definecolor{lightpink}{HTML}{FFB6C1}
\definecolor{lightsalmon}{HTML}{FFA07A}
\definecolor{lightseagreen}{HTML}{20B2AA}
\definecolor{lightskyblue}{HTML}{87CEFA}
\definecolor{lightslategray}{HTML}{778899}
\definecolor{lightslategrey}{HTML}{778899}
\definecolor{lightsteelblue}{HTML}{B0C4DE}
\definecolor{lightyellow}{HTML}{FFFFE0}
\definecolor{lime}{HTML}{00FF00}
\definecolor{limegreen}{HTML}{32CD32}
\definecolor{linen}{HTML}{FAF0E6}
\definecolor{magenta}{HTML}{FF00FF}
\definecolor{maroon}{HTML}{800000}
\definecolor{mediumaquamarine}{HTML}{66CDAA}
\definecolor{mediumblue}{HTML}{0000CD}
\definecolor{mediumorchid}{HTML}{BA55D3}
\definecolor{mediumpurple}{HTML}{9370DB}
\definecolor{mediumseagreen}{HTML}{3CB371}
\definecolor{mediumslateblue}{HTML}{7B68EE}
\definecolor{mediumspringgreen}{HTML}{00FA9A}
\definecolor{mediumturquoise}{HTML}{48D1CC}
\definecolor{mediumvioletred}{HTML}{C71585}
\definecolor{midnightblue}{HTML}{191970}
\definecolor{mintcream}{HTML}{F5FFFA}
\definecolor{mistyrose}{HTML}{FFE4E1}
\definecolor{moccasin}{HTML}{FFE4B5}
\definecolor{navajowhite}{HTML}{FFDEAD}
\definecolor{navy}{HTML}{000080}
\definecolor{oldlace}{HTML}{FDF5E6}
\definecolor{olive}{HTML}{808000}
\definecolor{olivedrab}{HTML}{6B8E23}
\definecolor{orange}{HTML}{FFA500}
\definecolor{orangered}{HTML}{FF4500}
\definecolor{orchid}{HTML}{DA70D6}
\definecolor{palegoldenrod}{HTML}{EEE8AA}
\definecolor{palegreen}{HTML}{98FB98}
\definecolor{paleturquoise}{HTML}{AFEEEE}
\definecolor{palevioletred}{HTML}{DB7093}
\definecolor{papayawhip}{HTML}{FFEFD5}
\definecolor{peachpuff}{HTML}{FFDAB9}
\definecolor{peru}{HTML}{CD853F}
\definecolor{pink}{HTML}{FFC0CB}
\definecolor{plum}{HTML}{DDA0DD}
\definecolor{powderblue}{HTML}{B0E0E6}
\definecolor{purple}{HTML}{800080}
\definecolor{red}{HTML}{FF0000}
\definecolor{rosybrown}{HTML}{BC8F8F}
\definecolor{royalblue}{HTML}{4169E1}
\definecolor{saddlebrown}{HTML}{8B4513}
\definecolor{salmon}{HTML}{FA8072}
\definecolor{sandybrown}{HTML}{F4A460}
\definecolor{seagreen}{HTML}{2E8B57}
\definecolor{seashell}{HTML}{FFF5EE}
\definecolor{sienna}{HTML}{A0522D}
\definecolor{silver}{HTML}{C0C0C0}
\definecolor{skyblue}{HTML}{87CEEB}
\definecolor{slateblue}{HTML}{6A5ACD}
\definecolor{slategray}{HTML}{708090}
\definecolor{slategrey}{HTML}{708090}
\definecolor{snow}{HTML}{FFFAFA}
\definecolor{springgreen}{HTML}{00FF7F}
\definecolor{steelblue}{HTML}{4682B4}
\definecolor{tan}{HTML}{D2B48C}
\definecolor{teal}{HTML}{008080}
\definecolor{thistle}{HTML}{D8BFD8}
\definecolor{tomato}{HTML}{FF6347}
\definecolor{turquoise}{HTML}{40E0D0}
\definecolor{violet}{HTML}{EE82EE}
\definecolor{wheat}{HTML}{F5DEB3}
\definecolor{white}{HTML}{FFFFFF}
\definecolor{whitesmoke}{HTML}{F5F5F5}
\definecolor{yellow}{HTML}{FFFF00}
\definecolor{yellowgreen}{HTML}{9ACD32}
\usepackage[most]{tcolorbox}

\usepackage{ifthen}
\provideboolean{admonitiontwoside}
\makeatletter%
\if@twoside%
\setboolean{admonitiontwoside}{true}
\else%
\setboolean{admonitiontwoside}{false}
\fi%
\makeatother%

\newenvironment{env-a1515c36-41d3-4443-a8f5-fb6185c16f0c}
{
    \savenotes\tcolorbox[blanker,breakable,left=5pt,borderline west={2pt}{-4pt}{firebrick}]
}
{
    \endtcolorbox\spewnotes
}
                

\newenvironment{env-d537f2d3-16df-45e9-b698-b1af599dfeb9}
{
    \savenotes\tcolorbox[blanker,breakable,left=5pt,borderline west={2pt}{-4pt}{blue}]
}
{
    \endtcolorbox\spewnotes
}
                

\newenvironment{env-38983e6d-d80a-4801-98c6-e0e09975cbc0}
{
    \savenotes\tcolorbox[blanker,breakable,left=5pt,borderline west={2pt}{-4pt}{green}]
}
{
    \endtcolorbox\spewnotes
}
                

\newenvironment{env-71a1f24c-dbec-4536-a7c9-409ef6b137ea}
{
    \savenotes\tcolorbox[blanker,breakable,left=5pt,borderline west={2pt}{-4pt}{aquamarine}]
}
{
    \endtcolorbox\spewnotes
}
                

\newenvironment{env-e84c7e8c-9705-4033-be67-5a22e449cd6e}
{
    \savenotes\tcolorbox[blanker,breakable,left=5pt,borderline west={2pt}{-4pt}{orange}]
}
{
    \endtcolorbox\spewnotes
}
                

\newenvironment{env-bc89d7c2-5a57-4d92-b55e-0d8d63f64f59}
{
    \savenotes\tcolorbox[blanker,breakable,left=5pt,borderline west={2pt}{-4pt}{blue}]
}
{
    \endtcolorbox\spewnotes
}
                

\newenvironment{env-adf63665-fb9b-4403-b251-cb5731aa1178}
{
    \savenotes\tcolorbox[blanker,breakable,left=5pt,borderline west={2pt}{-4pt}{gold}]
}
{
    \endtcolorbox\spewnotes
}
                

\newenvironment{env-f63514ce-31e9-4c8c-ae6a-14dbb7e0331f}
{
    \savenotes\tcolorbox[blanker,breakable,left=5pt,borderline west={2pt}{-4pt}{darkred}]
}
{
    \endtcolorbox\spewnotes
}
                

\newenvironment{env-cc396ffb-ce85-499a-875d-e3c13eebd921}
{
    \savenotes\tcolorbox[blanker,breakable,left=5pt,borderline west={2pt}{-4pt}{pink}]
}
{
    \endtcolorbox\spewnotes
}
                

\newenvironment{env-e01dac82-c7b3-4bdd-9805-eb8d18295759}
{
    \savenotes\tcolorbox[blanker,breakable,left=5pt,borderline west={2pt}{-4pt}{cyan}]
}
{
    \endtcolorbox\spewnotes
}
                

\newenvironment{env-5a35d948-f266-454a-b19c-9e87bc7eb868}
{
    \savenotes\tcolorbox[blanker,breakable,left=5pt,borderline west={2pt}{-4pt}{cyan}]
}
{
    \endtcolorbox\spewnotes
}
                

\newenvironment{env-218e1eaa-790b-4c21-bc9f-6d34443195ea}
{
    \savenotes\tcolorbox[blanker,breakable,left=5pt,borderline west={2pt}{-4pt}{purple}]
}
{
    \endtcolorbox\spewnotes
}
                

\newenvironment{env-8e95a2b0-0bf3-4454-97f2-615ca97bd11f}
{
    \savenotes\tcolorbox[blanker,breakable,left=5pt,borderline west={2pt}{-4pt}{darksalmon}]
}
{
    \endtcolorbox\spewnotes
}
                

\newenvironment{env-b0f201ac-ca38-4625-ace0-fe7382968fcd}
{
    \savenotes\tcolorbox[blanker,breakable,left=5pt,borderline west={2pt}{-4pt}{gray}]
}
{
    \endtcolorbox\spewnotes
}
                
\ifLuaTeX
  \usepackage{selnolig}  % disable illegal ligatures
\fi
\IfFileExists{bookmark.sty}{\usepackage{bookmark}}{\usepackage{hyperref}}
\IfFileExists{xurl.sty}{\usepackage{xurl}}{} % add URL line breaks if available
\urlstyle{same} % disable monospaced font for URLs
\hypersetup{
  hidelinks,
  pdfcreator={LaTeX via pandoc}}

\author{}
\date{}

\begin{document}

{
\setcounter{tocdepth}{3}
\tableofcontents
}
\begin{Shaded}
\begin{Highlighting}[]
\NormalTok{min\_depth: 1}
\end{Highlighting}
\end{Shaded}

\hypertarget{limited-liability-company}{%
\section{Limited Liability Company}\label{limited-liability-company}}

\hypertarget{constitution-and-documents}{%
\subsection{Constitution and
Documents}\label{constitution-and-documents}}

\begin{Shaded}
\begin{Highlighting}[]
\NormalTok{title: s 7 CA 2006}
\NormalTok{(1) A company is formed under this Act by one or more persons—}
\NormalTok{{-} (a) subscribing their names to a memorandum of association (see section 8), and}
\NormalTok{{-} (b) complying with the requirements of this Act as to registration (see sections 9 to 13).}

\NormalTok{(2) A company may not be so formed for an unlawful purpose.}
\end{Highlighting}
\end{Shaded}

s 20 CA 2006 provides that the
{[}\protect\hyperlink{articles-of-association-1}{Articles of
Association}{]} of a 2006 Act company are to be those set out in model
Articles created by statutory instrument, except to the extent the
{[}{[}Business Law and Practice/Company Fundamentals/Shareholders{]}{]}
substitute their own Articles. There is one set of model Articles for
private limited {[}{[}companies{]}{]} and another for public limited
{[}{[}companies{]}{]}. Both sets of model Articles are to be found in
the Companies (Model Articles) Regulations 2008.

For pre-CA 2006 companies, the memorandum is an additional
constitutional document. In the Memorandum of Association you will also
find a clause called the `authorised share capital clause'. This states:
the maximum number of shares into which ownership of the company could
be subdivided; and the face value of each share, known as the `nominal'
or `par' value. This figure is significant -- for example, it is
unlawful under the Companies Act 2006 for a company to charge less than
the nominal value when it issues new {[}{[}shares{]}{]}.

s 28 CA 2006 provides that such provisions are to be treated as if they
were contained in the company's Articles of Association. That is, they
are implied into the Articles of Association.

When a company is formed:

\begin{itemize}
\tightlist
\item
  The company has separate legal personality;
\item
  The members of a limited company have limited liability.

  \begin{itemize}
  \tightlist
  \item
    There is a limit on the amount of money a shareholder is liable to
    contribute to a limited company in insolvent liquidation. The extent
    of this liability is set out in section 74(2)(d) of the Insolvency
    Act 1986.
  \end{itemize}
\end{itemize}

\begin{Shaded}
\begin{Highlighting}[]
\NormalTok{title: s 74(2)(d) IA 1986}
\NormalTok{In the case of a company limited by shares, no contribution is required  }
\NormalTok{from any member exceeding the amount (if any) unpaid on the shares in  }
\NormalTok{respect of which he is liable as a present or past member.}
\end{Highlighting}
\end{Shaded}

\hypertarget{separate-personality}{%
\subsection{Separate Personality}\label{separate-personality}}

Although forming a company by registration without the need for a royal
charter or Act of Parliament had been permitted since the Joint Stock
Companies Registration Act of 1844, it was only following the decision
of the House of Lords in {[}{[}Salomon v A Salomon \& Co ltd {[}1897{]}
AC 22{]}{]} that the full extent of this principle was realised. The
House of Lords held in its judgment that the company's acts were its own
acts, not those of Mr Salomon personally.

\begin{itemize}
\tightlist
\item
  A company owns its own property, and it does not belong to the members
  (i.e., owners) of the company.

  \begin{itemize}
  \tightlist
  \item
    If the member were to insure the company's property, the insurance
    would not be valid and the insurer would not have to pay up if a
    claim was made, even if that member owned the whole company
    ({[}{[}Macaura v Northern Assurance Co Ltd {[}1925{]} AC 619{]}{]}).
  \end{itemize}
\item
  The company will enter into contracts with third parties and incur
  contractual rights/ obligations
\item
  A company is able to borrow money and give security over its assets.

  \begin{itemize}
  \tightlist
  \item
    There is nothing to stop a company from borrowing money from its
    members, or indeed its
    {[}\protect\hyperlink{directors-1}{directors}{]}.
  \end{itemize}
\item
  Company can incur debts in its own name.
\item
  If the company is wound up and comes to an end (e.g., because it can
  no longer pay its debts), under s 74(2)(d) of the Insolvency Act 1986
  (IA 1986) any unpaid amount of the nominal value of the shares may be
  claimed from the members. Any unpaid amount therefore represents the
  extent of the members' liability.
\item
  A company may sue and be sued in its own name.
\item
  A company may even be prosecuted for manslaughter in its own right
  under the Corporate Manslaughter and Corporate Homicide Act 2007.
\item
  A company as a separate legal person has its own lifespan independent
  of those who own or run it.
\item
  A company is taxed separately from the members and director.
\item
  As a separate legal person a company is also entitled to certain
  protections under the Human Rights Act 1998
\end{itemize}

\hypertarget{division-of-responsibility}{%
\subsubsection{Division of
Responsibility}\label{division-of-responsibility}}

\begin{itemize}
\tightlist
\item
  Workers are employees.
\item
  Members are owners.

  \begin{itemize}
  \tightlist
  \item
    In a company limited by shared, the members are known as
    shareholders.
  \item
    ``Member'' rather than ``shareholder'' used in CA 2006
  \end{itemize}
\item
  Directors run the company day-to-day

  \begin{itemize}
  \tightlist
  \item
    Take daily decisions such as entering into contracts
  \end{itemize}
\end{itemize}

\begin{Shaded}
\begin{Highlighting}[]
\NormalTok{title: s 112 CA 2006}
\NormalTok{(1) The subscribers of a company’s memorandum are deemed to have  }
\NormalTok{agreed to become members of the company, and on its registration  }
\NormalTok{become members and must be entered as such in its register of  }
\NormalTok{members.  }
\NormalTok{(2) Every other person who agrees to become a member of a company,  }
\NormalTok{and whose name is entered in its register of members, is a member of  }
\NormalTok{the company.}
\end{Highlighting}
\end{Shaded}

\hypertarget{group-of-companies}{%
\subsubsection{Group of Companies}\label{group-of-companies}}

A company can own property in its own name, and this property includes
shares in another company. This leads to structures of groups of
companies, with parents and subsidiaries. This is sensible risk
management.

\hypertarget{case-law}{%
\paragraph{Case Law}\label{case-law}}

In {[}{[}Adams v Cape Industries plc {[}1990{]} Ch 433{]}{]}, the
claimants for asbestos-related illnesses were unable to bring
proceedings against a parent company as they worked for one of the
parent's subsidiary companies. This was the case even though the
subsidiary company had gone insolvent. However, in what may be a shift
of emphasis, in {[}{[}Chandler v Cape plc {[}2012{]} EWCA 525 CA{]}{]},
a parent company was held liable for the asbestos-induced illness of an
employee of its subsidiary company. The Court of Appeal established a
number of principles for deciding the circumstances in which a parent
company will be liable for the subsidiary's actions in health and safety
matters. It seems as if such liability could extend to other areas, for
example environmental liability. (see GDL notes).

\hypertarget{separate-legal-personality-workarounds}{%
\subsection{Separate Legal Personality
Workarounds}\label{separate-legal-personality-workarounds}}

The consequences that flow from the principle of separate legal
personality can be problematic (e.g., evasion of responsibilities, tax
avoidance).

\hypertarget{giving-a-guarantee}{%
\subsubsection{Giving a Guarantee}\label{giving-a-guarantee}}

Banks can side-step separate legal personality by using contract law and
asking for a personal guarantee from the entrepreneur for a loan to a
business.

\hypertarget{corporate-veil}{%
\subsubsection{Corporate Veil}\label{corporate-veil}}

There are exceptional circumstances in which statute or case law
`pierces', or `lifts' or `looks behind' this corporate veil.

The corporate veil is lifted only in exceptional and limited
circumstances, when all other, more conventional, remedies have proved
to be of no assistance ({[}{[}Prest v Petrodel Resources Ltd {[}2013{]}
2 AC 415{]}{]}).

The courts will usually look behind the veil only in cases of fraud or
deliberate breach of trust, eg where companies are used to carry out
fraud ({[}{[}Re Darby, ex p Brougham {[}1911{]} 1 KB 95{]}{]}) or to
avoid existing obligations ({[}{[}Gilford Motor Co Ltd v Horne
{[}1933{]} Ch 935{]}{]}). The courts seem to ignore the principle of
separate legal personality in these cases in order to ensure that an
appropriate remedy is available against individuals who have committed a
wrong using a company that they control. The Supreme Court has held,
however, that even where fraud may exist, it will not pierce the veil in
order to hold a person controlling a company liable as a joint
contracting party for a contract which the company, and not the
controller, has entered into ({[}{[}VTB Capital plc v Nutritek
International Corp and others {[}2013{]} UKSC 5{]}{]}).

\hypertarget{single-economic-unit}{%
\subsubsection{Single Economic Unit}\label{single-economic-unit}}

A group of companies may be used to lessen the risks of a business
failure by taking advantage of separate legal personality. Previous
decisions ignoring separate legal personality among groups are quite
controversial and currently thought to be specific to their facts (i.e.,
wrong).

Nevertheless, there are statutory rules which apply to groups of
companies and treat them as if they are a single economic unit rather
than entirely separate legal persons.

\begin{itemize}
\tightlist
\item
  s 399 of the CA 2006 requires certain parent companies to prepare
  accounts for their groups of companies as if they were single
  companies, one aim being to ensure that shareholders of the parent
  company get to see the true financial state of their company.
\item
  There is also special tax treatment of groups of companies to reflect
  the fact that, although comprising separate legal entities, they are
  usually run together.
\end{itemize}

\hypertarget{liability-of-directors}{%
\subsubsection{Liability of Directors}\label{liability-of-directors}}

Directors are liable for their involvement in running a company (e.g.,
{[}{[}wrongful trading{]}{]}). But strictly this is not side-stepping
separate legal personality: affirms it by making the director pay to the
separate legal person, the company.

\hypertarget{other-principles}{%
\subsection{Other Principles}\label{other-principles}}

\hypertarget{transparency}{%
\subsubsection{Transparency}\label{transparency}}

Companies are required to make a large amount of information available
to the public, including:

\begin{itemize}
\item
  \begin{enumerate}
  \def\labelenumi{(\alph{enumi})}
  \tightlist
  \item
    the identity of the company's shareholders, and the number and type
    of shares that they own;
  \end{enumerate}
\item
  \begin{enumerate}
  \def\labelenumi{(\alph{enumi})}
  \setcounter{enumi}{1}
  \tightlist
  \item
    the identity and certain personal information about the company's
    directors;
  \end{enumerate}
\item
  \begin{enumerate}
  \def\labelenumi{(\alph{enumi})}
  \setcounter{enumi}{2}
  \tightlist
  \item
    the identity of the company secretary (if any);
  \end{enumerate}
\item
  \begin{enumerate}
  \def\labelenumi{(\alph{enumi})}
  \setcounter{enumi}{3}
  \tightlist
  \item
    the company's constitution (the internal rules which govern how the
    company should be run).
  \end{enumerate}
\item
  \begin{enumerate}
  \def\labelenumi{(\alph{enumi})}
  \setcounter{enumi}{4}
  \tightlist
  \item
    the company's accounts; and
  \end{enumerate}
\item
  \begin{enumerate}
  \def\labelenumi{(\alph{enumi})}
  \setcounter{enumi}{5}
  \tightlist
  \item
    certain decisions taken by the shareholders.
  \end{enumerate}
\end{itemize}

Designed to help third parties in deciding whether to deal with the
company.

\hypertarget{records}{%
\paragraph{Records}\label{records}}

\begin{itemize}
\tightlist
\item
  CA 2006 requires that this information must be sent, usually using a
  prescribed form, to the Registrar of Companies at Companies House.
\item
  Companies House is an agency of the Department of Business, Energy and
  Industrial Strategy (BEIS).
\item
  The information is stored on a register kept at Companies House in
  accordance with s 1080 CA 2006.
\item
  The Registrar of Companies is designated under s 1061 CA 2006 as
  having responsibility for keeping this register.
\end{itemize}

\hypertarget{maintenance-of-capital}{%
\paragraph{Maintenance of Capital}\label{maintenance-of-capital}}

\begin{Shaded}
\begin{Highlighting}[]
\NormalTok{The money invested by shareholders in a company is known as share capital.}
\end{Highlighting}
\end{Shaded}

It is a fundamental rule of company law that the share capital belongs
to the company and not to the shareholders. The share capital is treated
as a special fund within the company and

given extra protection by the law. This is so that it is available to
repay the company's creditors should the company be unable to pay its
debts. This is the maintenance of capital principle -- that a reserve of
money, the share capital, is maintained within the company, to be
available for creditors as a last resort.

Its effects are most clearly demonstrated in s 830 CA 2006. This
provision allows a company to make distributions to its shareholders
only out of distributable profits. Put simply, a company can pay money
to its shareholders only from available profits, which do not touch the
share capital.

\begin{Shaded}
\begin{Highlighting}[]
\NormalTok{Most private companies have a share capital of £100 or less.}
\end{Highlighting}
\end{Shaded}

\hypertarget{company-constitution}{%
\section{Company Constitution}\label{company-constitution}}

\hypertarget{importance}{%
\subsection{Importance}\label{importance}}

\begin{itemize}
\tightlist
\item
  Represents a special type of contract between the company, as a
  separate legal person, and the shareholders, and between the
  shareholders.
\item
  Evidences the existence of the company, states current share capital,
  and sets out rules governing how the company is run.
\end{itemize}

\hypertarget{definition}{%
\subsection{Definition}\label{definition}}

Constitution includes:

\begin{enumerate}
\def\labelenumi{\arabic{enumi}.}
\tightlist
\item
  Articles of Association (s 17)
\item
  Certificate of incorporation (s 32)
\item
  Current statement of capital (s 32)
\item
  Copies of any court orders and enactments altering constitution (s 32)
\item
  Resolutions affecting constitution (ss 17 \& 29)
\item
  Agreements involving shareholders affecting constitution (ss 17 \& 29)
\end{enumerate}

\hypertarget{articles-of-association}{%
\subsection{Articles of Association}\label{articles-of-association}}

\begin{Shaded}
\begin{Highlighting}[]
\NormalTok{title: s 18(1) CA 2006}
\NormalTok{A company must have articles of association prescribing regulations for the company.}
\end{Highlighting}
\end{Shaded}

\begin{longtable}[]{@{}
  >{\raggedright\arraybackslash}p{(\columnwidth - 2\tabcolsep) * \real{0.2193}}
  >{\raggedright\arraybackslash}p{(\columnwidth - 2\tabcolsep) * \real{0.7807}}@{}}
\toprule()
\begin{minipage}[b]{\linewidth}\raggedright
Aspect
\end{minipage} & \begin{minipage}[b]{\linewidth}\raggedright
Details
\end{minipage} \\
\midrule()
\endhead
Scope for varying CA 2006 & Articles can be used to exclude some but not
all provisions in CA 2006. \\
Availability & Available for inspection at Companies House. \\
Layout & Contained in a single document and divided into consecutive
numbered paragraphs (s 18(3)) \\
\bottomrule()
\end{longtable}

\hypertarget{unamended-model-articles}{%
\subsubsection{Unamended Model
Articles}\label{unamended-model-articles}}

\begin{Shaded}
\begin{Highlighting}[]
\NormalTok{title: s 20 CA 2006 {-} **Default** application of model articles}
\NormalTok{(1) On the formation of a limited company—}
\NormalTok{{-} (a) if articles are not registered, or}
\NormalTok{{-} (b) if articles are registered, in so far as they do not exclude or modify the relevant model articles,}

\NormalTok{the relevant model articles (so far as applicable) form part of the company\textquotesingle{}s articles in the same manner and to the same extent as if articles in the form of those articles had been duly registered.}

\NormalTok{(2) The “relevant model articles” means the model articles prescribed for a company of that description as in force at the date on which the company is registered.}
\end{Highlighting}
\end{Shaded}

Model articles comprise a set of \textbf{minimum basic rules} on running
the company. They keep regulation to a minimum.

\hypertarget{ma-index}{%
\subsubsection{MA Index}\label{ma-index}}

\begin{longtable}[]{@{}ll@{}}
\toprule()
Article numbers & Subject \\
\midrule()
\endhead
1-2 & Defined terms and liability of members \\
3-6 & Directors' powers and responsibilities \\
7-16 & Decision-making by directors \\
17-20 & Appointment of directors \\
21-29 & Shares \\
30-35 & Dividends and other distributions \\
36 & Capitalisation of profits \\
37-41 & Organisation of general meetings \\
42-47 & Voting at general meetings \\
48-51 & Administrative arrangements \\
52-53 & Directors' indemnity and insurance \\
\bottomrule()
\end{longtable}

When reading the model articles, always consider them in conjunction
with definitions in art 1.

\hypertarget{special-articles}{%
\subsubsection{Special Articles}\label{special-articles}}

\begin{Shaded}
\begin{Highlighting}[]
\NormalTok{Any amendments to MA / other provisions in the Articles are known as \textquotesingle{}special articles\textquotesingle{}.}
\end{Highlighting}
\end{Shaded}

Commonly included special articles:

\begin{longtable}[]{@{}
  >{\raggedright\arraybackslash}p{(\columnwidth - 2\tabcolsep) * \real{0.2022}}
  >{\raggedright\arraybackslash}p{(\columnwidth - 2\tabcolsep) * \real{0.7978}}@{}}
\toprule()
\begin{minipage}[b]{\linewidth}\raggedright
Article
\end{minipage} & \begin{minipage}[b]{\linewidth}\raggedright
Description
\end{minipage} \\
\midrule()
\endhead
Directors' meetings & Make calling directors' meetings more formal
(e.g., notice period) or limit directors' ability to take decisions by
electronic means. \\
Directors' interests in transactions & Allow directors' to vote on
matters when they have a personal interest. Useful in small
companies. \\
Directors' conflicts of interest & s 175 allows a private company to
permit directors to authorise a situation which would otherwise breach s
175. MA contains no such provision. \\
Number of directors & s 154: minimum is 1. This can be changed. \\
Absence of directors & MA does not give directors' power to appoint a
replacement if they are away. \\
Company secretary & Can amend model articles to include power of
secretary. But could also do this through directors' general powers. \\
Removal of pre-emption rights & s 561: existing shareholders are offered
the chance to buy any shares before they are offered to outsiders. \\
\bottomrule()
\end{longtable}

It is also possible to create completely bespoke Articles from scratch.

\hypertarget{effect-of-articles}{%
\subsubsection{Effect of Articles}\label{effect-of-articles}}

\begin{Shaded}
\begin{Highlighting}[]
\NormalTok{title: s 33 {-} Effect of company\textquotesingle{}s constitution}

\NormalTok{(1) The provisions of a company\textquotesingle{}s constitution bind the company and its members to the same extent as if there were covenants on the part of the company and of each member to observe those provisions.}

\NormalTok{(2) Money payable by a member to the company under its constitution is a debt due from him to the company. In England and Wales and Northern Ireland it is of the nature of an ordinary contract debt. }
\end{Highlighting}
\end{Shaded}

The constitution of a company forms:

\begin{enumerate}
\def\labelenumi{\arabic{enumi}.}
\tightlist
\item
  a contract between the company and each of its members, and
\item
  a contract between all the company's shareholders.
\end{enumerate}

\hypertarget{s-33-contract}{%
\paragraph{S 33 Contract}\label{s-33-contract}}

This creates contracts in circumstances when they would not otherwise
exist under contract law.

There is a contractual relationship between shareholders:

{[}{[}s 33 contract A.png{]}{]}

and between a company and a new shareholder:

{[}{[}s 33 contract B.png{]}{]}

\hypertarget{nature-of-contract}{%
\paragraph{Nature of Contract}\label{nature-of-contract}}

The s 33 contract is very different from a standard trading contract.

\begin{Shaded}
\begin{Highlighting}[]
\NormalTok{The contract allows for action to be taken only in so far as it deals with membership rights ([[Beattie v E and F Beattie [1938] Ch 708]]).}
\end{Highlighting}
\end{Shaded}

This includes:

\begin{itemize}
\tightlist
\item
  the right to vote,
\item
  the right to attend general meetings, and
\item
  the right to a dividend if one is declared.
\end{itemize}

Articles only create a contract between a company and the members
\textbf{in their capacity as members}, and not in any special or
personal capacity. They do not give rights to any person who is not a
member ({[}{[}Eley v Positive Government Security Life Assurance Company
(1876) 1 Ex D 88 (CA){]}{]}).

So not:

\begin{itemize}
\tightlist
\item
  The right to be a director ({[}{[}Read v Astoria Garage (Streatham)
  Ltd {[}1952{]} Ch 637{]}{]})
\item
  The right to be a solicitor (\emph{Eley})
\end{itemize}

Such rights will be dealt with in a \textbf{shareholders' agreement}.

\hypertarget{effect-of-contract}{%
\paragraph{Effect of Contract}\label{effect-of-contract}}

The members are bound by the s 33 contract ({[}{[}Hickman v Kent or
Romney Marsh Sheep-Breeders' Association {[}1915{]} 1 Ch 881
(Ch){]}{]}).

A member may sue another on the contract created by the articles without
joining the company as a party ({[}{[}Rayfield v Hands {[}1960{]} Ch 1
(Ch){]}{]}). But this is the only case in which an action between
shareholders was successful.

\hypertarget{additional-terms}{%
\paragraph{Additional Terms}\label{additional-terms}}

Since the articles are a contract, it is possible for additional terms
to be implied if necessary to give effect to the contract ({[}{[}Cream
Holdings Ltd v Stuart Davenport {[}2011{]} EWCA Civ 1287{]}{]}).

\hypertarget{amending-articles}{%
\subsection{Amending Articles}\label{amending-articles}}

\hypertarget{procedures}{%
\subsubsection{Procedures}\label{procedures}}

\hypertarget{special-resolution}{%
\paragraph{Special Resolution}\label{special-resolution}}

\begin{Shaded}
\begin{Highlighting}[]
\NormalTok{title: Procedure}
\NormalTok{1. Pass a **special resolution** to change the articles ([s 21 (1) CA 2006](https://www.legislation.gov.uk/ukpga/2006/46/section/21))}
\NormalTok{2. Registrar of Companies sent a copy of the articles as amended $\textbackslash{}leq 15$ days after articles amended (s 26(1)) and a copy of the special resolution itself (s 30(1))}
\end{Highlighting}
\end{Shaded}

\begin{Shaded}
\begin{Highlighting}[]
\NormalTok{Another reason s 33 contractual rights are different from normal contract rights; you can\textquotesingle{}t amend a contract without unanimous consent of 100\% of parties.}
\end{Highlighting}
\end{Shaded}

\hypertarget{ordinary-resolution-exceptions}{%
\paragraph{Ordinary Resolution
Exceptions}\label{ordinary-resolution-exceptions}}

Two exceptions, under which specific articles can be amended using
merely an ordinary resolution:

\begin{enumerate}
\def\labelenumi{\arabic{enumi}.}
\tightlist
\item
  Revoking, varying or renewing directors' authority to allot new shares
  in the company, if authority provided for in articles (s 551(8)).
\item
  Authorising directors to determine the T\&Cs of share redemption (s
  685(2))
\end{enumerate}

\hypertarget{court-rectification}{%
\paragraph{Court Rectification}\label{court-rectification}}

If necessary, the court can exceptionally rectify the articles
({[}{[}Folkes Group plc v Alexander {[}2002{]} 2 BCLC 254{]}{]}).

\hypertarget{restrictions}{%
\paragraph{Restrictions}\label{restrictions}}

\begin{longtable}[]{@{}
  >{\raggedright\arraybackslash}p{(\columnwidth - 2\tabcolsep) * \real{0.7773}}
  >{\raggedright\arraybackslash}p{(\columnwidth - 2\tabcolsep) * \real{0.2227}}@{}}
\toprule()
\begin{minipage}[b]{\linewidth}\raggedright
Restriction
\end{minipage} & \begin{minipage}[b]{\linewidth}\raggedright
Authority
\end{minipage} \\
\midrule()
\endhead
The shareholders cannot amend the articles so as to conflict with a
mandatory provision of CA 2006 & {[}{[}Allen v Gold Reefs of West Africa
{[}1900{]} 1 Ch 656{]}{]} \\
A shareholder is not bound by any change in the company's articles which
forces that shareholder to buy more shares in the company, unless they
expressly agree in writing to be so bound & s 25 CA 2006 \\
Shareholders can make a change to the articles only if it is `bona fide
for the benefit of the company as a whole' & {[}{[}Allen v Gold Reefs of
West Africa {[}1900{]} 1 Ch 656{]}{]} \\
\bottomrule()
\end{longtable}

\hypertarget{bona-fide-interest}{%
\subparagraph{Bona Fide Interest}\label{bona-fide-interest}}

\begin{itemize}
\tightlist
\item
  Whether a change is `bona fide for the benefit of the company as a
  whole' is a subjective matter for the company's shareholders to decide
  ({[}{[}Citco Banking Corporation NV v Pussers {[}2007{]} UKPC
  13{]}{]})
\item
  Amendment to Articles is not valid if no reasonable man could consider
  it to be for the benefit of the company ({[}{[}Shuttleworth v Cox Bros
  Ltd {[}1927{]} 2 KB 9{]}{]})
\item
  Usually involves situations where the intention was to discriminate
  against minority shareholders rather than benefit the company as a
  whole.
\end{itemize}

\hypertarget{entrenched-articles}{%
\paragraph{Entrenched Articles}\label{entrenched-articles}}

\href{https://www.legislation.gov.uk/ukpga/2006/46/section/22}{s 22 CA
2006}~permits entrenchment of specific provisions, though this is rare
in practice.

\begin{Shaded}
\begin{Highlighting}[]
\NormalTok{An entrenched article contains extra procedures or conditions, which makes it harder to change than other articles.}
\end{Highlighting}
\end{Shaded}

\hypertarget{restrictions-1}{%
\subparagraph{Restrictions}\label{restrictions-1}}

These entrenched provisions can always be amended by agreement of all
members, or by a court order
(\href{https://www.legislation.gov.uk/ukpga/2006/46/section/22}{s 22(3)
CA 2006}).

Under s 22(2) CA 2006 (\textbf{yet to enter force}), the entrepreneurs
forming a company will only be able to include entrenched rights in the
articles either when the company is first set up, or with the agreement
of all the shareholders once the company has been registered.

Currently, an entrenched article may be included by passing a s 21
special resolution.

\hypertarget{procedure}{%
\subparagraph{Procedure}\label{procedure}}

When dealing with entrenched articles, extra paperwork must be sent to
Companies House.

\begin{longtable}[]{@{}
  >{\raggedright\arraybackslash}p{(\columnwidth - 2\tabcolsep) * \real{0.8554}}
  >{\raggedright\arraybackslash}p{(\columnwidth - 2\tabcolsep) * \real{0.1446}}@{}}
\toprule()
\begin{minipage}[b]{\linewidth}\raggedright
Scenario
\end{minipage} & \begin{minipage}[b]{\linewidth}\raggedright
Form to send
\end{minipage} \\
\midrule()
\endhead
Articles amended to include entrenched article & Form CC01 \\
Articles amended to remove entrenched article & Form CC01 \\
Articles already include entrenched article and \textbf{any} article
altered & Form CC03 \\
\bottomrule()
\end{longtable}

\hypertarget{memorandum}{%
\subsection{Memorandum}\label{memorandum}}

Represents a snapshot of the company and its owners at the time the
company is formed.

\hypertarget{ca-1985-companies}{%
\subsection{CA 1985 Companies}\label{ca-1985-companies}}

Constitutional documents differ substantially from CA 2006 companies.

\hypertarget{memorandum-1}{%
\subsubsection{Memorandum}\label{memorandum-1}}

CA 1985 Memorandum was far more important. Consisted of:

\begin{itemize}
\tightlist
\item
  Name
\item
  Registered office
\item
  Objects
\item
  Shareholders' liability
\item
  Authorised share capital.
\end{itemize}

From 01/10/2009, the five main clauses of the memorandums of companies
automatically became provisions of their articles (CA 2006, s 28(1)).

\hypertarget{liability-clause}{%
\paragraph{Liability Clause}\label{liability-clause}}

The shareholders' liability clause of the old-style memorandum simply
stated: `The liability of\\
the members is limited.' This is now automatically a provision in the
articles due to s 28(1) of\\
the CA 2006.

\hypertarget{objects-clause}{%
\paragraph{Objects Clause}\label{objects-clause}}

All companies in existence before 1 October 2009 had an objects clause
in their memorandum. This clause set out the purposes for which the
company was formed and included a statement of what it was empowered to
do. Historically, if a company acted outside of its objects set out in
the objects clause then it was said to have acted `ultra vires'.

\hypertarget{ultra-vires-doctrine}{%
\subsubsection{Ultra Vires Doctrine}\label{ultra-vires-doctrine}}

Doctrine referring to the situation where a body purports to act outside
its power. Derives from public law: public bodies granted certain powers
by Parliament, beyond which they cannot stray. Doctrine applied to
registered companies to protect creditors and shareholders (e.g.,
{[}{[}Re German Date Coffee Co {[}1882{]} 20 Ch D 169{]}{]}).

\hypertarget{problems}{%
\paragraph{Problems}\label{problems}}

But doctrine was problematic:

\begin{itemize}
\tightlist
\item
  Objects clause initially not permitted to be altered. Could only be
  altered in very limited circumstances until 1991, when CA 1985
  amendments came through
\item
  Made diversification difficult
\item
  Doctrine of constructive notice meant that anyone dealing with a
  company was deemed to have knowledge of the contents of its objects
  clause, so would be unable to argue they did not know the company
  lacked capacity to enter into transaction. Doctrine applied to all
  publicly available documents.
\end{itemize}

Effect was super long objects clauses from 60s to 90s. Very general and
broad objects clauses accepted ({[}{[}Bell Houses Ltd v City Wall
Properties Ltd {[}1966{]} 2 QB 656{]}{]}). Post 1991 registered
companies specified in the memorandum that the company was a `general
commercial company', giving it power to carry on any trade or business
and had power to do all such things as were incidental or conducive
thereto. But issues of ultra vires still came before the courts:
{[}{[}Re Introductions Ltd v National Provincial Bank {[}1970{]} Ch
199{]}{]}.

There were recommendations for reform in the law to protect third
parties. After the UK joined the European Community, in 1973 the First
European Community Company Law Harmonisation Directive removed the
doctrine of constructive notice where it concerned the memorandum and
articles. It also contained a saving provision for ultra vires
transactions where the transaction was dealt with by directors, and a
third party was acting in good faith.

\hypertarget{reform}{%
\paragraph{Reform}\label{reform}}

CA 1985 introduced a reform that the memorandum could be altered by
special resolution, letting companies change their objects clause (s 4),
and allowed companies to have a general objects clause stating that the
company was to carry on business as a `general commercial company' (s
3A), allowing them to carry on any trade or business whatsoever.

CA 2006 brought further changes:

\begin{enumerate}
\def\labelenumi{\arabic{enumi}.}
\tightlist
\item
  s 35 CA 1985 had removed the doctrine of constructive notice in
  relation to a company's memorandum and articles. This was enshrined in
  \href{https://www.legislation.gov.uk/ukpga/2006/46/section/39}{s 39(1)
  CA 2006}: `The validity of an act done by a company shall not be
  called into question on the ground of lack of capacity by reason of
  anything in the company's constitution'
\item
  The requirement of an objects clause in the memorandum was completely
  removed.
  \href{https://www.legislation.gov.uk/ukpga/2006/46/section/31}{s 31 CA
  2006}: the default position is now that all companies have
  unrestricted objects.
\end{enumerate}

Many older companies still have an objects clause.

\begin{itemize}
\tightlist
\item
  These are treated as if they are a provision of the articles
  (\href{https://www.legislation.gov.uk/ukpga/2006/46/section/28}{s
  28(1) CA 2006}) and will continue to bind the company unless altered
  by special resolution.
\item
  If the company adopts new articles, then this treats the objects
  clause as having been removed
\item
  Any constitutional restrictions on a company's capacity have no
  bearing on {[}{[}Liability in tort and crime{]}{]}.
\item
  \textbf{s 39 CA 2006} effectively abolishes the `ultra vires' doctrine
  with regard to third parties dealing with the company.
\item
  \textbf{s 40(1)}: the powers of the directors to bind a company (e.g.,
  by entering a contact) are deemed to be free of any limitation under
  the company's constitution in favour of a person dealing with the
  company in good faith.
\end{itemize}

Note there are still some possible internal consequences if the
company's objects are exceeded.

\hypertarget{injunction}{%
\subparagraph{Injunction}\label{injunction}}

\begin{itemize}
\tightlist
\item
  If the company intends to act in breach of its objects, there is a
  right for a shareholder to go to court to seek an injunction under s
  40(4) to restrain the company from taking this action.
\item
  If the act has already been done (say, the contract has been signed),
  the shareholder can only take action against the directors (CA 2006, s
  40(5)) for breach of their duty to the company.
\item
  Directors are obliged under the CA 2006 to act within their powers,
  which includes observing any restrictions in the constitution (CA
  2006, s 171).
\end{itemize}

\hypertarget{directors-liability}{%
\subparagraph{Directors' Liability}\label{directors-liability}}

Any objects clause has, post-CA 2006, the effect of restricting the
directors rather than the company. Since the restrictions are now part
of the articles, they can be removed by special resolution (s 21).

\hypertarget{authorised-share-capital}{%
\paragraph{Authorised Share Capital}\label{authorised-share-capital}}

A company in existence before 1 October 2009 had to have a clause in its
memorandum\\
stating its authorised share capital. Now abolished: companies must
instead make statements of capital each time more shares are issued.

\hypertarget{articles}{%
\subsubsection{Articles}\label{articles}}

The old equivalent of MA was Table A, which was more comprehensive. MA
imposes a far lighter administrative burden. There is also a `Table A
2007' in force from 2007 to 2009.

\hypertarget{relevancy}{%
\subsubsection{Relevancy}\label{relevancy}}

Companies with Table A articles may continue to operate under these
rules (s 19(4) CA 2006). But major changes have been made to company law
by CA 2006 meaning many Table A articles will be redundant or impose an
unnecessary administrative burden.

\hypertarget{content}{%
\subsubsection{Content}\label{content}}

\begin{longtable}[]{@{}ll@{}}
\toprule()
Article numbers & Subject \\
\midrule()
\endhead
1 & Definitions \\
2-35 & Share capital and shares \\
36-63 & Shareholders' general meetings \\
64-98 & Directors and board meetings \\
99-101 & Company administration \\
102-110 & Company profits \\
111-116 & Notices \\
117-118 & Winding up and indemnity \\
\bottomrule()
\end{longtable}

\hypertarget{ma-vs-table-a}{%
\subsubsection{MA Vs Table A}\label{ma-vs-table-a}}

Key differences:

\begin{longtable}[]{@{}
  >{\raggedright\arraybackslash}p{(\columnwidth - 4\tabcolsep) * \real{0.5714}}
  >{\raggedright\arraybackslash}p{(\columnwidth - 4\tabcolsep) * \real{0.2971}}
  >{\raggedright\arraybackslash}p{(\columnwidth - 4\tabcolsep) * \real{0.1314}}@{}}
\toprule()
\begin{minipage}[b]{\linewidth}\raggedright
Provision
\end{minipage} & \begin{minipage}[b]{\linewidth}\raggedright
MA
\end{minipage} & \begin{minipage}[b]{\linewidth}\raggedright
Table A
\end{minipage} \\
\midrule()
\endhead
Limits liability of members & MA 2 & No \\
GMs & MA 37 refers only to shareholders `general' meetings & AGM \& EGM
distinguished \\
Requirement to hold AGM & No & Yes \\
Notice period for shareholder meetings & 14 clear days with exceptions &
21 clear days \\
Chairperson's casting vote in shareholder meetings & No & Yes \\
Deals with written resolutions & No & Yes \\
Allows proxies to vote on show of hands & MA 42 \& 45 & No \\
Makes provision for the use of alternate directors & No & Yes \\
Requires directors retire by rotation & No & Yes \\
Specify that directors' decisions are reached by majority decision or
unanimously & MA 7 & No \\
Specify that a unanimous decision of directors can be made by any means
(no formal meeting required) & MA 8 & No \\
Permit board meetings to be held by any method so long as all directors
can communicate & MA 9 & No \\
\bottomrule()
\end{longtable}

Notice the deregulatory trend. Many companies adopted special articles
amending Table A. CA 2006 came into force in stages until 01/10/09. An
interim solution was `Table A 2007', though this is adopted by very few
companies.

\hypertarget{document-requests}{%
\subsection{Document Requests}\label{document-requests}}

If a shareholder requests a company's constitutional documents, a
company must send them (s 32). This should include:

\begin{itemize}
\tightlist
\item
  Articles
\item
  Certificate of incorporation
\item
  Statement of capital.
\end{itemize}

\hypertarget{setting-up-company}{%
\section{Setting up Company}\label{setting-up-company}}

\hypertarget{from-scratch}{%
\subsection{From Scratch}\label{from-scratch}}

\begin{Shaded}
\begin{Highlighting}[]
\NormalTok{title: s 7: Method of forming company}

\NormalTok{(1) A company is formed under this Act by one or more persons—}
\NormalTok{{-} (a) subscribing their names to a memorandum of association (see section 8), and}
\NormalTok{{-} (b) complying with the requirements of this Act as to registration (see sections 9 to 13).}

\NormalTok{(2) A company may not be so formed for an unlawful purpose.}
\end{Highlighting}
\end{Shaded}

\hypertarget{registration}{%
\subsubsection{Registration}\label{registration}}

The documents which must be prepared and delivered to the Registrar for
a private company limited by shares are:

\begin{enumerate}
\def\labelenumi{\arabic{enumi}.}
\tightlist
\item
  an application for registration as a company;
\item
  a memorandum of association for the company; and
\item
  possibly articles of association for the company.
\end{enumerate}

If everything is in order, the Registrar for Companies will register the
documents and issue a certificate of incorporation.

\hypertarget{electronic-registration}{%
\paragraph{Electronic Registration}\label{electronic-registration}}

Can do this using:

\begin{enumerate}
\def\labelenumi{\arabic{enumi}.}
\tightlist
\item
  Companies House software filing service; or

  \begin{itemize}
  \tightlist
  \item
    Enter details with a third party, who then generate the
    documentation and send it through to Companies House.
  \end{itemize}
\item
  Companies House Web Incorporation Service

  \begin{itemize}
  \tightlist
  \item
    Register on the website and enter required information.
  \end{itemize}
\end{enumerate}

\hypertarget{paper-registration}{%
\paragraph{Paper Registration}\label{paper-registration}}

Hard copies of the necessary documents and the fee must be sent to the
Registrar of Companies at Companies House, either by post or in person.

Companies House is expected to be empowered to require companies to file
all documents digitally in a new Act of Parliament.

Hard-copy application for registration known as Form IN01.

\hypertarget{documentation}{%
\subsection{Documentation}\label{documentation}}

\hypertarget{incorporation}{%
\subsubsection{Incorporation}\label{incorporation}}

\begin{longtable}[]{@{}
  >{\raggedright\arraybackslash}p{(\columnwidth - 2\tabcolsep) * \real{0.8855}}
  >{\raggedright\arraybackslash}p{(\columnwidth - 2\tabcolsep) * \real{0.1145}}@{}}
\toprule()
\begin{minipage}[b]{\linewidth}\raggedright
Rule
\end{minipage} & \begin{minipage}[b]{\linewidth}\raggedright
Statutory authority
\end{minipage} \\
\midrule()
\endhead
With effect from the date of incorporation, the promoters become holders
of shares specified in the statement of capital and initial
shareholdings. & ss 16(1) \& (5) \\
Directors take their positions with effect from the date of
{[}\protect\hyperlink{incorporation}{Incorporation}{]} & ss 16(1) \&
(6) \\
A company can be incorporated with only one subscriber to the
{[}\protect\hyperlink{memorandum-1}{Memorandum}{]} & s 7(1) \\
The statement of initial capital and shareholdings should disclose the
number and nominal value of shares on
{[}\protect\hyperlink{incorporation}{Incorporation}{]} & s 10(4)(a) \\
\bottomrule()
\end{longtable}

\hypertarget{application-for-registration}{%
\subsubsection{Application for
Registration}\label{application-for-registration}}

The application for registration of the company is required by s 9(1).
Must include:

\begin{longtable}[]{@{}
  >{\raggedright\arraybackslash}p{(\columnwidth - 2\tabcolsep) * \real{0.7077}}
  >{\raggedright\arraybackslash}p{(\columnwidth - 2\tabcolsep) * \real{0.2923}}@{}}
\toprule()
\begin{minipage}[b]{\linewidth}\raggedright
Information to include
\end{minipage} & \begin{minipage}[b]{\linewidth}\raggedright
Statute
\end{minipage} \\
\midrule()
\endhead
Type of company being registered & ss 9(2)(c) \& d \\
Proposed names & s 9(2)(a) \\
Address of registered office & ss 9(2)(b), 9(5)(a) \\
Statement of capital and initial shareholdings & s 9(4)(a) \\
Statement of proposed officers & s 9(4)(c) \\
Statement of initial significant control & s 9(4)(d) \\
Possibly copy of Articles (if required) & s 9(5)(b) \\
Statement of compliance & s 9(1) \\
\bottomrule()
\end{longtable}

\hypertarget{company-type}{%
\subsubsection{Company Type}\label{company-type}}

The application must state whether the company is private or public,
limited or unlimited, and limited by shares or by guarantee.

\begin{Shaded}
\begin{Highlighting}[]
\NormalTok{THis registration process cannot be used to set up an LLP.}
\end{Highlighting}
\end{Shaded}

\hypertarget{company-name}{%
\subsubsection{Company Name}\label{company-name}}

\begin{itemize}
\tightlist
\item
  Ask client for a few alternatives, since there are statutory
  restrictions
\item
  Check index of company names to check availability of requested name.

  \begin{itemize}
  \tightlist
  \item
    Under the Company, Limited Liability Partnership and Business (Names
    and Trading Disclosures) Regulations 2015 (SI 2015/17), the `same'
    name includes not just an identical name but also names which would
    be essentially the same if simple elements were disregarded.
  \item
    Limited exceptions for when the new company will form part of the
    same group of companies.
  \end{itemize}
\item
  Name ending

  \begin{itemize}
  \tightlist
  \item
    Must end in Limited or Ltd or Welsh equivalents.
  \end{itemize}
\item
  Prohibited names

  \begin{itemize}
  \tightlist
  \item
    A company may not use a name which, in the opinion of the Secretary
    of State for BEIS, would amount to a criminal offence or is
    offensive (CA 2006, s 53).
  \item
    More rules on using punctuation in a company name exist.
  \end{itemize}
\end{itemize}

\hypertarget{names-requiring-approval}{%
\paragraph{Names Requiring Approval}\label{names-requiring-approval}}

Some names will require approval:

\begin{itemize}
\tightlist
\item
  Sensitive words or expressions
\item
  Names suggesting a connection to the government or devolved
  governments.
\item
  Names suggesting a profession may need prior approval from a body,
  e.g., from the General Dental Council, to include the word `dental' in
  the company name.
\item
  Request must be made in writing to SoS for BEIS or to the relevant
  body.
\end{itemize}

\hypertarget{challenging-name}{%
\paragraph{Challenging Name}\label{challenging-name}}

After the company has been formed, the use of the name may still be
challenged by third parties.

\hypertarget{sos-for-beis-powers}{%
\paragraph{SoS for BEIS Powers}\label{sos-for-beis-powers}}

\begin{itemize}
\tightlist
\item
  The Secretary of State may direct a company to change its name after
  the company has been registered if it is the same as or too like a
  name already on the index of company names under s 67.
\item
  Power to change the company name if misleading information was given
  for use of a particular name (s 75) or if the name gives a misleading
  impression of the nature of the company's activities, likely to cause
  harm to the public (s 76).
\end{itemize}

\hypertarget{s-69}{%
\paragraph{S 69}\label{s-69}}

Any person may make an application to the Company Names Tribunal (CNT)
to change a company's existing name if that company's name is the same
as one in which

\begin{itemize}
\tightlist
\item
  the applicant has goodwill (defined as reputation of any description),
  or
\item
  is sufficiently similar that it would be likely to mislead by
  suggesting a connection.
\end{itemize}

\hypertarget{tort-of-passing-off}{%
\paragraph{Tort of Passing Off}\label{tort-of-passing-off}}

A company may be liable for the tort of passing off if it uses a name
which suggests that the company is carrying on someone else's business.

\hypertarget{trade-mark}{%
\paragraph{Trade Mark}\label{trade-mark}}

If a company name includes the name protected by a trademark then the
holder of a trademark may bring a claim against the company for
infringement of that trademark. To avoid this risk, it is prudent to
carry out a search of the Trade Marks Register.

\hypertarget{business-name}{%
\paragraph{Business Name}\label{business-name}}

A company, once it has been registered, may choose to operate with a
trading or business name which is different from its registered name.

\hypertarget{restrictions-2}{%
\subparagraph{Restrictions}\label{restrictions-2}}

There are similar restrictions on the use of business names as apply to
the company name proper, although there is no need to register a
business name. The restrictions may be found in ss 1192 to 1199 of the
CA 2006, and in the Company, Limited Liability Partnership and Business
(Names and Trading Disclosures) Regulations 2015.

\hypertarget{registered-office}{%
\subsection{Registered Office}\label{registered-office}}

\begin{Shaded}
\begin{Highlighting}[]
\NormalTok{title: s 86}
\NormalTok{A company must at all times have a registered office to which all communications and notices may be addressed. }
\end{Highlighting}
\end{Shaded}

The registered office may be at any address, but it is usually either a
place where the company carries on its business (e.g.~offices or
factory), or the address of its solicitors or accountants.

The address must be provided on the application (CA 2006, s 9(5)(a)),
and must include the country in which the registered office is situated
(CA 2006, s 9(2)(b)).

\hypertarget{statement-of-capital}{%
\subsection{Statement of Capital}\label{statement-of-capital}}

The application for registration must include a statement of capital and
initial shareholdings (s 9(4)(a)). This statement must comply with the
requirements set out in s 10.

\begin{Shaded}
\begin{Highlighting}[]
\NormalTok{title: Statement of capital and initial shareholdings}

\NormalTok{(1) The statement of capital and initial shareholdings required to be delivered in the case of a company that is to have a share capital must comply with this section.}

\NormalTok{(2) It must state—}
\NormalTok{{-} (a) the **total number** of shares of the company to be taken on formation by the subscribers to the memorandum of association,}
\NormalTok{{-} (b) the **aggregate nominal value** of those shares,}
\NormalTok{{-} (ba) the aggregate **amount (if any) to be unpaid** on those shares (whether on account of their nominal value or by way of premium), and}
\NormalTok{{-} (c) for each class of shares—}
\NormalTok{    {-} (i) prescribed particulars of the **rights** attached to the shares,}
\NormalTok{    {-} (ii) the **total number** of shares of that class, and}
\NormalTok{    {-} (iii) the aggregate **nominal value** of shares of that class.}

\NormalTok{(3) It must contain such information as may be prescribed for the purpose of identifying the subscribers to the memorandum of association.}

\NormalTok{(4) It must state, **with respect to each subscriber** to the memorandum—}
\NormalTok{{-} (a) the number, nominal value (of each share) and class of shares to be taken by him on formation, and}
\NormalTok{{-} (b) the amount to be paid up and the amount (if any) to be unpaid on each share (whether on account of the nominal value of the share or by way of premium).}

\NormalTok{(5) Where a subscriber to the memorandum is to take shares of more than one class, the information required under subsection (4)(a) is required for each class.}
\end{Highlighting}
\end{Shaded}

s 8(1)(b): each subscriber to the memorandum must take at least one
share in the company.

\hypertarget{statement-of-proposed-officers}{%
\subsection{Statement of Proposed
Officers}\label{statement-of-proposed-officers}}

Required by s 9(4)(c), with further requirements imposed by s 12.

\hypertarget{directors}{%
\subsubsection{Directors}\label{directors}}

\begin{itemize}
\tightlist
\item
  In private companies, there must be at least one director (s 154(1)).
\item
  Key information on the directors to be appointed must be included on
  the statement contained in Form IN01 (s 12(1)).
\item
  For individuals this includes the director's name, home address
  (unless exempt), an address for service of documents, country of
  residence, nationality, business occupation and date of birth (s 163).
\item
  If a corporate director is to be appointed to the new company, a human
  director must also be appointed to the new company (s 155(1))
\item
  Different information must be provided on the application form for a
  corporate director (s 164).
\item
  In all cases the company must make a statement that the proposed
  director has consented to act as a director of the new company on Form
  IN01 (s 12(3)).
\end{itemize}

\hypertarget{directors-address}{%
\subsubsection{Directors' Address}\label{directors-address}}

\begin{itemize}
\tightlist
\item
  The director must provide an address for service of legal documents in
  the application (ss 12(2)(a) and 163(1)(b)).
\item
  If an address other than the director's home address is chosen then
  the director's home address must still be disclosed on the application
  (s 12(2)(a)), but it is not a part of the application which may be
  viewed by the public (s 242(1)).
\item
  A director who originally provides their home address when
  incorporating the company can subsequently make an application to the
  Registrar under s 1088 of the CA 2006 and the Companies (Disclosure of
  Address) Regulations 2009 (SI 2009/214) to remove it.
\item
  A proposed director of a company may apply to the Registrar for
  further privacy by having their home address removed from credit
  reference agencies' records under s 243 of the CA 2006 and the
  Companies (Disclosure of Address) Regulations 2009 (SI 2009/214)
\end{itemize}

\hypertarget{secretary}{%
\subsubsection{Secretary}\label{secretary}}

s 270(1): a Ltd is not required to appoint a company secretary.

If the client does decide to appoint a company secretary for the new
company then,

\begin{itemize}
\tightlist
\item
  For an individual, the name and an address for service must be
  included in the statement of proposed officers on the application for
  registration (ss 12(1) and 277)
\item
  See s 278 for corporate secretary registration requirements.
\end{itemize}

Unlike a private company, a public company must always have a company
secretary (s 271).

The company must make a statement that a proposed company secretary has
consented to act for the new company on the application for registration
(s 12(3)).

\hypertarget{statement-of-compliance}{%
\subsection{Statement of Compliance}\label{statement-of-compliance}}

Sections 9(1) and 13 require a statement of compliance:

\begin{quote}
`I confirm that the requirements as to registration under the Companies
Act 2006 have been complied with.'
\end{quote}

Each subscriber to the memorandum (original shareholder) of the company
must make the statement, or an agent (such as a solicitor) may make it
on the subscriber's behalf. The Registrar of Companies may accept this
statement as sufficient evidence of compliance with the CA registration
procedure (s 13(2)).

\hypertarget{articles-of-association-1}{%
\subsection{Articles of Association}\label{articles-of-association-1}}

Every company must have a set of articles (s 18). s 9(5)(b) may require
a copy of the articles to be included with the application for
registration of the new company being formed.

See {[}\protect\hyperlink{articles-of-association-1}{Articles of
Association}{]}.

\hypertarget{model-articles}{%
\subsubsection{Model Articles}\label{model-articles}}

3 sets:

\begin{enumerate}
\def\labelenumi{\arabic{enumi}.}
\tightlist
\item
  for a private company limited by shares;
\item
  for a private company limited by guarantee; and
\item
  for a public company.
\end{enumerate}

\hypertarget{amended-mas}{%
\subsubsection{Amended MAs}\label{amended-mas}}

Only changes to the model articles need to be submitted with the
application for registration (s 20(1)(b)).

\hypertarget{bespoke-articles}{%
\subsubsection{Bespoke Articles}\label{bespoke-articles}}

Replace MAs with bespoke or tailor-made articles. Submit complete set of
new articles in application form (s 18(2)).

\hypertarget{entrenchment}{%
\subsubsection{Entrenchment}\label{entrenchment}}

It is possible to include provisions in the company's articles which are
entrenched (s 22). The inclusion of such entrenched provisions must be
notified to the Registrar on the application (s 23).

\begin{Shaded}
\begin{Highlighting}[]
\NormalTok{title: Entrenched provisions of the articles}

\NormalTok{(1) A company\textquotesingle{}s articles may contain provision (“provision for entrenchment”) to the effect that specified provisions of the articles may be amended or repealed only if conditions are met, or procedures are complied with, that are more restrictive than those applicable in the case of a special resolution.}

\NormalTok{(2) Provision for entrenchment may only be made—}
\NormalTok{{-} (a) in the company\textquotesingle{}s articles on formation, or}
\NormalTok{{-} (b) by an amendment of the company\textquotesingle{}s articles agreed to by all the members of the company.}

\NormalTok{(3) Provision for entrenchment does not prevent amendment of the company\textquotesingle{}s articles—}
\NormalTok{{-} (a) by agreement of all the members of the company, or}
\NormalTok{{-} (b) by order of a court or other authority having power to alter the company\textquotesingle{}s articles.}

\NormalTok{(4) Nothing in this section affects any power of a court or other authority to alter a company\textquotesingle{}s articles.}
\end{Highlighting}
\end{Shaded}

\hypertarget{memorandum-of-association}{%
\subsection{Memorandum of Association}\label{memorandum-of-association}}

s 9(1) CA: must be submitted together with application for registration
and the articles (if required).

\begin{Shaded}
\begin{Highlighting}[]
\NormalTok{title: Memorandum}
\NormalTok{{-} Must state that the subscribers wish to form a company and that they agree to become be members of the company taking at least one share each (s 8).}
\NormalTok{{-} Must be in the form set out in Companies (Registration) Regulations 2008 (SI 2008/3014)}
\end{Highlighting}
\end{Shaded}

In a 1985 Act company, this document sets out the identity of the
proposed first shareholders and also the number of shares they propose
to take on incorporation.

\hypertarget{delivery-of-documents}{%
\subsection{Delivery of Documents}\label{delivery-of-documents}}

The application for registration, the memorandum of association and the
articles of association (if required) must be delivered to the Registrar
of Companies at Companies House (s 9(1)) together with the applicable
fee.

This can be done electronically. Authentication is also done
electronically. Or send to the Registrar for England and Wales (based in
Cardiff) using Form IN01, or deliver by hand to Companies House in
London or Cardiff.

\hypertarget{form-in01}{%
\subsubsection{Form IN01}\label{form-in01}}

Contains 9 parts:

\begin{enumerate}
\def\labelenumi{\arabic{enumi}.}
\tightlist
\item
  Company details
\item
  Proposed officers
\item
  Statement of capital (if limited by shares)
\item
  Statement of guarantee (if limited by guarantee)
\item
  Persons with Significant Control
\item
  Election to keep information on the public register
\item
  Consent to act
\item
  Statement about individual PSC particulars
\item
  Statement of compliance
\item
  Last page (option to include presenter information, choose address to
  which certificate of incorporation is sent).
\end{enumerate}

\hypertarget{role-of-registrar}{%
\subsection{Role of Registrar}\label{role-of-registrar}}

Checks the application and supporting documentation:

\begin{itemize}
\tightlist
\item
  Checks company name is not already taken/ does not need approval
\item
  Checks that none of the proposed directors are disqualified from
  acting
\item
  Relies on statement of compliance made in the application.
\end{itemize}

s 14: if everything is in order, the Registrar will issue a certificate
of incorporation for the new company (s 15(1)).

\hypertarget{certificate-of-incorporation}{%
\subsection{Certificate of
Incorporation}\label{certificate-of-incorporation}}

\begin{Shaded}
\begin{Highlighting}[]
\NormalTok{title: s 15 {-} Issue of certificate of incorporation}

\NormalTok{(1) On the registration of a company, the registrar of companies shall give a certificate that the company is incorporated.}

\NormalTok{(2) The certificate must state—}
\NormalTok{{-} (a) the name and registered number of the company,}
\NormalTok{{-} (b) the date of its incorporation,}
\NormalTok{{-} (c) whether it is a limited or unlimited company, and if it is limited whether it is limited by shares or limited by guarantee,}
\NormalTok{{-} (d) whether it is a private or a public company, and}
\NormalTok{{-} (e) whether the company\textquotesingle{}s registered office is situated in England and Wales (or in Wales), in Scotland or in Northern Ireland.}

\NormalTok{(3) The certificate must be signed by the registrar or authenticated by the registrar\textquotesingle{}s official seal.}

\NormalTok{(4) The certificate is conclusive evidence that the requirements of this Act as to registration have been complied with and that the company is duly registered under this Act.}
\end{Highlighting}
\end{Shaded}

\begin{itemize}
\tightlist
\item
  s 1066: the company is given a registered number.
\item
  A new public company must also be issued with a trading certificate by
  the Registrar of Companies before it can do business or use its
  borrowing powers.
\end{itemize}

\hypertarget{effect-of-registration}{%
\subsection{Effect of Registration}\label{effect-of-registration}}

The company becomes a \textbf{separate legal person}.

\begin{Shaded}
\begin{Highlighting}[]
\NormalTok{title: s 16 {-} Effect of registration}

\NormalTok{(1) The registration of a company has the following effects as from the date of incorporation.}

\NormalTok{(2) The subscribers to the memorandum, together with such other persons as may from time to time become members of the company, are a body corporate by the name stated in the certificate of incorporation.}

\NormalTok{(3) That body corporate is capable of exercising all the functions of an incorporated company.}

\NormalTok{(4) The status and registered office of the company are as stated in, or in connection with, the application for registration.}

\NormalTok{(5) In the case of a company having a share capital, the subscribers to the memorandum become holders of the shares specified in the statement of capital and initial shareholdings.}

\NormalTok{(6) The persons named in the statement of proposed officers—}
\NormalTok{{-} (a) as director, or}
\NormalTok{{-} (b) as secretary or joint secretary of the company,}

\NormalTok{are deemed to have been appointed to that office.}
\end{Highlighting}
\end{Shaded}

s 7(2): a company must not be formed for an unlawful purpose. If by
mistake the Registrar registers a company which breaches this rule, then
the company may be stuck off the register by the court and it will be
wound up.

\hypertarget{post-incorporation}{%
\subsection{Post-incorporation}\label{post-incorporation}}

\hypertarget{first-board-meeting}{%
\subsubsection{First Board Meeting}\label{first-board-meeting}}

\begin{itemize}
\tightlist
\item
  Chairperson of the board of directors usually elected
\item
  Make up a report on the incorporation of the company
\item
  Expense the incorporation of the company to the company if decided
\item
  Open company bank account.

  \begin{itemize}
  \tightlist
  \item
    Will require directors to sign a mandate form - specifying e.g.,
    that two directors' signatures are needed to sign checks over a
    certain amount.
  \end{itemize}
\item
  Company seal

  \begin{itemize}
  \tightlist
  \item
    s 45(1): a company seal can be adopted.
  \item
    The seal will have the company's name engraved on it in legible
    characters (CA 2006, s 45(2))
  \end{itemize}
\item
  Business name

  \begin{itemize}
  \tightlist
  \item
    Can choose to use a trading name rather than the company name.
  \end{itemize}
\item
  Accounting reference date (ARD)

  \begin{itemize}
  \tightlist
  \item
    When a new company is registered, its ARD initially will be the
    anniversary of the last day of the month in which the company was
    incorporated (s 391(4)).
  \item
    ARD can be changed by a decision of the directors and filing a form
    AA01 (s 392).
  \end{itemize}
\item
  Auditor

  \begin{itemize}
  \tightlist
  \item
    All companies must prepare accounts (s 394)
  \item
    Company accounts must be audited, unless the company is

    \begin{itemize}
    \tightlist
    \item
      dormant, i.e., not trading, (s 480) or
    \item
      a small company (s 77) -- see s 382 for definition of small
      company.
    \end{itemize}
  \end{itemize}
\item
  Directors' service contracts

  \begin{itemize}
  \tightlist
  \item
    The board of directors usually will take the decision to enter into
    such contracts on behalf of the company at the first board meeting,
    unless a particular director is to be appointed for a guaranteed
    term of more than two years.
  \end{itemize}
\item
  Company records

  \begin{itemize}
  \tightlist
  \item
    Defined s 1134
  \item
    Must be kept up to date.
  \end{itemize}
\item
  Tax registrations

  \begin{itemize}
  \tightlist
  \item
    Companies House automatically notifies HMRC of the registration of
    the new company.
  \item
    HMRC sends the new company an introductory pack about tax affairs

    \begin{itemize}
    \tightlist
    \item
      Includes Form CT41G (new company details)
    \item
      PAYE and national insurance

      \begin{itemize}
      \tightlist
      \item
        Company should register the company with HMRC to arrange for the
        deduction of income tax from salaries
      \end{itemize}
    \item
      VAT

      \begin{itemize}
      \tightlist
      \item
        Most businesses (except very small ones) must register for VAT
      \end{itemize}
    \end{itemize}
  \end{itemize}
\item
  Insurance?
\item
  Shareholders' agreement?
\end{itemize}

\hypertarget{disclosing-company-details}{%
\subsubsection{Disclosing Company
Details}\label{disclosing-company-details}}

Under powers contained in s 82 of the CA 2006, the Secretary of State
for BEIS has made the
\href{https://www.legislation.gov.uk/uksi/2015/17/contents/made}{Company,
Limited Liability Partnership and Business (Names and Trading
Disclosures) Regulations 2015} (SI 2015/17). The aim behind these
Regulations is to ensure that anyone dealing with the company, or
wishing to deal with it, knows its name, its legal status and where
further information about the company may be found.

\begin{itemize}
\tightlist
\item
  Include the company name visibly at all places of business, websites,
  business letters and notices.
\item
  All websites, letters and forms should state the company's country of
  registration, registration number and registered office.
\item
  Business names should contain the names of all or none of the
  directors.
\end{itemize}

\hypertarget{shelf-company}{%
\subsection{Shelf Company}\label{shelf-company}}

Another option is for the entrepreneur to buy a pre-existing

company -- a so-called `shelf' or `ready-made' company -- and then make
changes to it so that it is ready to run the entrepreneur's business.

Shelf companies are available to purchase from company formation agents
or law firms.

\begin{itemize}
\tightlist
\item
  Can be quicker than creating a company from scratch
\item
  May already include various amendments to the model articles.
\item
  Ownership

  \begin{itemize}
  \tightlist
  \item
    Must be transferred
  \end{itemize}
\item
  Name

  \begin{itemize}
  \tightlist
  \item
    Changed by shareholders by special resolution (s 78)
  \item
    Or by any alternative procedure specified in the articles (s 79)
  \end{itemize}
\item
  Articles

  \begin{itemize}
  \tightlist
  \item
    Consider amending (s 21)
  \end{itemize}
\item
  Registered office

  \begin{itemize}
  \tightlist
  \item
    Change by board resolution
  \end{itemize}
\item
  Accounting reference date

  \begin{itemize}
  \tightlist
  \item
    Change by board resolution
  \item
    Must comply with s 392 requirements.
  \end{itemize}
\item
  Issue more share capital?
\end{itemize}

\hypertarget{shareholders}{%
\section{Shareholders}\label{shareholders}}

\begin{itemize}
\tightlist
\item
  Shareholders are the owners of a company
\item
  The term `member' is more general and applies to company limited by
  guarantee too.
\end{itemize}

\hypertarget{promoters}{%
\subsection{Promoters}\label{promoters}}

\begin{Shaded}
\begin{Highlighting}[]
\NormalTok{title: Promoter}
\NormalTok{One who undertakes to form a company with reference to a given project and to set it going, and who takes the necessary steps to accomplish that purpose ([[Twycross v Grant (1877) 2 CPD 469]])}
\end{Highlighting}
\end{Shaded}

Note, this does not include professional advisers who help set up the
company.

\hypertarget{fiduciary-duty}{%
\subsubsection{Fiduciary Duty}\label{fiduciary-duty}}

\begin{Shaded}
\begin{Highlighting}[]
\NormalTok{A promoter is placed in a **fiduciary relationship** with the company once it has formed. }
\end{Highlighting}
\end{Shaded}

The primary fiduciary duty of the promoter is not to make a `secret
profit' when forming the company.

\hypertarget{pre-incorporation-contracts}{%
\subsubsection{Pre-incorporation
Contracts}\label{pre-incorporation-contracts}}

Prior to incorporation, the company does not exist, so it is impossible
to act on behalf of the company (as agent). If the promoter does this,
they are personally liable for any contract made, subject to any
agreement to the contrary (s 51); the company does not have to take over
the contract. If the company chooses to take the contract, a contract of
novation with the third party required.

MERMAID1

Possible solution: have the pre-incorporation contract contain a
provision ending personal liability if the newly-formed company agrees
to take over the contract on the same terms.

\hypertarget{joining-company}{%
\subsection{Joining Company}\label{joining-company}}

To become a shareholder in a private company (s 112):

\begin{enumerate}
\def\labelenumi{\arabic{enumi}.}
\tightlist
\item
  A person must agree to become a shareholder of the company
\item
  Their name must be entered in the register of members.
\end{enumerate}

\hypertarget{subscribers}{%
\subsubsection{Subscribers}\label{subscribers}}

Persons who signed the
{[}\protect\hyperlink{memorandum-1}{Memorandum}{]} of association as
`subscribers' automatically become the first shareholders of the company
when the Registrar of Companies issues the certificate of incorporation
(ss 112(1) \& 16(5)).

\begin{Shaded}
\begin{Highlighting}[]
\NormalTok{title: s 112 {-} The members of a company }
\NormalTok{(1) The subscribers of a company\textquotesingle{}s memorandum are deemed to have agreed to become members of the company, and on its registration become members and must be entered as such in its register of members.}
\NormalTok{(2) Every other person who agrees to become a member of a company, and whose name is entered in its register of members, is a member of the company.}
\end{Highlighting}
\end{Shaded}

\begin{Shaded}
\begin{Highlighting}[]
\NormalTok{title: s 16(5) {-} Effect of registration }
\NormalTok{In the case of a company having a share capital, the subscribers to the memorandum become holders of the shares specified in the statement of capital and initial shareholdings.}
\end{Highlighting}
\end{Shaded}

\hypertarget{buying-shares}{%
\subsubsection{Buying Shares}\label{buying-shares}}

The allottee (purchaser) agrees to become a shareholder by formally
applying to the company to buy the new shares, and becomes a shareholder
when their name is entered in the register of members (s 112(2)).

\hypertarget{transfer-of-shares}{%
\paragraph{Transfer of Shares}\label{transfer-of-shares}}

\begin{enumerate}
\def\labelenumi{\arabic{enumi}.}
\tightlist
\item
  The transferee (purchaser) agrees to become a shareholder by
  submitting the share transfer to the company, and becomes a
  shareholder when their name is entered in the register of members.
\item
  {[}\protect\hyperlink{directors-1}{Directors}{]} must enter their name
  in the register of members, unless company
  {[}\protect\hyperlink{articles-of-association-1}{Articles of
  Association}{]} give them discretion not to, in which case the shares
  remain legally owned by the seller.
\end{enumerate}

A gift of shares is also known as a transfer of shares.

\hypertarget{inheriting-shares}{%
\paragraph{Inheriting Shares}\label{inheriting-shares}}

May be bequeathed under a will or rules of intestacy (if there is no
will). The shares will, upon death of the shareholder, automatically
vest in the personal representatives of the deceased, in an operation of
law known as \textbf{transmission of shares}.

If the personal representative is the beneficiary of the shares, they
will become the shareholder. If not, they are a temporary shareholder,
and the articles will usually provide for limited rights of ownership
until the shares are transferred to the ultimate owner. The beneficiary
must apply to register as a new shareholder.

\hypertarget{shareholder-insolvency}{%
\paragraph{Shareholder Insolvency}\label{shareholder-insolvency}}

If a shareholder goes bankrupt, the bankrupt's property, including their
shares, automatically vest in a trustee in bankruptcy. Also known as
transmission of shares. The trustee will try to sell shares to raise
money to pay off the bankrupt's debts -- and they will usually have
limited rights of ownership. New owner must register.

\hypertarget{register-of-members}{%
\subsection{Register of Members}\label{register-of-members}}

\begin{Shaded}
\begin{Highlighting}[]
\NormalTok{title: s 113 {-} Register of members}

\NormalTok{(1) Every company must keep a register of its members.}

\NormalTok{(2) There must be entered in the register—}
\NormalTok{{-} (a) the names and addresses of the members,}
\NormalTok{{-} (b) the date on which each person was registered as a member, and}
\NormalTok{{-} (c) the date at which any person ceased to be a member.}

\NormalTok{(3) In the case of a company having a share capital, there must be entered in the register, with the names and addresses of the members, a statement of—}
\NormalTok{{-} (a) the shares held by each member, distinguishing each share—}
\NormalTok{    {-} (i) by its number (so long as the share has a number), and}
\NormalTok{    {-} (ii) where the company has more than one class of issued shares, by its class, and}
\NormalTok{{-} (b) the amount paid or agreed to be considered as paid on the shares of each member.}

\NormalTok{(4) If the company has converted any of its shares into stock, and given notice of the conversion to the registrar, the register of members must show the amount and class of stock held by each member instead of the amount of shares and the particulars relating to shares specified above.}

\NormalTok{(5) In the case of joint holders of shares or stock in a company, the company\textquotesingle{}s register of members must state the names of each joint holder.}

\NormalTok{In other respects joint holders are regarded for the purposes of this Chapter as a single member (so that the register must show a single address).}

\NormalTok{(6) In the case of a company that does not have a share capital but has more than one class of members, there must be entered in the register, with the names and addresses of the members, a statement of the class to which each member belongs.}

\NormalTok{(7) If a company makes default in complying with this section an offence is committed by—}
\NormalTok{{-} (a) the company, and}
\NormalTok{{-} (b) every officer of the company who is in default.}

\NormalTok{(8) A person guilty of an offence under this section is liable on summary conviction to a fine not exceeding level 3 on the standard scale and, for continued contravention, a daily default fine not exceeding one{-}tenth of level 3 on the standard scale.}
\end{Highlighting}
\end{Shaded}

Note that the system is very different for publicly traded companies,
which have thousands of shares held in electronic form.

\begin{Shaded}
\begin{Highlighting}[]
\NormalTok{title: Can [[Private companies]] elect not to keep their own register of members?}
\NormalTok{{-} Yes {-} s 128D(2) CA 2006 {-} if they ensure necessary information is filed and kept up{-}to{-}date on the Companies House central register. }
\NormalTok{{-} All shareholders must agree for this election to be valid (s 128B(2)(a)), and the Registrar of Companies must be notified (CA 2006, ss 128B(3) and 128C(1)). }
\NormalTok{{-} During the time that this election not to have a register of members is in force, the company must notify the Registrar of Companies as soon as is reasonably practicable of any information that otherwise would have gone into the register of members (s 128E(2)). }
\NormalTok{{-} This election can be withdrawn and the register of members reactivated following the procedure in s 128J.}
\end{Highlighting}
\end{Shaded}

\hypertarget{content-and-form}{%
\subsubsection{Content and Form}\label{content-and-form}}

\begin{itemize}
\tightlist
\item
  The register must be updated whenever necessary to reflect any changes
  in the membership of the company.
\item
  Can be hard copy or electronic (s 1135)
\item
  Criminal offence by the company and any officer in default, punishable
  by fine, if the register of members does not contain correct
  information (s 113(7) \& (8)).
\end{itemize}

\hypertarget{single-shareholder}{%
\paragraph{Single Shareholder}\label{single-shareholder}}

\begin{Shaded}
\begin{Highlighting}[]
\NormalTok{title: s 123 {-} Single member companies}
\NormalTok{(1) If a limited company is formed under this Act with only one member there shall be entered in the company\textquotesingle{}s register of members, with the name and address of the sole member, a statement that the company has only one member.}

\NormalTok{(2) If the number of members of a limited company falls to one, or if an unlimited company with only one member becomes a limited company on re{-}registration, there shall upon the occurrence of that event be entered in the company\textquotesingle{}s register of members, with the name and address of the sole member—}
\NormalTok{{-} (a) a statement that the company has only one member, and}
\NormalTok{{-} (b) the date on which the company became a company having only one member.}

\NormalTok{(3) If the membership of a limited company increases from one to two or more members, there shall upon the occurrence of that event be entered in the company\textquotesingle{}s register of members, with the name and address of the person who was formerly the sole member—}
\NormalTok{{-} (a) a statement that the company has ceased to have only one member, and}
\NormalTok{{-} (b) the date on which that event occurred.}
\end{Highlighting}
\end{Shaded}

\hypertarget{entering-information}{%
\paragraph{Entering Information}\label{entering-information}}

\hypertarget{promptness}{%
\subparagraph{Promptness}\label{promptness}}

\begin{Shaded}
\begin{Highlighting}[]
\NormalTok{title: s 771(1)}
\NormalTok{When a transfer of shares in or debentures of a company has been lodged with the company, the company must either—}
\NormalTok{{-} (a) register the transfer, or}
\NormalTok{{-} (b) give the transferee notice of refusal to register the transfer, together with its reasons for the refusal,}

\NormalTok{as soon as practicable and in any event within two months after the date on which the transfer is lodged with it. }
\end{Highlighting}
\end{Shaded}

\hypertarget{prospective-shareholder-status}{%
\subparagraph{Prospective Shareholder
Status}\label{prospective-shareholder-status}}

The prospective new shareholder's status between the date on which they
acquire the shares

and the date on which their name is entered on the register of members,
is that they are

beneficially entitled to the shares but are not the registered legal
owner of them.

The legal owner must vote at a meeting in accordance with the
instructions of the prospective shareholder, and must account to the
prospective shareholder for any dividends.

\hypertarget{court-power}{%
\subparagraph{Court Power}\label{court-power}}

The court has the power under s 125 to rectify the register of members
for an unwanted omission or entry, or for default or unnecessary delay
in removing a shareholder's name when leaving the company.

It may at the same time order the company to pay damages to a wronged
person and settle any disputes regarding ownership ({[}{[}Avenue Road
Developments Ltd v Reggies Co Ltd {[}2012{]} EWHC 1625 (Ch){]}{]}).

\hypertarget{inspection}{%
\paragraph{Inspection}\label{inspection}}

Under s 114(1), where the company does not keep this information on the
public register at Companies House, the register of members must be kept
either at the company's registered office, or at its `single alternative
inspection location' (SAIL) if specified under s 1136.

\begin{Shaded}
\begin{Highlighting}[]
\NormalTok{title: s 116 {-} Rights to inspect and require copies}

\NormalTok{(1) The register and the index of members\textquotesingle{} names must be open to the inspection—}
\NormalTok{{-} (a) of any member of the company without charge, and}
\NormalTok{{-} (b) of any other person on payment of such fee as may be prescribed.}

\NormalTok{(2) Any person may require a copy of a company\textquotesingle{}s register of members, or of any part of it, on payment of such fee as may be prescribed.}

\NormalTok{(3) A person seeking to exercise either of the rights conferred by this section must make a request to the company to that effect.}

\NormalTok{(4) The request must contain the following information—}
\NormalTok{{-} (a) in the case of an individual, his name and address;}
\NormalTok{{-} (b) in the case of an organisation, the name and address of an individual responsible for making the request on behalf of the organisation;}
\NormalTok{{-} (c) the purpose for which the information is to be used; and}
\NormalTok{{-} (d) whether the information will be disclosed to any other person, and if so—}
\NormalTok{    {-} (i) where that person is an individual, his name and address,}
\NormalTok{    {-} (ii) where that person is an organisation, the name and address of an individual responsible for receiving the information on its behalf, and}
\NormalTok{    {-} (iii) the purpose for which the information is to be used by that person.}
\end{Highlighting}
\end{Shaded}

\hypertarget{responding-to-request}{%
\subparagraph{Responding to Request}\label{responding-to-request}}

The company has five working days under s 117(1) of the CA 2006, either
to:

\begin{enumerate}
\def\labelenumi{\arabic{enumi}.}
\tightlist
\item
  Comply with the request

  \begin{itemize}
  \tightlist
  \item
    Must make clear the latest date on which amendments were made to the
    register (if any) and that there were no further amendments to be
    made (CA 2006, s 120(1)).
  \end{itemize}
\item
  Apply to the court to disallow the request (if it thinks the request
  is not for a `proper purpose') and thus prevent inspection or copying
  of the register.

  \begin{itemize}
  \tightlist
  \item
    Onus is on the company to show that the request was made for an
    improper purpose ({[}{[}Burberry Group plc v Richard Charles
    Fox-Davies {[}2014{]} EWCA Civ 604{]}{]})
  \end{itemize}
\end{enumerate}

\begin{Shaded}
\begin{Highlighting}[]
\NormalTok{title: Rationale }
\NormalTok{To protect privacy of shareholders where company is involved in controversial work.}
\end{Highlighting}
\end{Shaded}

\hypertarget{non-compliance-consequences}{%
\subparagraph{Non-compliance
Consequences}\label{non-compliance-consequences}}

The company and every officer in default (under s 118(1)) will commit an
offence for failure,

without a s 117 court order, to allow inspection or copying of the
register; and the person

making the request will commit an offence (under s 119(1)) if they
knowingly or recklessly

make a statement which is materially misleading, false or deceptive.

\hypertarget{psc-register}{%
\subsection{PSC Register}\label{psc-register}}

s 790M of the CA 2006 requires companies to keep a `PSC register'
(Persons of Significant Control). The register must be made available
for public inspection. Note that this requirement applies only to
private companies and public companies that are not publicly traded.

\begin{Shaded}
\begin{Highlighting}[]
\NormalTok{A person off significant control is an individual or ‘relevant legal entity’ (RLE) (s 790C(6)), who (s 790 \& Sch 1A):}
\NormalTok{1. owns $\textgreater{}25\textbackslash{}\%$ of the shares in the company (this would include non{-}voting shares, such as preference shares); or}
\NormalTok{2. owns or controls $\textgreater{}25\textbackslash{}\%$ of the voting rights in the company (this would usually be ordinary shares); or}
\NormalTok{3. has the right to appoint or remove a majority of the board of directors of the company; or}
\NormalTok{4. has the right to exercise, or who actually exercises, significant influence or control over the company.}
\end{Highlighting}
\end{Shaded}

\hypertarget{obligation-on-company}{%
\subsubsection{Obligation on Company}\label{obligation-on-company}}

Under ss 790D and 790E of the CA 2006, the company is placed under an
obligation to investigate, obtain and update information relevant to the
register. Additional obligations are placed on the individual or
individuals concerned by ss 790G and 790H, to notify the company of
their significant control.

\hypertarget{information-to-include}{%
\subsubsection{Information to Include}\label{information-to-include}}

\begin{longtable}[]{@{}
  >{\raggedright\arraybackslash}p{(\columnwidth - 2\tabcolsep) * \real{0.4328}}
  >{\raggedright\arraybackslash}p{(\columnwidth - 2\tabcolsep) * \real{0.5672}}@{}}
\toprule()
\begin{minipage}[b]{\linewidth}\raggedright
Individual
\end{minipage} & \begin{minipage}[b]{\linewidth}\raggedright
Relevant legal entity
\end{minipage} \\
\midrule()
\endhead
Name & Corporate or firm name \\
DOB & Registered or principal office \\
Nationality & Legal form of identity \\
Country of usual residence & Law by which governed \\
Address for service documents & Register of companies in which
entered \\
Residential address & Registration number (if relevant) \\
\bottomrule()
\end{longtable}

\hypertarget{levels-of-significant-control}{%
\subsubsection{Levels of Significant
Control}\label{levels-of-significant-control}}

Under the Register of People with Significant Control Regulations 2016
(SI 2016/339), there are \textbf{three different levels} of significant
control which need to be notified and which need to be included in the
PSC register, with percentage share holding \(S\):

\begin{enumerate}
\def\labelenumi{\arabic{enumi}.}
\tightlist
\item
  \(25\% < S \leq 50\%\)
\item
  \(50\% < S < 75\%\)
\item
  \(S > 75\%\)
\end{enumerate}

790E(5): the company must include this information on the PSC register
within 14 days from the day after it becomes aware of a change or has
reasonable cause to believe there has been a change.

\hypertarget{administration}{%
\subsubsection{Administration}\label{administration}}

\begin{itemize}
\tightlist
\item
  Can elect not to keep own register and just file with Companies House
  (s 790X)
\item
  The company must notify Companies House when someone becomes a PSC or
  is no longer a PSC by using forms PSC01 to PSC09
\item
  Deadline for form 14 days after deadline for completing register.
\end{itemize}

\hypertarget{shareholders-rights}{%
\subsection{Shareholders' Rights}\label{shareholders-rights}}

Aside from limited liability, shareholders also have contractual rights.

\hypertarget{s-33-contract-1}{%
\subsubsection{S 33 Contract}\label{s-33-contract-1}}

\begin{itemize}
\tightlist
\item
  `Membership only' rights enforceable by the shareholder.
\item
  Contract also imposes the obligation on the shareholder to observe the
  terms of the constitution.
\item
  The rights will as a minimum cover sharing in the profits of the
  company, sharing in any surplus on a profitable winding up of the
  company and any voting rights at shareholder meetings.
\end{itemize}

\hypertarget{shareholders-agreement}{%
\subsubsection{Shareholders' Agreement}\label{shareholders-agreement}}

See {[}{[}Shareholders' Agreements{]}{]}. This is a private contract
between shareholders, not involving the company.

It is not compulsory for the shareholders to enter the agreement (though
you can enforce by making anyone selling shares have as a term of the
contract that the new owner will join).

Note that a shareholders' agreement can also include non-shareholders as
parties.

\hypertarget{benefits}{%
\paragraph{Benefits}\label{benefits}}

Breach of contract remedies available if parties breach terms.
Advantages of putting terms in a shareholders' agreement rather than
articles:

\begin{itemize}
\tightlist
\item
  A shareholders' agreement can deal with matters which are personal to
  the shareholders, rather than just relating to their rights as
  members.
\item
  A shareholders' agreement can help protect minority shareholders

  \begin{itemize}
  \tightlist
  \item
    Unanimity needed for a change of contract.
  \item
    Can change things which wouldn't be possible under Articles; e.g.,
    get all the shareholders to agree to unanimity to change the
    Articles of the company.
  \end{itemize}
\item
  Confidentiality -- shareholders' agreement does not need to be made
  publicly available.
\end{itemize}

\hypertarget{limitations}{%
\paragraph{Limitations}\label{limitations}}

\begin{enumerate}
\def\labelenumi{\arabic{enumi}.}
\tightlist
\item
  Nothing may be included that fetters the company's powers to exercise
  its statutory duties.
\item
  The shareholders cannot agree to anything that would bind them as to
  how they might vote as directors, if they also held this position, as
  otherwise they might possibly breach their duties as directors.
\end{enumerate}

\hypertarget{statutory-rights}{%
\paragraph{Statutory Rights}\label{statutory-rights}}

\begin{longtable}[]{@{}
  >{\raggedright\arraybackslash}p{(\columnwidth - 2\tabcolsep) * \real{0.0693}}
  >{\raggedright\arraybackslash}p{(\columnwidth - 2\tabcolsep) * \real{0.9307}}@{}}
\toprule()
\begin{minipage}[b]{\linewidth}\raggedright
Right
\end{minipage} & \begin{minipage}[b]{\linewidth}\raggedright
Details
\end{minipage} \\
\midrule()
\endhead
Share certificate & A shareholder must receive from the company a share
certificate within two months either of allotment (if new shares are
being issued) under s 769(1) of the CA 2006, or of lodging the transfer
with the company (if existing shares are being transferred from an
existing shareholder) under s 776(1) of the Act. Share certificate prima
facie evidence of ownership (s 768(1)) \\
Register of members & New shareholder has a right to be entered in
register within 2 months (subject to any director discretion to
refuse) \\
Company documentation & Company must send constitutional documents/
current statement of capital to shareholder on request (s 32). Annual
accounts \& reports must be sent each year, no request required (s
423) \\
Inspect company documents & Minutes kept at registered office or SAIL,
which shareholders are permitted to read (s 358(1)). Other registers and
information also must be available on request. \\
Voting rights & Right to vote at shareholder general meetings, either in
person or by proxy (s 324(1)). Votes either by show of hands or poll
vote. Shareholders with 10\% of voting rights can demand poll vote. \\
Notice of GMs & If proper notice of GM not given to all shareholders, GM
not valid (ss 307-311). Notice must include sufficient information of
what is to be proposed, including exact wording of special resolutions
and sufficient info on ordinary resolutions. \\
Written resolution & Shareholders holding \(\geq 5\%\) voting shares (or
less if articles specify) have the right to circulate a written
resolution and accompanying 1000-word statement (s 292(1)). Request must
be authenticated (signed) and identify resolution (s 293). Private
company shareholder can be asked to pay the expenses of circulation (s
294(1)). \\
Written statement & s 314(1): shareholders have the right to circulate a
written 1000-word statement, if they hold \(\geq 5\%\) voting shares, or
at least 100 shareholders who hold at least an average of £100 paid up
share capital (s 314(2)). Request must be deposited at company's
registered office \(\geq 1\) week before GM (s 314(4)). Company must
circulate statement to all shareholders (s 315(1)). Can request expenses
of circulation from shareholder (s 316(2)). \\
Removal of director & s 168(1): director can be removed by ordinary
resolution. \\
Removal of auditor & s 510(1): auditor can be removed by ordinary
resolution. \\
Court proceedings & Shareholders have the right to go to court in
relation to most important transactions, e.g., to cancel a resolution
approving the buy-back of shares (s 721(1)). \\
Calling GM & Requisitioning: shareholders holding \(\geq 5\%\) voting
shares have the right to require directors to call GM (s 303(1)) by
depositing written request to company's registered office. If the
directors don't, shareholders can call the GM themselves/ apply the
court (s 306). \\
Court ordered GM & s 306: any shareholder may apply to court for an
order that GM be held. Court can order terms for GM, e.g., changing the
quorum ({[}{[}Smith v Butler {[}2011{]} EWHC 2301 (Ch){]}{]}) \\
Breach of constitution & Shareholder may seek to restrain a breach of
{[}{[}Business Law and Practice/Company Law/Directors' duties{]}{]}
before the event, or take action after under s 40 \\
{[}\protect\hyperlink{unfair-prejudice}{Unfair Prejudice}{]} & s 994(1):
shareholder can petition court for a remedy if she feels unfairly
prejudiced. Conduct may be unfairly prejudicial to all of the members or
to only some or one of them ({[}{[}Re a Company (No 004175 of 1986)
{[}1987{]} BCLC 574{]}{]}). Conduct must be both unfair and
prejudicial. \\
{[}{[}Just and equitable winding up{]}{]} & s 122(1)(g) IA 1986: any
shareholder may make an application to have the company wound up on the
ground that it is just and equitable to do so, provided they can prove a
`tangible interest'. Common examples: where management is in deadlock,
where shareholders no longer have confidence in management, where
company can no longer carry out objects, and in quasi-partnership where
a partner is excluded from management. \\
\bottomrule()
\end{longtable}

\hypertarget{unfair-prejudice}{%
\paragraph{Unfair Prejudice}\label{unfair-prejudice}}

\begin{itemize}
\tightlist
\item
  Whether {[}\protect\hyperlink{unfair-prejudice}{Unfair Prejudice}{]}
  has occurred is judged objectively from the perspective of an
  impartial outsider.
\item
  The essential element of `unfairness' is breach of the agreement
  between shareholders as to how the company is to be run ({[}{[}O'Neill
  v Phillips {[}1999{]} 1 WLR 1092{]}{]}).
\item
  The shareholder must also prove that they have been affected in their
  capacity as a shareholder, although this has been given a very wide
  interpretation ({[}{[}Gamlestaden Fastigheter v Baltic Partners
  Limited {[}2008{]} 1 BCLC 468{]}{]}).
\item
  It is not necessary to prove that the value of the shareholder's
  shares has been adversely affected, although frequently this will
  happened
\end{itemize}

\begin{Shaded}
\begin{Highlighting}[]
\NormalTok{1. non{-}payment of dividends;}
\NormalTok{2. directors awarding themselves excessive remuneration;}
\NormalTok{3. directors exercising their powers for an improper purpose (eg to ‘freeze out’ a minority shareholder); and}
\NormalTok{4. exclusion from management in a small company (eg one formed on the understanding that all those involved will share the running of the business and the profits).}
\end{Highlighting}
\end{Shaded}

\begin{longtable}[]{@{}
  >{\raggedright\arraybackslash}p{(\columnwidth - 2\tabcolsep) * \real{0.2778}}
  >{\raggedright\arraybackslash}p{(\columnwidth - 2\tabcolsep) * \real{0.7222}}@{}}
\toprule()
\begin{minipage}[b]{\linewidth}\raggedright
Case
\end{minipage} & \begin{minipage}[b]{\linewidth}\raggedright
Ratio
\end{minipage} \\
\midrule()
\endhead
{[}{[}VB Football Assets v Blackpool Football Club Ltd and Others
{[}2017{]} EWHC 2767 (Ch){]}{]} & Improper payments out of the company,
a failure to pay dividends and deliberate exclusion from company
decisions were unfairly prejudicial to a minority shareholder whereas
changes to the articles of association were not. \\
{[}{[}Re CF Booth Ltd (sub nom Booth v Booth) {[}2017{]} EWHC 457
(Ch){]}{]} & Excessive director remuneration when no dividends paid
unfairly prejudicial \\
{[}{[}Sharafi v Woven Rugs Ltd (in administration) {[}2010{]} EWHC 230
(Ch){]}{]} & Director's impropriety prejudicial. \\
\bottomrule()
\end{longtable}

\hypertarget{remedy}{%
\subparagraph{Remedy}\label{remedy}}

If the court finds that a shareholder has suffered unfair prejudice, it
may make any order it thinks appropriate under s 996(1) of the CA 2006.

The normal way of dealing with internal company disputes in small
private companies where unfairly prejudicial conduct had been proved is
an order for\\
a share purchase under s 996 ({[}{[}Grace v Biagioli {[}2005{]} EWCA Civ
1022{]}{]}).

\hypertarget{limitations-1}{%
\subparagraph{Limitations}\label{limitations-1}}

\begin{enumerate}
\def\labelenumi{\arabic{enumi}.}
\tightlist
\item
  In {[}{[}O'Neill v Phillips {[}1999{]} 1 WLR 1092 HL{]}{]}, the House
  of Lords made it clear that although the court's powers under s 996 of
  the CA 2006 were wide, that provision did not give an automatic right
  to withdrawal from a company where trust and confidence had broken
  down.
\item
  Where the parties had agreed to binding arbitration to settle a matter
  of unfair prejudice, proceedings would not be allowed under s 994 of
  the CA 2006 ({[}{[}Fulham Football Club (1987) Ltd v Richards
  {[}2011{]} EWCA Civ 855{]}{]}).
\end{enumerate}

\hypertarget{shareholders-role}{%
\subsection{Shareholders' Role}\label{shareholders-role}}

Generally limited: directors control things day-to-day. The shareholders
can change things indirectly by appointing/ removing directors and
voting at GMs.

\hypertarget{protection-of-minority-shareholders}{%
\subsection{Protection of Minority
Shareholders}\label{protection-of-minority-shareholders}}

\hypertarget{majority-rule}{%
\subsubsection{Majority Rule}\label{majority-rule}}

Generally, the principle of majority rule governs companies.

\begin{Shaded}
\begin{Highlighting}[]
\NormalTok{title: What happens if directors breach their duties to the company?}
\NormalTok{Rule in [[Foss v Harbottle [1843]]]:}
\NormalTok{1. Proper claimant rule: the company, rather than shareholder, must bring a claim for a wrong done to the company. }
\NormalTok{2. Internal management rule: courts will not interfere with the internal management of a company acting within its powers. So when a wrong is alleged, it is for a majority of shareholders to decide whether to make a claim on behalf of the company or to ratify the wrong by ordinary resolution. }
\end{Highlighting}
\end{Shaded}

\hypertarget{statutory-protection}{%
\subsubsection{Statutory Protection}\label{statutory-protection}}

These are pretty watered down. But still have:

\begin{itemize}
\tightlist
\item
  Unfair prejudice protections

  \begin{itemize}
  \tightlist
  \item
    Stops e.g., a director sanctioning their own misdemeanours.
  \end{itemize}
\item
  If you have 5\% voting rights, can circulate written resolution,
  requisition GM or circulate written statement.
\item
  Petition court for winding up (rare)
\item
  If 25\% can stop special resolutions.
\end{itemize}

\hypertarget{shareholders-agreement-1}{%
\subsubsection{Shareholders Agreement}\label{shareholders-agreement-1}}

Can protect minority shareholders.

\hypertarget{ratification}{%
\subsubsection{Ratification}\label{ratification}}

The right of the majority to pass an ordinary resolution to ratify
wrong-doing in the company cannot be exercised where the approval of
that action would be illegal or requires a special resolution under the
CA 2006.

\hypertarget{amending-articles-1}{%
\subsubsection{Amending Articles}\label{amending-articles-1}}

The company's articles must be `bona fide for the benefit of the company
as a whole', and that they may be challenged if they are made with the
intention of discriminating against minority shareholders.

\hypertarget{derivative-claims}{%
\subsubsection{Derivative Claims}\label{derivative-claims}}

See {[}\protect\hyperlink{derivative-claims}{Derivative claims}{]}.
There are circumstances in which a minority shareholder is permitted to
bring a `derivative claim' in the company's name for a wrong committed
against the company.

\begin{Shaded}
\begin{Highlighting}[]
\NormalTok{This is an exception to the rule in [[Foss v Harbottle [1843]]]. Can only be used where the minority shareholder wishes to make a claim in respect of a wrong committed against them, rather than against the property. }
\end{Highlighting}
\end{Shaded}

A claim may be brought by any shareholder under s 260(3) of the CA 2006
for an actual or proposed act or omission involving negligence, default,
breach of duty or breach of trust by a\\
director. There is no need to demonstrate any actual loss suffered by
the company, or indeed any benefit gained by the directors.

\hypertarget{stages}{%
\paragraph{Stages}\label{stages}}

There are two stages:

\begin{enumerate}
\def\labelenumi{\arabic{enumi}.}
\tightlist
\item
  Preliminary stage to decide whether the applicant is entitled to bring
  a claim

  \begin{enumerate}
  \def\labelenumii{\arabic{enumii}.}
  \tightlist
  \item
    s 261(1): claim dismissed if no prima facie case
  \item
    Then decide whether to hold a full hearing (s 261(4))
  \end{enumerate}
\item
  Hearing of the claim.
\end{enumerate}

Aim is to weed out `frivolous or vexatious' claims.

\hypertarget{relevant-factors}{%
\paragraph{Relevant Factors}\label{relevant-factors}}

The court will consider:

\begin{enumerate}
\def\labelenumi{\arabic{enumi}.}
\tightlist
\item
  whether or not the shareholder is acting in good faith in bringing the
  claim;
\item
  the importance a director (who is under a duty to promote the success
  of the company under s 172 of the CA 2006) would place on continuing
  the claim; and
\item
  whether authorisation or ratification of the wrong by the company
  would be likely.
\end{enumerate}

The court must also have particular regard to any evidence put before it
as to the views of shareholders who have no personal interest in the
matter (CA 2006, s 263(4)).

\hypertarget{automatic-refusal}{%
\paragraph{Automatic Refusal}\label{automatic-refusal}}

The court must refuse permission under s 263(2) if:

\begin{enumerate}
\def\labelenumi{\arabic{enumi}.}
\tightlist
\item
  a person acting in accordance with the director's duty under s 172 of
  the CA 2006 (duty to promote the success of the company) would not
  seek to continue the claim; or
\item
  the act or omission forming the basis of the claim has been authorised
  or ratified by the company. Under s 239 of the CA 2006, the
  ratification must be passed by the shareholders without the votes of
  the director concerned (assuming the director is a shareholder) or a
  person connected to them (e.g., their wife or civil partner).
\end{enumerate}

\hypertarget{remedy-1}{%
\paragraph{Remedy}\label{remedy-1}}

If this claim is successful, it will result in\\
a remedy being awarded to the company and not the shareholder, as a
derivative claim is being brought on behalf of the company.

\hypertarget{types-of-shareholder}{%
\subsection{Types of Shareholder}\label{types-of-shareholder}}

\begin{itemize}
\tightlist
\item
  Single member company (possible under s 7)
\item
  Holder of ordinary/ preference {[}{[}Shares{]}{]}
\item
  Possible to hold a share jointly! s 113(5)
\item
  Corporate shareholder

  \begin{itemize}
  \tightlist
  \item
    Required to appoint a human to act on its behalf at general
    meetings. Can be a proxy or corporate representative.
  \item
    s 323(1): corporate representative appointed by a resolution of the
    board of directors
  \item
    Representative then has the same powers as an individual shareholder
    (s 323(2)).
  \end{itemize}
\item
  Public company shareholder: more rights, more responsibilities.
\end{itemize}

\hypertarget{shareholder-power}{%
\subsection{Shareholder Power}\label{shareholder-power}}

\begin{longtable}[]{@{}
  >{\raggedright\arraybackslash}p{(\columnwidth - 4\tabcolsep) * \real{0.3333}}
  >{\raggedright\arraybackslash}p{(\columnwidth - 4\tabcolsep) * \real{0.3333}}
  >{\raggedright\arraybackslash}p{(\columnwidth - 4\tabcolsep) * \real{0.3333}}@{}}
\toprule()
\begin{minipage}[b]{\linewidth}\raggedright
Shareholding
\end{minipage} & \begin{minipage}[b]{\linewidth}\raggedright
What shareholders can do
\end{minipage} & \begin{minipage}[b]{\linewidth}\raggedright
Restrictions
\end{minipage} \\
\midrule()
\endhead
100\% & pass all resolutions at will & legality \\
75\% & pass special resolution & weighted voting rights (if in articles)
unfair prejudice petition (if wrongdoing) \\
over 50\% & pass ordinary resolution & weighted voting rights (if in
articles) unfair prejudice petition (if wrongdoing) \\
over 25\% & block special resolution & \\
10\% & demand poll vote & \\
5\% & circulate a written resolution requisition a general meeting
circulate a written statement & \\
\bottomrule()
\end{longtable}

\hypertarget{officers-of-the-company}{%
\section{Officers of the Company}\label{officers-of-the-company}}

Includes company secretary, directors, managers and auditors.

\hypertarget{company-secretary}{%
\subsection{Company Secretary}\label{company-secretary}}

\begin{Shaded}
\begin{Highlighting}[]
\NormalTok{title: s 270(1)}
\NormalTok{A private company is not required to have a secretary.}
\end{Highlighting}
\end{Shaded}

The company secretary is an officer of the company (s 1121).

\hypertarget{functions}{%
\subsubsection{Functions}\label{functions}}

\begin{itemize}
\tightlist
\item
  Generally centres around administration of the company (writing board
  minutes, keeping registers up-to-date, sending returns to Registrar of
  Companies).
\item
  Has apparent authority on behalf of the company to make contracts
  connected with the administrative side of the company's business,
  though not trading contracts.
\end{itemize}

\hypertarget{appointment}{%
\subsubsection{Appointment}\label{appointment}}

\begin{longtable}[]{@{}
  >{\raggedright\arraybackslash}p{(\columnwidth - 2\tabcolsep) * \real{0.1638}}
  >{\raggedright\arraybackslash}p{(\columnwidth - 2\tabcolsep) * \real{0.8362}}@{}}
\toprule()
\begin{minipage}[b]{\linewidth}\raggedright
Time appointed
\end{minipage} & \begin{minipage}[b]{\linewidth}\raggedright
Procedure
\end{minipage} \\
\midrule()
\endhead
When company first registered & Named on the statement of proposed
officers on Form IN01. Secretary automatically takes office when
certificate of incorporation issued (s 16(6)(b)) \\
Subsequently & Directors pass board resolution. Company must notify
Registrar of Companies within 14 days of appointment of secretary on
Form AP03 (human secretary) or Form AP04 (Corporate secretary). Then
update the register of secretaries if applicable (s 275(2)). \\
\bottomrule()
\end{longtable}

\hypertarget{register-of-secretaries}{%
\paragraph{Register of Secretaries}\label{register-of-secretaries}}

\begin{Shaded}
\begin{Highlighting}[]
\NormalTok{title: s 275 {-} Duty to keep register of secretaries}

\NormalTok{(1) A company must keep a register of its secretaries.}

\NormalTok{(2) The register must contain the required particulars (see sections 277 to 279) of the person who is, or persons who are, the secretary or joint secretaries of the company.}

\NormalTok{(3) The register must be kept available for inspection—}
\NormalTok{{-} (a) at the company\textquotesingle{}s registered office, or}
\NormalTok{{-} (b) at a place specified in regulations under section 1136.}

\NormalTok{(4) The company must give notice to the registrar—}
\NormalTok{{-} (a) of the place at which the register is kept available for inspection, and}
\NormalTok{{-} (b) of any change in that place,}

\NormalTok{unless it has at all times been kept at the company\textquotesingle{}s registered office.}

\NormalTok{(5) The register must be open to the inspection—}
\NormalTok{{-} (a) of any member of the company without charge, and}
\NormalTok{{-} (b) of any other person on payment of such fee as may be prescribed.}

\NormalTok{(6) If default is made in complying with subsection (1), (2) or (3), or if default is made for 14 days in complying with subsection (4), or if an inspection required under subsection (5) is refused, an offence is committed by—}
\NormalTok{{-} (a) the company, and}
\NormalTok{{-} (b) every officer of the company who is in default.}

\NormalTok{For this purpose a shadow director is treated as an officer of the company. }
\end{Highlighting}
\end{Shaded}

Private companies can elect not to keep their own register, but to hold
everything centrally with Companies House instead (s 279A).

\hypertarget{corporate-secretaries}{%
\paragraph{Corporate Secretaries}\label{corporate-secretaries}}

For corporate secretaries, the following must be entered into the
register:

\begin{enumerate}
\def\labelenumi{\arabic{enumi}.}
\tightlist
\item
  Corporate name and company's registered/ principal office
\item
  Register and registration number (EEA company)
\item
  Legal form and governing company + registration details (non-EEA).
\end{enumerate}

\hypertarget{terms-of-contract}{%
\paragraph{Terms of Contract}\label{terms-of-contract}}

Directors have the power to decide contractual terms on which secretary
holds office and remuneration.

\hypertarget{removal}{%
\paragraph{Removal}\label{removal}}

\begin{itemize}
\tightlist
\item
  Directors have power to remove at any times
\item
  May depend on employment contract and give rise to statutory
  employment law claims.
\item
  s 276(1)(a): notice Registrar of Companies within 14 days on Form
  TM02.
\end{itemize}

\hypertarget{change-in-details}{%
\paragraph{Change in Details}\label{change-in-details}}

\begin{itemize}
\tightlist
\item
  s 276(1)(b): notice Registrar of Companies within 14 days on Form CH03
  (human) or CH04 (corporate).
\end{itemize}

\hypertarget{public-companies}{%
\paragraph{Public Companies}\label{public-companies}}

Must have a secretary (s 271), who is qualified as specified in s 273 (3
years experience in the last 5/ has professional qualification etc.)

\hypertarget{directors-1}{%
\subsection{Directors}\label{directors-1}}

See {[}\protect\hyperlink{directors-1}{Directors}{]}.

Every private company must have at least one director (s 154(1)), and
every public

company must have at least two (s 154(2)).

\begin{Shaded}
\begin{Highlighting}[]
\NormalTok{title: s 250 {-} "Director"}
\NormalTok{In the Companies Acts, “director” includes any person occupying the position of director, by whatever name called. }
\end{Highlighting}
\end{Shaded}

\hypertarget{executive-director}{%
\subsubsection{Executive Director}\label{executive-director}}

Appointed to an executive office (finance, managing, \ldots). Both an
officer and employee of their company -- has a service contract under
which the director agrees to work for the company in return for a salary
and benefits.

\hypertarget{chief-executive}{%
\paragraph{Chief-executive}\label{chief-executive}}

The board of directors has the power to delegate their powers. They
usually do this by appointing a Chief Executive or Managing Director.

CEO will be formally appointed as a director with specific powers
granted by company articles (e.g., art 5 MA).

\hypertarget{non-executive-director-ned}{%
\paragraph{Non-executive Director
(NED)}\label{non-executive-director-ned}}

An officer of the company, but not an employee of the company. The role
involves independent guidance and advice to the board, and protecting
the interests of shareholders.

In a private company, there are often no NEDs. More relevant to
{[}\protect\hyperlink{public-companies}{Public companies}{]} -- certain
number mandatory for joining Main Market.

NED can still be liable to the company, for example, for breach of their
director's duties ({[}{[}Equitable Life v Bowley {[}2003{]} EWHC
2263{]}{]}).

\hypertarget{chairperson}{%
\paragraph{Chairperson}\label{chairperson}}

\begin{itemize}
\tightlist
\item
  Powers determined by the company's articles.
\item
  Private company with MA: no special powers other than their casting
  vote in the event of an equal vote for and against a board resolution
  (art 13).
\item
  For a public company, the chairperson is the figurehead.
\end{itemize}

\hypertarget{shadow-director}{%
\paragraph{Shadow Director}\label{shadow-director}}

\begin{Shaded}
\begin{Highlighting}[]
\NormalTok{title: Shadow director}
\NormalTok{A person in accordance with whose directions or instructions the directors of the company are accustomed to act.}
\end{Highlighting}
\end{Shaded}

\begin{longtable}[]{@{}
  >{\raggedright\arraybackslash}p{(\columnwidth - 2\tabcolsep) * \real{0.7622}}
  >{\raggedright\arraybackslash}p{(\columnwidth - 2\tabcolsep) * \real{0.2378}}@{}}
\toprule()
\begin{minipage}[b]{\linewidth}\raggedright
Case
\end{minipage} & \begin{minipage}[b]{\linewidth}\raggedright
Ratio
\end{minipage} \\
\midrule()
\endhead
A shadow director was anyone, other than a professional adviser, who
exercises real influence in the corporate affairs of a company but who
does not necessarily give directions or instructions on every matter
decided by the directors. & {[}{[}Secretary of State for Trade and
Industry v Deverell {[}2000{]} 2 WLR 907{]}{]} \\
A person becomes a shadow director only when the directors act on the
instructions given. The mere act of giving instructions is not enough. &
{[}{[}Ultraframe (UK) Ltd v Fielding {[}2005{]} EWHC 1638{]}{]} \\
It is not necessary that all of the directors acted in accordance with
the shadow director's directions but that a majority of them did &
{[}{[}In the matter of Coroin Ltd, Ch D, 5 March 2012{]}{]} \\
\bottomrule()
\end{longtable}

Shadow directors are subject to the same directors' duties as far as
capable of applying (s 170(5)).

The rule aims to stop people avoiding the onerous duties imposed on
directors by not being formally appointed, even though they are running
the company on a daily basis. But a person is not deemed a shadow
director by reason only that the directors act on advice given by them
in a professional capacity (s 251(2) CA 2006).

\hypertarget{de-facto-director}{%
\paragraph{De Facto Director}\label{de-facto-director}}

Someone who assumes to act as a director but has in fact not been
validly appointed, and therefore is not a de jure director.

Case law has held that such a person may fall within the definition of a
`director' under s 250(1) and under the IA 1986 (e.g., {[}{[}Re Hydrodam
(Corby) Ltd {[}1994{]} 2 BCLC 180{]}{]}).

\begin{Shaded}
\begin{Highlighting}[]
\NormalTok{title: De facto director relevant factors}
\NormalTok{1. Whether they have assumed responsibility to act as a director.}
\NormalTok{2. Whether or not they acted as a director must be determined objectively. It does not matter whether the individual thought they were acting as a director.}
\NormalTok{3. Whether the company considered the individual to be a director and held them out as such, and whether third parties considered that they were a director}
\NormalTok{4. Look at the acts in their context and determine their cumulative effect.}
\NormalTok{5. The fact that a person is consulted about directorial decisions, or is asked for approval, does not in general make them a director because they are not making the decision.}
\end{Highlighting}
\end{Shaded}

An individual director of a corporate director of another company is not
a de facto director of that other company, and could therefore not be
held personally liable for the unlawful payment of a dividend
({[}{[}Holland v The Commissioners for HMRC {[}2010{]} UKSC 51{]}{]}) --
corporate veil.

\hypertarget{alternate-director}{%
\paragraph{Alternate Director}\label{alternate-director}}

Companies may provide in their Articles for the appointment of alternate
directors (not a provision in MA). An alternate director attends board
meetings and acts in place of the director, if the actual director is
incapacitated/ away. Usually approved by resolution of board of
directors.

Usually, another director can be appointed as an alternate director.

\hypertarget{corporate-director}{%
\paragraph{Corporate Director}\label{corporate-director}}

It is possible to have a corporate director of a company {[}{[}(Holland
v The Commissioners for HMRC {[}2010{]} UKSC 51{]}{]}). But subject to:

\begin{Shaded}
\begin{Highlighting}[]
\NormalTok{title: s 155 {-} Companies required to have at least one director who is a natural person}
\NormalTok{(1) A company must have at least one director who is a natural person.}

\NormalTok{(2) This requirement is met if the office of director is held by a natural person as a corporation sole or otherwise by virtue of an office.}
\end{Highlighting}
\end{Shaded}

\hypertarget{upcoming-changes}{%
\subparagraph{Upcoming Changes}\label{upcoming-changes}}

BUT the law is set to change in this area. The Small Business,
Enterprise and Employment Act 2015, which received Royal Assent in March
2015, has inserted a new s 156A into the Companies Act 2006. This
provision requires every director to be a human and prohibits corporate
directors. A new CA 2006, s 156B allows the Secretary of State for BEIS
to make regulations setting out exceptions to this ban on corporate
directors.

\hypertarget{appointment-of-directors}{%
\paragraph{Appointment of Directors}\label{appointment-of-directors}}

Directors appointed in accordance with the process set out in the
articles.

\hypertarget{statutory-restrictions}{%
\subparagraph{Statutory Restrictions}\label{statutory-restrictions}}

\begin{enumerate}
\def\labelenumi{\arabic{enumi}.}
\tightlist
\item
  Director must be at least 16 when they take office (s 157)
\item
  A director disqualified under the CDDA 1986 will commit a criminal
  offence under s 13 CDDA if they act as a director of a company during
  the period of the disqualification.
\item
  An undischarged bankrupt cannot act as a director of a company (CDDA
  1986, s 11)
\item
  s 1184: SoS for BEIS has power to prevent someone from being a
  director of a UK company if they are subject to restrictions under
  foreign law.
\end{enumerate}

\hypertarget{restrictions-in-articles}{%
\subparagraph{Restrictions in Articles}\label{restrictions-in-articles}}

Articles may provide for the ineligibility of a director.

\begin{Shaded}
\begin{Highlighting}[]
\NormalTok{title: Art 18(d) MA}
\NormalTok{A person ceases to be a director as soon as a registered medical practitioner who is treating that person gives a written opinion to the company stating that that person has become physically or mentally incapable of acting as a director and may remain so for more than three months.}
\end{Highlighting}
\end{Shaded}

\hypertarget{procedure-for-appointment}{%
\subparagraph{Procedure for
Appointment}\label{procedure-for-appointment}}

CA 2006 does not stipulate a procedure for appointing directors, so this
is generally governed by the Articles of the company. MA deals with this
as follows:

\begin{Shaded}
\begin{Highlighting}[]
\NormalTok{title: MA 17}
\NormalTok{(1) Any person who is willing to act as a director, and is permitted by law to do so, may be appointed to be a director—}
\NormalTok{{-} (a) by ordinary resolution, or}
\NormalTok{{-} (b) by a decision of the directors.}
\end{Highlighting}
\end{Shaded}

Decision of directors is usually used since this is easier. But ALWAYS
check the Articles for any modifications.

Ordinary Resolution

\begin{Shaded}
\begin{Highlighting}[]
\NormalTok{1. Special notice of the resolution to appoint a new director must be given by the proposing shareholders, as required by s 168(2).}
\NormalTok{2. Ordinary resolution}
\NormalTok{3. Send to registrar a statement by the company that the person has consented to act in that capacity (s 167(2)(b)).}
\NormalTok{    {-} For the first director or directors of the company, this is set out on Form IN01 (see 3.8.1 above). }
\NormalTok{    {-} For directors appointed after incorporation, the company must sign the statement on Form AP01 (individual) or on Form AP02 (corporate)}
\NormalTok{4. As soon as reasonably practicable after the notification of appointment of a new director has been made, the Registrar of Companies must send information on the roles and duties of a director to that director (s 1079B)}
\NormalTok{5. Also if the director is an employee, need to satisfy service contract formalities. }
\end{Highlighting}
\end{Shaded}

MERMAID2

\hypertarget{number-of-directors}{%
\paragraph{Number of Directors}\label{number-of-directors}}

The MA 11 assumes that there are two directors, as this is the minimum
number required for valid board meetings to be held (known as the
`quorum'). If a company with model articles has just a sole director,
then that director can still validly take decisions because of MA 7(2).

\hypertarget{nature-of-office}{%
\subsection{Nature of Office}\label{nature-of-office}}

There is a multi-layered legal relationship between directors and their
company (it's complicated!)

\begin{longtable}[]{@{}
  >{\raggedright\arraybackslash}p{(\columnwidth - 2\tabcolsep) * \real{0.0698}}
  >{\raggedright\arraybackslash}p{(\columnwidth - 2\tabcolsep) * \real{0.9302}}@{}}
\toprule()
\begin{minipage}[b]{\linewidth}\raggedright
Relationship
\end{minipage} & \begin{minipage}[b]{\linewidth}\raggedright
Details
\end{minipage} \\
\midrule()
\endhead
Agency & Company is principal, directors are agents, given actual and
apparent authority to bind the company. \\
Fiduciary & Not trustees, but in a trustee-like position and owe special
duties to the company. So directors must show utmost good faith in their
dealings with the company. \\
Employment & Usually only executive directors, who have a service
contract and are paid a salary. Governed by employment law and contract
of employment. \\
\bottomrule()
\end{longtable}

\hypertarget{directors-powers}{%
\subsection{Directors' Powers}\label{directors-powers}}

\hypertarget{general}{%
\subsubsection{General}\label{general}}

Broad general authority, e.g.,

\begin{Shaded}
\begin{Highlighting}[]
\NormalTok{title: MA 3 {-} Directors\textquotesingle{} General Authority}
\NormalTok{Subject to the articles, the directors are responsible for the management of the company’s business, for which purpose they may exercise all the powers of the company.}
\end{Highlighting}
\end{Shaded}

\hypertarget{joint}{%
\paragraph{Joint}\label{joint}}

Directors generally act jointly. Can take decisions by a majority at a
board meeting (MA 7) or unanimously without a meeting being held (MA 8).

\hypertarget{shareholder-input}{%
\paragraph{Shareholder Input}\label{shareholder-input}}

There are certain decisions for which shareholders hold a veto, e.g.,
substantial property transaction between company and director, payment
of dividends under MA.

Shareholders have a reserve power:

\begin{Shaded}
\begin{Highlighting}[]
\NormalTok{title: MA 4}
\NormalTok{(1) The shareholders may, by special resolution, direct the directors to take, or refrain from taking, specified action.}

\NormalTok{(2) No such special resolution invalidates anything which the directors have done before the passing of the resolution.}
\end{Highlighting}
\end{Shaded}

\hypertarget{delegation-of-power}{%
\paragraph{Delegation of Power}\label{delegation-of-power}}

MA 5 gives extremely broad powers to delegate: can even delegate the
power to delegate. Managing director/ CEO may be given all the powers of
the board, but these may be varied/ withdrawn at any time.

\hypertarget{directors-authority}{%
\subsection{Directors' Authority}\label{directors-authority}}

The ability of a director to bind a company in a contractual
relationship with a third party is based on the director's position as
\textbf{agent} of the company.

MERMAID3

\begin{itemize}
\tightlist
\item
  Directors will bind the company if they act with either
  \textbf{actual} or \textbf{apparent} authority.
\item
  If they exceed this authority, they will not bind the company.

  \begin{itemize}
  \tightlist
  \item
    They will be personally liable for breach of warranty of authority
    to any third party with whom they were dealing.
  \item
    They will also be personally liable on the contract to the third
    party, as they would have failed to tie the company into the
    contractual obligation.
  \end{itemize}
\end{itemize}

\hypertarget{actual-authority}{%
\subsubsection{Actual Authority}\label{actual-authority}}

\begin{Shaded}
\begin{Highlighting}[]
\NormalTok{Actual authority is where the principal (the company) gives the agent (the director) specific prior consent to the agent’s actions.}
\end{Highlighting}
\end{Shaded}

\begin{quote}
An `actual' authority is a legal relationship between principal and
agent created by a consensual agreement to which they alone are the
parties. \ldots{} To this agreement the third party is a stranger; he
may be totally ignorant of the existence of any authority on the part of
the agent. Nevertheless, if the agent does enter into a contract
pursuant to the `actual' authority, it does create contractual rights
and liabilities between the principal and the third party.\\
{[}{[}Freeman and Lockyer v Buckhurst Park Properties (Mangal) Ltd
{[}1964{]} 2 QB 480{]}{]}
\end{quote}

May be express or implied.

\hypertarget{implied-actual-authority}{%
\paragraph{Implied Actual Authority}\label{implied-actual-authority}}

\begin{enumerate}
\def\labelenumi{\arabic{enumi}.}
\tightlist
\item
  From appointment to a specific role in a company

  \begin{itemize}
  \tightlist
  \item
    It is only the relationship between the principal and the agent
    which is relevant to determining whether implied actual authority
    exists
  \item
    e.g., appointment of a managing director carries with it the powers
    of that person to do all such acts necessary to manage the company,
    including those not expressly stated in their contract ({[}{[}Smith
    v Butler {[}2012{]} EWCA Civ 314{]}{]})
  \end{itemize}
\item
  From a course of dealing\\
  e.g., where director or agent continually enters into specific
  transactions and board of directors acquiesces or agrees to this
  ({[}{[}Hely-Hutchinson v Brayhead Ltd {[}1968{]} 1 QB 549{]}{]})
\end{enumerate}

\hypertarget{apparent-authority}{%
\subsubsection{Apparent Authority}\label{apparent-authority}}

\begin{Shaded}
\begin{Highlighting}[]
\NormalTok{Apparent authority is where the agent (the director) acts without the principal’s (the company) prior consent but still binds the principal (the company) in the contract with the third party. **The principal (the company) is estopped from denying the agent’s (the director) authority.**}
\end{Highlighting}
\end{Shaded}

\begin{quote}
An `apparent' authority, on the other hand, is a legal relationship
between the principal and the third party created by a representation,
made by the principal to the third party, intended to be and in fact
acted on by the third party, that the agent has authority \ldots{} To
the relationship so created, the agent is a stranger.\\
{[}{[}Freeman and Lockyer v Buckhurst Park Properties (Mangal) Ltd
{[}1964{]} 2 QB 480{]}{]}
\end{quote}

Apparent authority cannot arise by the agent's own actions, only by
those of the company. Without the estoppel, the agent would not bind
themselves, but would bind themselves alone to the third party.

\textbf{Ostensible or apparent authority arises where there has been
some holding out or representation on which a third party has relied
that the agent acting on the company's behalf had authority to enter
into the contract in question.}

\begin{Shaded}
\begin{Highlighting}[]
\NormalTok{If a company appoints A as a finance director, from the perspective of a third party, A almost certainly has apparent authority to enter into a contract for a purchase order, deriving from his position as finance director.}
\end{Highlighting}
\end{Shaded}

The agent has no actual authority but can still bind the principal.
Three cases:

\hypertarget{statutory-deemed-authority-s-40-ca-2006}{%
\paragraph{Statutory Deemed Authority (s 40 CA
2006)}\label{statutory-deemed-authority-s-40-ca-2006}}

\begin{itemize}
\tightlist
\item
  The purpose is to protect third parties where there are restrictions
  on the power of the company's agents to bind the company set out in
  the company's constitution
\item
  So the company cannot claim not to be bound by acts of its directors
  by asserting these are unconstitutional -- even when the company's
  articles require specific shareholder or board approval for a
  particular act
\end{itemize}

\begin{quote}
Power of directors to bind the company\\
(1) In favour of a person dealing with a company in good faith, the
power of the directors to bind the company, or authorise others to do
so, is deemed to be free of any limitation under the company's
constitution.\\
(2) For this purpose -\\
(a) a person ``deals with'' a company if he is a party to any
transaction or other act to which the company is a party,\\
(b) a person dealing with a company -\\
(i) is not bound to enquire as to any limitation on the powers of
directors to bind the company or authorise others to do so,\\
(ii) is presumed to have acted in good faith unless the contrary is
proved, and\\
(iii) is not to be regarded as acting in bad faith by reason only of his
knowing that an act is beyond the powers of the directors under the
company's constitution.\\
(3) The references above to limitations on the directors' powers under
the company's constitution include limitations deriving -\\
(a) from a resolution of the company or of any class of shareholders,
or\\
(b) from any agreement between the members of the company or of any
class of shareholders.
\end{quote}

Note, this protects third parties, not directors. Director can be sued
by company for losses/ disqualified.

\hypertarget{deemed-authority-at-common-law-ostensible-authority}{%
\paragraph{Deemed Authority at Common Law -- Ostensible
Authority}\label{deemed-authority-at-common-law-ostensible-authority}}

Determined by looking at the relationship between the principal and
third party. Refers to the authority of an agent as it appears to the
third party.

Upheld in {[}{[}Freeman and Lockyer v Buckhurst Park Properties (Mangal)
Ltd {[}1964{]} 2 QB 480{]}{]}.

An agent is said to have apparent or ostensible authority if:

\begin{enumerate}
\def\labelenumi{\arabic{enumi}.}
\tightlist
\item
  The principal has made a representation (by words or conduct) to the
  third party to the effect that the agent has the authority to act for
  him, although the agent does not in fact have such authority.
\item
  The third party has in fact relied on such representation to deal with
  the agent.
\item
  The third party has altered his position resulting from such reliance,
  for example, assuming obligations under a contract with the agent.
\end{enumerate}

\hypertarget{deemed-authority-at-common-law-indoor-management-rule}{%
\paragraph{Deemed Authority at Common Law -- `indoor Management'
Rule}\label{deemed-authority-at-common-law-indoor-management-rule}}

A rule of less significance due to s40 CA 2006 but still applies where
the third party has not dealt directly with the board or a question of
whether the agent was authorised by the board applies. Derived from
{[}{[}Royal British Bank v Turquand (1856) 6 E \& B 327{]}{]}: outsiders
are entitled to assume that a company's internal procedures have been
complied with.

But this does not apply where the third party has actual notice of the
irregularity or is not acting in good faith ({[}{[}Rolled Steel Ltd v
British Steel Corpn {[}1986{]} Ch 246{]}{]}) or when the third party is
an insider entering a contract with the company ({[}{[}Morris v Kanssen
{[}1946{]} AC 459{]}{]}).

Given s 40 CA 2006: the rules in Turquand's case now apply

\begin{itemize}
\tightlist
\item
  where there are potential questions over the execution of documents,
  the passing of authorising resolutions and the regularity of
  appointments, where nothing has occurred which would be contrary to
  company's constitution
\item
  where director appointed Managing Director, but relevant formalities
  of that appointment not complied with
\end{itemize}

\hypertarget{ratification-1}{%
\subsubsection{Ratification}\label{ratification-1}}

Company is able to ratify acts beyond the actual authority of its
agents. Involves passing a resolution to approve the act and agreeing
that the company will be bound to it.

e.g., {[}{[}New Falmouth Resorts Ltd v International Hotels Jamaica Ltd
{[}2013{]} UKPC 11{]}{]}

\hypertarget{summary}{%
\subsubsection{Summary}\label{summary}}

MERMAID4

LPC purposes:

\begin{enumerate}
\def\labelenumi{\arabic{enumi}.}
\tightlist
\item
  The courts will allow apparent authority arguments where a plausible
  representative of the company had apparent authority to third parties,
  in the absence of information from the company to correct this
  impression.
\item
  But an improper agreement entered into with a single director/
  representative where it was implausible that the representative had
  authority to carry out the transaction will be unenforceable against
  the company ({[}{[}Criterion Properties v Stratford UK Properties LLC
  {[}2004{]} UKHL 28,{]}{]}).
\end{enumerate}

\hypertarget{directors-annual-responsibilities}{%
\subsection{Directors' Annual
Responsibilities}\label{directors-annual-responsibilities}}

\hypertarget{company-accounts}{%
\subsubsection{Company Accounts}\label{company-accounts}}

\begin{itemize}
\tightlist
\item
  A company must keep adequate accounting records under s 386(1) of the
  CA 2006, otherwise it commits an offence.
\item
  It is the directors' responsibility to ensure that full accounts are
  produced for each financial year (CA 2006, s 394).
\item
  These accounts must give a `true and fair view of the state of affairs
  of the company as at the end of the financial year' (CA 2006, s
  396(2)).
\item
  The directors must not approve the accounts unless they are satisfied
  that they give a true and fair view of the assets, liabilities,
  financial position, and profit and loss of the company (CA 2006, s
  393(1)).
\end{itemize}

\hypertarget{directors-report}{%
\paragraph{Directors' Report}\label{directors-report}}

\begin{itemize}
\tightlist
\item
  By s 415, every company must prepare a directors' report for each
  financial year to accompany the accounts.
\item
  s 417 requires the directors' report to include a business review
  including details on development, performance, risks, and position of
  the company.

  \begin{itemize}
  \tightlist
  \item
    This does not apply to `small' companies.
  \end{itemize}
\end{itemize}

\hypertarget{filing}{%
\paragraph{Filing}\label{filing}}

Under s 441, companies must also file the accounts and directors' report
for each financial year at Companies House.

However, so-called `micro-entities', `small' and `medium-sized'
companies may file an abbreviated version of the year-end accounts
(Small Companies (Micro-Entities' Accounts) Regulations 2013 (SI
2013/3008), CA 2006, s 444 and s 445 respectively).

The time limit for filing accounts is nine months from the end of the
accounting reference period for a private company (CA 2006, s 442(2)).

\hypertarget{company-sizes}{%
\paragraph{Company Sizes}\label{company-sizes}}

To satisfy the definition, the company must fulfil any two of the
provisions in a given row.

\begin{longtable}[]{@{}
  >{\raggedright\arraybackslash}p{(\columnwidth - 6\tabcolsep) * \real{0.3205}}
  >{\raggedright\arraybackslash}p{(\columnwidth - 6\tabcolsep) * \real{0.1923}}
  >{\raggedright\arraybackslash}p{(\columnwidth - 6\tabcolsep) * \real{0.2436}}
  >{\raggedright\arraybackslash}p{(\columnwidth - 6\tabcolsep) * \real{0.2436}}@{}}
\toprule()
\begin{minipage}[b]{\linewidth}\raggedright
Company
\end{minipage} & \begin{minipage}[b]{\linewidth}\raggedright
Annual turnover
\end{minipage} & \begin{minipage}[b]{\linewidth}\raggedright
Balance sheet total
\end{minipage} & \begin{minipage}[b]{\linewidth}\raggedright
Number of employees
\end{minipage} \\
\midrule()
\endhead
Micro-entity (s 384A) & \(\leq £632k\) & \(\leq £316k\) & \(\leq 10\) \\
Small company (s 382(3)) & \(\leq £10.2m\) & \(\leq £5.1m\) &
\(\leq 50\) \\
Medium company (s 465(3)) & \(\leq £36m\) & \(\leq £18m\) &
\(\leq 250\) \\
\bottomrule()
\end{longtable}

\hypertarget{confirmation-statement}{%
\paragraph{Confirmation Statement}\label{confirmation-statement}}

\begin{itemize}
\tightlist
\item
  Every company must submit a confirmation statement to the Registrar of
  Companies once in every 12-month period.
\item
  The directors are responsible for doing this within 14 days after the
  company's `confirmation date' (the date to which the confirmation
  statement is made up) (s 853A(1) and (3)).
\item
  A company's first confirmation date is the anniversary of the date of
  its incorporation. Subsequently, it is the anniversary of the previous
  confirmation date (s 853A(5)).
\item
  A failure to submit the confirmation statement on time is a criminal
  offence.
\end{itemize}

\hypertarget{dividends}{%
\paragraph{Dividends}\label{dividends}}

The directors in managing the company have the power to recommend
payment of a dividend (a share of profits) to the company's shareholders
if there are sufficient distributable profits available.

\hypertarget{service-contracts-and-removal}{%
\section{Service Contracts and
Removal}\label{service-contracts-and-removal}}

\hypertarget{directors-service-contracts}{%
\subsection{Directors' Service
Contracts}\label{directors-service-contracts}}

An executive director is an employee of the company, so should be given
a written employment contract --- a service contract. No automatic
entitlement for directors to be paid for services.

\begin{itemize}
\tightlist
\item
  Company must keep copy of directors' service contracts/ memoranda of
  the terms
  (\href{https://www.legislation.gov.uk/ukpga/2006/46/section/228}{s 228
  CA 2006}).
\item
  Shareholders have right to inspect copies of directors' service
  contracts/ memoranda
  (\href{https://www.legislation.gov.uk/ukpga/2006/46/section/229}{s 229
  CA 2006}), must be provided within 7 days of request.
\end{itemize}

\hypertarget{long-term-service-contracts}{%
\subsubsection{Long Term Service
Contracts}\label{long-term-service-contracts}}

\begin{itemize}
\tightlist
\item
  \href{https://www.gov.uk/government/publications/model-articles-for-private-companies-limited-by-shares/model-articles-for-private-companies-limited-by-shares\#remuneration}{Art
  19 MA} means that terms of an individual director's service contract
  are for the board to determine. Only requires approval of resolution
  of board of directors, though shareholder approval required
  (\href{https://www.legislation.gov.uk/ukpga/2006/46/section/188}{s 188
  CA 2006}) to enter long-term service contracts (where service contract
  has guaranteed term which is or may be \textgreater{} 2 years).
\item
  If shareholder approval not given, term incorporated into service
  agreement \textbf{void} under s 189(a) CA 2006. Additionally, under s
  189(b) CA 2006, the service contract will be deemed to contain a term
  entitling company to terminate the contract at any time, by giving
  reasonable notice.
\end{itemize}

\hypertarget{awarded-by-board}{%
\subsubsection{Awarded by Board}\label{awarded-by-board}}

The board has authority to enter service contracts with directors (e.g.,
MA 3 \& 19).

\begin{Shaded}
\begin{Highlighting}[]
\NormalTok{title: Does the director need to declare a conflict of interest when their service contract is discussed?}
\NormalTok{No, a formal declaration to the board of a personal interest will not be necessary due to s 177(6).}
\end{Highlighting}
\end{Shaded}

\begin{itemize}
\tightlist
\item
  The director will often be prevented under the articles from voting
  and counting in the quorum for that board meeting for reasons of
  fairness.
\item
  But this can create problems if there are only two directors, and the
  quorum for board meetings is two.
\item
  Potential solution: change articles by special resolution (s 21(1)).
\item
  More robust solution: pass an ordinary resolution at GM to temporarily
  relax the rules on directors voting and counting in the quorum.

  \begin{itemize}
  \tightlist
  \item
    Permitted under MA 14(3).
  \end{itemize}
\end{itemize}

\hypertarget{guaranteed-term-contracts}{%
\subsubsection{Guaranteed-term
Contracts}\label{guaranteed-term-contracts}}

Very advantageous for directors, but financially risky for the company.
So shareholders are protected by CA 2006: allowed to veto certain
guaranteed-term contracts between the company (negotiated by the board)
and the individual directors.

\hypertarget{procedure-1}{%
\paragraph{Procedure}\label{procedure-1}}

\begin{itemize}
\tightlist
\item
  Proposed service contract with a guaranteed term \(>2\) years:
  ordinary resolution needed (s 188).
\item
  If GM called to pass such a resolution, a memorandum setting out the
  proposed service contract in question must be available for inspection
  by members at GM and for 15 days prior to GM at the registered office.
\item
  If a written resolution of the shareholders is to be used, the
  memorandum of the proposed service contracts must be sent out to
  members with the written resolution itself (s 188(5)).
\end{itemize}

\hypertarget{if-not-approved}{%
\subsubsection{If Not Approved}\label{if-not-approved}}

If shareholder approval not given, term incorporated into service
agreement~\textbf{void}~under s 189(a) CA 2006. Additionally, under s
189(b) CA 2006, the service contract will be deemed to contain a term
entitling company to terminate the contract at any time, by giving
reasonable notice.

\hypertarget{personal-services}{%
\subsubsection{Personal Services}\label{personal-services}}

A service contract is defined in s 227(1) to include a contract under
which the director undertakes to perform services personally for the
company. Under s 227(2), this may be within or outside the scope of the
ordinary duties of a director.

\hypertarget{inspection-1}{%
\subsubsection{Inspection}\label{inspection-1}}

\begin{itemize}
\tightlist
\item
  s 228: copies of all directors' service contracts (or a written
  memorandum if the contract is not in writing) must be kept at the
  registered office - for at least 1 year after termination of contract,
  or its single alternative inspection location (SAIL) under s 1136.
\item
  Must be open for inspection by the company's shareholders (s 229).
\item
  This applies to all directors' service contracts.
\end{itemize}

\hypertarget{notification-requirements}{%
\subsection{Notification Requirements}\label{notification-requirements}}

\hypertarget{register-of-directors}{%
\subsubsection{Register of Directors}\label{register-of-directors}}

\begin{Shaded}
\begin{Highlighting}[]
\NormalTok{title: s 162 {-} Register of directors}

\NormalTok{(1) Every company must keep a register of its directors.}

\NormalTok{(2) The register must contain the required particulars (see sections 163, 164 and 166) of each person who is a director of the company.}

\NormalTok{(3) The register must be kept available for inspection—}
\NormalTok{{-} (a) at the company\textquotesingle{}s registered office, or}
\NormalTok{{-} (b) at a place specified in regulations under section 1136.}

\NormalTok{(4) The company must give notice to the registrar—}
\NormalTok{{-} (a) of the place at which the register is kept available for inspection, and}
\NormalTok{{-} (b) of any change in that place,}

\NormalTok{unless it has at all times been kept at the company\textquotesingle{}s registered office.}

\NormalTok{(5) The register must be open to the inspection—}
\NormalTok{{-} (a) of any member of the company without charge, and}
\NormalTok{{-} (b) of any other person on payment of such fee as may be prescribed.}

\NormalTok{(6) If default is made in complying with subsection (1), (2) or (3) or if default is made for 14 days in complying with subsection (4), or if an inspection required under subsection (5) is refused, an offence is committed by—}
\NormalTok{{-} (a) the company, and}
\NormalTok{{-} (b) every officer of the company who is in default.}

\NormalTok{For this purpose a shadow director is treated as an officer of the company. }
\end{Highlighting}
\end{Shaded}

There are different particulars to be entered depending on whether the
director is a human or corporate director -- see ss 163 \& 164.

\hypertarget{register-of-directors-residential-addresses}{%
\subsubsection{Register of Directors' Residential
Addresses}\label{register-of-directors-residential-addresses}}

\begin{itemize}
\tightlist
\item
  Every company must also keep a register of (human) directors'
  residential addresses (s 165(1)).
\item
  s 167A: private companies can elect not to keep it themselves, and
  just keep up-to-date information at Companies House.
\item
  The register of directors' residential addresses obviously not open to
  inspection.
\item
  Failure to keep a register of directors' residential addresses is a
  criminal offence committed by\\
  the company and every officer in default (CA 2006, s 165(4)).
\end{itemize}

\hypertarget{companies-house}{%
\subsubsection{Companies House}\label{companies-house}}

\begin{longtable}[]{@{}
  >{\raggedright\arraybackslash}p{(\columnwidth - 6\tabcolsep) * \real{0.4045}}
  >{\raggedright\arraybackslash}p{(\columnwidth - 6\tabcolsep) * \real{0.1348}}
  >{\raggedright\arraybackslash}p{(\columnwidth - 6\tabcolsep) * \real{0.1798}}
  >{\raggedright\arraybackslash}p{(\columnwidth - 6\tabcolsep) * \real{0.2809}}@{}}
\toprule()
\begin{minipage}[b]{\linewidth}\raggedright
Event to file
\end{minipage} & \begin{minipage}[b]{\linewidth}\raggedright
Form (human)
\end{minipage} & \begin{minipage}[b]{\linewidth}\raggedright
Form (corporate)
\end{minipage} & \begin{minipage}[b]{\linewidth}\raggedright
Time limit
\end{minipage} \\
\midrule()
\endhead
First director(s) of company & IN01 & IN01 & Upon company
registration \\
Subsequently appointment of director & AP01 & AP02 & 14 days after
appointment \\
Change to particulars of director & CH01 & CH02 & 14 days after
change \\
Director leaving office & TM01 & TM02 & 14 days after leaving \\
\bottomrule()
\end{longtable}

\hypertarget{stationery}{%
\subsubsection{Stationery (!)}\label{stationery}}

All company business letters must contain the names of either all the
directors or none of the director (except as signatory/ in the text) --
reg 26 Company, Limited Liability Partnership and Business (Names and
Trading Disclosures) Regulations 2015 (SI 2015/17).

\hypertarget{termination-of-directorship}{%
\subsection{Termination of
Directorship}\label{termination-of-directorship}}

There are multiple ways in which a director can leave office and cease
to be a director.

\hypertarget{resignation}{%
\subsubsection{Resignation}\label{resignation}}

A director may resign at any time by giving notice to the company (MA
18(f)). If subject to an employment contract, the director must take
account of any notice periods/ other procedures to avoid breach of
contract liability.

\hypertarget{removal-by-board}{%
\subsubsection{Removal by Board}\label{removal-by-board}}

The articles may give the board power to dismiss a director by majority
vote. When doing so, directors must act bona fide in the best interests
of the company.

\hypertarget{removal-by-shareholders-under-ca-2006}{%
\subsubsection{Removal by Shareholders Under CA
2006}\label{removal-by-shareholders-under-ca-2006}}

\begin{itemize}
\tightlist
\item
  Directors can be removed by ordinary resolution (s 168(1)) and this
  power cannot be taken away by the service contract/ articles.
\item
  Any shareholder wanting to propose a resolution to remove a director
  must give the company special notice (s 168(2)).

  \begin{itemize}
  \tightlist
  \item
    s 312(1): special notice means giving formal notice to the company
    at its registered office of their intention to propose the
    resolution at least 28 clear days before a GM.
  \item
    If a replacement director is to be appointed, special notice of the
    appointment must also be given.
  \end{itemize}
\item
  A written resolution cannot be used (s 288(2)(a)).
\end{itemize}

\hypertarget{cooperative-board}{%
\paragraph{Cooperative Board}\label{cooperative-board}}

A cooperative board can respond to the s 169 notice by calling a GM in
the usual way (s 312(2)). Only 14 clear days' notice of the meeting is
required from the company to the shareholders. The fact it is held
within 28 clear days of the notice being served does not prevent the
shareholder's special notice being valid. Shareholders are deemed to
have given proper notice under s 312(4).

Where a GM has already been called and shareholders serve the s 168
notice to have the ordinary resolution considered at the GM, directors
can agree to add the resolution to the meeting agenda, but only if they
have time to give \textbf{\(\geq 14\) clear days' notice} of the meeting
by an advertisement in a newspaper/ other means allowed in the articles
(s 312(3)).

\hypertarget{directors-rights}{%
\paragraph{Director's Rights}\label{directors-rights}}

When s 168 special notice is received, the board must send a copy to the
director concerned immediately. The director then has the right to make
written representations to the company, which the company must circulate
to shareholders (s 169(3)). Director may speak at the meeting (s
169(2)).

\hypertarget{bushell-v-faith}{%
\paragraph{Bushell V Faith}\label{bushell-v-faith}}

Possible for company to insert a `Bushell v Faith' clause into its
articles: see {[}{[}Bushell v Faith {[}1970{]} AC 1099 (House of
Lords){]}{]}. These give weighted voting rights, allowing director/
shareholders to block such resolutions.

Might seem contrary to s 168 CA 2006. But such weighted voting clauses
allowed because requirement for ordinary resolution not changed, just
that way votes are amassed make it easier for director/ shareholder to
survive. They represent an internal agreement amongst shareholders as to
the weight their vote carries on a specific resolution.

\begin{Shaded}
\begin{Highlighting}[]
\NormalTok{Check whether it is possible to remove the Bushell v Faith clause by passing a special resolution under s 21 CA 2006 at a GM, or whether the director has weighted voting rights for this. }
\end{Highlighting}
\end{Shaded}

Also check directors' service contract -- can be expensive to get rid of
the director.

\hypertarget{effect-of-removal}{%
\paragraph{Effect of Removal}\label{effect-of-removal}}

\begin{itemize}
\tightlist
\item
  Director automatically loses any executive role, if the executive role
  is dependent on being a director.
\item
  Retains any accrued employment rights, including those under the
  service contract.
\item
  Service contract not rendered invalid by removal as a director.

  \begin{itemize}
  \tightlist
  \item
    Ex-director may be entitled to claim against the company for
    wrongful dismissal.
  \item
    May also have a claim for unfair dismissal or redundancy
  \item
    If termination as a director occurs in accordance with the terms of
    the service contract, unlikely to be unfair dismissal ({[}{[}Cobley
    v Forward Technology Industries plc {[}2003{]} EWCA Civ 646{]}{]})
  \item
    Non-executive director does not usually have a claim for breach of
    employment rights
  \end{itemize}
\item
  Admin: update register of directors and register of directors'
  residential addresses.
\end{itemize}

\hypertarget{removal-by-shareholders-under-articles}{%
\subsubsection{Removal by Shareholders Under
Articles}\label{removal-by-shareholders-under-articles}}

Under Table A, art 73, directors must retire by rotation.
\(\frac{1}{3}\) of directors must retire from office and be subject to
re-election each AGM. The requirement to retire by rotation is included
in MA Public Companies.

\hypertarget{protecting-against-dismissal}{%
\subsubsection{Protecting Against
Dismissal}\label{protecting-against-dismissal}}

Possibilities include:

\begin{enumerate}
\def\labelenumi{\arabic{enumi}.}
\tightlist
\item
  Bushell v Faith clause
\item
  Fixed-term service contract of long duration and without a break
  clause (subject to shareholder approval for fixed term contract \(>2\)
  years).

  \begin{itemize}
  \tightlist
  \item
    Claims for wrongful dismissal and unfair dismissal can make it more
    expensive to remove a director.
  \end{itemize}
\item
  Shareholders' agreement, where shareholders agree not to vote against
  specific directors on a motion to dismiss.
\item
  If the director makes a loan to the company, express it to be
  repayable if the director loses their position.
\end{enumerate}

\hypertarget{employment-and-regulatory-concerns}{%
\subsection{Employment and Regulatory
Concerns}\label{employment-and-regulatory-concerns}}

Common law: an employer is free to offer employment to whoever they
choose ({[}{[}Allen v Flood and Taylor {[}1898{]} AC 1{]}{]}). This is
restricted by statute; the employer may not discriminate against a
person on obvious grounds (race, sex, religion, sexual orientation,
disability etc.).

s 39 Equality Act 2010: unlawful to discriminate in recruitment,
employment terms, opportunities for promotion and training, dismissal
etc.

Types of discrimination:

\begin{enumerate}
\def\labelenumi{\arabic{enumi}.}
\tightlist
\item
  Direct discrimination (s 13)
\item
  Indirect discrimination (s 19)
\item
  Harassment (s 26)
\item
  Victimisation (s 27).
\end{enumerate}

\hypertarget{written-statement-of-terms}{%
\subsubsection{Written Statement of
Terms}\label{written-statement-of-terms}}

s 1 Employment Rights Act 1996: the employer must, within 2 months of
employment, give the employee a written statement of terms and
conditions. This should include:

\begin{itemize}
\tightlist
\item
  Identity of parties
\item
  Commencement date
\item
  Remuneration details
\item
  Hours of work
\item
  Holiday, sick pay, pensions
\item
  Notice required
\item
  Job description, etc.
\end{itemize}

\hypertarget{employee-obligations}{%
\subsubsection{Employee Obligations}\label{employee-obligations}}

Statutory obligations to employees include

\begin{itemize}
\tightlist
\item
  Allowing employees time off work

  \begin{itemize}
  \tightlist
  \item
    Ante-natal care
  \item
    Trade union duties
  \item
    Public duties
  \item
    Maternity, paternity, adoption and parental leave
  \item
    Caring for dependents
  \end{itemize}
\item
  Allowing employees to sometimes request flexible working
\item
  Allowing an employee to return after maternity leave
\item
  Reasonable care of employees' health and safety at work.
\end{itemize}

\hypertarget{informing-and-consulting}{%
\subsubsection{Informing and
Consulting}\label{informing-and-consulting}}

Information and Consultation of Employees Regulations 2004 (SI
2004/3426): employees of companies with \(\geq 50\) employees have the
right to be informed and consulted about certain key decisions.

\hypertarget{dismissal-of-employees}{%
\subsubsection{Dismissal of Employees}\label{dismissal-of-employees}}

Potential claims include the common law claim for wrongful dismissal and
statutory claims of unfair dismissal and claim for a statutory
redundancy payment.

\hypertarget{wrongful-dismissal}{%
\subsubsection{Wrongful Dismissal}\label{wrongful-dismissal}}

Likely to apply if the employer terminates a contract for an indefinite
term with no notice/ inadequate notice, or if a fixed-term contract is
terminated before its expiry date.

Most employment contracts are for an indefinite term and terminable, by
either side giving correct contractual notice.

In a fixed-term contract, the contract is not usually terminable by
notice.

\hypertarget{notice-period}{%
\paragraph{Notice Period}\label{notice-period}}

\begin{itemize}
\tightlist
\item
  The applicable notice period will usually be expressly agreed in the
  contract.
\item
  If an expressly agreed notice period is shorter than the statutory
  minimum required by \textbf{s 86 ERA 1996}, the longer statutory
  period must be given.
\item
  If there is no express provision, there is an implied term that the
  employee is entitled to reasonable notice
\item
  For more senior employees, a longer period will be implied.
\end{itemize}

For time of continuous employment \(T\), the statutory minimum notice
period is \(P\) weeks, where:

\[P = \begin{cases}  
1 , & \text{if } 1 \text{ month} \leq T < 2 \text{ years}\\  
T , & \text{if } 2 \text{ years} \leq T < 13 \text{ years},\\  
12, & \text{if } T \geq 13 \text{ years}  
\end{cases}  
\]

A claim for wrongful dismissal requires a dismissal in breach of
contract.

Where an employee resigns, they will have no claim. If they resign
without sufficient notice/ terminate before the expiry date of a
fixed-term contract, the employee will be in breach.

\hypertarget{repudiatory-breach}{%
\paragraph{Repudiatory Breach}\label{repudiatory-breach}}

If the employer has committed a repudiatory breach of contract, the
employee is entitled to treat the contract as discharged. They can leave
with or without notice and bring a claim for wrongful dismissal, since
they have been ``constructively dismissed'' by breach of contract.

The employee must leave within a reasonable period of the employer's
breach, otherwise they will be deemed to have affirmed the contract.

{[}\protect\hyperlink{repudiatory-breach}{Repudiatory breach}{]} can be
caused by:

\begin{itemize}
\tightlist
\item
  Unilaterally altering the employee's contract
\item
  Breaching an implied duty of good faith towards employees

  \begin{itemize}
  \tightlist
  \item
    e.g., imposing unreasonable work demands on the employee, publicly
    humiliating them.
  \end{itemize}
\item
  Employee revealing confidential information/ wilfully disobeying
  orders.

  \begin{itemize}
  \tightlist
  \item
    So it is a defence available to an employer in a wrongful dismissal
    claim that the employee committed a repudiatory breach of an express
    or implied term of the contract
  \item
    Defence available even if the employer did not know of the
    employee's breach at the time of termination of the contract.
  \end{itemize}
\end{itemize}

\hypertarget{damages}{%
\paragraph{Damages}\label{damages}}

\begin{itemize}
\tightlist
\item
  Normal contractual rules apply.
\item
  The aim is to put the employee in \textbf{the position they would have
  been in}, so far as money can do this, had the contract not been
  broken.
\item
  Starting point is the salary/ wages which would have been earned
  during the proper notice period/ remainder of the fixed term.
\item
  Damages for other fringe benefits can also be claimed (e.g., pension
  rights, use of a company car).
\item
  Damages will not be awarded for loss of future prospects.
\item
  The employee is under a duty to mitigate their loss (by applying for
  other jobs).
\item
  Claim can be brought in the High Court/ County Court/ Employment
  Tribunal.
\end{itemize}

\hypertarget{unfair-dismissal}{%
\subsubsection{Unfair Dismissal}\label{unfair-dismissal}}

\begin{itemize}
\tightlist
\item
  s 94 ERA 1996: an employee has the right not to be unfairly dismissed.
\item
  Claim before employment tribunal
\item
  Employee must prove

  \begin{itemize}
  \tightlist
  \item
    They are a ``qualifying employee'' (2 years' continuous employment
    ending with the effective date of termination). - s 108 ERA 1996
  \item
    They have been dismissed

    \begin{itemize}
    \tightlist
    \item
      Includes actual and constructive dismissal.
    \end{itemize}
  \end{itemize}
\item
  Burden of proof moves to the employer

  \begin{itemize}
  \tightlist
  \item
    must show that the reason for the dismissal was one of 5 permitted
    reasons:

    \begin{itemize}
    \tightlist
    \item
      Capability or qualifications of the employee for doing work of the
      kind employed to do
    \item
      Conduct of the employee
    \item
      Employee was redundant
    \item
      Employee could not continue to work without contravening a
      statutory enactment (e.g., bus driver losing driving licence)
    \item
      Some other substantial reason (e.g., personality clash between
      employees).
    \end{itemize}
  \end{itemize}
\item
  s 98(4) ERA 1996: if employer demonstrates a fair reason, the tribunal
  must decide whether in the circumstances, the employer acted
  reasonably.
\item
  Any procedural defects will be considered

  \begin{itemize}
  \tightlist
  \item
    In capability cases, the employer should have warned the employee
    about their standard of work.
  \item
    In conduct cases, the employer should allow the employee to state
    their case.
  \end{itemize}
\end{itemize}

\hypertarget{remedies}{%
\paragraph{Remedies}\label{remedies}}

\begin{itemize}
\tightlist
\item
  Reinstatement

  \begin{itemize}
  \tightlist
  \item
    Being given same job back
  \end{itemize}
\item
  Re-engagement

  \begin{itemize}
  \tightlist
  \item
    Being given another comparable/ suitable job with the same/ an
    associated employer.
  \end{itemize}
\item
  Compensation

  \begin{itemize}
  \tightlist
  \item
    Basic award

    \begin{itemize}
    \tightlist
    \item
      Calculated by reference to a statutory formula, including age, pay
      and length of service.
    \end{itemize}
  \item
    Compensatory award

    \begin{itemize}
    \tightlist
    \item
      A further amount as the tribunal considers just and equitable,
      considering loss of immediate and future wages etc.
    \item
      Maximum of £93,878, excl. fringe benefits.
    \item
      Adjusted if either party unreasonably failed to follow ACAS
      recommended code of discipline and grievance.
    \end{itemize}
  \end{itemize}
\end{itemize}

\hypertarget{redundancy}{%
\subsubsection{Redundancy}\label{redundancy}}

If the employer does not pay a redundancy payment, or the employee
disputes the calculation, the employee may refer the matter to an
employment tribunal (within a 6 month time limit).

\hypertarget{conditions}{%
\paragraph{Conditions}\label{conditions}}

To claim a statutory redundancy payment, an employee must prove:

\begin{enumerate}
\def\labelenumi{\arabic{enumi}.}
\tightlist
\item
  Dismissal -- actually, constructively or by failure to renew a
  fixed-term contract
\item
  2 years' continuous employment
\end{enumerate}

This raises a presumption that they have been dismissed for redundancy.
An employee may be show a reason other than redundancy, but then this
may lead to a claim for unfair dismissal.

\begin{Shaded}
\begin{Highlighting}[]
\NormalTok{title: Redundancy}
\NormalTok{s 139 ERA 1996: redundancy involves either:}
\NormalTok{1. Complete closedown of the business}
\NormalTok{2. Partial closedown of the business}
\NormalTok{3. Overmanning or a change in the type of work undertaken.}
\end{Highlighting}
\end{Shaded}

An employee may lose their entitlement if they unreasonably refuse an
offer of suitable alternative employment.

\hypertarget{overlapping-claims}{%
\subsubsection{Overlapping Claims}\label{overlapping-claims}}

Multiple claims may be made, depending on the circumstances. If multiple
claims against the employer are successful: compensation should not be
awarded for the same loss twice.

\hypertarget{discriminatory-dismissals}{%
\paragraph{Discriminatory Dismissals}\label{discriminatory-dismissals}}

If an employer unlawfully discriminates when dismissing an employee, the
employee can claim uncapped compensation, which can include compensation
for injured feelings.

\hypertarget{settlement-agreements}{%
\subsubsection{Settlement Agreements}\label{settlement-agreements}}

Many complaints are agreed, usually because the employer pays the
employee a sum of money.

\begin{longtable}[]{@{}
  >{\raggedright\arraybackslash}p{(\columnwidth - 2\tabcolsep) * \real{0.1000}}
  >{\raggedright\arraybackslash}p{(\columnwidth - 2\tabcolsep) * \real{0.9000}}@{}}
\toprule()
\begin{minipage}[b]{\linewidth}\raggedright
Statute
\end{minipage} & \begin{minipage}[b]{\linewidth}\raggedright
Details
\end{minipage} \\
\midrule()
\endhead
s 203 ERA 1996 & Any provision in an agreement is void so far as it
seeks to exclude or limit ERA 1996, or to stop someone bringing
proceedings \\
s 18 ERA 1996 & Someone can be stopped from bringing proceedings if a
settlement agreement has been entered into. \\
\bottomrule()
\end{longtable}

For a settlement to be binding:

\begin{enumerate}
\def\labelenumi{\arabic{enumi}.}
\tightlist
\item
  In writing, identify adviser, relate to complaint and state the
  relevant statutory conditions are satisfied.
\item
  Employee/ worker must have received advice from a relevant independent
  adviser as to the terms and effects of the proposed agreement.
\item
  Must be a contract of insurance or an indemnity provided.
\end{enumerate}

s 203 ERA 1996: must relate only to the matter in dispute and cannot
purport to exclude all possible claims.

\end{document}
