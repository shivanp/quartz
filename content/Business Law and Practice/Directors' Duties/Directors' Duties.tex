% Options for packages loaded elsewhere
\PassOptionsToPackage{unicode}{hyperref}
\PassOptionsToPackage{hyphens}{url}
%
\documentclass[
]{article}
\usepackage{amsmath,amssymb}
\usepackage{lmodern}
\usepackage{iftex}
\ifPDFTeX
  \usepackage[T1]{fontenc}
  \usepackage[utf8]{inputenc}
  \usepackage{textcomp} % provide euro and other symbols
\else % if luatex or xetex
  \usepackage{unicode-math}
  \defaultfontfeatures{Scale=MatchLowercase}
  \defaultfontfeatures[\rmfamily]{Ligatures=TeX,Scale=1}
\fi
% Use upquote if available, for straight quotes in verbatim environments
\IfFileExists{upquote.sty}{\usepackage{upquote}}{}
\IfFileExists{microtype.sty}{% use microtype if available
  \usepackage[]{microtype}
  \UseMicrotypeSet[protrusion]{basicmath} % disable protrusion for tt fonts
}{}
\makeatletter
\@ifundefined{KOMAClassName}{% if non-KOMA class
  \IfFileExists{parskip.sty}{%
    \usepackage{parskip}
  }{% else
    \setlength{\parindent}{0pt}
    \setlength{\parskip}{6pt plus 2pt minus 1pt}}
}{% if KOMA class
  \KOMAoptions{parskip=half}}
\makeatother
\usepackage{xcolor}
\usepackage[margin=1in]{geometry}
\usepackage{longtable,booktabs,array}
\usepackage{calc} % for calculating minipage widths
% Correct order of tables after \paragraph or \subparagraph
\usepackage{etoolbox}
\makeatletter
\patchcmd\longtable{\par}{\if@noskipsec\mbox{}\fi\par}{}{}
\makeatother
% Allow footnotes in longtable head/foot
\IfFileExists{footnotehyper.sty}{\usepackage{footnotehyper}}{\usepackage{footnote}}
\makesavenoteenv{longtable}
\setlength{\emergencystretch}{3em} % prevent overfull lines
\providecommand{\tightlist}{%
  \setlength{\itemsep}{0pt}\setlength{\parskip}{0pt}}
\setcounter{secnumdepth}{-\maxdimen} % remove section numbering
\usepackage{xcolor}
\definecolor{aliceblue}{HTML}{F0F8FF}
\definecolor{antiquewhite}{HTML}{FAEBD7}
\definecolor{aqua}{HTML}{00FFFF}
\definecolor{aquamarine}{HTML}{7FFFD4}
\definecolor{azure}{HTML}{F0FFFF}
\definecolor{beige}{HTML}{F5F5DC}
\definecolor{bisque}{HTML}{FFE4C4}
\definecolor{black}{HTML}{000000}
\definecolor{blanchedalmond}{HTML}{FFEBCD}
\definecolor{blue}{HTML}{0000FF}
\definecolor{blueviolet}{HTML}{8A2BE2}
\definecolor{brown}{HTML}{A52A2A}
\definecolor{burlywood}{HTML}{DEB887}
\definecolor{cadetblue}{HTML}{5F9EA0}
\definecolor{chartreuse}{HTML}{7FFF00}
\definecolor{chocolate}{HTML}{D2691E}
\definecolor{coral}{HTML}{FF7F50}
\definecolor{cornflowerblue}{HTML}{6495ED}
\definecolor{cornsilk}{HTML}{FFF8DC}
\definecolor{crimson}{HTML}{DC143C}
\definecolor{cyan}{HTML}{00FFFF}
\definecolor{darkblue}{HTML}{00008B}
\definecolor{darkcyan}{HTML}{008B8B}
\definecolor{darkgoldenrod}{HTML}{B8860B}
\definecolor{darkgray}{HTML}{A9A9A9}
\definecolor{darkgreen}{HTML}{006400}
\definecolor{darkgrey}{HTML}{A9A9A9}
\definecolor{darkkhaki}{HTML}{BDB76B}
\definecolor{darkmagenta}{HTML}{8B008B}
\definecolor{darkolivegreen}{HTML}{556B2F}
\definecolor{darkorange}{HTML}{FF8C00}
\definecolor{darkorchid}{HTML}{9932CC}
\definecolor{darkred}{HTML}{8B0000}
\definecolor{darksalmon}{HTML}{E9967A}
\definecolor{darkseagreen}{HTML}{8FBC8F}
\definecolor{darkslateblue}{HTML}{483D8B}
\definecolor{darkslategray}{HTML}{2F4F4F}
\definecolor{darkslategrey}{HTML}{2F4F4F}
\definecolor{darkturquoise}{HTML}{00CED1}
\definecolor{darkviolet}{HTML}{9400D3}
\definecolor{deeppink}{HTML}{FF1493}
\definecolor{deepskyblue}{HTML}{00BFFF}
\definecolor{dimgray}{HTML}{696969}
\definecolor{dimgrey}{HTML}{696969}
\definecolor{dodgerblue}{HTML}{1E90FF}
\definecolor{firebrick}{HTML}{B22222}
\definecolor{floralwhite}{HTML}{FFFAF0}
\definecolor{forestgreen}{HTML}{228B22}
\definecolor{fuchsia}{HTML}{FF00FF}
\definecolor{gainsboro}{HTML}{DCDCDC}
\definecolor{ghostwhite}{HTML}{F8F8FF}
\definecolor{gold}{HTML}{FFD700}
\definecolor{goldenrod}{HTML}{DAA520}
\definecolor{gray}{HTML}{808080}
\definecolor{green}{HTML}{008000}
\definecolor{greenyellow}{HTML}{ADFF2F}
\definecolor{grey}{HTML}{808080}
\definecolor{honeydew}{HTML}{F0FFF0}
\definecolor{hotpink}{HTML}{FF69B4}
\definecolor{indianred}{HTML}{CD5C5C}
\definecolor{indigo}{HTML}{4B0082}
\definecolor{ivory}{HTML}{FFFFF0}
\definecolor{khaki}{HTML}{F0E68C}
\definecolor{lavender}{HTML}{E6E6FA}
\definecolor{lavenderblush}{HTML}{FFF0F5}
\definecolor{lawngreen}{HTML}{7CFC00}
\definecolor{lemonchiffon}{HTML}{FFFACD}
\definecolor{lightblue}{HTML}{ADD8E6}
\definecolor{lightcoral}{HTML}{F08080}
\definecolor{lightcyan}{HTML}{E0FFFF}
\definecolor{lightgoldenrodyellow}{HTML}{FAFAD2}
\definecolor{lightgray}{HTML}{D3D3D3}
\definecolor{lightgreen}{HTML}{90EE90}
\definecolor{lightgrey}{HTML}{D3D3D3}
\definecolor{lightpink}{HTML}{FFB6C1}
\definecolor{lightsalmon}{HTML}{FFA07A}
\definecolor{lightseagreen}{HTML}{20B2AA}
\definecolor{lightskyblue}{HTML}{87CEFA}
\definecolor{lightslategray}{HTML}{778899}
\definecolor{lightslategrey}{HTML}{778899}
\definecolor{lightsteelblue}{HTML}{B0C4DE}
\definecolor{lightyellow}{HTML}{FFFFE0}
\definecolor{lime}{HTML}{00FF00}
\definecolor{limegreen}{HTML}{32CD32}
\definecolor{linen}{HTML}{FAF0E6}
\definecolor{magenta}{HTML}{FF00FF}
\definecolor{maroon}{HTML}{800000}
\definecolor{mediumaquamarine}{HTML}{66CDAA}
\definecolor{mediumblue}{HTML}{0000CD}
\definecolor{mediumorchid}{HTML}{BA55D3}
\definecolor{mediumpurple}{HTML}{9370DB}
\definecolor{mediumseagreen}{HTML}{3CB371}
\definecolor{mediumslateblue}{HTML}{7B68EE}
\definecolor{mediumspringgreen}{HTML}{00FA9A}
\definecolor{mediumturquoise}{HTML}{48D1CC}
\definecolor{mediumvioletred}{HTML}{C71585}
\definecolor{midnightblue}{HTML}{191970}
\definecolor{mintcream}{HTML}{F5FFFA}
\definecolor{mistyrose}{HTML}{FFE4E1}
\definecolor{moccasin}{HTML}{FFE4B5}
\definecolor{navajowhite}{HTML}{FFDEAD}
\definecolor{navy}{HTML}{000080}
\definecolor{oldlace}{HTML}{FDF5E6}
\definecolor{olive}{HTML}{808000}
\definecolor{olivedrab}{HTML}{6B8E23}
\definecolor{orange}{HTML}{FFA500}
\definecolor{orangered}{HTML}{FF4500}
\definecolor{orchid}{HTML}{DA70D6}
\definecolor{palegoldenrod}{HTML}{EEE8AA}
\definecolor{palegreen}{HTML}{98FB98}
\definecolor{paleturquoise}{HTML}{AFEEEE}
\definecolor{palevioletred}{HTML}{DB7093}
\definecolor{papayawhip}{HTML}{FFEFD5}
\definecolor{peachpuff}{HTML}{FFDAB9}
\definecolor{peru}{HTML}{CD853F}
\definecolor{pink}{HTML}{FFC0CB}
\definecolor{plum}{HTML}{DDA0DD}
\definecolor{powderblue}{HTML}{B0E0E6}
\definecolor{purple}{HTML}{800080}
\definecolor{red}{HTML}{FF0000}
\definecolor{rosybrown}{HTML}{BC8F8F}
\definecolor{royalblue}{HTML}{4169E1}
\definecolor{saddlebrown}{HTML}{8B4513}
\definecolor{salmon}{HTML}{FA8072}
\definecolor{sandybrown}{HTML}{F4A460}
\definecolor{seagreen}{HTML}{2E8B57}
\definecolor{seashell}{HTML}{FFF5EE}
\definecolor{sienna}{HTML}{A0522D}
\definecolor{silver}{HTML}{C0C0C0}
\definecolor{skyblue}{HTML}{87CEEB}
\definecolor{slateblue}{HTML}{6A5ACD}
\definecolor{slategray}{HTML}{708090}
\definecolor{slategrey}{HTML}{708090}
\definecolor{snow}{HTML}{FFFAFA}
\definecolor{springgreen}{HTML}{00FF7F}
\definecolor{steelblue}{HTML}{4682B4}
\definecolor{tan}{HTML}{D2B48C}
\definecolor{teal}{HTML}{008080}
\definecolor{thistle}{HTML}{D8BFD8}
\definecolor{tomato}{HTML}{FF6347}
\definecolor{turquoise}{HTML}{40E0D0}
\definecolor{violet}{HTML}{EE82EE}
\definecolor{wheat}{HTML}{F5DEB3}
\definecolor{white}{HTML}{FFFFFF}
\definecolor{whitesmoke}{HTML}{F5F5F5}
\definecolor{yellow}{HTML}{FFFF00}
\definecolor{yellowgreen}{HTML}{9ACD32}
\usepackage[most]{tcolorbox}

\usepackage{ifthen}
\provideboolean{admonitiontwoside}
\makeatletter%
\if@twoside%
\setboolean{admonitiontwoside}{true}
\else%
\setboolean{admonitiontwoside}{false}
\fi%
\makeatother%

\newenvironment{env-3b44c50d-4e1d-4123-9050-8dedcab86a0d}
{
    \savenotes\tcolorbox[blanker,breakable,left=5pt,borderline west={2pt}{-4pt}{firebrick}]
}
{
    \endtcolorbox\spewnotes
}
                

\newenvironment{env-1331fea0-f6a9-490d-abd5-c6befd8e7bfb}
{
    \savenotes\tcolorbox[blanker,breakable,left=5pt,borderline west={2pt}{-4pt}{blue}]
}
{
    \endtcolorbox\spewnotes
}
                

\newenvironment{env-136e818f-6bd8-45c9-b7d8-e6b8925f9e62}
{
    \savenotes\tcolorbox[blanker,breakable,left=5pt,borderline west={2pt}{-4pt}{green}]
}
{
    \endtcolorbox\spewnotes
}
                

\newenvironment{env-0f352b19-b6a1-4ac9-bf2a-bb8bde02fe73}
{
    \savenotes\tcolorbox[blanker,breakable,left=5pt,borderline west={2pt}{-4pt}{aquamarine}]
}
{
    \endtcolorbox\spewnotes
}
                

\newenvironment{env-bdb0c8db-783b-4759-8f75-e5414791f120}
{
    \savenotes\tcolorbox[blanker,breakable,left=5pt,borderline west={2pt}{-4pt}{orange}]
}
{
    \endtcolorbox\spewnotes
}
                

\newenvironment{env-ceffa806-b34f-4c58-9e42-a9e2da25850f}
{
    \savenotes\tcolorbox[blanker,breakable,left=5pt,borderline west={2pt}{-4pt}{blue}]
}
{
    \endtcolorbox\spewnotes
}
                

\newenvironment{env-246868e5-1b14-46c9-8aed-b83e61569202}
{
    \savenotes\tcolorbox[blanker,breakable,left=5pt,borderline west={2pt}{-4pt}{yellow}]
}
{
    \endtcolorbox\spewnotes
}
                

\newenvironment{env-7dee983f-32e5-4156-bbbe-38a2603301dc}
{
    \savenotes\tcolorbox[blanker,breakable,left=5pt,borderline west={2pt}{-4pt}{darkred}]
}
{
    \endtcolorbox\spewnotes
}
                

\newenvironment{env-d7dfd3c8-6d94-4eed-a91e-ca5b45830c05}
{
    \savenotes\tcolorbox[blanker,breakable,left=5pt,borderline west={2pt}{-4pt}{pink}]
}
{
    \endtcolorbox\spewnotes
}
                

\newenvironment{env-54374eb3-1bc6-43b0-a970-303ecc8fc079}
{
    \savenotes\tcolorbox[blanker,breakable,left=5pt,borderline west={2pt}{-4pt}{cyan}]
}
{
    \endtcolorbox\spewnotes
}
                

\newenvironment{env-dea8832f-c94c-45f2-8eee-8412fb4b0ef1}
{
    \savenotes\tcolorbox[blanker,breakable,left=5pt,borderline west={2pt}{-4pt}{cyan}]
}
{
    \endtcolorbox\spewnotes
}
                

\newenvironment{env-d3768c7e-89fb-4dd9-8f3d-05734b982232}
{
    \savenotes\tcolorbox[blanker,breakable,left=5pt,borderline west={2pt}{-4pt}{purple}]
}
{
    \endtcolorbox\spewnotes
}
                

\newenvironment{env-978626eb-980e-49c8-816b-1548bf60c3d0}
{
    \savenotes\tcolorbox[blanker,breakable,left=5pt,borderline west={2pt}{-4pt}{darksalmon}]
}
{
    \endtcolorbox\spewnotes
}
                

\newenvironment{env-55e08dc8-8a40-49e4-9d74-9e25a9308fa4}
{
    \savenotes\tcolorbox[blanker,breakable,left=5pt,borderline west={2pt}{-4pt}{gray}]
}
{
    \endtcolorbox\spewnotes
}
                
\ifLuaTeX
  \usepackage{selnolig}  % disable illegal ligatures
\fi
\IfFileExists{bookmark.sty}{\usepackage{bookmark}}{\usepackage{hyperref}}
\IfFileExists{xurl.sty}{\usepackage{xurl}}{} % add URL line breaks if available
\urlstyle{same} % disable monospaced font for URLs
\hypersetup{
  pdftitle={Directors' Duties},
  hidelinks,
  pdfcreator={LaTeX via pandoc}}

\title{Directors' Duties}
\author{}
\date{}

\begin{document}
\maketitle

{
\setcounter{tocdepth}{3}
\tableofcontents
}
\hypertarget{duties}{%
\section{Duties}\label{duties}}

\hypertarget{introduction}{%
\subsection{Introduction}\label{introduction}}

\hypertarget{pre-ca-2006}{%
\subsubsection{Pre-CA 2006}\label{pre-ca-2006}}

Directors subject to duties deriving from equitable principles and
common law rules:

\begin{itemize}
\tightlist
\item
  The fiduciary duties imposed on directors therefore arose out of
  equitable principles (e.g., acting in good faith).
\item
  Directors were also subject to the common law duty of skill and care.
\end{itemize}

\hypertarget{ca-2006}{%
\subsubsection{CA 2006}\label{ca-2006}}

New duties: ss 171-181. More than one duty may apply to the directors in
any one situation (CA 2006, s 179(1)).

\hypertarget{relevance-of-pre-ca-2006-rules}{%
\subsubsection{Relevance of pre-CA 2006
Rules}\label{relevance-of-pre-ca-2006-rules}}

Former regime (under common law and equity) still applies to the extent
not expressly provided for in CA 2006:

\begin{longtable}[]{@{}ll@{}}
\toprule()
Section & Rule \\
\midrule()
\endhead
s 170(3) & General duties are based on the old common law rules and
equitable principles as they apply in relation to directors and have
effect in place of those rules and principles as regards the duties owed
to a company by a director. \\
s 170(4) & New duties shall be interpreted and applied in the same way
as the common law rules and equitable principles \\
s 178 & The remedies for breach of the general duties will be the same
as under the corresponding equitable principles or common law rules. \\
\bottomrule()
\end{longtable}

\hypertarget{who-owes-duties}{%
\subsubsection{Who Owes Duties?}\label{who-owes-duties}}

Directors. Recall s 250(1): includes any person occupying the position
of director by whatever name called.

\hypertarget{to-whom}{%
\subsubsection{To Whom?}\label{to-whom}}

Directors owe their duties to the company (CA 2006, s 170(1)).

\begin{env-55e08dc8-8a40-49e4-9d74-9e25a9308fa4}

Who can take action?

\begin{itemize}
\tightlist
\item
  The company itself (acting through the board of directors) must take
  action against a director for breach of these duties, as any wrong is
  committed against the company itself.
\item
  In limited circumstances, the shareholders may bring a derivative
  action on behalf of the company to enforce the company's rights.
\item
  Exceptionally, the directors might owe a duty to shareholders as well
  as to the company where, for example, they agreed to act as agents for
  the shareholders, or if they assumed responsibility for giving advice
  to the shareholders on their shareholdings, but this would seem to be
  relevant only in small family companies (Platt v Platt {[}1999{]} 2
  BCLC 745)
\end{itemize}

\end{env-55e08dc8-8a40-49e4-9d74-9e25a9308fa4}

A director may also owe duties to the creditors of their company rather
than to the shareholders if the company is insolvent or approaching
insolvency (s 172(3)).

\hypertarget{summary}{%
\subsubsection{Summary}\label{summary}}

\begin{longtable}[]{@{}ll@{}}
\toprule()
Section & Duty \\
\midrule()
\endhead
s 171 CA 2006 & Duty to act within powers \\
s 172 CA 2006 & Duty to promote the success of the company for the
benefit of the members as a whole \\
s 173 CA 2006 & Duty to exercise independent judgment \\
s 174 CA 2006 & Duty to exercise reasonable care, skill, and
diligence \\
s 175 CA 2006 & Duty to avoid conflicts of interest \\
s 176 CA 2006 & Duty not to accept benefits from third parties \\
s 177 CA 2006 & Duty to declare any interest in a proposed
transaction \\
\bottomrule()
\end{longtable}

Note that s 175-177 replace/ supersede equitable duties stated as 'no
conflict' and 'no profit/ misuse of position' rules. These were
summarised in the following cases:

\begin{itemize}
\tightlist
\item
  Aberdeen Rly Co v Blaikie Bros {[}1854{]} 1 Macq 61
\item
  Bray v Ford {[}1896{]} AC 44
\end{itemize}

The way in which directors' duties are framed results in an overlap
between them. s 179 emphasises that the effect of duties is cumulative:
necessary for directors to comply with every duty which may apply.

Note that s 172 and 175 duties overlap in many cases.

\hypertarget{s-171}{%
\subsection{S 171}\label{s-171}}

\begin{env-136e818f-6bd8-45c9-b7d8-e6b8925f9e62}

s 171 - Duty to act within powers

A director of a company must---

\begin{itemize}
\tightlist
\item
  (a) act in accordance with the company's constitution, and
\item
  (b) only exercise powers for the purposes for which they are
  conferred.
\end{itemize}

\end{env-136e818f-6bd8-45c9-b7d8-e6b8925f9e62}

Details:

\begin{enumerate}
\tightlist
\item
  Act in accordance with the company's constitution

  \begin{itemize}
  \tightlist
  \item
    The company's constitution is defined in
    \href{https://www.legislation.gov.uk/ukpga/2006/46/section/257}{s
    257 CA 2006} and includes everything set out in the company's
    articles of association (s 17) and decisions taken in accordance
    with the articles (i.e., shareholder resolutions).
  \item
    A director is in breach of this duty if he acts without authority,
    e.g. commits the company to borrow more than the articles allow
    without prior shareholder approval.
  \end{itemize}
\item
  Only exercise powers for the purposes for which they are conferred.

  \begin{itemize}
  \tightlist
  \item
    Directors should exercise their powers ``bona fide in what they
    consider--not what a court may consider--is in the interests of the
    company, and not for any collateral purpose.'' (Re Smith and Fawcett
    Ltd. {[}1942{]} Ch 304). This duty codified the \textbf{'proper
    purposes' doctrine}.
  \end{itemize}
\end{enumerate}

\hypertarget{proper-purposes-doctrine}{%
\subsubsection{Proper Purposes
Doctrine}\label{proper-purposes-doctrine}}

Outlined in relevant case law:

\begin{itemize}
\tightlist
\item
  Hogg v Cramphorn {[}1967{]} Ch 254: whether purpose of a new share
  issue was malicious
\item
  Howard Smith Ltd v Ampol Petroleum Ltd {[}1974{]} AC 821:

  \begin{itemize}
  \tightlist
  \item
    If directors exercise their powers for several purposes, then courts
    will only consider the dominant purpose.
  \item
    If the dominant purpose is proper, even if the other purposes are
    improper, there will be no breach of duty.
  \item
    The courts ascertain objective the purpose for which the power was
    conferred.
  \end{itemize}
\item
  Eclairs Group Ltd v JKX Oil \& Gas plc {[}2015{]} UKSC 71: powers of
  directors to issue disclosure notice.
\item
  Extrasure Travel Insurances Ltd v Scattergood {[}2003{]} 1 BCLC 598:
  the test of improper purpose is an objective one. See this case for a
  nice staged test.
\item
  Teck Corporation v Millar (1972) 33 DLR (3d) 288: good faith vs proper
  purposes doctrine
\end{itemize}

Note the overlap between the duty to promote the success of the company
(s 172) and the proper purposes' doctrine (s 171). The proper purposes
doctrine operates to limit the authority of directors, even if their
action was carried out in what they bona fide believed to be the best
interests of the company. If the power is exercised primarily for some
collateral purpose (construed objectively), then the action may be set
aside.

\hypertarget{s-172}{%
\subsection{S 172}\label{s-172}}

Success = long-term increase in company's value.

\begin{env-136e818f-6bd8-45c9-b7d8-e6b8925f9e62}

s 172 - Duty to promote the success of the company

(1) A director of a company must act in the way he considers, in good
faith, would be most likely to promote the success of the company for
the benefit of its members as a whole, and in doing so have regard
(amongst other matters) to---

\begin{itemize}
\tightlist
\item
  (a) the likely consequences of any decision in the long term,
\item
  (b) the interests of the company's employees,
\item
  (c) the need to foster the company's business relationships with
  suppliers, customers and others,
\item
  (d) the impact of the company's operations on the community and the
  environment,
\item
  (e) the desirability of the company maintaining a reputation for high
  standards of business conduct, and
\item
  (f) the need to act fairly as between members of the company.
\end{itemize}

(2) Where or to the extent that the purposes of the company consist of
or include purposes other than the benefit of its members, subsection
(1) has effect as if the reference to promoting the success of the
company for the benefit of its members were to achieving those purposes.

(3) The duty imposed by this section has effect subject to any enactment
or rule of law requiring directors, in certain circumstances, to
consider or act in the interests of creditors of the company.

\end{env-136e818f-6bd8-45c9-b7d8-e6b8925f9e62}

\hypertarget{interpretation}{%
\subsubsection{Interpretation}\label{interpretation}}

Codifies the long-settled duty of a director to act honestly and in good
faith, in the best interests of the company.

Lord Greene MR in Re Smith and Fawcett Ltd. {[}1942{]} Ch 304, who
stated that

\begin{quote}
``{[}Directors{]} must exercise their discretion \emph{bona fide} in
what they consider---not what a court may consider---is in the interests
of the company''.
\end{quote}

Similarly, in Regentcrest v Cohen {[}2001{]} 2 BCLC 80 it was held that
the question for the court was not whether a director's actions were
objectively in the best interests of the company, but whether
\textbf{subjectively} the director honestly believed they were acting in
the best interests of the company.

Although the test is \textbf{subjective}, the determination of whether a
director has complied with s 172 will involve an element of objectivity.

Pennycuick J in Charterbridge Corpn Ltd v Lloyd's Bank Ltd (1970) stated
that the test for determining this duty is

\begin{quote}
``whether an intelligent and honest man in the position of a director of
the company concerned, could, in the whole of the existing
circumstances, have reasonably believed that the transactions were for
the benefit of the company.''
\end{quote}

In Re Southern Counties Fresh Food Ltd (2009), Warren J stated that

\begin{quote}
``it is accepted that a breach will have occurred if it is established
that the relevant exercise of power is one which could not be considered
by any reasonable director to be in the interests of the company''.
\end{quote}

In Madoff Securities International Ltd v Raven \& Others {[}2013{]} EWHC
3147, it was stated that a director may legitimately defer to his fellow
directors' views if he thinks they believe they are acting in the best
interests of the company, even if he is not himself in agreement.

This test, in practice, makes it very difficult to challenge directors'
decision-making.

The approach requires directors to recognise that shareholder value for
the company depends on successful management of relationship with other
stakeholders. \textbf{Two factors}: directors' good faith and interests
of company.

'Success' is defined as long term increase in value (s 172(1)(a)) and
takes into account the effect of the directors' decision-making on the
members as a whole and the other stakeholders.

\hypertarget{duty-to-creditors}{%
\subsubsection{Duty to Creditors}\label{duty-to-creditors}}

Where a company is insolvent, duty extends to directors acting in the
best interests of creditors:

\begin{itemize}
\tightlist
\item
  Item Software (UK) Ltd v Fassihi {[}2004{]} EWCA Civ 1244: in this
  case it was held that the duty to act bona fide in the interests of
  the company includes the duty to \textbf{disclose misconduct} by the
  director to the company.
\item
  Re HLC Environmental Projects Ltd {[}2013{]} EWHC 2876 Ch: the court
  held that a director who \textbf{prefers one creditor to another} by
  deliberately paying one over the others (here a bank whose debt was
  guaranteed by the parent company, so as to assist the parent company)
  may be in breach of the duty to act in the interests of the creditors
  as a whole.
\end{itemize}

N.B. if this seems confusing, get even more confused by
\href{https://usir.salford.ac.uk/id/eprint/3112/1/CA_2006.pdf}{an
acadmic opinion}.

\hypertarget{s-173}{%
\subsection{S 173}\label{s-173}}

\begin{env-136e818f-6bd8-45c9-b7d8-e6b8925f9e62}

s 173 - Duty to exercise independent judgment

(1) A director of a company must exercise independent judgment.\\
(2) This duty is not infringed by his acting --

\begin{itemize}
\tightlist
\item
  (a) in accordance with an agreement duly entered into by the company
  that restricts the future exercise of discretion by its directors, or
\item
  (b) in a way authorised by the company's constitution.
\end{itemize}

\end{env-136e818f-6bd8-45c9-b7d8-e6b8925f9e62}

Where the board is able to establish that it was in the best interests
of the company to enter into an agreement which fetters the discretion
of the directors, this will not be a breach of s 173.

The duty is a restatement of the previous fiduciary duty that a director
must not fetter their own discretion.

\hypertarget{cases}{%
\subsubsection{Cases}\label{cases}}

\begin{itemize}
\tightlist
\item
  Fulham Football Club Ltd v Cabra Estates plc {[}1994{]} 1 BCLC 363
  (Court of Appeal): directors have the right to enter into a contract
  on behalf of the company that may in future require the directors to
  act in a particular way, provided it is done in good faith in the
  interests of the company.
\item
  Madoff Securities International Ltd v Raven {[}2013{]} EWHC 3147:
  directors must be mindful of the individual nature of their duties.
  Can't be dominated, bamboozled or manipulated.
\end{itemize}

\hypertarget{s-174}{%
\subsection{S 174}\label{s-174}}

\begin{env-136e818f-6bd8-45c9-b7d8-e6b8925f9e62}

s 174 - Duty to exercise reasonable care, skill and diligence

(1) A director of a company must exercise reasonable care, skill and
diligence.\\
(2) This means the care, skill and diligence that would be exercised by
a reasonably diligent person with-

\begin{itemize}
\tightlist
\item
  (a) the general knowledge, skill and experience that may reasonably be
  expected of a person carrying out the functions carried out by the
  director in relation to the company, and
\item
  (b) the general knowledge, skill and experience that the director has.
\end{itemize}

\end{env-136e818f-6bd8-45c9-b7d8-e6b8925f9e62}

\hypertarget{interpretation-1}{%
\subsubsection{Interpretation}\label{interpretation-1}}

Codifies the common law duties of care and skill (Gregson v HAE Trustees
Ltd {[}2008{]} EWHC 1006).

In early cases, the standard expected of directors was subjective, so
poor directors could escape liability for company losses.

But Donoghue v Stevenson {[}1932{]} AC 562 brought recognition of
general duty of care and raised standard expected of directors to
objective/ subjective test. This standard was borrowed on the Wrongful
trading test used in s 214(4) IA 1986.

\begin{env-d7dfd3c8-6d94-4eed-a91e-ca5b45830c05}

Test

The minimum standard expected of a director is that objectively expected
of a director in that position, but then this can be subjectively raised
if a particular director has special knowledge/ skill and experience.

\end{env-d7dfd3c8-6d94-4eed-a91e-ca5b45830c05}

Re D'Jan of London Ltd {[}1994{]} 1 BCLC 561 ChD: director's actions
will be measured against the conduct expected of a reasonably diligent
person.

The standard applicable to director of a profitable company will change
if it experiences financial difficulties (Roberts v Frohlich {[}2011{]}
EWHC 257 (Ch)).

\hypertarget{delegation}{%
\subsubsection{Delegation}\label{delegation}}

s 174 shares similarities with cases brought under Company Directors
Disqualification Act 1986 in relation to delegation.

Principle: directors must be proactive in monitoring actions of
delegates and other directors, and must keep themselves informed (Re
Barings plc (No 5) {[}1999{]} 1 BCLC 433).

A director who is completely inactive will be in breach of this duty and
may be held liable for wrongs committed by the company (Lexi Holdings
plc (In Administration) v Luqman {[}2009{]} EWCA Civ 117---other
directors responsible for not realising that one of them was stealing).

But directors able to legitimately defer to the views of a fellow
director with greater experience when reasonable, and this did not
breach the duty to exercise independent judgement: Madoff Securities
International Ltd v Raven {[}2013{]} EWHC 3147.

\hypertarget{contract}{%
\subsubsection{Contract}\label{contract}}

There will usually also be a term in the director's service contract
that they will perform their duties with reasonable skill and care. So a
company can usually also seek damages for breach of contract.

\hypertarget{s-175}{%
\subsection{S 175}\label{s-175}}

\begin{env-136e818f-6bd8-45c9-b7d8-e6b8925f9e62}

s 175 - Duty to avoid conflicts of interest

(1) A director of a company must avoid a situation in which he has, or
can have, a direct or indirect interest that conflicts, or possibly may
conflict, with the interests of the company.

(2) This applies in particular to the exploitation of any property,
information or opportunity (and it is immaterial whether the company
could take advantage of the property, information or opportunity).

(3) This duty does not apply to a conflict of interest arising in
relation to a transaction or arrangement with the company.

(4) This duty is not infringed---

\begin{itemize}
\tightlist
\item
  (a) if the situation cannot reasonably be regarded as likely to give
  rise to a conflict of interest; or
\item
  (b) if the matter has been authorised by the directors.
\end{itemize}

(5) Authorisation may be given by the directors---

\begin{itemize}
\tightlist
\item
  (a) where the company is a private company and nothing in the
  company's constitution invalidates such authorisation, by the matter
  being proposed to and authorised by the directors; or
\item
  (b) where the company is a public company and its constitution
  includes provision enabling the directors to authorise the matter, by
  the matter being proposed to and authorised by them in accordance with
  the constitution.
\end{itemize}

(6) The authorisation is effective only if---

\begin{itemize}
\tightlist
\item
  (a) any requirement as to the quorum at the meeting at which the
  matter is considered is met without counting the director in question
  or any other interested director, and
\item
  (b) the matter was agreed to without their voting or would have been
  agreed to if their votes had not been counted.
\end{itemize}

(7) Any reference in this section to a conflict of interest includes a
conflict of interest and duty and a conflict of duties.

\end{env-136e818f-6bd8-45c9-b7d8-e6b8925f9e62}

Note that the Explanatory Notes to s 175 comment that the duty under s
175(4) to (6) is not infringed if

\begin{enumerate}
\tightlist
\item
  The situation cannot reasonably be regarded as likely to give rise to
  a conflict
\item
  Authorisation (in private or public companies) has been given by
  independent directors (who have no direct or indirect interest).
\end{enumerate}

\hypertarget{interpretation-2}{%
\subsubsection{Interpretation}\label{interpretation-2}}

Reasonable man test for what constitutes conflict of interest: Boardman
v Phipps {[}1967{]} 2 A.C. 46. Note that the point of this was to make
law more chill, Boardman acknowledged as unfair.

s 175(3) expressly \textbf{excludes} conflicts of interest arising in
relation to \textbf{transactions or arrangements} with the company;
these are instead subject to disclosure duties in s 177. Practical rule:
many directors will have interests in multiple companies.

\hypertarget{corporate-opportunities}{%
\paragraph{Corporate Opportunities}\label{corporate-opportunities}}

Corporate opportunity viewed as asset which may not be misappropriated
by the directors:

\begin{itemize}
\tightlist
\item
  Cook v Deeks {[}1916{]} 1 AC 554 (Privy Council)
\item
  Regal (Hastings) Ltd v Gulliver {[}1942{]} 1 All ER 378
\item
  Bhullar v Bhullar {[}2003{]} EWCA Civ 424 (Court of Appeal)
\end{itemize}

It is irrelevant that the company could not/ would not have pursued that
opportunity. If a director makes a personal profit out of their
position, they are liable to account to the company for this. Principle
expressly stated in s 175(2).

\hypertarget{resigning-directors}{%
\paragraph{Resigning Directors}\label{resigning-directors}}

A director cannot resign from a company just to then exploit a corporate
opportunity, which would otherwise be a breach of s 175.

s 170(2)(a):

\begin{quote}
``a person who ceases to be a director continues to be subject (a) to
the duty in section 175 (duty to avoid conflicts of interest) as regards
the exploitation of any property, information or opportunity of which he
became aware when he was a director.''
\end{quote}

But need to consider all the facts to ascertain whether s 175 has been
breached or not:

\begin{itemize}
\tightlist
\item
  Shepherds Investments Ltd v Walters {[}2006{]} EWHC 836
\item
  CMS Dolphin Ltd v Simonet {[}2001{]} EWHC 415
\item
  See also Foster Bryant Surveying v Bryant {[}2007{]} EWCA Civ 200, but
  note this is a pre-CA 2006 case and questionable whether it would be
  decided the same way today.
\end{itemize}

In Towers v Premier Waste Management Ltd {[}2012{]} BCC, 72, a director
borrowed equipment from one of the company's customers for his personal
interest without disclosing his personal interest to the board. Found to
have breached s 175 despite no bad faith, no actual conflict and no
quantifiable loss to the company. Director ordered to pay rental costs
to the company.

\hypertarget{competing-directorships}{%
\paragraph{Competing Directorships}\label{competing-directorships}}

A general equitable rule prevents a fiduciary from entering into a
position which gives rise to conflicting fiduciary duties to another
person without the informed consent of both principals (Clark Boyce v
Mouat {[}1994{]} 1 AC 428). Company will need to authorise the conflict
under the process in s 175(5) and 175(6).

But depends on facts of the case: in In Plus Group Ltd v Pyke {[}2002{]}
EWCA Civ 370 (Court of Appeal) no breach was fiduciary duty was found.

\hypertarget{consequences}{%
\subsubsection{Consequences}\label{consequences}}

If a director breaches this duty, they will be obliged to account for
any profit made unless authorised by the company.

\hypertarget{s-176}{%
\subsection{S 176}\label{s-176}}

\begin{env-136e818f-6bd8-45c9-b7d8-e6b8925f9e62}

s 176 - Duty not to accept benefits from third parties

(1) A director of a company must not accept a benefit from a third party
conferred by reason of---

\begin{itemize}
\tightlist
\item
  (a) his being a director, or
\item
  (b) his doing (or not doing) anything as director.
\end{itemize}

(2) A ``third party'' means a person other than the company, an
associated body corporate or a person acting on behalf of the company or
an associated body corporate.

(3) Benefits received by a director from a person by whom his services
(as a director or otherwise) are provided to the company are not
regarded as conferred by a third party.

(4) This duty is not infringed if the acceptance of the benefit cannot
reasonably be regarded as likely to give rise to a conflict of interest.

(5) Any reference in this section to a conflict of interest includes a
conflict of interest and duty and a conflict of duties.

\end{env-136e818f-6bd8-45c9-b7d8-e6b8925f9e62}

\hypertarget{interpretation-3}{%
\subsubsection{Interpretation}\label{interpretation-3}}

This duty replaces the equitable principle that fiduciaries must not
accept secret commissions or bribes (Attorney General for Hong Kong v
Reid {[}1994{]} 1 AC 324). Should be read in conjunction with wider
no-conflict duty in s 175 CA 2006.

s 176(4) states that duty is not infringed if acceptance of the benefit
cannot reasonably be regarded as likely to give rise to a conflict of
interest. ``Benefits'' construed in the ordinary meaning of the word. No
provision for authorisation by the board here; ratification would need
to be sought by shareholders under s 239 CA 2006.

Note that in general, this duty is construed widely, since the old
interpretation was driven by bribes.

Note that this duty is absolute and there is \textbf{no} provision for
authorisation by the board of directors for such a breach.

\hypertarget{s-177}{%
\subsection{S 177}\label{s-177}}

\begin{env-136e818f-6bd8-45c9-b7d8-e6b8925f9e62}

s 177 - Duty to declare interest in proposed transaction or arrangement

(1) If a director is in any way, directly or indirectly, interested in a
proposed transaction or arrangement with the company, he must declare
the nature and extent of that interest to the other directors.

(2) The declaration may (but need not) be made---

\begin{itemize}
\tightlist
\item
  (a) at a meeting of the directors, or
\item
  (b) by notice to the directors in accordance with---

  \begin{itemize}
  \tightlist
  \item
    (i) section 184 (notice in writing), or
  \item
    (ii) section 185 (general notice).
  \end{itemize}
\end{itemize}

(3) If a declaration of interest under this section proves to be, or
becomes, inaccurate or incomplete, a further declaration must be made.

(4) Any declaration required by this section must be made before the
company enters into the transaction or arrangement.

(5) This section does not require a declaration of an interest of which
the director is not aware or where the director is not aware of the
transaction or arrangement in question. For this purpose a director is
treated as being aware of matters of which he ought reasonably to be
aware.

(6) A director need not declare an interest---

\begin{itemize}
\tightlist
\item
  (a) if it cannot reasonably be regarded as likely to give rise to a
  conflict of interest;
\item
  (b) if, or to the extent that, the other directors are already aware
  of it (and for this purpose the other directors are treated as aware
  of anything of which they ought reasonably to be aware); or
\item
  (c) if, or to the extent that, it concerns terms of his service
  contract that have been or are to be considered---

  \begin{itemize}
  \tightlist
  \item
    (i) by a meeting of the directors, or
  \item
    (ii) by a committee of the directors appointed for the purpose under
    the company's constitution.
  \end{itemize}
\end{itemize}

\end{env-136e818f-6bd8-45c9-b7d8-e6b8925f9e62}

\hypertarget{aim}{%
\subsubsection{Aim}\label{aim}}

To ensure that the board of directors (who will act on behalf of the
company) have full disclosure of a possible conflict of interest before
deciding whether to enter into the transaction with one of their own
directors.

\hypertarget{making-declaration}{%
\subsubsection{Making Declaration}\label{making-declaration}}

\begin{itemize}
\tightlist
\item
  The declaration must be made before the company enters into the
  contract (CA 2006, s 177(4)).
\item
  s 177(2)(a): may be made at a directors' board meeting; or

  \begin{itemize}
  \tightlist
  \item
    Disclosure need not be formally in writing (Lee Panavision Ltd v Lee
    Lighting Ltd (1992))--orally is fine
  \end{itemize}
\item
  s 177(2)(b)(i): may be made by notice to the directors in writing

  \begin{itemize}
  \tightlist
  \item
    Must comply with the requirements of s 184 of the CA 2006, namely,
    it must be in paper or electronic form and be sent by hand, post or,
    if agreed by the recipient, by electronic means to all of the other
    directors.
  \end{itemize}
\item
  s 177(2)(b)(ii): declaration can be made by general notice.

  \begin{itemize}
  \tightlist
  \item
    Should be done if declaration arises repeatedly over time, e.g.,

    \begin{itemize}
    \tightlist
    \item
      s 185(2)(a): if the director of the company has an interest in
      another company.
    \item
      s 185(2)(b): if the director made a general notice regarding the
      director's connection with a specified person
    \end{itemize}
  \item
    Notice must:

    \begin{itemize}
    \tightlist
    \item
      be given to the other directors at a board meeting or read at the
      next board meeting after it was given (s 185(4)); and
    \item
      (b) state the nature and extent of the director's interest in the
      company or the nature of his connection with the specified person
      (s 185(3)).
    \end{itemize}
  \end{itemize}
\end{itemize}

Disclosure need not be formally in writing (Lee Panavision Ltd v Lee
Lighting Ltd (1992)). Under s 180, if directors comply with duty to
disclose, the transaction not liable to be set aside (though this is
subject to other provisions in the company's constitution).

The idea of this section is to invite scrutiny of decisions. Even more
checks and balances. Note already board authorisation under s 175 CA
2006.

\hypertarget{impact}{%
\subsubsection{Impact}\label{impact}}

MA 14(1): the director who made the declaration cannot vote as a
director regarding the contract, nor form part of the quorum (minimum
number of directors required) for a board meeting.

\hypertarget{exceptions}{%
\subsubsection{Exceptions}\label{exceptions}}

See ss 177(5) \& 177(6).

\begin{env-1331fea0-f6a9-490d-abd5-c6befd8e7bfb}

Note

If one of the two exceptions applies and the company has unamended MA,
the director concerned will still not be able to vote or count in the
quorum on the matter under discussion as art 14 of the model articles of
private companies is still applicable.

\end{env-1331fea0-f6a9-490d-abd5-c6befd8e7bfb}

A declaration does not need to be made where there is just one director
of a company whose articles require only one director.

\hypertarget{s-177-vs-s-182}{%
\subsection{S 177 Vs S 182}\label{s-177-vs-s-182}}

\href{https://www.legislation.gov.uk/ukpga/2006/46/section/177}{s 177 CA
2006} places a duty on directors to disclose an interest in a
\textbf{proposed} transaction with the company.
\href{https://www.legislation.gov.uk/ukpga/2006/46/section/182}{s 182 CA
2006} applies to cases where a director has an interest in a transaction
\textbf{after it has been entered into by the company,} and also
requires directors to disclose such interests.

Under s 187(1), the requirements under s 182 also apply to shadow
directors, but with some amendments.

Breach of the duty under s 177 leads to civil consequences under s 178,
which are the same as for breach of all the other duties (other than s
174), whereas breach of s 182 leads to criminal sanctions under s 183.

Note that these provisions do not apply to substantial property
transactions, loans, quasi-loans and credit transactions which require
the approval of the company's members (ss 190--203).

\hypertarget{remedies-and-liability}{%
\section{Remedies and Liability}\label{remedies-and-liability}}

Directors' duties are owed to the company and not to shareholders as
individuals (s 170). But since company inanimate, s 260 CA 2006 allows
shareholders to bring Derivative claims on behalf of the company where
the directors have acted in breach of their duties.

\hypertarget{remedies}{%
\subsection{Remedies}\label{remedies}}

Recall that duties of directors were codified in CA 2006, but remedies
for breach were not. Just have s 178 saying that:

\begin{itemize}
\tightlist
\item
  Consequences of breach (or threatened breach) of sections 171-177 are
  \textbf{the same as would apply if the corresponding common law rule
  or equitable principle applied}.
\item
  Duties in those sections (except s 174) are enforceable in the same
  way as any other fiduciary duty owed to a company by its directors.
\end{itemize}

Breach of duty to exercise reasonable care, skill, and diligence (s 174)
is a common law rather than fiduciary duty, so the only remedy for
breach is damages. Only a failure to make a declaration under s 182 is a
criminal offence, punishable by fine (s 183(1)).

s 179: more than one of the general duties may apply in any given case.
So common to sue a director for breach of multiple duties.

Remedial options:

\begin{itemize}
\tightlist
\item
  Account for profits (personal remedy)

  \begin{itemize}
  \tightlist
  \item
    Directors held liable to account to the company for profits received
    from breach of their duties, regardless of whether they acted
    honestly
  \item
    Only profits received as a result of the breach are payable
  \item
    In Regal (Hastings) Ltd v Gulliver {[}1942{]} 1 All ER 378 directors
    bought cinemas and sold them for a personal profit.
  \item
    Here suing the director for the physical cash (can't get it back if
    director bankrupt etc.)
  \end{itemize}
\item
  Account for profits (proprietary remedy)

  \begin{itemize}
  \tightlist
  \item
    'Follow the money': can trace money through to the assets and take
    those instead.
  \item
    Debate in the courts as to whether the director held the profits of
    their breach of duty on constructive trust for the company or not.
  \item
    Settled in FHR European Ventures LLP v Cedar Capital Partners LLC
    {[}2014{]} UKSC 45 {[}2015{]} AC 250, which held that profits are
    held on constructive trust in all cases where profits can be
    identified as an asset capable of being held on constructive trust.
    (This basically means if profits were invested in shares, entitled
    to the dividends and stuff too).
  \item
    Returning company property.
  \end{itemize}
\item
  Payment of equitable compensation by the directors.
\item
  Rescission

  \begin{itemize}
  \tightlist
  \item
    Rescind contract to before contract made, if possible
  \end{itemize}
\item
  Injunction against a director (usually when the breach is threatened).
\end{itemize}

The aim of these remedies is not to compensate the company for the
breach of duty by the director, but to confiscate any profit made by the
director in breach and give it to the company (Murad v Al-Saraj
{[}2005{]} EWCA Civ 959).

\hypertarget{avoiding-liability}{%
\subsection{Avoiding Liability}\label{avoiding-liability}}

\hypertarget{prior-authorisation-by-board-ss-175-177}{%
\subsubsection{Prior Authorisation by Board: Ss 175 \&
177}\label{prior-authorisation-by-board-ss-175-177}}

Where directors are able to authorise conduct which would otherwise
constitute a breach of duty under \textbf{s 175 or 177}, the effect is
that no breach of duty will take place and the transaction is not liable
to be set aside (no longer voidable) (s 180(1)).

\begin{longtable}[]{@{}ll@{}}
\toprule()
Company & s175 authorisation procedure \\
\midrule()
\endhead
Private company & Authorisation may be given provided the company's
articles do not prevent it. \\
Public company & Articles must specifically permit the directors to
authorise the matter. \\
\bottomrule()
\end{longtable}

\hypertarget{prior-authorisation-by-shareholders-s-180}{%
\subsubsection{Prior Authorisation by Shareholders (s
180)}\label{prior-authorisation-by-shareholders-s-180}}

\begin{itemize}
\tightlist
\item
  \href{https://www.legislation.gov.uk/ukpga/2006/46/section/180}{s
  180(4) CA 2006} allows shareholders to authorise in advance conduct
  which would otherwise constitute a breach of duty by the directors
\item
  Shareholders can authorise breaches of duty or even negligence, but
  not acts which are unlawful
\item
  Sharma v Sharma {[}2013{]} EWCA Civ 1287 (Court of Appeal)
\end{itemize}

\hypertarget{ratification-by-shareholders-s-239}{%
\subsubsection{Ratification by Shareholders (s
239)}\label{ratification-by-shareholders-s-239}}

\begin{env-136e818f-6bd8-45c9-b7d8-e6b8925f9e62}

s 139 - Ratification of acts of directors

(1) This section applies to the ratification by a company of conduct by
a director amounting to negligence, default, breach of duty or breach of
trust in relation to the company.

(2) The decision of the company to ratify such conduct must be made by
resolution of the members of the company.

(3) Where the resolution is proposed as a written resolution neither the
director (if a member of the company) nor any member connected with him
is an eligible member.

(4) Where the resolution is proposed at a meeting, it is passed only if
the necessary majority is obtained disregarding votes in favour of the
resolution by the director (if a member of the company) and any member
connected with him.

This does not prevent the director or any such member from attending,
being counted towards the quorum and taking part in the proceedings at
any meeting at which the decision is considered.

(5) For the purposes of this section---

\begin{itemize}
\tightlist
\item
  (a) ``conduct'' includes acts and omissions;
\item
  (b) ``director'' includes a former director;
\item
  (c) a shadow director is treated as a director; and
\item
  (d) in section 252 (meaning of ``connected person''), subsection (3)
  does not apply (exclusion of person who is himself a director).\\
  ...
\end{itemize}

\end{env-136e818f-6bd8-45c9-b7d8-e6b8925f9e62}

\begin{itemize}
\tightlist
\item
  The conduct of a director which amounts to breach of duty, negligence,
  default or breach of trust can be ratified by shareholders
  (disregarding the votes of directors involved, if also a shareholder,
  or any connected persons
  (\href{https://www.legislation.gov.uk/ukpga/2006/46/section/239}{s
  239(4) CA 2006}))
\item
  The company cannot then take any action against the director for the
  breach.
\item
  The voting restrictions do not apply if the matter is decided by
  unanimous consent (s 239(6)(a)).
\item
  Neither approval under s 180 nor ratification under s 239 will be
  effective unless the decision is honest, bona fide and in the best
  interests of the company (Madoff Securities International Ltd v Raven
  {[}2013{]} EWHC 3147).
\item
  Ratification is not possible if done unfairly or improperly, or if it
  is illegal or oppressive towards a minority (North-West Transportation
  Co Ltd v Beatty (1887) 12 App Cas 589). If there is unfair
  ratification, an Unfair Prejudice claim could be brought.
\end{itemize}

\hypertarget{relief-granted-by-court-s-1157}{%
\subsubsection{Relief Granted by Court (s
1157)}\label{relief-granted-by-court-s-1157}}

Basically an equity thing.
\href{https://www.legislation.gov.uk/ukpga/2006/46/section/1157}{s
1157(1) CA 2006}:

\begin{quote}
If in proceedings for negligence, default, breach of duty or breach of
trust against\\
(a) an officer of the company, or\\
(b) a person employed by a company as an auditor,\\
it appears to the court that the officer or person is or may be liable,
but that he acted honestly and reasonably, and that having regard to all
the circumstances of the case, he ought fairly to be excused, the court
may relieve him from his liability on such terms as it thinks fit.
\end{quote}

The director may apply to court for relief before a claim is made (s
1157(2)).

\begin{env-d7dfd3c8-6d94-4eed-a91e-ca5b45830c05}

Court granting equitable relief

In Re HLC Environmental Projects Ltd {[}2013{]} EWHC 2876 Ch, a three
part test was established. Need that the director:

\begin{itemize}
\tightlist
\item
  Acted honestly,
\item
  acted reasonably, and
\item
  considering the circumstances of case, ought fairly to be excused.
\end{itemize}

\end{env-d7dfd3c8-6d94-4eed-a91e-ca5b45830c05}

The burden of proving honesty and reasonableness lies on the director.
It is only if both of these are established that the court needs to
consider the third element.

Such relief granted in Re D'Jan of London Ltd {[}1994{]} 1 BCLC 561 ChD,
where it was satisfied that non-disclosure breach was a minor error. But
in other cases (e.g., Smith v Butler {[}2012{]} EWCA Civ 314),
application for relief refused.

In Coleman Taymar Ltd v Oakes {[}2001{]} 2 BCLC 249 the court held that
relief under s 1157 can be granted in relation to a liability to account
for profits as well as a liability to pay damages. \textbf{Honesty is a
subjective requirement, whereas reasonableness is objective.}

\hypertarget{insurance}{%
\subsubsection{Insurance}\label{insurance}}

\href{https://www.legislation.gov.uk/ukpga/2006/46/section/232}{s 232 CA
2006}: a company cannot exempt a director to any extent from
\textbf{liability} in negligence, default, breach of duty or breach of
trust, and any provision that purports to do this will be \textbf{void}.

Companies may provide their directors with:

\begin{itemize}
\tightlist
\item
  Insurance against such liability under s 233
\item
  Qualifying third party indemnity provisions under s 234
\item
  Qualifying pension scheme indemnity provisions under s 235
\end{itemize}

\begin{longtable}[]{@{}llll@{}}
\toprule()
Section breached/to be breached & Approval? & Ratification? &
Conditions \\
\midrule()
\endhead
s 175 CA 2006 or s 177 CA 2006 & Directors or shareholders can authorise
(s 180) & Yes, by ordinary resolution (s 239) & Decision must be honest,
bona fide and in the best interests of the company \\
Any other breach of duty & Shareholders can authorise (so long as not
unlawful) & Yes, by ordinary resolution & As above. \\
Negligence, default or breach of trust & n/a & Yes, by ordinary
resolution & As above. \\
\bottomrule()
\end{longtable}

More Controls on Directors

\hypertarget{directors-liability}{%
\subsection{Directors' Liability}\label{directors-liability}}

\hypertarget{agency}{%
\subsubsection{Agency}\label{agency}}

A director is an agent of a company. Contracts are made in the name of
the company, so directors are not personally liable for breach of
contract. But liability can arise in two situations:

\begin{enumerate}
\tightlist
\item
  Exceeding actual authority

  \begin{itemize}
  \tightlist
  \item
    If actual authority \textless{} director's actions \textless{}
    apparent authority, director liable to indemnify company for any
    loss suffered and to account for any profit.
  \item
    But shareholders can ratify breach (s 239)
  \end{itemize}
\item
  Breach of warranty of authority

  \begin{itemize}
  \tightlist
  \item
    Else if apparent authority \textless{} director's actions, director
    does not bind company, but personally liable to third party on the
    contract for breach of warranty of authority.
  \item
    s 40(2): an outsider dealing with a company is not bound to check on
    the powers of the directors or of the company.
  \item
    s 40(1): as regards an outsider, the power of the directors to bind
    the company shall not be limited by the company's constitution
    (provided outsider acting in good faith).
  \item
    See Capacity and Authority \textgreater{} Statutory Deemed Authority
    s 40 CA 2006.
  \end{itemize}
\end{enumerate}

\hypertarget{torts}{%
\subsubsection{Torts}\label{torts}}

Any tort committed by the company does not usually give rise to
liability for the directors, even if they caused the tort to be
committed (because agency). But there are exceptions, if the director:

\begin{enumerate}
\tightlist
\item
  Commits a tort separate from that of the company (Standard Chartered
  Bank v Pakistan National Shipping Corporation (No 2) {[}2003{]} 1 All
  ER 173);
\item
  Voluntarily assumes personal responsibility for the tort by creating a
  special relationship between the director and the third party
  (Williams v Natural Life Health Foods Ltd {[}1998{]} 1 WLR 830); or
\item
  Procures or induces the company to commit a tort where the director
  acts beyond the director's constitutional role in the company (MCA
  Records Inc v Charly Records Ltd {[}2003{]} 1 BCLC 93).
\end{enumerate}

\hypertarget{debts}{%
\subsubsection{Debts}\label{debts}}

Generally, directors have no liability whatsoever for any debts they
incur on the company's behalf.

Exceptions:

\begin{enumerate}
\tightlist
\item
  If the director engages in misconduct (Contex Drouzbha v Wiseman and
  another {[}2007{]} EWCA Civ 120). Punitive liability.
\item
  If directors have personally guaranteed a loan to the company and the
  company defaults under the terms of that loan. Personal contractual
  liability.
\end{enumerate}

\hypertarget{failure-to-maintain-company-records}{%
\subsubsection{Failure to Maintain Company
Records}\label{failure-to-maintain-company-records}}

\begin{itemize}
\tightlist
\item
  Failure to keep proper company records: those in default (including
  directors) liable to a fine (s 1135(3)).
\item
  Failure to keep proper accounting records: the directors in default
  (and other officers) may be sentenced to up to two years' imprisonment
  (s 389).
\item
  There are specific offences which will be committed for failure to
  keep a particular register up to date - more fines.
\end{itemize}

\hypertarget{financial-records}{%
\subsubsection{Financial Records}\label{financial-records}}

Liability for failures can be civil or criminal.

\hypertarget{share-capital-transactions}{%
\subsubsection{Share Capital
Transactions}\label{share-capital-transactions}}

The directors of a company may be liable for breaches of the statutory
rules seeking to protect the company's share capital; e.g.,

\begin{itemize}
\tightlist
\item
  if there is prohibited Financial assistance (s 680(1)), or
\item
  if a required directors' statement is unreasonably given for a
  redemption or buy-back of shares out of capital (s 715).
\item
  Punishable by up to two years' imprisonment and a fine.
\end{itemize}

\hypertarget{health-and-safety-breaches}{%
\subsubsection{Health and Safety
Breaches}\label{health-and-safety-breaches}}

The directors (and other officers) may be criminally liable for breaches
of health and safety legislation.

\begin{itemize}
\tightlist
\item
  Where the company's actions have led to the death of an individual
  (e.g., an employee or a customer) and management failure is to blame,
  the directors may be prosecuted for the common law offence of gross
  negligence manslaughter.
\item
  The Corporate Manslaughter and Corporate Homicide Act 2007 (CMCHA
  2007) may be used only against a company--see Liability in tort and
  crime.
\end{itemize}

\hypertarget{bribery}{%
\subsubsection{Bribery}\label{bribery}}

A maximum prison sentence of 10 years and/or an unlimited fine may be
imposed on directors under the Bribery Act 2010 for four criminal
offences relating to bribery.

\hypertarget{political-donations}{%
\subsubsection{Political Donations}\label{political-donations}}

If a company makes a donation to a political party under Part 14 of the
CA 2006 without shareholder approval (where required), the directors
will be personally liable under s 369(2) of the Act to reimburse the
company to the amount of that donation.

\hypertarget{substantial-property-transactions-ss-190-196}{%
\section{Substantial Property Transactions (ss
190-196)}\label{substantial-property-transactions-ss-190-196}}

If a director in their personal capacity, or someone "connected" to the
director, buys something from or sells something to the company, the
consent of the shareholders by resolution \textsuperscript{{[}1{]}} is
necessary if the asset being bought or sold is a substantial non-cash
asset.

\begin{env-246868e5-1b14-46c9-8aed-b83e61569202}

Substantial (s 191)

An asset is a substantial asset in relation to a company if its value:

\begin{itemize}
\tightlist
\item
  Exceeds 10\% of the company's asset value and is more than £5,000, or
\item
  Exceeds £100,000
\end{itemize}

These values should be determined at the date the contract is entered
into (s 191(5)). The value of a series of contracts will be aggregated
to determine if the threshold has been met (s 190(5)).

\end{env-246868e5-1b14-46c9-8aed-b83e61569202}

\begin{env-246868e5-1b14-46c9-8aed-b83e61569202}

Non-cash

Any property, or interest in property, other than cash (s 1163). So note
this inclues property rights (e.g., lease) and security (e.g., mortgage,
floating charge).

\end{env-246868e5-1b14-46c9-8aed-b83e61569202}

\hypertarget{requirements}{%
\subsection{Requirements}\label{requirements}}

Must be:

\begin{enumerate}
\tightlist
\item
  an arrangement (also referred to in the CA 2006 at times as a
  `transaction');

  \begin{itemize}
  \tightlist
  \item
    This will usually involve a contract between the parties, but also
    covers looser arrangements.
  \end{itemize}
\item
  between the company and:

  \begin{itemize}
  \tightlist
  \item
    one of its directors, or
  \item
    a person `connected' with such a director, or
  \item
    one of its holding company directors, or
  \item
    a person `connected' with a holding company director;
  \end{itemize}
\item
  involving the acquisition of a `non-cash' asset;
\item
  that is `substantial'.
\end{enumerate}

\hypertarget{connected-persons}{%
\subsection{Connected Persons}\label{connected-persons}}

The definition of ``persons connected with a director'' is set out in s
252-254 CA 2006. Complicated, but key categories are:

\begin{enumerate}
\tightlist
\item
  Members of director's family:

  \begin{itemize}
  \tightlist
  \item
    spouse or civil partner,
  \item
    partner with whom the director lives in an 'enduring family
    relationship',
  \item
    parents,
  \item
    children or step-children (s 253),
  \item
    partner's children or step-children if they are under 18 and live
    with the director.
  \end{itemize}
\item
  Companies in which the director (and others connected with them) hold
  {\(> 20\%\)} of shares (s 254)
\item
  Business partner of the director/ those connected with them (s
  252(2)(d))
\item
  Trustees of a trust the beneficiaries of which include the director or
  those connected with them (s 252(2)(c))
\end{enumerate}

References to a director above also covers shadow directors (s
223(1)(b)).

\hypertarget{effect}{%
\subsection{Effect}\label{effect}}

Shareholder approval must be given either before the transaction is
entered into, or after, provided that the transaction is made
conditional on approval being obtained. This can be an ordinary
resolution at a general meeting or written resolution.

Possible for shareholders to approve the transaction within a reasonable
time after the event (s 196) but this does not absolve directors of
potential liability under s 195.

\hypertarget{exceptions-1}{%
\subsection{Exceptions}\label{exceptions-1}}

Approval by shareholders is not required:

\begin{enumerate}
\tightlist
\item
  Where the company is a wholly-owned subsidiary of another;
\item
  A contract between a company and a director in their capacity as a
  director of the company (s 192(a));
\item
  A contract between a holding company and one of its subsidiaries (s
  192(b)(i));
\item
  A contract between two wholly-owned subsidiaries of the same holding
  company (s 192(b)(ii));
\item
  If the company is insolvent and in administration/ liquidation (s
  193(1)).
\end{enumerate}

\hypertarget{remedies-1}{%
\subsubsection{Remedies}\label{remedies-1}}

s 195 sets out the consequences of a substantial property transaction
being entered into without shareholder approval.

\begin{env-136e818f-6bd8-45c9-b7d8-e6b8925f9e62}

s 195(2)

The arrangement, and any transaction entered into in pursuance of the
arrangement (whether by the company or any other person), is voidable at
the instance of the company, unless---

\begin{itemize}
\tightlist
\item
  (a) restitution of any money or other asset that was the subject
  matter of the arrangement or transaction is no longer possible,
\item
  (b) the company has been indemnified in pursuance of this section by
  any other persons for the loss or damage suffered by it, or
\item
  (c) rights acquired in good faith, for value and without actual notice
  of the contravention by a person who is not a party to the arrangement
  or transaction would be affected by the avoidance.
\end{itemize}

\end{env-136e818f-6bd8-45c9-b7d8-e6b8925f9e62}

For the directors involved (and those connected under s 195(4)),
consequences set out in s195(3). They are liable to account to the
company for any profits made and to indemnify company for any losses
incurred.

\begin{env-136e818f-6bd8-45c9-b7d8-e6b8925f9e62}

s 195(3)

Whether or not the arrangement or any such transaction has been avoided,
each of the persons specified in subsection (4) is liable---

\begin{itemize}
\tightlist
\item
  (a) to account to the company for any gain that he has made directly
  or indirectly by the arrangement or transaction, and
\item
  (b) (jointly and severally with any other person so liable under this
  section) to indemnify the company for any loss or damage resulting
  from the arrangement or transaction.
\end{itemize}

\end{env-136e818f-6bd8-45c9-b7d8-e6b8925f9e62}

\begin{env-136e818f-6bd8-45c9-b7d8-e6b8925f9e62}

s 195(4)

The persons so liable are---

\begin{itemize}
\tightlist
\item
  (a) any director of the company or of its holding company with whom
  the company entered into the arrangement in contravention of section
  190,
\item
  (b) any person with whom the company entered into the arrangement in
  contravention of that section who is connected with a director of the
  company or of its holding company,
\item
  (c) the director of the company or of its holding company with whom
  any such person is connected, and
\item
  (d) any other director of the company who authorised the arrangement
  or any transaction entered into in pursuance of such an arrangement.
\end{itemize}

\end{env-136e818f-6bd8-45c9-b7d8-e6b8925f9e62}

MERMAID

\hypertarget{loans-to-directors}{%
\section{Loans to Directors}\label{loans-to-directors}}

\begin{env-136e818f-6bd8-45c9-b7d8-e6b8925f9e62}

s 197 - Loans to directors: requirement of members' approval

(1) A company may not---

\begin{itemize}
\tightlist
\item
  (a) make a loan to a director of the company or of its holding
  company, or
\item
  (b) give a guarantee or provide security in connection with a loan
  made by any person to such a director,
\end{itemize}

unless the transaction has been approved by a resolution of the members
of the company.

(2) If the director is a director of the company's holding company, the
transaction must also have been approved by a resolution of the members
of the holding company.

(3) A resolution approving a transaction to which this section applies
must not be passed unless a memorandum setting out the matters mentioned
in subsection (4) is made available to members---

\begin{itemize}
\tightlist
\item
  (a) in the case of a written resolution, by being sent or submitted to
  every eligible member at or before the time at which the proposed
  resolution is sent or submitted to him;
\item
  (b) in the case of a resolution at a meeting, by being made available
  for inspection by members of the company both---

  \begin{itemize}
  \tightlist
  \item
    (i) at the company's registered office for not less than 15 days
    ending with the date of the meeting, and
  \item
    (ii) at the meeting itself.
  \end{itemize}
\end{itemize}

(4)The matters to be disclosed are---

\begin{itemize}
\tightlist
\item
  (a) the nature of the transaction,
\item
  (b) the amount of the loan and the purpose for which it is required,
  and
\item
  (c) the extent of the company's liability under any transaction
  connected with the loan.
\end{itemize}

(5)No approval is required under this section on the part of the members
of a body corporate that---

\begin{itemize}
\tightlist
\item
  (a) is not a UK-registered company, or
\item
  (b) is a wholly-owned subsidiary of another body corporate.
\end{itemize}

\end{env-136e818f-6bd8-45c9-b7d8-e6b8925f9e62}

Company loans to directors, although permitted, may also be subject to
requirement of shareholder approval by ordinary resolution. Before
shareholders can be asked to approve a loan to a director (or a director
of a holding company) they must be given information about nature of
transaction, amount of loan, purpose of loan and company's liability (s
197(4) in the form of a memorandum
(\href{https://www.legislation.gov.uk/ukpga/2006/46/section/197}{s
197(3) CA 2006}).

Approval is also needed when the loan/ security arrangements are entered
into with a person connected with a director and the company is a public
company/ company associated with a public company (ss 198201).

Shareholders can approve the transaction after the event (s 214) but
does not absolve directors of potential liability under s 213.

``Loan'' only defined through the case law: in Champagne Perrier-Jouet
SA v HH Finch Ltd {[}1982{]} 1 WLR 1359 held to mean ``a sum of money
lent for a period of time, to be returned in money or money's worth''.

Loan vs remuneration: Currencies Direct Ltd v Ellis {[}2003{]} BCLC 482,
CA.

\hypertarget{quasi-loans-and-credit-transactions}{%
\subsection{Quasi-loans and Credit
Transactions}\label{quasi-loans-and-credit-transactions}}

\begin{env-246868e5-1b14-46c9-8aed-b83e61569202}

Quasi-loan

Arises where the company meets some financial obligation of the director
on the understanding that the director will reimburse it later (s
199(1)).

\end{env-246868e5-1b14-46c9-8aed-b83e61569202}

\begin{env-246868e5-1b14-46c9-8aed-b83e61569202}

Credit transaction

The company provides the director with goods or services and the
director pays the company for these in instalments over time.

\end{env-246868e5-1b14-46c9-8aed-b83e61569202}

For a public company, or a company associated with a public company,
prior shareholder approval is needed before entering into either a
quasi-loan (s 198(3)) or credit transaction (s 201(3)). The restrictions
extend to where these arrangements are made to a connected person of a
public company or a company associated with a public company (CA 2006,
ss 200(2) and 201(2)).

\hypertarget{exceptions-2}{%
\subsubsection{Exceptions}\label{exceptions-2}}

Exceptions to the requirement for shareholder approval:

\begin{longtable}[]{@{}ll@{}}
\toprule()
Section of CA 2006 & Rule \\
\midrule()
\endhead
s 204 & Expenditure on company business ({\(\leq \pounds 50,000\)}) \\
s 205 & Loans for defending proceedings brought against a director \\
s 206 & Loans for defending regulatory actions or investigations \\
s 207 & Minor and business transactions---loans
{\(\leq \pounds 10,000\)} do not require shareholder approval \\
s 208 & Intra group transactions \\
s 209 & Money lending companies (where loan is made in the ordinary
course of business for the company) \\
\bottomrule()
\end{longtable}

\hypertarget{remedies-s-213}{%
\subsubsection{Remedies (s 213)}\label{remedies-s-213}}

In relation to transaction, the arrangement is voidable at the instance
of the company, subject to certain exceptions:

\begin{env-136e818f-6bd8-45c9-b7d8-e6b8925f9e62}

s 213(2)

The transaction or arrangement is voidable at the instance of the
company, unless---

\begin{itemize}
\tightlist
\item
  (a) restitution of any money or other asset that was the subject
  matter of the transaction or arrangement is no longer possible,
\item
  (b) the company has been indemnified for any loss or damage resulting
  from the transaction or arrangement, or
\item
  (c) rights acquired in good faith, for value and without actual notice
  of the contravention by a person who is not a party to the transaction
  or arrangement would be affected by the avoidance.
\end{itemize}

\end{env-136e818f-6bd8-45c9-b7d8-e6b8925f9e62}

Directors involved (and those connected) liable to account to the
company for any profits made and to indemnify the company for any loss
incurred. It is possible for shareholders to ratify the failure to
obtain shareholder approval (s 214).

\hypertarget{service-contracts-ss-188-189}{%
\subsubsection{Service Contracts (ss
188-189)}\label{service-contracts-ss-188-189}}

An executive director is an employee of the company, so should be given
a written employment contract --- a service contract. No automatic
entitlement for directors to be paid for services.

\begin{itemize}
\tightlist
\item
  Company must keep copy of directors' service contracts/ memoranda of
  the terms
  (\href{https://www.legislation.gov.uk/ukpga/2006/46/section/228}{s 228
  CA 2006}).
\item
  Shareholders have right to inspect copies of directors' service
  contracts/ memoranda
  (\href{https://www.legislation.gov.uk/ukpga/2006/46/section/229}{s 229
  CA 2006}), must be provided within 7 days of request.
\end{itemize}

\hypertarget{long-term-service-contracts}{%
\paragraph{Long Term Service
Contracts}\label{long-term-service-contracts}}

\begin{itemize}
\tightlist
\item
  \href{https://www.gov.uk/government/publications/model-articles-for-private-companies-limited-by-shares/model-articles-for-private-companies-limited-by-shares\#remuneration}{Art
  19 MA} means that terms of an individual director's service contract
  are for the board to determine. Only requires approval of resolution
  of board of directors, though shareholder approval required
  (\href{https://www.legislation.gov.uk/ukpga/2006/46/section/188}{s 188
  CA 2006}) to enter long-term service contracts (where service contract
  has guaranteed term which is or may be \textgreater{} 2 years).
\item
  If shareholder approval not given, term incorporated into service
  agreement \textbf{void} under s 189(a) CA 2006. Additionally, under s
  189(b) CA 2006, the service contract will be deemed to contain a term
  entitling company to terminate the contract at any time, by giving
  reasonable notice.
\end{itemize}

\hypertarget{payments-for-loss-of-office-ss-215-222}{%
\subsubsection{Payments for Loss of Office (ss
215-222)}\label{payments-for-loss-of-office-ss-215-222}}

s 217: any payment for loss of office to a director needs to be approved
by shareholders by way of an ordinary resolution.

\begin{env-136e818f-6bd8-45c9-b7d8-e6b8925f9e62}

s 217 - Payment by company: requirement of members' approval

(1) A company may not make a payment for loss of office to a director of
the company unless the payment has been approved by a resolution of the
members of the company.

(2) A company may not make a payment for loss of office to a director of
its holding company unless the payment has been approved by a resolution
of the members of each of those companies.

(3) A resolution approving a payment to which this section applies must
not be passed unless a memorandum setting out particulars of the
proposed payment (including its amount) is made available to the members
of the company whose approval is sought---

\begin{itemize}
\tightlist
\item
  (a) in the case of a written resolution, by being sent or submitted to
  every eligible member at or before the time at which the proposed
  resolution is sent or submitted to him;
\item
  (b) in the case of a resolution at a meeting, by being made available
  for inspection by the members both---

  \begin{itemize}
  \tightlist
  \item
    (i) at the company's registered office for not less than 15 days
    ending with the date of the meeting, and
  \item
    (ii) at the meeting itself.
  \end{itemize}
\end{itemize}

(4) No approval is required under this section on the part of the
members of a body corporate that---

\begin{itemize}
\tightlist
\item
  (a) is not a UK-registered company, or
\item
  (b) is a wholly-owned subsidiary of another body corporate.
\end{itemize}

\end{env-136e818f-6bd8-45c9-b7d8-e6b8925f9e62}

Two exceptions:

\begin{enumerate}
\tightlist
\item
  When the payment is made in good faith (s 220)

  \begin{enumerate}
  \tightlist
  \item
    In discharge of an existing legal obligation
  \item
    By way of damages for breach of such an obligation
  \item
    By way of settlement or compromise of a claim arising in connection
    with the termination of a person's office or employment
  \item
    By way of pension in respect of past services
  \end{enumerate}
\item
  When the payment is less than £200 (s 221)
\end{enumerate}

If no shareholder approval obtained, the director holds payment on trust
for the company (s 222), and any director authorising payment is jointly
and severally liable to the company for any resulting loss.

\hypertarget{summary-1}{%
\subsection{Summary}\label{summary-1}}

\begin{longtable}[]{@{}llllll@{}}
\toprule()
& Substantial property transaction & Loan to directors ltd & Loan to
directors plc & Termination payment & Long-term service contract \\
\midrule()
\endhead
Documents must be circulated to shareholders with the written resolution
or made available at/ before GM & N & Y & Y & Y & Y \\
Transaction void/ voidable if not properly approved & Y & Y & Y & N &
N \\
Director concerned and other directors who approved the transaction are
liable to the company if it is not properly approved & Y & Y & Y & Y &
N \\
Rules also apply to transactions involving connected persons & Y & N & Y
& Y & N \\
Shareholders of a wholly owned subsidiary don't have to pass a
resolution & Y & Y & Y & Y & Y \\
\bottomrule()
\end{longtable}

\begin{center}\rule{0.5\linewidth}{0.5pt}\end{center}

\begin{enumerate}
\tightlist
\item
  \protect\hypertarget{fn-1-ded2dac9c9f91efc}{}{Strictly, Article 190
  doesn't specify resolution required, so by s 281(3) can pass an
  ordinary resolution}
\end{enumerate}

\end{document}
