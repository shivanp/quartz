% Options for packages loaded elsewhere
\PassOptionsToPackage{unicode}{hyperref}
\PassOptionsToPackage{hyphens}{url}
%
\documentclass[
]{article}
\usepackage{amsmath,amssymb}
\usepackage{lmodern}
\usepackage{iftex}
\ifPDFTeX
  \usepackage[T1]{fontenc}
  \usepackage[utf8]{inputenc}
  \usepackage{textcomp} % provide euro and other symbols
\else % if luatex or xetex
  \usepackage{unicode-math}
  \defaultfontfeatures{Scale=MatchLowercase}
  \defaultfontfeatures[\rmfamily]{Ligatures=TeX,Scale=1}
\fi
% Use upquote if available, for straight quotes in verbatim environments
\IfFileExists{upquote.sty}{\usepackage{upquote}}{}
\IfFileExists{microtype.sty}{% use microtype if available
  \usepackage[]{microtype}
  \UseMicrotypeSet[protrusion]{basicmath} % disable protrusion for tt fonts
}{}
\makeatletter
\@ifundefined{KOMAClassName}{% if non-KOMA class
  \IfFileExists{parskip.sty}{%
    \usepackage{parskip}
  }{% else
    \setlength{\parindent}{0pt}
    \setlength{\parskip}{6pt plus 2pt minus 1pt}}
}{% if KOMA class
  \KOMAoptions{parskip=half}}
\makeatother
\usepackage{xcolor}
\usepackage[margin=1in]{geometry}
\usepackage{longtable,booktabs,array}
\usepackage{calc} % for calculating minipage widths
% Correct order of tables after \paragraph or \subparagraph
\usepackage{etoolbox}
\makeatletter
\patchcmd\longtable{\par}{\if@noskipsec\mbox{}\fi\par}{}{}
\makeatother
% Allow footnotes in longtable head/foot
\IfFileExists{footnotehyper.sty}{\usepackage{footnotehyper}}{\usepackage{footnote}}
\makesavenoteenv{longtable}
\setlength{\emergencystretch}{3em} % prevent overfull lines
\providecommand{\tightlist}{%
  \setlength{\itemsep}{0pt}\setlength{\parskip}{0pt}}
\setcounter{secnumdepth}{-\maxdimen} % remove section numbering
\usepackage{xcolor}
\definecolor{aliceblue}{HTML}{F0F8FF}
\definecolor{antiquewhite}{HTML}{FAEBD7}
\definecolor{aqua}{HTML}{00FFFF}
\definecolor{aquamarine}{HTML}{7FFFD4}
\definecolor{azure}{HTML}{F0FFFF}
\definecolor{beige}{HTML}{F5F5DC}
\definecolor{bisque}{HTML}{FFE4C4}
\definecolor{black}{HTML}{000000}
\definecolor{blanchedalmond}{HTML}{FFEBCD}
\definecolor{blue}{HTML}{0000FF}
\definecolor{blueviolet}{HTML}{8A2BE2}
\definecolor{brown}{HTML}{A52A2A}
\definecolor{burlywood}{HTML}{DEB887}
\definecolor{cadetblue}{HTML}{5F9EA0}
\definecolor{chartreuse}{HTML}{7FFF00}
\definecolor{chocolate}{HTML}{D2691E}
\definecolor{coral}{HTML}{FF7F50}
\definecolor{cornflowerblue}{HTML}{6495ED}
\definecolor{cornsilk}{HTML}{FFF8DC}
\definecolor{crimson}{HTML}{DC143C}
\definecolor{cyan}{HTML}{00FFFF}
\definecolor{darkblue}{HTML}{00008B}
\definecolor{darkcyan}{HTML}{008B8B}
\definecolor{darkgoldenrod}{HTML}{B8860B}
\definecolor{darkgray}{HTML}{A9A9A9}
\definecolor{darkgreen}{HTML}{006400}
\definecolor{darkgrey}{HTML}{A9A9A9}
\definecolor{darkkhaki}{HTML}{BDB76B}
\definecolor{darkmagenta}{HTML}{8B008B}
\definecolor{darkolivegreen}{HTML}{556B2F}
\definecolor{darkorange}{HTML}{FF8C00}
\definecolor{darkorchid}{HTML}{9932CC}
\definecolor{darkred}{HTML}{8B0000}
\definecolor{darksalmon}{HTML}{E9967A}
\definecolor{darkseagreen}{HTML}{8FBC8F}
\definecolor{darkslateblue}{HTML}{483D8B}
\definecolor{darkslategray}{HTML}{2F4F4F}
\definecolor{darkslategrey}{HTML}{2F4F4F}
\definecolor{darkturquoise}{HTML}{00CED1}
\definecolor{darkviolet}{HTML}{9400D3}
\definecolor{deeppink}{HTML}{FF1493}
\definecolor{deepskyblue}{HTML}{00BFFF}
\definecolor{dimgray}{HTML}{696969}
\definecolor{dimgrey}{HTML}{696969}
\definecolor{dodgerblue}{HTML}{1E90FF}
\definecolor{firebrick}{HTML}{B22222}
\definecolor{floralwhite}{HTML}{FFFAF0}
\definecolor{forestgreen}{HTML}{228B22}
\definecolor{fuchsia}{HTML}{FF00FF}
\definecolor{gainsboro}{HTML}{DCDCDC}
\definecolor{ghostwhite}{HTML}{F8F8FF}
\definecolor{gold}{HTML}{FFD700}
\definecolor{goldenrod}{HTML}{DAA520}
\definecolor{gray}{HTML}{808080}
\definecolor{green}{HTML}{008000}
\definecolor{greenyellow}{HTML}{ADFF2F}
\definecolor{grey}{HTML}{808080}
\definecolor{honeydew}{HTML}{F0FFF0}
\definecolor{hotpink}{HTML}{FF69B4}
\definecolor{indianred}{HTML}{CD5C5C}
\definecolor{indigo}{HTML}{4B0082}
\definecolor{ivory}{HTML}{FFFFF0}
\definecolor{khaki}{HTML}{F0E68C}
\definecolor{lavender}{HTML}{E6E6FA}
\definecolor{lavenderblush}{HTML}{FFF0F5}
\definecolor{lawngreen}{HTML}{7CFC00}
\definecolor{lemonchiffon}{HTML}{FFFACD}
\definecolor{lightblue}{HTML}{ADD8E6}
\definecolor{lightcoral}{HTML}{F08080}
\definecolor{lightcyan}{HTML}{E0FFFF}
\definecolor{lightgoldenrodyellow}{HTML}{FAFAD2}
\definecolor{lightgray}{HTML}{D3D3D3}
\definecolor{lightgreen}{HTML}{90EE90}
\definecolor{lightgrey}{HTML}{D3D3D3}
\definecolor{lightpink}{HTML}{FFB6C1}
\definecolor{lightsalmon}{HTML}{FFA07A}
\definecolor{lightseagreen}{HTML}{20B2AA}
\definecolor{lightskyblue}{HTML}{87CEFA}
\definecolor{lightslategray}{HTML}{778899}
\definecolor{lightslategrey}{HTML}{778899}
\definecolor{lightsteelblue}{HTML}{B0C4DE}
\definecolor{lightyellow}{HTML}{FFFFE0}
\definecolor{lime}{HTML}{00FF00}
\definecolor{limegreen}{HTML}{32CD32}
\definecolor{linen}{HTML}{FAF0E6}
\definecolor{magenta}{HTML}{FF00FF}
\definecolor{maroon}{HTML}{800000}
\definecolor{mediumaquamarine}{HTML}{66CDAA}
\definecolor{mediumblue}{HTML}{0000CD}
\definecolor{mediumorchid}{HTML}{BA55D3}
\definecolor{mediumpurple}{HTML}{9370DB}
\definecolor{mediumseagreen}{HTML}{3CB371}
\definecolor{mediumslateblue}{HTML}{7B68EE}
\definecolor{mediumspringgreen}{HTML}{00FA9A}
\definecolor{mediumturquoise}{HTML}{48D1CC}
\definecolor{mediumvioletred}{HTML}{C71585}
\definecolor{midnightblue}{HTML}{191970}
\definecolor{mintcream}{HTML}{F5FFFA}
\definecolor{mistyrose}{HTML}{FFE4E1}
\definecolor{moccasin}{HTML}{FFE4B5}
\definecolor{navajowhite}{HTML}{FFDEAD}
\definecolor{navy}{HTML}{000080}
\definecolor{oldlace}{HTML}{FDF5E6}
\definecolor{olive}{HTML}{808000}
\definecolor{olivedrab}{HTML}{6B8E23}
\definecolor{orange}{HTML}{FFA500}
\definecolor{orangered}{HTML}{FF4500}
\definecolor{orchid}{HTML}{DA70D6}
\definecolor{palegoldenrod}{HTML}{EEE8AA}
\definecolor{palegreen}{HTML}{98FB98}
\definecolor{paleturquoise}{HTML}{AFEEEE}
\definecolor{palevioletred}{HTML}{DB7093}
\definecolor{papayawhip}{HTML}{FFEFD5}
\definecolor{peachpuff}{HTML}{FFDAB9}
\definecolor{peru}{HTML}{CD853F}
\definecolor{pink}{HTML}{FFC0CB}
\definecolor{plum}{HTML}{DDA0DD}
\definecolor{powderblue}{HTML}{B0E0E6}
\definecolor{purple}{HTML}{800080}
\definecolor{red}{HTML}{FF0000}
\definecolor{rosybrown}{HTML}{BC8F8F}
\definecolor{royalblue}{HTML}{4169E1}
\definecolor{saddlebrown}{HTML}{8B4513}
\definecolor{salmon}{HTML}{FA8072}
\definecolor{sandybrown}{HTML}{F4A460}
\definecolor{seagreen}{HTML}{2E8B57}
\definecolor{seashell}{HTML}{FFF5EE}
\definecolor{sienna}{HTML}{A0522D}
\definecolor{silver}{HTML}{C0C0C0}
\definecolor{skyblue}{HTML}{87CEEB}
\definecolor{slateblue}{HTML}{6A5ACD}
\definecolor{slategray}{HTML}{708090}
\definecolor{slategrey}{HTML}{708090}
\definecolor{snow}{HTML}{FFFAFA}
\definecolor{springgreen}{HTML}{00FF7F}
\definecolor{steelblue}{HTML}{4682B4}
\definecolor{tan}{HTML}{D2B48C}
\definecolor{teal}{HTML}{008080}
\definecolor{thistle}{HTML}{D8BFD8}
\definecolor{tomato}{HTML}{FF6347}
\definecolor{turquoise}{HTML}{40E0D0}
\definecolor{violet}{HTML}{EE82EE}
\definecolor{wheat}{HTML}{F5DEB3}
\definecolor{white}{HTML}{FFFFFF}
\definecolor{whitesmoke}{HTML}{F5F5F5}
\definecolor{yellow}{HTML}{FFFF00}
\definecolor{yellowgreen}{HTML}{9ACD32}
\usepackage[most]{tcolorbox}

\usepackage{ifthen}
\provideboolean{admonitiontwoside}
\makeatletter%
\if@twoside%
\setboolean{admonitiontwoside}{true}
\else%
\setboolean{admonitiontwoside}{false}
\fi%
\makeatother%

\newenvironment{env-26031ed7-b728-4e96-9a1e-354c9202051b}
{
    \savenotes\tcolorbox[blanker,breakable,left=5pt,borderline west={2pt}{-4pt}{firebrick}]
}
{
    \endtcolorbox\spewnotes
}
                

\newenvironment{env-440d3209-5a60-4bb6-8a13-881ba9304f97}
{
    \savenotes\tcolorbox[blanker,breakable,left=5pt,borderline west={2pt}{-4pt}{blue}]
}
{
    \endtcolorbox\spewnotes
}
                

\newenvironment{env-99928e31-e9c7-4b95-96ac-39649baaf1f5}
{
    \savenotes\tcolorbox[blanker,breakable,left=5pt,borderline west={2pt}{-4pt}{green}]
}
{
    \endtcolorbox\spewnotes
}
                

\newenvironment{env-862347ee-a221-4f52-a3b4-3eedf85d51d9}
{
    \savenotes\tcolorbox[blanker,breakable,left=5pt,borderline west={2pt}{-4pt}{aquamarine}]
}
{
    \endtcolorbox\spewnotes
}
                

\newenvironment{env-c2ea7781-55a7-40ef-9219-ca9616c6b82e}
{
    \savenotes\tcolorbox[blanker,breakable,left=5pt,borderline west={2pt}{-4pt}{orange}]
}
{
    \endtcolorbox\spewnotes
}
                

\newenvironment{env-03b4608f-c228-4088-bd8a-ec46ef93ef4f}
{
    \savenotes\tcolorbox[blanker,breakable,left=5pt,borderline west={2pt}{-4pt}{blue}]
}
{
    \endtcolorbox\spewnotes
}
                

\newenvironment{env-751485c0-9cbb-4b3b-8497-6bad2ef120e5}
{
    \savenotes\tcolorbox[blanker,breakable,left=5pt,borderline west={2pt}{-4pt}{gold}]
}
{
    \endtcolorbox\spewnotes
}
                

\newenvironment{env-23835a2c-94f4-4b0e-afdc-60cc8d24b348}
{
    \savenotes\tcolorbox[blanker,breakable,left=5pt,borderline west={2pt}{-4pt}{darkred}]
}
{
    \endtcolorbox\spewnotes
}
                

\newenvironment{env-0630ef60-d677-4921-bba8-06d563fa419c}
{
    \savenotes\tcolorbox[blanker,breakable,left=5pt,borderline west={2pt}{-4pt}{pink}]
}
{
    \endtcolorbox\spewnotes
}
                

\newenvironment{env-f9980082-0043-4b24-aae8-b6167468a442}
{
    \savenotes\tcolorbox[blanker,breakable,left=5pt,borderline west={2pt}{-4pt}{cyan}]
}
{
    \endtcolorbox\spewnotes
}
                

\newenvironment{env-26bb6fb1-abd0-489e-8e80-3f170ecd0052}
{
    \savenotes\tcolorbox[blanker,breakable,left=5pt,borderline west={2pt}{-4pt}{cyan}]
}
{
    \endtcolorbox\spewnotes
}
                

\newenvironment{env-0a540a95-48c3-4466-84d1-6ddd726278e0}
{
    \savenotes\tcolorbox[blanker,breakable,left=5pt,borderline west={2pt}{-4pt}{purple}]
}
{
    \endtcolorbox\spewnotes
}
                

\newenvironment{env-fb3a5115-52d0-44e7-be8e-116757f48c54}
{
    \savenotes\tcolorbox[blanker,breakable,left=5pt,borderline west={2pt}{-4pt}{darksalmon}]
}
{
    \endtcolorbox\spewnotes
}
                

\newenvironment{env-76b80ee8-bee9-4b5a-b97a-3d134ffa5488}
{
    \savenotes\tcolorbox[blanker,breakable,left=5pt,borderline west={2pt}{-4pt}{gray}]
}
{
    \endtcolorbox\spewnotes
}
                
\ifLuaTeX
  \usepackage{selnolig}  % disable illegal ligatures
\fi
\IfFileExists{bookmark.sty}{\usepackage{bookmark}}{\usepackage{hyperref}}
\IfFileExists{xurl.sty}{\usepackage{xurl}}{} % add URL line breaks if available
\urlstyle{same} % disable monospaced font for URLs
\hypersetup{
  pdftitle={Insolvency},
  hidelinks,
  pdfcreator={LaTeX via pandoc}}

\title{Insolvency}
\author{}
\date{}

\begin{document}
\maketitle

{
\setcounter{tocdepth}{3}
\tableofcontents
}
\hypertarget{proving-insolvency}{%
\section{Proving Insolvency}\label{proving-insolvency}}

Legislation: IA 1986. Aims of insolvency law:

\begin{enumerate}
\tightlist
\item
  Protect creditors of the company
\item
  Balance interests of competing groups of creditors
\item
  Promote corporate rescues
\item
  Control/ punish directors.
\end{enumerate}

\begin{env-99928e31-e9c7-4b95-96ac-39649baaf1f5}

s 122(1) IA 1986 - Circumstances in which company may be wound up by the
court.

A company may be wound up by the court if---

\begin{itemize}
\tightlist
\item
  (a) the company has by special resolution resolved that the company be
  wound up by the court,
\item
  (b) being a public company which was registered as such on its
  original incorporation, the company has not been issued with a trading
  certificate under section 761 of the Companies Act 2006 (requirement
  as to minimum share capital) and more than a year has expired since it
  was so registered,
\item
  (c) it is an old public company, within the meaning of the Schedule 3
  to the Companies Act 2006 (Consequential Amendments, Transitional
  Provisions and Savings) Order 2009,
\item
  (d) the company does not commence its business within a year from its
  incorporation or suspends its business for a whole year;
\item
  (f) \textbf{the company is unable to pay its debts,}
\item
  (g) the court is of the opinion that it is just and equitable that the
  company should be wound up.
\end{itemize}

\end{env-99928e31-e9c7-4b95-96ac-39649baaf1f5}

\begin{longtable}[]{@{}ll@{}}
\toprule()
Test & Description \\
\midrule()
\endhead
Cash flow test & An inability to pay debts as they fall due
(\href{https://www.legislation.gov.uk/ukpga/1986/45/section/123}{s
123(1)(e) IA 1985}) \\
Balance sheet test & Company's liabilities are greater than its assets
(s 123(2)) \\
Failure to comply with a statutory demand for a debt of over £750 within
3 weeks & s 123(1)(a) \\
Failure to satisfy enforcement of a judgment debt & A creditor has
enforced judgment against the company and attempted to execute the
judgment and the debt is still unsatisfied in full or in part (s
123(1)(b)) \\
\bottomrule()
\end{longtable}

\hypertarget{options}{%
\subsection{Options}\label{options}}

\begin{itemize}
\tightlist
\item
  Take steps to put the company into liquidation themselves
\item
  Talk to creditors to come to a compromise
\item
  Enter a CVA
\item
  Appoint administrator to take over the running of the company
\item
  Utilise moratorium procedure (CIGA 2020)
\item
  Enter into restructuring plan (CIGA 2020).
\item
  Enter into a scheme of arrangement.
\end{itemize}

\hypertarget{unpaid-creditor-options}{%
\subsection{Unpaid Creditor Options}\label{unpaid-creditor-options}}

\hypertarget{unsecured-creditor}{%
\subsubsection{Unsecured Creditor}\label{unsecured-creditor}}

\begin{itemize}
\tightlist
\item
  Serve a statutory demand for debts {\(> \pounds 750\)} (s 123(1)(a)),
  wait 3 weeks, then present a petition to court to put the company into
  liquidation.
\item
  Sue company, obtain judgment, execute judgment (s 123(1)(b) IA 1986),
  then present a petition to court to put the company into liquidation.
\item
  Suggest a CVA
\item
  Apply to court to put the company into administration.
\end{itemize}

\hypertarget{secured-creditor}{%
\subsubsection{Secured Creditor}\label{secured-creditor}}

As well as the above, a secured creditor can:

\begin{itemize}
\tightlist
\item
  Appoint an administrator out of court
\item
  Appoint an LPA receiver
\item
  Appoint an administrative receiver (if security for the debt was
  created before 15/09/03)
\end{itemize}

\begin{env-c2ea7781-55a7-40ef-9219-ca9616c6b82e}

Warning

If serving a statutory demand, the secured creditor must be careful to
only serve this for the unsecured element of the debt, else it may lose
its security.

\end{env-c2ea7781-55a7-40ef-9219-ca9616c6b82e}

\hypertarget{directors-duties-on-insolvency}{%
\section{Directors' Duties on
Insolvency}\label{directors-duties-on-insolvency}}

Note that under s 172(3), the duty to promote the success of the company
for the benefit of members as a whole shifts to the benefit of
creditors.

\hypertarget{misfeasance}{%
\subsection{Misfeasance}\label{misfeasance}}

Directors owe duties to the company under s 171178 CA 2006. Breach of
any such duties is generally actionable by the company, though
shareholders may be able to bring a claim for Unfair Prejudice, just and
equitable winding up, or derivative claims on behalf of the company.

On a winding up, typically the liquidator brings an action against the
directors under
\href{https://www.legislation.gov.uk/ukpga/1986/45/section/212}{s 212 IA
1986} for any breaches of duty committed by them.

s 212 does not create any new liability or rights, but simply provides a
summary procedure to enable the company (acting by its liquidators) to
pursue claims against directors who have breached their duties.

Where a person's liability is established, the court may order that
person to compensate the company in respect of the money or property
misapplied.

\hypertarget{bringing-a-claim}{%
\subsubsection{Bringing a Claim}\label{bringing-a-claim}}

A claim may be brought by:

\begin{itemize}
\tightlist
\item
  A liquidator
\item
  The Official Receiver
\item
  Any creditor or contributory
\end{itemize}

The burden of proof is on the claimants to establish misfeasance on the
part of the director/ other defendant (Mullarkey v Broad {[}2008{]} 1
BCLC 638).

\hypertarget{against-whom-may-a-claim-be-brought}{%
\subsubsection{Against Whom May a Claim Be
Brought?}\label{against-whom-may-a-claim-be-brought}}

s 212(1): a claim in misfeasance may be brought against:

\begin{enumerate}
\tightlist
\item
  Any person who is or has been an officer of the company (including
  present or former directors, managers or secretaries of the company)
\item
  Any others who acted in the promotion, formation or management of the
  company; and
\item
  A liquidator or administrative receiver (a claim for misfeasance can
  also be brought against an administrator under Schedule B1 to the IA
  1986).
\end{enumerate}

See Re Centralcrest Engineering Co Ltd {[}2000{]} BCC 727 for claim
against liquidator.

\hypertarget{scope}{%
\subsubsection{Scope}\label{scope}}

Covers the whole spectrum of directors' duties. Includes:

\begin{itemize}
\tightlist
\item
  Misapplication of any money or assets of the company
\item
  Breach of a statutory provision or duty, e.g.,

  \begin{itemize}
  \tightlist
  \item
    Unlawful loans to a director;
  \item
    Director entering contract with his own company and failing to
    notify board
    (\href{https://www.legislation.gov.uk/ukpga/2006/46/section/177}{s
    177 CA 2006});
  \item
    Failing to seek prior General Meeting approval where a director has
    entered into a substantial property transaction
    (\href{https://www.legislation.gov.uk/ukpga/2006/46/section/190}{s
    190 CA 2006}); and
  \item
    A director failing to act within his powers
    (\href{https://www.legislation.gov.uk/ukpga/2006/46/section/171}{s
    171 CA 2006})
  \item
    Directors responsible for transactions at an undervalue as provided
    in s 238 or preferences as provided in s 239 may thereby commit a
    misfeasance;
  \item
    Breach of the duty to exercise reasonable care, skill and diligence,
    i.e., in negligence
    (\href{https://www.legislation.gov.uk/ukpga/2006/46/section/174}{s
    174 CA 2006}).
  \end{itemize}
\end{itemize}

\hypertarget{remedies}{%
\subsubsection{Remedies}\label{remedies}}

The court will examine the conduct of the director/ other person against
whom the claim for misfeasance has been brought, and make an order for
repayment, restoration or contribution to the company's assets as it
thinks just.

The director may claim relief under
\href{https://www.legislation.gov.uk/ukpga/2006/46/section/1157}{s 1157
CA 2006}, where the court is satisfied that the director acted honestly
and reasonably, and, having regard to all the circumstances of the case,
ought fairly to be excused.

A finding of misfeasance is also a relevant factor when the court
considers whether to make a disqualification order against a director
for unfitness under
\href{https://www.legislation.gov.uk/ukpga/1986/46/section/6}{s 6
Company Directors' Disqualification Act 1986}.

\hypertarget{ratification}{%
\subsubsection{Ratification}\label{ratification}}

Ratification by the shareholders under
\href{https://www.legislation.gov.uk/ukpga/2006/46/section/239}{s 239 CA
2006} can usually absolve the directors from personal liability for
breach of duty. Ratification at a time when the company is solvent
should preclude misfeasance proceedings.

When a company is facing insolvency, duties of directors shift towards
the company's creditors, and away from the members as a whole (per case
law). Rationale: in these circumstances, creditors are the ones who
stand to lose if the directors breach their duties.

\textbf{Not} possible for shareholders to ratify breach of directors'
duties at a time when the company's fortunes have declined to the extent
that there is a reasonable possibility of the company going insolvent (s
239(7) CA 2006).

\hypertarget{fraudulent-trading}{%
\subsection{Fraudulent Trading}\label{fraudulent-trading}}

The provisions on fraudulent and wrongful trading in IA 1986 were
enacted to prevent the reckless and negligent conduct on the part of
those running companies. They are examples of claims in which the
corporate veil may be pierced and directors held liable for losses made
by the company.

\hypertarget{claim}{%
\subsubsection{Claim}\label{claim}}

A claim can be brought under s 213/ 246ZA IA 1986 against:

\begin{quote}
Any person (s 213(2), s 246ZA) who is knowingly party to the carrying on
of any business of the company, with intent to defraud creditors or for
any fraudulent purpose (s 213(1), s 246ZA(1)).
\end{quote}

\href{https://www.legislation.gov.uk/ukpga/1986/45/section/213}{s 213 IA
1986} (in Liquidation) and
\href{https://www.legislation.gov.uk/ukpga/1986/45/section/246ZA}{s
246ZA IA 1986} (in administration) impose a civil liability to
contribute to the funds available to the general body of unsecured
creditors suffering loss caused by carrying on of the company's business
with intent to defraud.

There is a corresponding criminal claim for fraudulent trading under
\href{https://www.legislation.gov.uk/ukpga/2006/46/section/993}{s 993 CA
2006}. The claim may be brought by a liquidator (s 213) or by an
administrator (s 246ZA), though court approval is required.

\hypertarget{dishonesty}{%
\subsubsection{Dishonesty}\label{dishonesty}}

Actual dishonesty must be proven for a claim for fraudulent trading to
succeed. This is assessed on a subjective basis, i.e., what the
particular person knew or believed.

Knowledge includes blind-eye knowledge, which requires a suspicion of
the relevant facts together with a deliberate decision to avoid
confirming that they did exist (Morris v State Bank of India {[}2005{]}
2 BCLC 328)

Fraud is defined for the purposes of s 213 as requiring

\begin{quote}
``Real dishonesty involving, according to current notions of fair
trading among commercial men at the present day, real moral blame.'' (Re
Patrick and Lyon Ltd {[}1933{]} Ch 786)
\end{quote}

Not necessary to show that all the company's creditors have been
defrauded (Re Gerald Cooper Chemicals Ltd {[}1978{]} Ch 262).

\hypertarget{remedies-1}{%
\subsubsection{Remedies}\label{remedies-1}}

A person found to be liable under s 213 / 246ZA can be ordered to make a
contribution to the company's assets as the court thinks proper. The
court does not have the power to include a punitive element in the
amount awarded; the contribution should only reflect and compensate for
the loss caused to the creditors (Morphitis v Bernasconi {[}2003{]} 2
BCLC 53).

Any sums recovered are held on trust for the unsecured creditors
generally, and not for the defrauded creditor (Re Esal (Commodities) Ltd
{[}1997{]} 1 BCLC 705).

\hypertarget{director-fraudulent-trading}{%
\paragraph{Director Fraudulent
Trading}\label{director-fraudulent-trading}}

Where the court makes an order against a person under s 213 / 246ZA, and
that person is also a director, the court is likely also to make a
disqualification order under
\href{https://www.legislation.gov.uk/ukpga/1986/46/section/10}{s 10 CDDA
1986}.

\hypertarget{criminal-sanctions}{%
\paragraph{Criminal Sanctions}\label{criminal-sanctions}}

Criminal sanctions can be imposed by the court under
\href{https://www.legislation.gov.uk/ukpga/2006/46/section/993}{s 993 CA
2006}, to punish a person knowingly party to fraudulent trading, whether
or not the company is being wound up. The penalties are imprisonment (of
up to 10 years on indictment) and/or fines.

\hypertarget{fraudulent-vs-wrongful}{%
\paragraph{Fraudulent Vs Wrongful}\label{fraudulent-vs-wrongful}}

A very high standard of proof is required in practice for a successful
claim in fraudulent trading. So claims are rare, and claims for Wrongful
trading under s 214 / 246ZB IA 1986 are more often brought against
directors.

\hypertarget{wrongful-trading}{%
\subsection{Wrongful Trading}\label{wrongful-trading}}

\hypertarget{rationale}{%
\subsubsection{Rationale}\label{rationale}}

Liability for wrongful trading is a much more recent innovation than the
introduction of liability for fraudulent trading. It was introduced to
address criticism of the ineffectiveness of the fraudulent trading
provisions, given the high bar of proof required for such claims.

\hypertarget{purpose}{%
\subsubsection{Purpose}\label{purpose}}

The purpose of wrongful trading liability is to ensure that directors
become aware that when an insolvent Liquidation/ administration is
inevitable, they are under a duty to take every step possible to
minimise the potential losses to the company's creditors.

If they fail, the court can order the directors to contribute to the
insolvent estate by way of compensation for the losses that the general
body of creditors have suffered as a result. This will increase the
funds available for distribution to the unsecured creditors upon
insolvency.

Wrongful trading liability imposes a personal liability on directors,
and is an exception to the principle of limited liability, under which
those who run a company cannot be liable for its unpaid debts.

There is no requirement to show intent or dishonesty.

\hypertarget{who-may-bring-a-claim}{%
\subsubsection{Who May Bring a Claim}\label{who-may-bring-a-claim}}

A civil claim for wrongful trading can be brought against a director by
a liquidator
(\href{https://www.legislation.gov.uk/ukpga/1986/45/section/214}{s 214
IA 1986}) or an administrator
(\href{https://www.legislation.gov.uk/ukpga/1986/45/section/246ZB}{s
246ZB IA 1986}). There are no criminal provisions for wrongful trading.

Wrongful trading is a major risk run by directors of a company trading
on the brink of insolvency.

Administrators and liquidators can also now under
\href{https://www.legislation.gov.uk/ukpga/2015/26/contents/enacted}{SBEEA
2015} assign wrongful trading claims to a third party as a way of
raising funds for the insolvent estate and thereby avoid the risk of
litigation.

\hypertarget{against-whom-can-a-claim-be-brought}{%
\subsubsection{Against Whom Can a Claim Be
Brought}\label{against-whom-can-a-claim-be-brought}}

Can be brought against any person who was at the relevant time a
director.

This includes shadow directors (defined in
\href{https://www.legislation.gov.uk/ukpga/2006/46/section/251}{s 251 CA
2006}), de facto and non-executive directors (Re Hydrodam (Corby) Ltd
{[}1994{]} 2 BCLC 180).

Recall that under fraudulent trading, a claim could be brought against
any person who had the intention to commit a fraud.

\hypertarget{requirements-for-liability}{%
\subsubsection{Requirements for
Liability}\label{requirements-for-liability}}

\begin{env-f9980082-0043-4b24-aae8-b6167468a442}

Wrongful Trading (s 214(2)/246ZB(2))

For a director to be liable for wrongful trading, the court must be
satisfied that the company has gone into insolvent Liquidation or
insolvent administration, and

\begin{enumerate}
\tightlist
\item
  At some time before the commencement of the winding up or insolvent
  administration (for convenience, that time is referred to as the
  `point of no return')
\item
  The director knew or ought to have concluded that
\item
  There was no reasonable prospect that the company would avoid going
  into insolvent liquidation (or insolvent administration).
\end{enumerate}

\end{env-f9980082-0043-4b24-aae8-b6167468a442}

For these purposes, a company goes into insolvent liquidation/
administration at a time when its assets are insufficient for the
payment of its debts and other liabilities, and the expenses of the
winding up/ administration (s 214(6) / 246ZB(6)).

Insolvency for wrongful trading purposes is judged solely on the balance
sheet test, not on the cash flow test
(\href{https://www.legislation.gov.uk/ukpga/1986/45/section/214}{s
214(6) IA 1986}).

\hypertarget{continued-trading}{%
\paragraph{Continued Trading}\label{continued-trading}}

It must be proven that the director in question allowed the company to
continue trading during the period in which they knew/ ought to have
known that there was no reasonable prospect that the company would avoid
going into insolvent liquidation/ administration, and that the continued
trading made the company's position worse (Re Continental Assurance Co
of London plc {[}2001{]} BPIR 733).

But if the company has not reached 'the point of no return', then
wrongful trading liability cannot arise. Only if this point has been
reached should the defence below be considered.

Re Produce Marketing Consortium Ltd {[}1989{]} BCLC 513, ChD: there is
an obligation laid on companies to keep accounting records which
disclose with \textbf{reasonable accuracy} the financial position of the
company. This minimum standard holds regardless of the size of the
company, though a higher standard will be imposed for larger companies.

\hypertarget{the-every-step-defence}{%
\subsubsection{The 'every Step' Defence}\label{the-every-step-defence}}

This defence is laid out in s 214(3) / 246ZB(3).

Assuming the company has reached the point of no return, a director may
be able to escape liability if they can satisfy the court that, after
they first knew/ ought to have concluded that there was no reasonable
prospect of the company avoiding an insolvent administration/
liquidation, they took every step with a view to minimising the
potential loss to the company's creditors.

Evidence of this:

\begin{itemize}
\tightlist
\item
  Voicing concerns at board meetings;
\item
  Seeking independent financial/ legal advice;
\item
  Ensuring adequate, up-to-date financial information was available;
\item
  Suggesting reductions in overheads/liabilities;
\item
  Not incurring further credit; and
\item
  Consulting a lawyer and/or an insolvency practitioner for advice on
  continued trading and the different insolvency procedures.
\end{itemize}

The burden of proof for this is on the directors: Brooks v Armstrong
{[}2015{]} EWHC 2289.

\hypertarget{reasonably-diligent-person-test}{%
\subsubsection{Reasonably Diligent Person
Test}\label{reasonably-diligent-person-test}}

The 'reasonably diligent person' test (s 214(4) / 246ZB(4)) is applied
to determine:

\begin{itemize}
\tightlist
\item
  Whether a liquidator or administrator has established that a director
  ought to have concluded that there was no reasonable prospect of
  avoiding an insolvent liquidation or administration (the s 214(2) /
  246ZB(2) liability), and
\item
  Whether the director then took every step to minimise the potential
  loss to the company's creditors (the s 214(3) / 246ZB(3) defence).
\end{itemize}

Test:

\begin{itemize}
\tightlist
\item
  The facts which a director ought to have known or ascertained,
\item
  The conclusions which they ought to have reached and
\item
  The steps which they ought to have taken are those which would have
  been known or ascertained, or reached or taken, by a reasonably
  diligent person having both:

  \begin{itemize}
  \tightlist
  \item
    The general knowledge, skill and experience that may reasonably be
    expected of a person carrying out the same functions as are carried
    out by the director in question (an objective test); and
  \item
    The actual knowledge, skill and experience of that particular
    director (a subjective test).
  \end{itemize}
\end{itemize}

The court then applies the \textbf{higher} of these standards.

\hypertarget{advice-to-directors}{%
\subsubsection{Advice to Directors}\label{advice-to-directors}}

\begin{itemize}
\tightlist
\item
  Directors should hold board meetings to review the company's financial
  position, and consider whether it is appropriate to incur new credit/
  liabilities.
\item
  Minutes of each meeting should be written up.
\item
  Lawyers advising a company often take an active role in helping
  directors to prepare minutes and ensure that board meetings consider
  all relevant issues.
\item
  A director cannot escape liability by simply resigning, since a claim
  for wrongful trading can be brought against any person who was a
  director at the relevant time (Re Purpoint Ltd {[}1991{]} BCLC 491).
\item
  The best course of action is for the company to seek professional
  advice as soon as possible (Re Continental Assurance Co of London plc
  {[}2001{]} BPIR 733).
\item
  But the absence of warnings from advisors does not relieve directors
  of the responsibility to review the company's position critically (Re
  Brian D Pierson (Contractors) Ltd {[}2001{]} 1 BCLC).
\end{itemize}

\hypertarget{remedies-2}{%
\subsubsection{Remedies}\label{remedies-2}}

If a director is found liable for wrongful trading, the court can order
that director to make such contribution to the assets of the company as
the court thinks fit.

The court has wide discretion to determine the extent of the directors'
liability. The contribution will ordinarily be based on the additional
depletion of the company's assets caused by the directors' conduct from
the date that the directors ought to have concluded that the company
could not have avoided an insolvent administration/ liquidation (i.e.,
the point of no return).

\hypertarget{apportioning-contributions}{%
\paragraph{Apportioning
Contributions}\label{apportioning-contributions}}

An order by the court for a director to contribute to the company's
assets is compensatory, not penal. An order to contribute can be made
against the directors on a joint and several basis. But the court also
has discretion to apportion liability between directors based on
culpability by ordering the more culpable directors to pay less.

Where the court makes a contribution order against a director, the court
also has discretion to make a disqualification order against them under
\href{https://www.legislation.gov.uk/ukpga/1986/46/section/10}{s 10 CDDA
1986}.

\hypertarget{no-relief}{%
\paragraph{No Relief}\label{no-relief}}

Under \href{https://www.legislation.gov.uk/ukpga/2006/46/section/1157}{s
1157 CA 2006} the court may ordinarily relieve a director from liability
in proceedings for negligence, breach of duty or breach of trust, if it
is satisfied that they acted honestly and reasonably. And having regard
to all the circumstances of the case, the director ought fairly to be
excused. But this relief is \textbf{not} available in wrongful trading
proceedings (Re Produce Marketing Consortium Ltd {[}1989{]} BCLC 513,
ChD).

The risk of action for wrongful trading can be of concern to directors,
particularly in a difficult economic climate. For this reason, the
government suspended the wrongful trading provisions from
01/03/20-30/09/20 to allow company directors to attempt to keep their
business going during COVID without risking personal liability. A
further suspension was announced 26/11/20-30/04/21.

\hypertarget{liquidation}{%
\section{Liquidation}\label{liquidation}}

\begin{env-751485c0-9cbb-4b3b-8497-6bad2ef120e5}

Definition

Liquidation is the process by which a company's business is wound up,
and its assets transferred to creditors and (if there is a surplus of
assets over liabilities) to its members.

\end{env-751485c0-9cbb-4b3b-8497-6bad2ef120e5}

Basic steps:

\begin{enumerate}
\tightlist
\item
  Liquidation proceedings commenced
\item
  Liquidator appointed
\item
  Liquidator collects company's assets and may review past transactions.
\item
  Liquidator distributes assets in the statutory order to the creditors
\item
  Company dissolved.
\end{enumerate}

3 types of liquidation:

\begin{longtable}[]{@{}ll@{}}
\toprule()
Type & Description \\
\midrule()
\endhead
Compulsory Liquidation & Commenced against an insolvent company by a 3rd
party \\
Creditors voluntary liquidation & Commenced by an insolvent company,
usually in response to creditor pressure. \\
Members' voluntary liquidation & Commenced by a solvent company that
wishes to cease trading/ is dormant. \\
\bottomrule()
\end{longtable}

\hypertarget{compulsory-liquidation}{%
\subsection{Compulsory Liquidation}\label{compulsory-liquidation}}

This is a court-based process.

To begin the process, an applicant presents a winding up petition to the
court under which the applicant requests the court to make a winding up
order against the company on a number of statutory grounds.

When the court grants a petition for compulsory liquidation, the order
operates in favour of all the creditors and contributories (members and
some former members) of the company.

The Official Receiver will become the liquidator and continue in office
until another person is appointed
(\href{https://www.legislation.gov.uk/ukpga/1986/45/section/136}{s
136(2) IA 1986}). The Official Receiver will notify Companies House and
all known creditors of the liquidation. The Official Receiver has the
power to summon separate meetings of the company's creditors and
contributories for the purpose of choosing a person to become the
liquidator of the company in their place (s 136(4)).

\hypertarget{applying-for-a-winding-up-order}{%
\subsubsection{Applying for a Winding up
Order}\label{applying-for-a-winding-up-order}}

The following can apply to the court for the issue of a winding up
petition:

\begin{enumerate}
\tightlist
\item
  a creditor;
\item
  the company (acting by the shareholders; this would happen where there
  are insufficient assets in the company to fund a voluntary
  liquidation);
\item
  the directors (by board resolution); again, this would happen where
  there are insufficient assets to fund a voluntary liquidation;
\item
  an administrator;
\item
  an administrative receiver;
\item
  the supervisor of a CVA; and
\item
  The Secretary of State for Business, Energy \& Industrial Strategy (on
  public policy grounds).
\end{enumerate}

\hypertarget{inability-to-pay-debts}{%
\subsubsection{Inability to Pay Debts}\label{inability-to-pay-debts}}

The \textbf{most common ground} for a winding up petition is the
company's inability to pay its debts under
\href{https://www.legislation.gov.uk/ukpga/1986/45/section/122}{s
122(1)(f) IA 1986}. This can be evidenced in several ways (
\href{https://www.legislation.gov.uk/ukpga/1986/45/section/123}{s 123 IA
1986}):

\begin{enumerate}
\tightlist
\item
  Failure by the company to comply with a creditor's statutory demand (s
  123(1)(a))

  \begin{itemize}
  \tightlist
  \item
    A statutory demand is a written demand in a prescribed form
    requiring the company to pay a specific debt.
  \item
    The statutory demand can only be used if the debt exceeds £750 and
    is not disputed on substantial grounds.
  \item
    The company has 21 days in which to pay the debt, failing which the
    creditor has the right to petition the court to wind up the company.
  \end{itemize}
\item
  The creditor sues the company, obtains judgment and fails in an
  attempt to execute the judgment debt (s 123(1)(b) IA 1986)
\item
  Proof to the satisfaction of the court that the company is unable to
  pay its debts as they fall due (the ``cash-flow test'') (s 123(1)(c)
  IA 1986).

  \begin{itemize}
  \tightlist
  \item
    The cash flow test is usually satisfied by going through the
    statutory demand process in 1 above
  \item
    Alternatively, the company admitting it cannot pay the debt in open
    correspondence.
  \end{itemize}
\item
  Proof to the satisfaction of the court that the value of the company's
  assets is less than the amount of its liabilities, taking into account
  contingent and prospective liabilities (the 'balance sheet test') (s
  123(2))

  \begin{itemize}
  \tightlist
  \item
    Re Cheyne Finance plc {[}2007{]} EWHC 2402 (Ch)
  \end{itemize}
\end{enumerate}

The cash-flow test must include a consideration of debts falling due in
the reasonably near future (BNY Corporate Trustee Services Ltd v
Eurosail-UK 2007-3BL plc {[}2013{]}).

The court will consider all relevant factors and may dismiss the
petition if the debtor company can show that it may recover its
financial position or if the debt is disputed in any way by the debtor
company (Tallington Lakes Ltd v Ancasta International Boat Sales Ltd
{[}2012{]} EWCA Civ 1712).

\begin{itemize}
\tightlist
\item
  There must be a genuine and substantial dispute (Misra Ventures Ltd v
  LDX International Group LLP {[}2018{]} EWCA Civ 330).
\end{itemize}

If the winding-up petition is accepted and a winding-up order is made,
the Official Receiver, a civil servant or court official will
automatically become a liquidator. Must decide whether it is appropriate
to appoint an insolvency practitioner as liquidator in their place. This
will generally be done if {\(> 50\%\)} creditors by value choose a
private practice insolvency practitioner.

\begin{env-440d3209-5a60-4bb6-8a13-881ba9304f97}

Note

Where a commercial debt is owed to a landlord and the Commercial Rent
(Coronavirus) Act 2020 applies, a legally binding arbitration procss
will be available to seek to resolve disputes about pandemic related
debt.

\end{env-440d3209-5a60-4bb6-8a13-881ba9304f97}

\hypertarget{consequences-of-winding-up-order}{%
\subsubsection{Consequences of Winding up
Order}\label{consequences-of-winding-up-order}}

To prevent an insolvent company from transferring its assets to third
parties at the expense of its creditors, under
\href{https://www.legislation.gov.uk/ukpga/1986/45/section/127}{s 127 IA
1986} certain dispositions of a company's property, transfers of its
shares and changes to its members will be void if made after the
commencement of the winding up.

So if these dispositions etc. were made during the period between the
presentation of the winding up petition and a winding up order being
made, they will be void. This includes:

\begin{itemize}
\tightlist
\item
  Disposition of the company's property;
\item
  Transfer of the company's shares;
\item
  Altering the status of the company's members.
\end{itemize}

Once a compulsory winding up order has been made:

\begin{itemize}
\tightlist
\item
  An automatic stay will be granted on commencing or continuing with
  proceedings against the company;
\item
  all employees will be automatically dismissed, and
\item
  the directors lose their powers, and they are automatically dismissed
  from office.
\end{itemize}

\hypertarget{voluntary-winding-up}{%
\subsection{Voluntary Winding up}\label{voluntary-winding-up}}

\href{https://www.legislation.gov.uk/ukpga/1986/45/section/84}{s 84(1)
IA 1986} allows for the company to be wound up without a court order in
3 situations:

\begin{longtable}[]{@{}ll@{}}
\toprule()
Situation & Description \\
\midrule()
\endhead
Shareholder resolution & Where the company's purpose according to the
articles has expired and resolution of the shareholders (rare) \\
Members' voluntary liquidation (MVL) & Where the company resolves by
special resolution to wind up the company. The company must be
solvent. \\
Creditors' voluntary liquidation (CVL) & Where the company resolves that
it is advisable to wind up the company due to its inability to carry on
its business. Here the company is insolvent. \\
\bottomrule()
\end{longtable}

\hypertarget{members-voluntary-liquidation}{%
\subsubsection{Members' Voluntary
Liquidation}\label{members-voluntary-liquidation}}

Can only be used for companies which are solvent. The directors are
required to swear a \textbf{declaration of solvency}, stating that they
have made a full enquiry into the solvency, and they have formed the
opinion that the company will be able to pay its creditors in full,
together with interest at the official rate, within a period not
exceeding 12 months from the commencement of the winding up
(\href{https://www.legislation.gov.uk/ukpga/1986/45/section/89}{s 89(1)
IA 1986}). The declaration must also contain a statement of the
company's assets and liabilities as at the latest practicable date
before making the declaration.

Any director making a declaration of solvency who does not have
reasonable grounds for their opinion is liable to a \textbf{fine or
imprisonment}
(\href{https://www.legislation.gov.uk/ukpga/1986/45/section/89}{s 89(4)
IA 1986}). If the debts are not actually paid in full within the
specified period, it will be presumed that the director did not have
reasonable grounds for their opinion.

The members must then pass a Special resolution to place the company
into MVL and an Ordinary resolution to appoint a liquidator. The winding
up commences when the special resolution is passed
(\href{https://www.legislation.gov.uk/ukpga/1986/45/section/84}{s 84(1)
IA 1986} and
\href{https://www.legislation.gov.uk/ukpga/1986/45/section/86}{s 86 IA
1986}).

\hypertarget{conversion-of-mvl-to-cvl}{%
\subsubsection{Conversion of MVL to
CVL}\label{conversion-of-mvl-to-cvl}}

On a MVL, if the liquidator considers that the company will be unable to
pay its debts in full together with interest within the period stated in
the directors' declaration, they must change the members' winding up
into a creditors' winding up by going through the procedural conditions
in \href{https://www.legislation.gov.uk/ukpga/1986/45/section/95}{s 95
IA 1986}. This involves the liquidator preparing and sending a statement
of the company's affairs to the company's creditors.

The company's creditors may nominate a person to be liquidator. In most
cases, this will be the insolvency practitioner who was appointed to
deal with the MVL. The creditors' voluntary liquidation takes effect
from the date of nomination of the liquidator.

\hypertarget{creditors-voluntary-liquidation}{%
\subsubsection{Creditors' Voluntary
Liquidation}\label{creditors-voluntary-liquidation}}

This is a form of insolvent liquidation commenced by resolution of the
shareholders, but under the effective control of the creditors, who can
choose the liquidator.

Where a directors' declaration of solvency has not been made, the
liquidation will be a CVL. The procedure is for the shareholders to pass
a special resolution to place the company into a CVL.

The shareholders may also nominate a person to be liquidator, but in any
event within \textbf{14 days} of the special resolution being passed the
directors of the company must ask the company's creditors to either
approve the nominated liquidator or put forward their own choice of
liquidator. Where the creditors' choice of liquidator differs from that
of the company's shareholders, the creditors' nomination will take
precedence.

The directors must also draw up a statement of the company's affairs
(setting out the company's assets and liabilities) and send it to the
company's creditors.

\hypertarget{role-of-the-liquidator}{%
\subsection{Role of the Liquidator}\label{role-of-the-liquidator}}

The appointment of a liquidator terminates the management powers of the
company's directors, and these powers are transferred to the liquidator,
together with their fiduciary duties. So liquidators must act in good
faith, avoid conflicts of interest and not make a secret profit
(Silkstone and Haigh Moore Coal Co v Edey {[}1900{]}).

The liquidator must be a qualified Insolvency Practitioner
(\href{https://www.legislation.gov.uk/ukpga/1986/45/section/230}{s 230
IA 1986}) or the Official Receiver (appointed by the court in the short
term) and acts as an officer of the court.

The liquidator in both a CVL and a compulsory liquidation has extensive
statutory powers. Principle functions are:

\begin{itemize}
\tightlist
\item
  To secure and realise the assets of the company, then distribute to
  the company's creditors
  (\href{https://www.legislation.gov.uk/ukpga/1986/45/section/143}{s 143
  IA 1986});
\item
  To take into their custody or under their control all the property of
  the company
  (\href{https://www.legislation.gov.uk/ukpga/1986/45/section/144}{s 144
  IA 1986})
\end{itemize}

The liquidator's powers to manage the company are set out in Part I to
III \href{https://www.legislation.gov.uk/ukpga/1986/45/schedule/4}{Sch 4
IA 1986}, and include powers to

\begin{itemize}
\tightlist
\item
  Sell any of the company's property;
\item
  Execute deeds and other documents in the name of the company;
\item
  Raise money on the security of the company's assets;
\item
  Make or draw a bill of exchange or promissory note in the name of the
  company;
\item
  Appoint an agent to do any business that the liquidator is unable to
  do;
\item
  Do all other things that may be necessary to wind up the company's
  affairs and to distribute its assets.
\item
  Carry on the business of the company, but only to the extent that is
  necessary for the beneficial winding up of the company.
\item
  Commence or defend court proceedings in the name of the company, for
  example to recover debts owed to it or dispute debts alleged to be
  owed by the company.
\item
  Pay debts and compromise claims.
\end{itemize}

Additional duties are imposed on the Official Receiver in compulsory
liquidations, including presenting a report to the creditors and the
court (s 132 IA 1986).

The liquidator will also review the actions of directors. If the
director has transferred property after winding up has begun, or in
anticipation of it up to one year beforehand, the director may be liable
for fraud under s 206 IA 1986. If it appears to the liquidator during
the course of a MVL that the company is insolvent, the liquidator must
convert the liquidation to a CVL (s 95 IA 1986).

\hypertarget{effect-of-liquidation}{%
\subsubsection{Effect of Liquidation}\label{effect-of-liquidation}}

\begin{itemize}
\tightlist
\item
  Directors' powers cease and the liquidator takes over the running of
  the company (ss 91 \& 103 IA 1986).
\item
  In a compulsory liquidation, directors' appointments are terminated
  (Measures Brothers Ltd v Measures {[}1910{]} 2 CH 248).
\end{itemize}

The liquidation is concluded when:

\begin{itemize}
\tightlist
\item
  Available assets are sold
\item
  Money distributed to the creditors
\item
  Liquidator released
\item
  Accounts forwarded to Companies House and the court
\item
  3 months later, the company is dissolved by the Registrar of
  Companies.
\end{itemize}

\hypertarget{liquidators-powers-to-avoid-certain-transactions}{%
\subsubsection{Liquidator's Powers to Avoid Certain
Transactions}\label{liquidators-powers-to-avoid-certain-transactions}}

Liquidators have a duty to preserve the company's property and to
maximise the value of the company's assets available for distribution.
Empowered to avoid certain antecedent transactions in order to maximise
assets available for distribution to creditors:

\begin{itemize}
\tightlist
\item
  Disclaim onerous property
  (\href{https://www.legislation.gov.uk/ukpga/1986/45/section/178}{s178
  IA 1986});
\item
  Apply to court to set aside a transaction at an undervalue
  (\href{https://www.legislation.gov.uk/ukpga/1986/45/section/238}{s238
  IA 1986});
\item
  Apply to court to set aside a preference
  (\href{https://www.legislation.gov.uk/ukpga/1986/45/section/239}{s 239
  IA 1986});
\item
  Apply to court to set aside, or vary the terms of, an extortionate
  credit transaction
  (\href{https://www.legislation.gov.uk/ukpga/1986/45/section/244}{s 244
  IA 1986});
\item
  Claim that a floating charge created for no new, or inadequate,
  consideration is invalid
  (\href{https://www.legislation.gov.uk/ukpga/1986/45/section/245}{s 245
  IA 1986});
\item
  Apply to court to set aside a transaction that will defraud creditors
  (\href{https://www.legislation.gov.uk/ukpga/1986/45/section/423}{s 423
  IA 1986}).
\end{itemize}

Many of these powers also apply to administrators.

\hypertarget{distribution-of-assets}{%
\subsection{Distribution of Assets}\label{distribution-of-assets}}

\begin{itemize}
\tightlist
\item
  Liquidator asks creditors to send them details of their debt on a
  standard form ("proving the debt").
\item
  Liquidator approves/ rejects proof of debt.
\item
  {\(\exists\)} simplified procedure for small debts (\textless£1,000)
\end{itemize}

Then the assets are distributed; see Statutory order of priority on
winding up and IA 1986, ss 175, 176ZA and 176A.

Essentially:

\begin{itemize}
\tightlist
\item
  Expenses of the winding up
\item
  Preferential debts
\item
  Monies secured by floating charges by the order of priority (subject
  to ring-fencing of a prescribed part for the unsecured creditors).
\item
  Surplus distributed to shareholders.
\item
  Unsecured creditors (ranking and abating equally)
\end{itemize}

\hypertarget{administration}{%
\section{Administration}\label{administration}}

Administration is a procedure aiming to rescue a company which is
insolvent if at all possible, or to achieve a better result for
creditors if not.

It is a ``collective procedure'', meaning the administrator acts in the
interests of the creditors as a whole rather than on behalf of a
particular creditor.

Administrators are qualified insolvency practitioners, who can be
appointed by the court or under the out-of-court procedure.

The main advantage is that the administrator has the benefit of a
moratorium, allowing them to operate almost unhindered by creditors from
the outset.

\hypertarget{purpose-1}{%
\subsection{Purpose}\label{purpose-1}}

\href{https://www.legislation.gov.uk/ukpga/1986/45/section/8}{s 8 IA
1986} and
\href{https://www.legislation.gov.uk/ukpga/1986/45/schedule/B1}{Schedule
B1 IA 1986} set out the objectives of the administration.

State that an administrator

\begin{quote}
``\ldots must perform his functions with the objective of:\\
(a) rescuing the company as a going concern, or\\
(b) achieving a better result for the company's creditors as a whole
than would be likely if the company were wound up\ldots,\\
(c) realising the property in order to make a distribution to one or
more secure or preferential creditors.''
\end{quote}

These cascading objectives guide the actions of the administrator
throughout the process. Objective (b) is most likely to be achieved.

The administrator can only move down the list if the alternative is not
reasonably practicable, or if (b) is better for the creditors as a whole
and (c) will not unnecessarily harm them ((IA 1986, Sch B1, para 3(3)
and (4))).

The administrator has to perform all their duties in the interests of
the company's creditors as a whole (IA 1986, Sch B1, para 3(2)).

\hypertarget{appointment}{%
\subsection{Appointment}\label{appointment}}

\hypertarget{court-procedure}{%
\subsubsection{Court Procedure}\label{court-procedure}}

The court may appoint an administrator where the company is or is likely
to become unable to pay its debts (IA 1986, Sch B1 para 10) on the
application of (IA 1986, Sch B1 para 12):

\begin{itemize}
\tightlist
\item
  The company
\item
  The directors
\item
  One or more creditors.
\end{itemize}

The appointment may only be made where the order is reasonably likely to
achieve the purpose of the administration (Sch B1 para 11(b)):

\begin{quote}
The court may make an administration order in relation to a company only
if satisfied---\\
(a) that the company is or is likely to become unable to pay its debts,
and\\
(b) that the administration order is reasonably likely to achieve the
purpose of administration
\end{quote}

See AA Mutual International Insurance Co Ltd {[}2004{]} EWHC Ch 1.

\begin{env-862347ee-a221-4f52-a3b4-3eedf85d51d9}

Action

As soon as reasonably practicable after making the application, the
applicant must notify:

\begin{enumerate}
\tightlist
\item
  any person who has appointed (or is entitled to appoint) an
  administrative receiver of the company;
\item
  any qualifying floating charge holder (QFCH) who may be entitled to
  appoint an administrator; and
\item
  such other persons as may be prescribed by the IR 2016.
\end{enumerate}

\end{env-862347ee-a221-4f52-a3b4-3eedf85d51d9}

\hypertarget{out-of-court-procedure}{%
\subsubsection{Out of Court Procedure}\label{out-of-court-procedure}}

The following parties may appoint an administrator using the
out-of-court procedure:

\begin{itemize}
\tightlist
\item
  The company (Sch B1 para 22(1))
\item
  The directors (Sch B1 para 22(2)); or
\item
  A Qualifying Floating Charge holder (Sch B1 para 14)
\end{itemize}

\begin{env-751485c0-9cbb-4b3b-8497-6bad2ef120e5}

Qualifying Floating Charge Holder 'QFCH'

TLDR: The holder of a floating charge created after 15 September 2003
relating to the whole or substantially the whole of the company's
property
(\href{https://www.legislation.gov.uk/ukpga/1986/45/schedule/B1}{Sch B1
para 14 IA 1986}).

\begin{center}\rule{0.5\linewidth}{0.5pt}\end{center}

A qualifying floating charge (QFC) is one where the charge document:

\begin{itemize}
\tightlist
\item
  (a) states that para 14 of Sch B1 to the IA 1986 applies to it and--

  \begin{itemize}
  \tightlist
  \item
    (i) purports to empower the holder of the floating charge to appoint
    an administrator of the company, or
  \item
    (ii) purports to empower the holder of the floating charge to make
    an appointment which would be the appointment of an administrative
    receiver within the meaning given by s 29(2) of the IA 1986; and
  \end{itemize}
\item
  (b) is a QFC which, on its own or with other security held by the same
  lender, relates to the whole, or substantially the whole, of the
  company's property (para 14(3)).
\end{itemize}

\end{env-751485c0-9cbb-4b3b-8497-6bad2ef120e5}

The most common method of appointing an administrator is by directors
using the out-of-court procedure.

\begin{env-c2ea7781-55a7-40ef-9219-ca9616c6b82e}

Exception

Directors \textbf{cannot} use the out-of-court procedure where a
creditor has presented a petition for the winding up of the company. The
directors can then apply to court for an administration order, or the
qualifying floating charge holder can use the out-of-court procedure to
appoint an administrator.

\end{env-c2ea7781-55a7-40ef-9219-ca9616c6b82e}

\hypertarget{statutory-moratorium}{%
\subsubsection{Statutory Moratorium}\label{statutory-moratorium}}

Key benefit: during administration, the company is given a
\textbf{moratorium} (Sch B1 para 42-44 IA 1986). During this time, all
business documents and the company website must state that the company
is in administration.

During the moratorium (except with the consent of the court/
administrator):

\begin{enumerate}
\tightlist
\item
  No order or resolution to wind up the company can be made/ passed
\item
  No administrative receiver of the company can be appointed
\item
  No steps can be taken to enforce any security over the company's
  property, or to repossess goods subject to security, hire-purchase and
  retention of title;
\item
  No legal proceedings, execution or other process can be commenced or
  continued against the company or its property, and
\item
  A landlord cannot forfeit a lease of the company's premises by means
  of peaceable re-entry.
\end{enumerate}

\hypertarget{appointment-by-a-qfch}{%
\subsubsection{Appointment by a QFCH}\label{appointment-by-a-qfch}}

Lenders with a QFC can easily appoint administrators.

If there is another QFCH whose charge would have priority, the lender
cannot appoint an administrator until it has given 2 business days'
written notice of its intention. Allows the holder of prior charge time
to consider whether to appoint an administrator itself (effectively a
very short interim moratorium).

The floating charge must be enforceable, due to the occurrence of some
trigger event, e.g., late payment by the borrower.

It is not possible for a QFCH to appoint an administrator out-of-court
under IA 1986, Sch B1, para 14 where there is a provisional liquidator
or an administrative receiver already appointed, or the company is
already being wound up.

\begin{env-862347ee-a221-4f52-a3b4-3eedf85d51d9}

Action

Notice of appointment has to be filed at court by the lender together
with various documents. The notice must include a statutory declaration
by the lender to the effect that:

\begin{enumerate}
\tightlist
\item
  the lender is the holder of a QFC in respect of the company's
  property;
\item
  the floating charge has become enforceable; and
\item
  the appointment is in accordance with IA 1986, Sch B1.
\end{enumerate}

\end{env-862347ee-a221-4f52-a3b4-3eedf85d51d9}

Once these documents have been filed at court (and the other parts of
Sch B1, para 18 are complied with) the administration begins.

An 'old' floating charge before 15 September 2003, which complies with
the para 14(3) conditions will enable the lender to choose between
appointing an administrator and appointing an administrative receiver.

\hypertarget{process}{%
\subsubsection{Process}\label{process}}

\begin{enumerate}
\tightlist
\item
  Administrator appointed
\item
  Main moratorium comes into effect
\item
  Administrator puts forward proposals

  \begin{enumerate}
  \tightlist
  \item
    Creditors may seek further details or amend proposals (IA 1986, Sch
    B1, para 53(1)(b)).
  \item
    Creditors have final say on this: the proposal is approved if a
    majority in value of creditors present and voting vote in favour (so
    long as unconnected with company).
  \end{enumerate}
\end{enumerate}

\hypertarget{effects}{%
\subsubsection{Effects}\label{effects}}

The effects of the administration order are:

\begin{enumerate}
\tightlist
\item
  The company is managed by the administrator
\item
  The directors' powers cease, though they are still in office
\item
  Moratorium continues
\item
  Administrator controls the company's assets; and
\item
  Administrator carries out administrator's proposals, which have been
  approved by creditors.
\end{enumerate}

\hypertarget{powers-and-duties}{%
\subsubsection{Powers and Duties}\label{powers-and-duties}}

Administrators have wide powers to

\begin{quote}
``do all such things as may be necessary for the management of the
affairs, business and property of the company''
\end{quote}

These include powers to:

\begin{itemize}
\tightlist
\item
  Remove and appoint directors (s 14, Sch 1 and para 61 Sch B1)
\item
  Call a meeting of creditors or administrators, if required (para 62
  Sch B1)
\item
  Apply to the court for directions (para 63 Sch B1)
\item
  Pay money to a creditor, but only with court permission if they are an
  unsecured creditor (para 65)
\item
  Pay money to any party if likely to assist the administration (para
  66)
\item
  Dispose of property subject to a floating charge (para 70 Sch B1);
\item
  Dispose of property subject to a fixed charge (with the court's
  consent) (Para 71 Sch B1).
\item
  Deal with property that is subject to a hire-purchase agreement (para
  72)
\end{itemize}

Additionally, the Small Business, Enterprise and Employment Act 2015
(SBEEA 2015) granted additional powers to administrators to allow them
to bring proceedings against directors for fraudulent and Wrongful
trading.

\hypertarget{approach-of-the-court}{%
\paragraph{Approach of the Court}\label{approach-of-the-court}}

In Re T \& D Industries Ltd {[}2000{]} 1 LWR 646, held that
administration was meant to be a cheaper, more commercial alternative to
liquidation, so the administrator had substantial autonomy to act
without consulting the court.

\hypertarget{end-of-administration}{%
\subsubsection{End of Administration}\label{end-of-administration}}

The administration may be ended (Sch B1 IA 1986):

\begin{enumerate}
\tightlist
\item
  Automatically after 1 year from the date the administration took
  effect (may be extended) (para 76).
\item
  On application by the administrator to the court (para 79) if:

  \begin{enumerate}
  \tightlist
  \item
    They think the purpose of administration cannot be achieved
  \item
    They think the company should not have entered administration.
  \item
    Creditors' meeting requires them to make an application, or
  \item
    They think the purpose of administration has been sufficiently
    achieved in relation to the company.
  \end{enumerate}
\item
  By termination where the object has been achieved (para 80)
\item
  By the court ending the administration on the application of a
  creditor (para 81)
\item
  By the court converting the administration into a liquidation in the
  public interest (para 82)
\item
  By the administrator converting the administration into a CVL (para
  83)
\item
  By the administrator dissolving the company where they believe there
  is no property which might permit a distribution to creditors (para
  84);
\item
  When the administrator resigns (para 87), is removed (para 88), ceases
  to be qualified (para 89) or is replaced by those who appointed the
  administrator in the first place (paras 92, 93, 94).
\end{enumerate}

\hypertarget{pre-pack-administration}{%
\subsubsection{Pre-pack Administration}\label{pre-pack-administration}}

A pre-packaged administration is where the business of an insolvent
company is prepared for sale to a selected buyer, prior to the company's
entry into administration. The agreed sale is carried out by an
insolvency practitioner shortly after their appointment. Often, the
pre-pack purchaser will be one or more of the existing owners or
directors of the insolvent company.

Pre-packaged sales are controversial, particularly where the sale is to
existing members or management. The concern is that often creditors are
given insufficient information to determine whether the sale was in
their best interests. The unsecured creditors are not consulted and are
unlikely to be paid their debts.

This concern led to calls for greater transparency, and the Association
of Business Recovery Professionals issued a Statement of Insolvency
Practice (SIP) in 2013, requiring clear, comprehensive and timely
explanations to creditors following pre-packaged sales. The
administrator must disclose to unsecured creditors:

\begin{itemize}
\tightlist
\item
  Source of administrator's introduction
\item
  Extent of administrator's involvement prior to appointment
\item
  Valuations of the business or the underlying assets
\item
  Alternative courses of action considered
\item
  Why it was not appropriate to keep the business trading and sell it as
  a going concern
\item
  Whether efforts were made to consult major creditors
\item
  Identities of purchasers and any connections to former management.
\end{itemize}

The government have since passed Administration (Restrictions on
Disposal etc to Connected Persons) Regulations 2021 (SI 2021/427) which
came into force on 30 April 2021. These impose more onerous obligations.

\hypertarget{other-alternatives}{%
\section{Other Alternatives}\label{other-alternatives}}

The company may try to avoid liquidation using:

\begin{enumerate}
\tightlist
\item
  Administration (IA 1986, Sch B1).
\item
  Company voluntary arrangements (CVAs) (IA 1986, Part I).
\item
  The moratorium procedure (CIGA 2020, s 1, inserting a new Part A1 into
  the IA 1986).
\item
  The restructuring plan procedure (CIGA 2020, s 7 and Sch 9, inserting
  a new Part 26A (ss 901A--910L) into CA 2006)
\item
  Schemes of arrangement (CA 2006, ss 895--901)
\item
  Informal agreements with creditors.
\end{enumerate}

MERMAID1

These are not mutually exclusive procedures.

Administration

\hypertarget{creditors-voluntary-arrangement}{%
\subsection{Creditors Voluntary
Arrangement}\label{creditors-voluntary-arrangement}}

\hypertarget{definition}{%
\subsubsection{Definition}\label{definition}}

The CVA is a compromise between a company and its creditors. Defined in
\href{https://www.legislation.gov.uk/ukpga/1986/45/section/1}{s 1(1) IA
1986} as

\begin{quote}
``A composition in satisfaction of its debts or a scheme of arrangement
of its affairs''.
\end{quote}

The essence of a CVA is that the creditors agree to part payment of the
debts or to a new timetable for repayment. The agreement must be
reported to court, but there is no requirement for the court to approve
the arrangement.

The CVA is supervised and implemented by an Insolvency Practitioner. But
the company's directors remain in post and are involved in the
implementation. Can also be used together with administration or
liquidation.

\hypertarget{setting-up-cva}{%
\subsubsection{Setting up CVA}\label{setting-up-cva}}

\begin{itemize}
\tightlist
\item
  If the company is not in liquidation or administration, the directors
  draft the written proposals and appoint a nominee (an insolvency
  practitioner).
\item
  The directors submit the written proposals to the creditors and a
  statement of the company's affairs to the nominee (s 2(3) IA).
\item
  The nominee considers the proposals, and, within 28 days, must report
  to court on whether to call a meeting of company and creditors
  (\href{https://www.legislation.gov.uk/ukpga/1986/45/section/2}{s 2(1)
  and 2(2) IA 1986}).

  \begin{itemize}
  \tightlist
  \item
    May call a meeting of creditors under the "Decision Procedure" in IR
    2016
  \item
    May also call a meeting of members.
  \end{itemize}
\item
  Nominee gives 14 days' notice of meeting to creditors. A meeting of
  the members must take place within 5 days of the creditors' decision.
\item
  Voting: proposals must be approved by

  \begin{itemize}
  \tightlist
  \item
    75\% in value of creditors (excluding secured creditors); and
  \item
    A majority in value of unconnected creditors (investors not
    connected to the company) (s 4A IA); and

    \begin{itemize}
    \tightlist
    \item
      It is for the convenor or chair of the meeting to decide if a
      creditor is connected.
    \end{itemize}
  \item
    A simple majority of members.
  \end{itemize}
\item
  All creditors (except secured and preferential who disagree) are
  bound, even if they did not take part in the Decision Procedure (s 5
  IA 1986)
\item
  Nominee becomes "supervisor" and reports to court on approval.
\item
  Directors remain in office and run company, but supervisor checks
  implementation of proposals.
\item
  Supervisor makes final report to creditors and members.
\end{itemize}

\hypertarget{effect-of-a-cva}{%
\subsubsection{Effect of a CVA}\label{effect-of-a-cva}}

A CVA is binding on all unsecured creditors (as regards past but not
future debts - s 5 IA 1986), including those who did not vote/ voted
against it.

But secured/ preferential creditors are \textbf{not} bound unless they
unanimously consent to the CVA
(\href{https://www.legislation.gov.uk/ukpga/1986/45/section/4}{s 4 IA
1986}).

\begin{env-440d3209-5a60-4bb6-8a13-881ba9304f97}

Note

There is no statutory moratorium under a CVA, but a company could use
the moratorium created by CIGA 2020 while creating a CVA proposal to be
voted on.

\end{env-440d3209-5a60-4bb6-8a13-881ba9304f97}

\hypertarget{using-cvas}{%
\subsubsection{Using CVAs}\label{using-cvas}}

CVAs are used, either alone or within administration, in order to
attempt to reach a compromise with creditors. They are particularly used
when trying to get landlords to agree to a rent reduction in order to
allow the company wiggle room to keep trading.

\begin{longtable}[]{@{}ll@{}}
\toprule()
Advantages & Disadvantages \\
\midrule()
\endhead
Directors retain control of the company, which is able to continue
trading under a successful CVA and may exit this procedure as a solvent,
profitable company. & Secured or preferential creditors are not bound
unless they unanimously consent to the CVA (s 4(3) IA). \\
Binding on unsecured creditors (s 5(2)(b) IA) irrespective of how they
voted. & Relatively complex procedure \\
No court approval required, so costs less and is quicker than other
insolvency procedures. Can be used alongside administration or
liquidation. & No longer offers a moratorium \\
\bottomrule()
\end{longtable}

Overall, CVAs are relatively rarely used due to the complexity of the
procedure, and the fact that secured/ preferential creditors are not
bound. This will probably fall further and be replaced by the
Restructuring Plan introduced by CIGA 2020.

\begin{itemize}
\tightlist
\item
  \href{https://www.theguardian.com/business/2020/nov/20/landlords-accuse-clarks-over-legal-deal-to-cut-shop-rents}{Clarks'
  use of CVA}
\item
  \href{https://www.theguardian.com/business/2020/sep/15/new-look-strikes-deal-on-rent-cuts-and-payment-holidays}{New
  Look's use of CVA}
\end{itemize}

\hypertarget{restructuring-plan-ciga-2020}{%
\subsubsection{Restructuring Plan (CIGA
2020)}\label{restructuring-plan-ciga-2020}}

The purpose of the Plan is to compromise a company's creditors and
shareholders, and restructure its liabilities so that a company can
return to insolvency. The plan can bind secured creditors. Acts as a
compromise between the company and its creditors and/or shareholders.

Relevant provisions:
\href{https://www.legislation.gov.uk/ukpga/2006/46/part/26A}{Part 26A CA
2006} (as amended by CIGA 2020). A plan can only be used by companies
which have or are likely to encounter financial difficulties, and
propose to eliminate, reduce, prevent or mitigate the effect of these
financial difficulties through the procedure. Lots of similarities with
US Chapter 11 bankruptcy proceedings.

A company just needs to establish a 'sufficient connection' to the
English jurisdiction to use the procedure - likely to present a low bar.

Typically, directors of a company, insolvency practitioners and
restructuring lawyers prepare a restructuring plan proposal and apply to
court for approval to convene meetings with the company's creditors and
members.

\hypertarget{key-features}{%
\paragraph{Key Features}\label{key-features}}

\begin{itemize}
\tightlist
\item
  Creditors and members must be divided into classes. Each class which
  votes on the Plan must be asked to approve it. The Plan must be
  approved by \textbf{{\(\geq 75\%\)} in value of each class voting}.
\item
  The court must sanction the Plan, and it will then bind all creditors.
\item
  The parties who can apply to the court for sanction of a restructuring
  plan are (s 901C(2) CA):

  \begin{itemize}
  \tightlist
  \item
    The company
  \item
    Any creditor
  \item
    Any member
  \item
    The liquidator (if company is in liquidation)
  \item
    The administrator (if company is in administration)
  \end{itemize}
\end{itemize}

There are usually two court hearings:

\begin{enumerate}
\tightlist
\item
  Convening meeting

  \begin{itemize}
  \tightlist
  \item
    Court considers whether:

    \begin{itemize}
    \tightlist
    \item
      Company is eligible
    \item
      Creditor classes correctly formulated
    \item
      Any creditors to be excluded from voting (no genuine economic
      interest)
    \item
      Jurisdictional issues.
    \end{itemize}
  \item
    Creditors can make representations
  \item
    If court is satisfied, it will make an order convening creditor and
    member meetings.
  \item
    All creditors and members sent notice of the meetings and
    documentation outlining the restructuring plan.
  \item
    Voting thresholds: approval of 75\% in gross value of debt or equity
    that votes in each class of creditors or members.
  \end{itemize}
\item
  Sanctioning hearing

  \begin{itemize}
  \tightlist
  \item
    Court decides whether to sanction the plan, in its absolute
    discretion.
  \end{itemize}
\end{enumerate}

\hypertarget{advantages}{%
\paragraph{Advantages}\label{advantages}}

The court can sanction a Plan if it is \textbf{just and equitable} to do
so, even if:

\begin{itemize}
\tightlist
\item
  One or more classes do not vote to approve the plan
\item
  It brings about a cross class cramdown;

  \begin{itemize}
  \tightlist
  \item
    Where a class of creditor can force the Plan on another class of
    creditor who has voted against the Plan
  \end{itemize}
\item
  It brings about a cramdown of shareholders;

  \begin{itemize}
  \tightlist
  \item
    Shareholders are forced to accept the Plan, even if it means
    diluting equity by creating debt-for-equity swaps.
  \end{itemize}
\end{itemize}

\begin{env-0630ef60-d677-4921-bba8-06d563fa419c}

Cross-class cram down

Will be ordered when:

\begin{enumerate}
\tightlist
\item
  Court satisfied that no member of the dissenting classes will be any
  worse off under the plan than they would have been under the 'relevant
  alternative'
\item
  At least one class that would benefit from the relevant alternative
  has voted in favour of the plan
\item
  The court is prepared to sanction the plan.
\end{enumerate}

\end{env-0630ef60-d677-4921-bba8-06d563fa419c}

\begin{env-751485c0-9cbb-4b3b-8497-6bad2ef120e5}

Relevant alternative

What the court considers to be most likely to occur if the plan is not
sanctioned by the court (usually liquidation).

\end{env-751485c0-9cbb-4b3b-8497-6bad2ef120e5}

\hypertarget{usage}{%
\paragraph{Usage}\label{usage}}

\begin{itemize}
\tightlist
\item
  A plan is likely to be used by directors following the use of a
  pre-insolvency moratorium. The company may apply for the
  pre-insolvency moratorium to protect itself, whilst the arrangements
  are made for the implementation of the Plan. The moratorium will end
  once the Plan received court sanction.
\item
  Can also be used by administrators and liquidators
\item
  May be better than a CVA because it can compromise the rights and
  claims of secured creditors and shareholders (which a CVA can't do).
\end{itemize}

See
\href{https://uk.practicallaw.thomsonreuters.com/w-027-5147?transitionType=Default\&contextData=(sc.Default)\&firstPage=true}{Virgin
Atlantic restructuring plan}.

\hypertarget{schemes-of-arrangement}{%
\subsubsection{Schemes of Arrangement}\label{schemes-of-arrangement}}

Similar to CVAs, but can be undertaken at any stage in the lifetime of a
company. 2 court hearings needed, as well as meetings of creditors and
shareholders. Often used to restructure large companies.

Need 75\% in value + a majority in number of the creditors to agree. So
this is harder to get agreed.

\hypertarget{receivership}{%
\subsection{Receivership}\label{receivership}}

Secured creditors may be able to appoint an administrative receiver.
Usually appointed where the company is not complying with the terms of
the charge holder's loan agreement. The company does not need to be
insolvent -- a receiver may be appointed by a charge holder whenever the
charge allows it.

\begin{longtable}[]{@{}ll@{}}
\toprule()
Receiver & Task \\
\midrule()
\endhead
Appointed by fixed charge holder & Take possession of the charged
property and deal with it for the benefit of the charge holder only
(usually, sell it) \\
Appointed by floating charge holder & As above, but also have a duty to
pay preferential creditors (s 40 IA 1986). \\
\bottomrule()
\end{longtable}

\hypertarget{lpa-receivers}{%
\subsubsection{LPA Receivers}\label{lpa-receivers}}

A fixed charge holder was traditionally appointed under LPA 1925, so
referred to as an LPA Receiver. Now, an express power to appoint is
usually conferred in the charge document. Does not need to be a licensed
insolvency practitioner.

Fixed charge receivers are appointed by the holders of a fixed charge
pursuant to the terms of the security documentation. They are appointed
to enforce the security, and recover the debt that is owing to their
appointor, often a bank. Owe duties primarily and exclusively to the
appointor. They owe their duties primarily and exclusively to the
appointor. The duty to the chargor is to act in good faith in the course
of their appointment.

It is a legal anomaly that fixed charge receivers are usually an agent
of the chargor. Usually have extensive powers set out in the security
documentation and some limited powers under the Law of Property Act
1925. These typically include the ability to sell, mortgage and collect
rent.

\hypertarget{administrative-receivership}{%
\subsubsection{Administrative
Receivership}\label{administrative-receivership}}

The new administration procedure can into effect on 15/09/03, replacing
administrative receivership. Administrative receivership still exists
for charges created before 15 September 2003. s 29 IA 1986:
administrative receiver can be appointed by a floating charge holder if
the floating charge was granted before 15 September 2003 (ss 72A-72H IA
1986).

Whilst administration is a collective procedure; in contrast,
receivership is an individual enforcement procedure which benefits only
the appointing creditor.

When a particular creditor (for example, a bank) have significant
control of a company's assets, they have the power to call in `an
administrative receiver' if a floating charge that was granted before 15
September 2003 is breached. Whereas administration offers a possible
route towards a return to profitability, receivership means the end of a
company, often ending in liquidation.

Main defects:

\begin{itemize}
\tightlist
\item
  Administrative receivership seen to be too slanted towards the
  interests of the secured creditor who appointed the receiver.
\item
  Unwieldy and expensive.
\end{itemize}

\hypertarget{phoenix-companies}{%
\subsection{Phoenix Companies}\label{phoenix-companies}}

s 216 IA 1986: prohibits the re-use of a company's name where the
company has gone into insolvent liquidation. This is to prevent
insolvency being used to clear debts and start afresh.

Directors of the new company may be held personally liable for its debts
(s 216) unless they consent of the court to be directors of the new
company. May also face criminal penalties.

Exceptions:

\begin{enumerate}
\tightlist
\item
  Where the successor company acquires the whole, or substantially the
  whole, of the business of an insolvent company, under arrangements
  made by an insolvency practitioner acting as its liquidator or
  administrator.
\item
  Directors of the successor phoenix company applying to court for leave
  to re-use the name of the insolvent company. Application within 7 days
  of the date on which the company went into liquidation, and leave
  granted by court within 6 weeks of the date.
\item
  A company can re-use the name of an insolvent company if;

  \begin{enumerate}
  \tightlist
  \item
    It has been known by that name for 12 months ending with the day
    before the liquidating company went into liquidation.
  \item
    It has not at any time in those 12 months been 'dormant' (defined s
    252(5) CA 2006).
  \end{enumerate}
\end{enumerate}

\hypertarget{statutory-order-of-priority-on-winding-up}{%
\section{Statutory Order of Priority on Winding
up}\label{statutory-order-of-priority-on-winding-up}}

A liquidator will (and an administrator may) be required to distribute
the assets of the company to its creditors by way of a dividend. This
must be done in a specified order of priority, according to rules which
have no single source, but are found in different parts of IA 1986, IR
2016 and general law.

The order presented is simplified and assumes there is a qualifying
floating charge (QFC) granted on or after the relevant date
(15/09/2003).

Administrators may also pay dividends to unsecured creditors if they
have court permission to do so. The statutory order of distribution can
be affected by priority or subordination agreements entered into by
creditors, under which one class of creditor agrees to rank behind
another.

Administrators may also pay dividends to unsecured creditors if they
have court permission to do so. The statutory order of distribution can
be affected by priority or subordination agreements entered into by
creditors, under which one class of creditor agrees to rank behind
another.

\hypertarget{order}{%
\subsection{Order}\label{order}}

\begin{enumerate}
\tightlist
\item
  Liquidator's fees and expenses of preserving and realising assets
  subject to fixed charges.
\item
  Amount due to fixed charge creditor out of the proceeds of selling
  assets subject to the fixed charge.
\item
  Other costs and expenses of the Liquidation.
\item
  Preferential creditors (the first tier and then the secondary tier).
\item
  Creation of the prescribed part fund (if available) for unsecured
  creditors.
\item
  Amount due to creditors with floating charges.
\item
  Unsecured/trade creditors (including payment of the prescribed part).
\item
  Interest owed to unsecured creditors.
\item
  Business Law and Practice/Company Law/Shareholders.
\end{enumerate}

\hypertarget{fixed-charge-assets}{%
\subsubsection{Fixed Charge Assets}\label{fixed-charge-assets}}

Assets are divided into two funds which are applied separately: the
assets subject to a fixed charge, and the remaining assets. The
remaining assets will be subject to the floating charge in this example,
since assuming there is a QFC (a charge over the whole of the company's
assets).

Assets subject to fixed charges are realised first by the liquidator.
The proceeds from these assets are applied as:

\begin{enumerate}
\tightlist
\item
  Liquidator's costs of preserving and realising assets subject to a
  fixed charge;
\item
  Fixed charge creditors (in respect of assets subject to a fixed
  charge)
\end{enumerate}

The proceeds of selling assets which are subject to a fixed charge must
first be used to pay off the debt secured by such charge (or mortgage).
The proceeds will be paid net of the liquidator's costs and associated
fees of realising the assets. If the proceeds are not sufficient to
discharge the debt in full, then the creditor must await payment of the
balance at an appropriate later point in the order of priority. Payment
of this balance will depend on whether or not the same debt was also
secured by a floating charge.

\hypertarget{assets-subject-to-a-floating-charge}{%
\subsubsection{Assets Subject to a Floating
Charge}\label{assets-subject-to-a-floating-charge}}

Here it is presumed there is a QFC, so all the remaining assets of the
company are subject to this. The remaining assets are realised, and the
proceeds applied as follows:

\hypertarget{other-costs-and-expenses-of-the-liquidation}{%
\subsubsection{Other Costs and Expenses of the
Liquidation}\label{other-costs-and-expenses-of-the-liquidation}}

This includes the cost of selling assets secured by a floating charge,
and the costs and expenses incurred in pursuing litigation (such as
actions concerning Wrongful trading or voidable transactions). Such
litigation will require prior approval from preferential creditors and
floating charge folders, or alternatively from the Court. Otherwise, the
liquidator cannot claim the costs of litigation.

\hypertarget{preferential-debts-schedule-6}{%
\subsubsection{Preferential Debts (Schedule
6)}\label{preferential-debts-schedule-6}}

\href{https://www.legislation.gov.uk/ukpga/2002/40/contents}{Enterprise
Act 2002} removed the preferential status of certain Crown debts,
previously payable ahead of other creditors. But this has since been
instated. For insolvencies occurring on or after 1/12/20, there are two
tiers of preferential debts:

\begin{longtable}[]{@{}ll@{}}
\toprule()
Tier & Description \\
\midrule()
\endhead
First tier & Wages and salary of employees for work done in the four
months before the insolvency date (IA 1986, Sch 6, para 9), up to a
maximum of £800 per person (Insolvency Proceedings (Monetary Limits)
Order 1986 (SI 1986/1996), art 4). Holiday pay due to any employee whose
contract has been terminated, whether that termination takes place
before or after the insolvency date (IA 1986, Sch 6, para 10). \\
Second tier & Debts due within certain periods to HMRC in respect of
PAYE, employee national insurance contributions and VAT. Note HMRC
remains an unsecured creditor for corporation taxes and other taxes owed
directly by the company. \\
\bottomrule()
\end{longtable}

\hypertarget{prescribed-part-fund}{%
\subsubsection{Prescribed Part Fund}\label{prescribed-part-fund}}

\href{https://www.legislation.gov.uk/ukpga/2002/40/contents}{Enterprise
Act 2002} introduced the ``prescribed part'' fund into the IA 1986 to
increase the chance that unsecured creditors would get paid something in
a liquidation. The idea is that money previously payable preferentially
to the Crown is reserved for the unsecured creditors and does not flow
into the pocket of floating charge holders. The prescribed part fund is
sometimes referred to as the ``ring-fenced'' fund.

The prescribed part fund is calculated by reference to a percentage of
the company's net property which is subject to floating charges (50\% of
the first £10,000 of net floating charge realisations, plus 20\% of
anything thereafter, up to £800,000 limit). This is set aside for
distribution to the company's unsecured creditors
(\href{https://www.legislation.gov.uk/ukpga/1986/45/section/176}{s 176A
IA 1986}). `Net property' means the proceeds of selling property other
than that which is subject to a fixed charge, after deduction of the
liquidator's expenses and any preferential debts.

\begin{env-26bb6fb1-abd0-489e-8e80-3f170ecd0052}

Important

Secured creditors are not allowed to have access to the ring-fenced fund
for any unsecured portion of their debts (Re Airbase (UK) Ltd v HMRC
{[}2008{]} EWHC 124 (Ch)).

\end{env-26bb6fb1-abd0-489e-8e80-3f170ecd0052}

The pot of money is reserved to be shared rateably among the unsecured
creditors when they are paid. For this purpose, a floating charge holder
who suffers a shortfall on floating charge realisations does
\textbf{not} share in the prescribed part fund, although the shortfall
does constitute an unsecured claim against the company.

The ring-fencing provisions only apply to realisations from floating
charges created on or after the Relevant Date.

\hypertarget{floating-charge-creditors}{%
\subsubsection{Floating Charge
Creditors}\label{floating-charge-creditors}}

The liquidator next pays any remaining realisations from assets subject
to floating charges to the floating charge holders themselves (according
to the priority of their security, if there is more than one floating
charge holder).

\hypertarget{unsecured-creditors}{%
\subsubsection{Unsecured Creditors}\label{unsecured-creditors}}

These could include:

\begin{itemize}
\tightlist
\item
  Ordinary trade creditors who have not been paid
\item
  Secured creditors, to the extent that security is invalid or assets
  subject to the security have not realised sufficient funds to pay off
  the secured debt.
\end{itemize}

All the unsecured creditors rank and abate equally (``pari passu''
rule). So if not enough money is available, an equal percentage of the
total claim of each creditor will be paid. Secured creditors who have
not been paid in full from the realisation of assets subject to their
security can only claim as unsecured creditors against realisations from
unsecured assets, so they are \textbf{not} eligible for any payment from
the prescribed part fund.

\hypertarget{interest-on-unsecured-including-preferential-debts}{%
\subsubsection{Interest on Unsecured (including preferential)
Debts}\label{interest-on-unsecured-including-preferential-debts}}

Interest accruing on unsecured debts from the commencement of the
winding up.

\hypertarget{shareholders}{%
\subsubsection{Shareholders}\label{shareholders}}

The shareholders who participate in the equity of the company will rank
last. Their rights, as between themselves, will depend on the rights
attributable to their particular class/ classes of shares. This will be
written into the Articles of Association. Preferential shareholders may
have preferential rights to a return of their capital on a winding up in
priority to ordinary shareholders.

In most insolvent liquidations, the shareholders are unlikely to receive
any value from their shares.

\begin{env-751485c0-9cbb-4b3b-8497-6bad2ef120e5}

Rank and abate equally

Means that all creditors in a particular category share the money
available.

\end{env-751485c0-9cbb-4b3b-8497-6bad2ef120e5}

\hypertarget{voidable-transactions}{%
\section{Voidable Transactions}\label{voidable-transactions}}

\hypertarget{introduction}{%
\subsection{Introduction}\label{introduction}}

IA 1986 gives both a liquidator and an administrator the ability to
challenge certain transactions that have taken place within specified
statutory periods prior to company insolvency. These are known as
'voidable'/ 'antecedent' transactions.

The aim of a challenge is to restore the company to the same position it
would have been in had the transaction not taken place, and thereby,
increase the funds available to the insolvent estate for the benefit of
creditors.

Known as 'clawback' provisions, since they can result in an order
reversing transactions, or provision for financial restitution to be
paid. The beneficiary of the transaction with the insolvent company is
the target of the proceedings, rather than the directors responsible for
entering the transaction.

Liquidators or administrators may investigate fraudulent trading or
wrongful trading. On liquidation only, a liquidator may also disclaim
onerous transactions (IA 1986, s 178) and\\
investigate misfeasance (IA 1986, s 212).

\hypertarget{types}{%
\subsection{Types}\label{types}}

Four different voidable transactions may be challenged by either a
liquidator or administrator under relevant provisions of IA 1986:

\begin{enumerate}
\tightlist
\item
  Transactions at an undervalue
  (\href{https://www.legislation.gov.uk/ukpga/1986/45/section/238}{s 238
  IA 1986})
\item
  Transactions defrauding creditors
  (\href{https://www.legislation.gov.uk/ukpga/1986/45/section/238}{s 423
  IA 1986})
\item
  Preferences
  (\href{https://www.legislation.gov.uk/ukpga/1986/45/section/238}{s 239
  IA 1986})
\item
  Avoidance of floating charges
  (\href{https://www.legislation.gov.uk/ukpga/1986/45/section/238}{s 245
  IA 1986})
\end{enumerate}

\begin{longtable}[]{@{}lllll@{}}
\toprule()
Type of voidable transaction & Relevant time prior to the onset of
insolvency & Is insolvency required at the time or as a result of the
transaction? & What other elements are required? & Defences? \\
\midrule()
\endhead
Transaction at an undervalue -- s 238 & 2 years & Yes (presumed if
connected person) & Gift or consideration `significantly' less & Company
acted in good faith and transaction benefited the company \\
Preference -- s 239 & 6 months (2 years if connected person) & Yes &
Payment put creditor in a better position + Desire to prefer (desire to
prefer is presumed if connected person) & No desire to prefer e.g
commercial pressure. \\
Avoidance of floating charge -- s 245 & 12 months (2 years if connected)
& Yes (No need to prove if connected person) & Automatically void if no
`new' monies/'fresh' consideration & Valid for `new' monies/'fresh'
consideration \\
Transaction defrauding creditors -- s 423 & None & No & Transaction at
an undervalue and intention to defraud creditors & No intention/TUV \\
\bottomrule()
\end{longtable}

\hypertarget{challenging-voidable-transactions}{%
\subsection{Challenging Voidable
Transactions}\label{challenging-voidable-transactions}}

A liquidator or administrator seeking to challenge any voidable
transaction (except for a transaction defrauding creditors under s 423)
will need to ask the following questions in each case:

\begin{enumerate}
\tightlist
\item
  Did the transaction involve a `connected person' or `associate'?
\item
  Did the transaction take place within the `relevant time'?
\item
  Was the company insolvent at the time of the transaction, or did it
  become insolvent as a result of the transaction?
\item
  Is there a presumption available which shifts the burden of proof from
  the liquidator/administrator to the other party?
\end{enumerate}

\hypertarget{transactions-by-a-company-at-an-undervalue-tuv}{%
\subsection{Transactions by a Company at an Undervalue
(TUV)}\label{transactions-by-a-company-at-an-undervalue-tuv}}

The provisions under s 238 concern loss of value from a company, whether
through gifts or a significant inequality in consideration, to the
company's detriment, where the company is insolvent.

Insolvency here means 'inability to pay debts under s 123', i.e., either
on a cash flow or balance sheet basis. Note, this is a wider definition
than 'insolvency' for Wrongful trading purposes.

\hypertarget{bringing-a-claim-1}{%
\subsubsection{Bringing a Claim}\label{bringing-a-claim-1}}

A claim may be brought under s 238(1) by:

\begin{itemize}
\tightlist
\item
  A liquidator
\item
  An administrator
\end{itemize}

A transaction at an undervalue is either (s 238(4)):

\begin{enumerate}
\tightlist
\item
  A gift
\item
  A transaction for a consideration the value of which, in money or
  money's worth, is significantly less in value than the consideration
  provided by the company.
\end{enumerate}

This involves a monetary comparison between what the company gave away
and what it received under the transaction.

\hypertarget{granting-security-payment-of-a-dividend}{%
\subsubsection{Granting security/ Payment of a
Dividend}\label{granting-security-payment-of-a-dividend}}

It was generally thought that the granting of security by a company
cannot amount to a transaction at an undervalue on the basis that the
security does not itself deplete the assets of the company/ diminish
their value (Re MC Bacon Ltd {[}1990{]} BCLC 324).

But then in Hill v Spread Trustee Company Limited {[}2006{]} EWCA Civ
542 it was found that the granting of security for no consideration can
be challenged as a transaction at an undervalue.

So the law is now uncertain given the differences in views between these
cases.

\begin{env-440d3209-5a60-4bb6-8a13-881ba9304f97}

TLDR

Granting of security is unlikely to be a transactino at an undervalue.

\end{env-440d3209-5a60-4bb6-8a13-881ba9304f97}

Similar uncertainty exists around whether a dividend, lawfully paid,
could amount to a transaction at any undervalue. BTI 2014 LLC v Sequana
SA \& others {[}2019{]} EWCA Civ 112 suggests that a dividend can be
attacked as a transaction at an undervalue.

\hypertarget{when-and-how-can-the-transaction-be-avoided}{%
\subsubsection{When and How Can the Transaction Be
Avoided}\label{when-and-how-can-the-transaction-be-avoided}}

The court may set aside a transaction as a transaction at an undervalue
if:

\begin{itemize}
\tightlist
\item
  The company made a gift or otherwise entered into a transaction for a
  consideration, the value of which in money or money's worth is
  significantly less in value than the consideration provided by the
  company.
\item
  It took place within the `relevant time' (s 238(2)) -- in the
  \textbf{two years} preceding the onset of insolvency (s 240(1)(a))

  \begin{itemize}
  \tightlist
  \item
    "Onset of insolvency"

    \begin{itemize}
    \tightlist
    \item
      For CL, the date of presentation of the petition
    \item
      For CVL, the date it formally enters liquidation
    \item
      For administration, the date when the company files a Notice of
      Intention to Appoint an Administrator, or the date when it
      actually goes into administration (whichever is earlier).
    \item
      The commencement of the relevant insolvency procedure
      (administration or Liquidation) (s 240(3)).
    \end{itemize}
  \item
    Note that the relevant time is two years, regardless of whether the
    transaction took place with a connected person or not.
  \end{itemize}
\item
  It is proved by the applicant that the company was \textbf{insolvent}
  at the time of the transaction or became so as a result of it (s
  240(2)). Where a transaction at an undervalue is entered into with a
  person connected with the company, insolvency is presumed unless the
  connected person proves otherwise (s 240(2)).

  \begin{itemize}
  \tightlist
  \item
    Here connected person means director/shadow director, or an
    associate.
  \end{itemize}
\item
  Sections 249 and 435 set out the definitions of `connected persons'
  and `associates' respectively.
\end{itemize}

\hypertarget{defences}{%
\subsubsection{Defences}\label{defences}}

No order will be made to set aside the transaction if the court is
satisfied that (s 238(5) IA 1986):

\begin{enumerate}
\tightlist
\item
  The company entered into the transaction in \textbf{good faith} and
  for the purpose of carrying on its business; and
\item
  At the time there were \textbf{reasonable grounds} for believing that
  the transaction would benefit the company.
\end{enumerate}

This is often relied on in practice.

\hypertarget{sanctions}{%
\subsubsection{Sanctions}\label{sanctions}}

The court has a \textbf{discretion} to make such order as it thinks fit
to restore the position as if the company had not entered the
transaction (s 238(3)).

s 241(1): provides a non-exhaustive list of the types of restoration
order that the court might make under s 238 (and under s 239 in relation
to voidable preference). These include:

\begin{itemize}
\tightlist
\item
  Returning property to the company (s 241(1)(a))
\item
  Returning proceeds of sale to the company (s 241(1)(b))
\item
  Discharge of any security (s 241(1)(c)).
\end{itemize}

Any such order should not prejudice a subsequent purchaser from the
party which transacted at an undervalue with (or received a preference
from) the company, provided they were acting 'in good faith and for
value' (s 241(2)).

But under s 241(2A) there is a rebuttable presumption that an
acquisition by a subsequent purchaser was \textbf{not} in good faith,
where the subsequent purchaser either:

\begin{itemize}
\tightlist
\item
  Had notice of the relevant surrounding circumstances (i.e., the
  transaction at an undervalue or preference) and of the relevant
  proceedings; or
\item
  Was connected with or was an associate of either the company or the
  party which transacted at an undervalue with (or received a preference
  from) the company.
\end{itemize}

In such circumstances, the burden of proof shifts to the subsequent
purchaser to show good faith.

\hypertarget{transactions-defrauding-creditors-tdc}{%
\subsection{Transactions Defrauding Creditors
(TDC)}\label{transactions-defrauding-creditors-tdc}}

Such claims can be made under
\href{https://www.legislation.gov.uk/ukpga/1986/45/section/423}{s 423 IA
1986}. These claims do not necessarily relate to insolvency, can also be
brought by a victim of the transaction in question where the company is
solvent.

Requirements for the claim:

\begin{enumerate}
\tightlist
\item
  There has been a transaction at an undervalue (s 238 IA 1986)
\item
  The intention/ purpose of the transaction was to put assets beyond the
  reach of creditors of the company, or otherwise prejudice their
  interests (s 423(3) IA 1986).
\end{enumerate}

Then this transaction can be avoided under s 423(3). This includes
future creditors who were unknown at the time of transaction.

Insolvency practitioners may prefer to bring claims under s 238 (TUV)
rather than s 423. Under s 238, it need not be proved that the purpose
of the transaction was to put the assets beyond the reach of creditors,
or otherwise prejudice them.

Where the challenge is made by an administrator or liquidator, it may
therefore be easier to establish the claim under s 238, assuming that
the claim satisfies the criteria for challenging an undervalue
transaction under the above sections (i.e. `relevant time' and
insolvency).

\hypertarget{who-can-claim}{%
\subsubsection{Who Can Claim}\label{who-can-claim}}

An application to the court to set aside the transaction can be made by
any of the following (s 424):

\begin{enumerate}
\tightlist
\item
  A liquidator or an administrator;
\item
  A supervisor of a voluntary arrangement; or
\item
  A victim of the transaction in question.
\end{enumerate}

There is \textbf{no} `relevant time' or period within which the
transaction must have taken place. However, generally speaking, the more
recent the transaction, the more likely it is that the applicant will be
able to show the necessary intent.

The court may make such order as it thinks fit to restore the position
to what it would have been but for the transaction in question (s
423(2)). A non-exhaustive list of orders is set out in s 425(1).

The main advantage of a claim under s 423 is that it does not face the
risk of becoming time-barred in the same way as a claim under s 238.

\hypertarget{preferences-by-a-company}{%
\subsection{Preferences by a Company}\label{preferences-by-a-company}}

The purpose of
\href{https://www.legislation.gov.uk/ukpga/1986/45/section/239}{s 239 IA
1986} is to prevent a creditor obtaining an improper advantage over
other creditors of a company, at a time when the company is insolvent. A
claim may be brought under s 239(1) by:

\begin{itemize}
\tightlist
\item
  A liquidator
\item
  An administrator.
\end{itemize}

The company gives preference to a person if:

\begin{itemize}
\tightlist
\item
  That person is a creditor of the company (or a surety or guarantor of
  any of the company's debts or liabilities); and
\item
  The company does anything or allows anything to be done which has the
  effect of putting that person in a better position in the event of the
  company going into insolvent liquidation than he/she would otherwise
  have been in.
\end{itemize}

\hypertarget{when-can-preference-be-avoided}{%
\subsubsection{When Can Preference Be
Avoided}\label{when-can-preference-be-avoided}}

The preference is voidable if:

\begin{itemize}
\tightlist
\item
  It was given within the `relevant time' (s 239(2)) -- in the \textbf{6
  months} preceding the `onset of insolvency' (s 240(1)(b)), being the
  commencement of the relevant insolvency procedure (s 240(3)). The
  relevant time is extended to \textbf{2 years} for preferences to
  connected persons and associates (s 240(1)(a)). (See sections 249 and
  435 for definitions of `connected persons' and `associates').
\item
  It is proved that the company was insolvent (on either a cash flow or
  balance sheet basis) at the time of the transaction or became so as a
  result of it (s 240(2)).
\item
  It is proved that the company was `influenced \ldots{} by a desire' to
  prefer the creditor (s 239(5)). This is a \textbf{subjective} test (as
  opposed to an intention to prefer, which is objective). The company
  must have positively wished to put the party in a better position (Re
  MC Bacon Ltd {[}1990{]} BCLC 324).
\end{itemize}

\begin{env-c2ea7781-55a7-40ef-9219-ca9616c6b82e}

Warning

Here there is no statutory presumption of knowledge of insolvency where
the preference is given to a person who is connected with the company
(in contrast to TUV).

\end{env-c2ea7781-55a7-40ef-9219-ca9616c6b82e}

\begin{env-0a540a95-48c3-4466-84d1-6ddd726278e0}

Preference examples

\begin{itemize}
\tightlist
\item
  Making an unsecured creditor secured;
\item
  Paying one unsecured creditor before other unsecured creditors
\item
  Allowing a supplier of goods to change its T\&Cs to include a
  retention of title clause where none existed previously.
\item
  Allowing a creditor to enter judgment against the company when the
  company has a good defence of the claim.
\end{itemize}

\end{env-0a540a95-48c3-4466-84d1-6ddd726278e0}

\hypertarget{connected-persons}{%
\subsubsection{Connected Persons}\label{connected-persons}}

If the preference is given to a connected person or associate, there is
a rebuttable presumption that the company was influenced by the desire
to protect the creditor (s 239(6)). This shifts the burden of proof from
the liquidator or administrator to the preferred person to rebut the
statutory presumption.

So it is for the preferred person to rebut this presumption.

\hypertarget{defences-1}{%
\subsubsection{Defences}\label{defences-1}}

The defence available is an absence of the desire to prefer, required by
s 239(5).

\hypertarget{sanctions-1}{%
\subsubsection{Sanctions}\label{sanctions-1}}

The court has a discretion to make an order to restore the position as
if the company had not given the preference (s 239(3)).

Section 241(1) provides a non-exhaustive list of the types of
restoration order that the court may make.

Note that s 241(2) and s 241(2A) apply to both preferences and
transactions at an undervalue.

\hypertarget{exorbitant-credit-transactions}{%
\paragraph{Exorbitant Credit
Transactions}\label{exorbitant-credit-transactions}}

A liquidator or an administrator may apply to reopen credit transactions
made within three years from when the company went into liquidation (IA
1986, s 244(2)) if the credit transaction was "extortionate", meaning
the transaction grossly contravenes ordinary principles of fair dealing.

\hypertarget{when-can-floating-charges-be-avoided}{%
\subsection{When Can Floating Charges Be
Avoided}\label{when-can-floating-charges-be-avoided}}

For the floating charge to be invalid:

\begin{itemize}
\tightlist
\item
  Must have been created within the 'relevant time'

  \begin{itemize}
  \tightlist
  \item
    This is \textbf{12 months} preceding the onset of insolvency (i.e.,
    commencement of administration or liquidation (s 245(2) \&
    245(3)(b))
  \item
    Relevant time is extended to \textbf{two years} in the case of a
    floating charge granted to a connected person (s 245(3)(a)).
  \end{itemize}
\item
  Unless the floating charge was granted to a 'connected person' or
  'associate', it must be proved that the company:

  \begin{itemize}
  \tightlist
  \item
    Was insolvent (on a cash flow or balance sheet basis) at the time of
    the floating charge's creation, or
  \item
    Became insolvent as a consequence of the transaction under which the
    charge was created (s 245(4)).
  \end{itemize}
\end{itemize}

\hypertarget{when-are-new-floating-charges-valid}{%
\subsubsection{When Are New Floating Charges
Valid}\label{when-are-new-floating-charges-valid}}

Even if the above requirements are met, a floating charge will be valid
to the extent that 'new money' or other fresh consideration is provided
to the company (or existing debts extinguished), in return for the grant
of the floating charge on or after its creation (s 245(2)).

s 245(2) effect: if a floating charge is granted to secure the repayment
of a new loan made on or after the creation of the charge, then it will
be valid.

\begin{env-0a540a95-48c3-4466-84d1-6ddd726278e0}

Example

An example of when a floating charge would be void is where an existing
unsecured creditor is granted a floating charge by a company which is
insolvent and the charge purports to secure the repayment of existing
monies owed to that creditor.

If s 245 did not apply, such an unsecured creditor would thereby improve
its position in the order of priority if the company later went into an
insolvency procedure, which would be unfair on the company's other
unsecured creditors. However, if (and to the extent that) an existing
unsecured creditor provides further credit to the company (or to the
extent that any other new credit is given by a new creditor) then that
creditor is entitled to have the protection of a valid floating charge.

\end{env-0a540a95-48c3-4466-84d1-6ddd726278e0}

\hypertarget{avoidance-of-floating-charges}{%
\subsubsection{Avoidance of Floating
Charges}\label{avoidance-of-floating-charges}}

Where a floating charge is void under s 245, only the security (and its
advantage to a floating charge creditor in the order of priority) is
void, and not the debt itself. A floating charge is also void against
the liquidator, administrator, and other creditors, if it is not duly
registered with Companies House under
\href{https://www.legislation.gov.uk/ukpga/2006/46/section/859H}{s 859H
CA 2006}.

A floating charge granted to a creditor may also be voidable as a
transaction at an undervalue or a preference, under s 238 and 239.

\end{document}
