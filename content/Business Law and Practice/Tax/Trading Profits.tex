% Options for packages loaded elsewhere
\PassOptionsToPackage{unicode}{hyperref}
\PassOptionsToPackage{hyphens}{url}
%
\documentclass[
]{article}
\usepackage{amsmath,amssymb}
\usepackage{lmodern}
\usepackage{iftex}
\ifPDFTeX
  \usepackage[T1]{fontenc}
  \usepackage[utf8]{inputenc}
  \usepackage{textcomp} % provide euro and other symbols
\else % if luatex or xetex
  \usepackage{unicode-math}
  \defaultfontfeatures{Scale=MatchLowercase}
  \defaultfontfeatures[\rmfamily]{Ligatures=TeX,Scale=1}
\fi
% Use upquote if available, for straight quotes in verbatim environments
\IfFileExists{upquote.sty}{\usepackage{upquote}}{}
\IfFileExists{microtype.sty}{% use microtype if available
  \usepackage[]{microtype}
  \UseMicrotypeSet[protrusion]{basicmath} % disable protrusion for tt fonts
}{}
\makeatletter
\@ifundefined{KOMAClassName}{% if non-KOMA class
  \IfFileExists{parskip.sty}{%
    \usepackage{parskip}
  }{% else
    \setlength{\parindent}{0pt}
    \setlength{\parskip}{6pt plus 2pt minus 1pt}}
}{% if KOMA class
  \KOMAoptions{parskip=half}}
\makeatother
\usepackage{xcolor}
\usepackage[margin=1in]{geometry}
\usepackage{longtable,booktabs,array}
\usepackage{calc} % for calculating minipage widths
% Correct order of tables after \paragraph or \subparagraph
\usepackage{etoolbox}
\makeatletter
\patchcmd\longtable{\par}{\if@noskipsec\mbox{}\fi\par}{}{}
\makeatother
% Allow footnotes in longtable head/foot
\IfFileExists{footnotehyper.sty}{\usepackage{footnotehyper}}{\usepackage{footnote}}
\makesavenoteenv{longtable}
\setlength{\emergencystretch}{3em} % prevent overfull lines
\providecommand{\tightlist}{%
  \setlength{\itemsep}{0pt}\setlength{\parskip}{0pt}}
\setcounter{secnumdepth}{-\maxdimen} % remove section numbering
\usepackage{xcolor}
\definecolor{aliceblue}{HTML}{F0F8FF}
\definecolor{antiquewhite}{HTML}{FAEBD7}
\definecolor{aqua}{HTML}{00FFFF}
\definecolor{aquamarine}{HTML}{7FFFD4}
\definecolor{azure}{HTML}{F0FFFF}
\definecolor{beige}{HTML}{F5F5DC}
\definecolor{bisque}{HTML}{FFE4C4}
\definecolor{black}{HTML}{000000}
\definecolor{blanchedalmond}{HTML}{FFEBCD}
\definecolor{blue}{HTML}{0000FF}
\definecolor{blueviolet}{HTML}{8A2BE2}
\definecolor{brown}{HTML}{A52A2A}
\definecolor{burlywood}{HTML}{DEB887}
\definecolor{cadetblue}{HTML}{5F9EA0}
\definecolor{chartreuse}{HTML}{7FFF00}
\definecolor{chocolate}{HTML}{D2691E}
\definecolor{coral}{HTML}{FF7F50}
\definecolor{cornflowerblue}{HTML}{6495ED}
\definecolor{cornsilk}{HTML}{FFF8DC}
\definecolor{crimson}{HTML}{DC143C}
\definecolor{cyan}{HTML}{00FFFF}
\definecolor{darkblue}{HTML}{00008B}
\definecolor{darkcyan}{HTML}{008B8B}
\definecolor{darkgoldenrod}{HTML}{B8860B}
\definecolor{darkgray}{HTML}{A9A9A9}
\definecolor{darkgreen}{HTML}{006400}
\definecolor{darkgrey}{HTML}{A9A9A9}
\definecolor{darkkhaki}{HTML}{BDB76B}
\definecolor{darkmagenta}{HTML}{8B008B}
\definecolor{darkolivegreen}{HTML}{556B2F}
\definecolor{darkorange}{HTML}{FF8C00}
\definecolor{darkorchid}{HTML}{9932CC}
\definecolor{darkred}{HTML}{8B0000}
\definecolor{darksalmon}{HTML}{E9967A}
\definecolor{darkseagreen}{HTML}{8FBC8F}
\definecolor{darkslateblue}{HTML}{483D8B}
\definecolor{darkslategray}{HTML}{2F4F4F}
\definecolor{darkslategrey}{HTML}{2F4F4F}
\definecolor{darkturquoise}{HTML}{00CED1}
\definecolor{darkviolet}{HTML}{9400D3}
\definecolor{deeppink}{HTML}{FF1493}
\definecolor{deepskyblue}{HTML}{00BFFF}
\definecolor{dimgray}{HTML}{696969}
\definecolor{dimgrey}{HTML}{696969}
\definecolor{dodgerblue}{HTML}{1E90FF}
\definecolor{firebrick}{HTML}{B22222}
\definecolor{floralwhite}{HTML}{FFFAF0}
\definecolor{forestgreen}{HTML}{228B22}
\definecolor{fuchsia}{HTML}{FF00FF}
\definecolor{gainsboro}{HTML}{DCDCDC}
\definecolor{ghostwhite}{HTML}{F8F8FF}
\definecolor{gold}{HTML}{FFD700}
\definecolor{goldenrod}{HTML}{DAA520}
\definecolor{gray}{HTML}{808080}
\definecolor{green}{HTML}{008000}
\definecolor{greenyellow}{HTML}{ADFF2F}
\definecolor{grey}{HTML}{808080}
\definecolor{honeydew}{HTML}{F0FFF0}
\definecolor{hotpink}{HTML}{FF69B4}
\definecolor{indianred}{HTML}{CD5C5C}
\definecolor{indigo}{HTML}{4B0082}
\definecolor{ivory}{HTML}{FFFFF0}
\definecolor{khaki}{HTML}{F0E68C}
\definecolor{lavender}{HTML}{E6E6FA}
\definecolor{lavenderblush}{HTML}{FFF0F5}
\definecolor{lawngreen}{HTML}{7CFC00}
\definecolor{lemonchiffon}{HTML}{FFFACD}
\definecolor{lightblue}{HTML}{ADD8E6}
\definecolor{lightcoral}{HTML}{F08080}
\definecolor{lightcyan}{HTML}{E0FFFF}
\definecolor{lightgoldenrodyellow}{HTML}{FAFAD2}
\definecolor{lightgray}{HTML}{D3D3D3}
\definecolor{lightgreen}{HTML}{90EE90}
\definecolor{lightgrey}{HTML}{D3D3D3}
\definecolor{lightpink}{HTML}{FFB6C1}
\definecolor{lightsalmon}{HTML}{FFA07A}
\definecolor{lightseagreen}{HTML}{20B2AA}
\definecolor{lightskyblue}{HTML}{87CEFA}
\definecolor{lightslategray}{HTML}{778899}
\definecolor{lightslategrey}{HTML}{778899}
\definecolor{lightsteelblue}{HTML}{B0C4DE}
\definecolor{lightyellow}{HTML}{FFFFE0}
\definecolor{lime}{HTML}{00FF00}
\definecolor{limegreen}{HTML}{32CD32}
\definecolor{linen}{HTML}{FAF0E6}
\definecolor{magenta}{HTML}{FF00FF}
\definecolor{maroon}{HTML}{800000}
\definecolor{mediumaquamarine}{HTML}{66CDAA}
\definecolor{mediumblue}{HTML}{0000CD}
\definecolor{mediumorchid}{HTML}{BA55D3}
\definecolor{mediumpurple}{HTML}{9370DB}
\definecolor{mediumseagreen}{HTML}{3CB371}
\definecolor{mediumslateblue}{HTML}{7B68EE}
\definecolor{mediumspringgreen}{HTML}{00FA9A}
\definecolor{mediumturquoise}{HTML}{48D1CC}
\definecolor{mediumvioletred}{HTML}{C71585}
\definecolor{midnightblue}{HTML}{191970}
\definecolor{mintcream}{HTML}{F5FFFA}
\definecolor{mistyrose}{HTML}{FFE4E1}
\definecolor{moccasin}{HTML}{FFE4B5}
\definecolor{navajowhite}{HTML}{FFDEAD}
\definecolor{navy}{HTML}{000080}
\definecolor{oldlace}{HTML}{FDF5E6}
\definecolor{olive}{HTML}{808000}
\definecolor{olivedrab}{HTML}{6B8E23}
\definecolor{orange}{HTML}{FFA500}
\definecolor{orangered}{HTML}{FF4500}
\definecolor{orchid}{HTML}{DA70D6}
\definecolor{palegoldenrod}{HTML}{EEE8AA}
\definecolor{palegreen}{HTML}{98FB98}
\definecolor{paleturquoise}{HTML}{AFEEEE}
\definecolor{palevioletred}{HTML}{DB7093}
\definecolor{papayawhip}{HTML}{FFEFD5}
\definecolor{peachpuff}{HTML}{FFDAB9}
\definecolor{peru}{HTML}{CD853F}
\definecolor{pink}{HTML}{FFC0CB}
\definecolor{plum}{HTML}{DDA0DD}
\definecolor{powderblue}{HTML}{B0E0E6}
\definecolor{purple}{HTML}{800080}
\definecolor{red}{HTML}{FF0000}
\definecolor{rosybrown}{HTML}{BC8F8F}
\definecolor{royalblue}{HTML}{4169E1}
\definecolor{saddlebrown}{HTML}{8B4513}
\definecolor{salmon}{HTML}{FA8072}
\definecolor{sandybrown}{HTML}{F4A460}
\definecolor{seagreen}{HTML}{2E8B57}
\definecolor{seashell}{HTML}{FFF5EE}
\definecolor{sienna}{HTML}{A0522D}
\definecolor{silver}{HTML}{C0C0C0}
\definecolor{skyblue}{HTML}{87CEEB}
\definecolor{slateblue}{HTML}{6A5ACD}
\definecolor{slategray}{HTML}{708090}
\definecolor{slategrey}{HTML}{708090}
\definecolor{snow}{HTML}{FFFAFA}
\definecolor{springgreen}{HTML}{00FF7F}
\definecolor{steelblue}{HTML}{4682B4}
\definecolor{tan}{HTML}{D2B48C}
\definecolor{teal}{HTML}{008080}
\definecolor{thistle}{HTML}{D8BFD8}
\definecolor{tomato}{HTML}{FF6347}
\definecolor{turquoise}{HTML}{40E0D0}
\definecolor{violet}{HTML}{EE82EE}
\definecolor{wheat}{HTML}{F5DEB3}
\definecolor{white}{HTML}{FFFFFF}
\definecolor{whitesmoke}{HTML}{F5F5F5}
\definecolor{yellow}{HTML}{FFFF00}
\definecolor{yellowgreen}{HTML}{9ACD32}
\usepackage[most]{tcolorbox}

\usepackage{ifthen}
\provideboolean{admonitiontwoside}
\makeatletter%
\if@twoside%
\setboolean{admonitiontwoside}{true}
\else%
\setboolean{admonitiontwoside}{false}
\fi%
\makeatother%

\newenvironment{env-3ae98384-e2ec-4d09-ac55-0ea7db18e4d9}
{
    \savenotes\tcolorbox[blanker,breakable,left=5pt,borderline west={2pt}{-4pt}{firebrick}]
}
{
    \endtcolorbox\spewnotes
}
                

\newenvironment{env-e05fd1ac-83a5-41ce-95fd-28930cb87831}
{
    \savenotes\tcolorbox[blanker,breakable,left=5pt,borderline west={2pt}{-4pt}{blue}]
}
{
    \endtcolorbox\spewnotes
}
                

\newenvironment{env-b05156de-272c-4941-ad2d-f95253330136}
{
    \savenotes\tcolorbox[blanker,breakable,left=5pt,borderline west={2pt}{-4pt}{green}]
}
{
    \endtcolorbox\spewnotes
}
                

\newenvironment{env-eaa47f00-a623-48fb-b749-5dda9cde5dea}
{
    \savenotes\tcolorbox[blanker,breakable,left=5pt,borderline west={2pt}{-4pt}{aquamarine}]
}
{
    \endtcolorbox\spewnotes
}
                

\newenvironment{env-cb77b424-888b-4192-8bb4-4a31d57bbcb7}
{
    \savenotes\tcolorbox[blanker,breakable,left=5pt,borderline west={2pt}{-4pt}{orange}]
}
{
    \endtcolorbox\spewnotes
}
                

\newenvironment{env-e20bae03-a75a-491d-88a0-5d0510ce977c}
{
    \savenotes\tcolorbox[blanker,breakable,left=5pt,borderline west={2pt}{-4pt}{blue}]
}
{
    \endtcolorbox\spewnotes
}
                

\newenvironment{env-2f161b0c-3bbb-42f2-a685-2a4a488d7eb3}
{
    \savenotes\tcolorbox[blanker,breakable,left=5pt,borderline west={2pt}{-4pt}{yellow}]
}
{
    \endtcolorbox\spewnotes
}
                

\newenvironment{env-efc06023-58aa-4c2f-b0a9-4e44a4470353}
{
    \savenotes\tcolorbox[blanker,breakable,left=5pt,borderline west={2pt}{-4pt}{darkred}]
}
{
    \endtcolorbox\spewnotes
}
                

\newenvironment{env-1a66adc4-dc6c-4a2d-a867-62a6a9970cae}
{
    \savenotes\tcolorbox[blanker,breakable,left=5pt,borderline west={2pt}{-4pt}{pink}]
}
{
    \endtcolorbox\spewnotes
}
                

\newenvironment{env-069142a4-c114-4ef1-a52b-99fe622a8735}
{
    \savenotes\tcolorbox[blanker,breakable,left=5pt,borderline west={2pt}{-4pt}{cyan}]
}
{
    \endtcolorbox\spewnotes
}
                

\newenvironment{env-b61b3e6f-3f12-4601-9a57-a38348b0a5b7}
{
    \savenotes\tcolorbox[blanker,breakable,left=5pt,borderline west={2pt}{-4pt}{cyan}]
}
{
    \endtcolorbox\spewnotes
}
                

\newenvironment{env-4d7b0451-0460-4d48-9d7f-d4cfd102b52c}
{
    \savenotes\tcolorbox[blanker,breakable,left=5pt,borderline west={2pt}{-4pt}{purple}]
}
{
    \endtcolorbox\spewnotes
}
                

\newenvironment{env-4c715519-3f47-4942-b40b-cfb73c4b348c}
{
    \savenotes\tcolorbox[blanker,breakable,left=5pt,borderline west={2pt}{-4pt}{darksalmon}]
}
{
    \endtcolorbox\spewnotes
}
                

\newenvironment{env-fccd4df8-632d-4617-b2b2-7da032db8c20}
{
    \savenotes\tcolorbox[blanker,breakable,left=5pt,borderline west={2pt}{-4pt}{gray}]
}
{
    \endtcolorbox\spewnotes
}
                
\ifLuaTeX
  \usepackage{selnolig}  % disable illegal ligatures
\fi
\IfFileExists{bookmark.sty}{\usepackage{bookmark}}{\usepackage{hyperref}}
\IfFileExists{xurl.sty}{\usepackage{xurl}}{} % add URL line breaks if available
\urlstyle{same} % disable monospaced font for URLs
\hypersetup{
  pdftitle={Trading Profits},
  hidelinks,
  pdfcreator={LaTeX via pandoc}}

\title{Trading Profits}
\author{}
\date{}

\begin{document}
\maketitle

{
\setcounter{tocdepth}{3}
\tableofcontents
}
\hypertarget{trading-profits}{%
\section{Trading Profits}\label{trading-profits}}

\begin{longtable}[]{@{}ll@{}}
\toprule()
Corporate structure & Income profits taxation \\
\midrule()
\endhead
Sole trader & Income tax \\
Company & Corporation tax \\
Partnership & Income tax of partners/ corporation tax of corporate
partners. \\
\bottomrule()
\end{longtable}

Relevant legislation:

\begin{itemize}
\tightlist
\item
  Income Tax (Trading and Other Income) Act 2005 (ITTOIA 2005)
\item
  Corporation Tax Act 2009 (CTA 2009)
\item
  Corporation Tax Act 2010 (CTA 2010)
\end{itemize}

\hypertarget{calculation}{%
\subsection{Calculation}\label{calculation}}

{\(\text{Trading\ profits} = \text{Chargeable\ receipts} - \text{Deductible\ expenditure} - \text{Capital\ allowance}\)}

Very small businesses may be allowed to work on a cash basis, but the
vast majority of businesses must calculate trading profits.

\hypertarget{chargeable-receipts}{%
\subsection{Chargeable Receipts}\label{chargeable-receipts}}

``Trade'' is given its ordinary meaning of operations of a commercial
character.

\hypertarget{deductible-expenditure}{%
\subsection{Deductible Expenditure}\label{deductible-expenditure}}

\begin{env-b05156de-272c-4941-ad2d-f95253330136}

s 24 ITTOIA 2005 - Expenses not wholly and exclusively for trade and
unconnected losses

(1) In calculating the profits of a trade, no deduction is allowed
for---

\begin{itemize}
\tightlist
\item
  (a) expenses not incurred wholly and exclusively for the purposes of
  the trade, or
\item
  (b) losses not connected with or arising out of the trade.
\end{itemize}

(2) If an expense is incurred for more than one purpose, this section
does not prohibit a deduction for any identifiable part or identifiable
proportion of the expense which is incurred wholly and exclusively for
the purposes of the trade.

\end{env-b05156de-272c-4941-ad2d-f95253330136}

Must deduct:

\begin{itemize}
\tightlist
\item
  any expenditure of an income nature,

  \begin{itemize}
  \tightlist
  \item
    Expenditure incurred for the purpose of enabling the trader to sell
    the item at a profit.
  \item
    Income expenditure usually has the quality of recurrence.
  \end{itemize}
\item
  which has been incurred 'wholly and exclusively' for the trade, and

  \begin{itemize}
  \tightlist
  \item
    So cannot be for a dual purpose.
  \item
    e.g., having a restaurant meal while working away from home is for
    the dual purpose of eating (Caillebotte v Quinn {[}1975{]} 1 WLR
    731).
  \item
    But realistically, HMRC allows some expenses to be apportioned, so
    part is deductible.
  \end{itemize}
\item
  deduction of which is not prohibited by statute.
\end{itemize}

Commonly deducted expenses include salaries, rent, utility charges,
stock, business rates and stationery.

\hypertarget{capital-allowances}{%
\subsection{Capital Allowances}\label{capital-allowances}}

Expenditure on capital items cannot in principle be deducted from
chargeable receipts because it is not of a capital nature.

But this could cause cash flow difficulties, so legislation allows a
specified amount of the cost to be deducted each year in calculating
trading profits. Principally applies to plant and machinery.

This is the equivalent of amortisation in accounting.

\begin{env-2f161b0c-3bbb-42f2-a685-2a4a488d7eb3}

Capital allowances

Under the Capital Allowances Act 2001 (CAA 2001), where expenditure is
incurred on certain assets or activities, a percentage of the capital
expenditure will be allowed as a deduction in calculating trading
profits in a given accounting period. These deductions are known as
capital allowances.

\end{env-2f161b0c-3bbb-42f2-a685-2a4a488d7eb3}

\hypertarget{plant-and-machinery} of the reducing balance of the cost of the asset
in calculating their trading profits (CAA 2001, ss 52--59).

\hypertarget{sale-of-assets}{%
\paragraph{Sale of Assets}\label{sale-of-assets}}

If plant/ machinery is sold:

\begin{itemize}
\tightlist
\item
  At a profit, there may be a balancing charge
\item
  At a loss, there may be a balancing allowance.
\end{itemize}

These ensure that the taxpayer had relief for exactly the amount by
which the asset diminished.

\hypertarget{pooling}{%
\paragraph{Pooling}\label{pooling}}

Generally, all expenditure on plant and machinery is pooled together and
the writing down allowance given each year on the balance of
expenditure.

\begin{env-b05156de-272c-4941-ad2d-f95253330136}

s 55 CAA 2001 - Determination of entitlement or liability

(1) Whether a person is entitled to a writing-down allowance or a
balancing allowance, or liable to a balancing charge, for a chargeable
period is determined separately for each pool of qualifying expenditure
and depends on---

\begin{itemize}
\tightlist
\item
  (a) the available qualifying expenditure in that pool for that period
  (``AQE''), and
\item
  (b) the total of any disposal receipts to be brought into account in
  that pool for that period (``TDR'').
\end{itemize}

(2) If AQE exceeds TDR, the person is entitled to a writing-down
allowance or a balancing allowance for the period.

(3) If TDR exceeds AQE, the person is liable to a balancing charge for
the period.

(4) The entitlement under subsection (2) is to a writing-down allowance
except for the final chargeable period when it is to a balancing
allowance.

\end{env-b05156de-272c-4941-ad2d-f95253330136}

Generally, no balancing allowance or charge should occur until the whole
pool is sold/ trade discontinued.

If the written-down value of the pool being used for an accounting
period is ever £1,000 or less, the trader has the option of claiming a
writing down allowance big enough to eliminate the pool.

\hypertarget{annual-investment-allowance}{%
\paragraph{Annual Investment
Allowance}\label{annual-investment-allowance}}

Every ongoing business receives an annual investment allowance up to
£1,000,000. So the first £1,000,000 of fresh \textbf{qualifying}
expenditure on plant and machinery will be fully deductible.

\begin{env-b61b3e6f-3f12-4601-9a57-a38348b0a5b7}

Important

The key here is "qualifying". So if you're not qualifying (meaning would
usually qualify for 18\% deduction, use the annual investment
allowance).

\end{env-b61b3e6f-3f12-4601-9a57-a38348b0a5b7}

Note that this was meant to be a temporary measure from £200,000, but
hasn't been changed.

Note this can apply to second hand machinery being purchased. So either
apply AIA or super-deduction, but never both.

\hypertarget{super-deduction}{%
\paragraph{Super-deduction}\label{super-deduction}}

'Temporary' (until 31/03/23) capital allowances for \textbf{companies}
investing in \textbf{new} \textbf{qualifying} plant and machinery over
COVID-19: 130\% super-deduction rather than the normal 18\% deduction.

Expenditure on assets usually attracting a 6\% (e.g., life-long assets)
receive first year allowance of 50\%.

\begin{env-cb77b424-888b-4192-8bb4-4a31d57bbcb7}

Warning

Only applies to companies.

\end{env-cb77b424-888b-4192-8bb4-4a31d57bbcb7}

\hypertarget{special-rules}{%
\paragraph{Special Rules}\label{special-rules}}

\begin{itemize}
\tightlist
\item
  Structures and buildings

  \begin{itemize}
  \tightlist
  \item
    Capital Allowances (Structures and Buildings Allowances) Regulations
    2019
  \item
    Flat 3\% allowance on qualifying expenditure incurred on
    construction, renovation or conversion of commercial structures.
  \item
    Confined to physical construction, does not include cost of land.
  \end{itemize}
\item
  Short life assets (CAA 2001, ss 83-89)

  \begin{itemize}
  \tightlist
  \item
    If a business thinks an asset will have a useful working life of
    {\(< 8\)} years, election can be made (within 2 years) to treat the
    asset as a short-life asset.
  \end{itemize}
\item
  Life-long assets and integral features (CAA 2001, s 33A and ss
  90-104E)

  \begin{itemize}
  \tightlist
  \item
    Expected working life of {\(\geq 25\)} years.
  \item
    ``Integral features'' includes escalators, electrical and lighting
    etc.
  \item
    Only qualify for 6\% writing down allowance.
  \item
    Eligible for AIA.
  \end{itemize}
\end{itemize}

\hypertarget{trading-loss-relief}{%
\subsection{Trading Loss Relief}\label{trading-loss-relief}}

If there is a trading loss, there is no trading income. Relief on
trading loss available for tax purposes.

Where relief can be claimed under more than one provision, the taxpayer
can elect which to claim, or can claim more than one if there is a big
enough loss.

\hypertarget{start-up-loss-relief-ss-72-81-ita-2007}{%
\subsubsection{Start-up Loss Relief (ss 72-81 ITA
2007)}\label{start-up-loss-relief-ss-72-81-ita-2007}}

Start-up relief applies if the taxpayer suffers a loss which is assessed
in any of the first 4 years of a business. The loss can be carried back
and set against the taxpayer's total income in 3 years prior to the tax
year of the loss.

The loss must be set against earlier years before later years. there is
a limit of £50,000.

\hypertarget{carry-across-relief-ss-64-71-ita-2007}{%
\subsubsection{Carry-across Relief (ss 64-71 ITA
2007)}\label{carry-across-relief-ss-64-71-ita-2007}}

\hypertarget{set-off-against-total-income-s-64-ita-2007}{%
\paragraph{Set Off Against Total Income (s 64 ITA
2007)}\label{set-off-against-total-income-s-64-ita-2007}}

Trading loss arising in an accounting period is treated as a loss of the
tax year in which the accounting period ends. Can be set against:

\begin{itemize}
\tightlist
\item
  Taxpayer's total income in the loss-making year, or
\item
  Taxpayer's total income in preceding year.
\end{itemize}

If the loss is big enough, can also set against a combination of these
two years (but using up one first).

COVID-19 extension: trading losses can be carried back for 3 years. Must
be set against later years first, and carry-back to 2nd and 3rd years is
£2 million.

\hypertarget{set-off-against-capital-gains-s-71-ita-2007}{%
\paragraph{Set Off Against Capital Gains (s 71 ITA
2007)}\label{set-off-against-capital-gains-s-71-ita-2007}}

After s 64 relief, further relief can be obtained by setting against a
taxpayer's chargeable capital gains for that tax year.

\hypertarget{carry-forward-relief-ss-83-85-ita-2007}{%
\subsubsection{Carry-forward Relief (ss 83-85 ITA
2007)}\label{carry-forward-relief-ss-83-85-ita-2007}}

A loss can be carried forward and set against subsequent
\textbf{profits} which the trade produces, taking earlier years first.

Note the qualifications:

\begin{itemize}
\tightlist
\item
  Must wait for future profits before relief can be claimed
\item
  Can only set against profits, not income or capital.
\item
  BUT can carry forward indefinitely until a profit arises.
\item
  No cap on the amount which can be claimed.
\end{itemize}

\hypertarget{carry-back-of-terminal-trading-loss-ss-89-94-ita-2007}{%
\paragraph{Carry Back of Terminal Trading Loss (ss 89-94 ITA
2007)}\label{carry-back-of-terminal-trading-loss-ss-89-94-ita-2007}}

If there is a trading loss in the last 12 months of the trade, it can be
carried across trading profits made in the final tax year and then
carried back to the 3 years prior.

Does not allow relief against non-trading income or capital gains.

4 years in which to claim relief. No cap.

\hypertarget{carry-forward-relief-on-incorporation}{%
\subsubsection{Carry-forward Relief on
Incorporation}\label{carry-forward-relief-on-incorporation}}

Losses can be carried forward and set against income which the taxpayer
receives from the company. At least 80\% of the consideration for the
transfer must consist of shares in the company.

Can be carried forward indefinitely, no cap.

\hypertarget{restrictions}{%
\subsubsection{Restrictions}\label{restrictions}}

Under Finance Act 2013, there is a cap on relief which can be claimed in
any tax year, set as the greater of:

\begin{itemize}
\tightlist
\item
  £50,000, or
\item
  25\% of taxpayer's total income.
\end{itemize}

The cap only relates to income from sources other than the trade which
produced the loss.

Applies to start-up relief, carry-across/ carry-back relief, interest
relief on loans to close companies and interest relief for partnership
investment.

\begin{longtable}[]{@{}llll@{}}
\toprule()
ITA 2007 section & When will loss have occurred & Against what will loss
be set & Relevant period? \\
\midrule()
\endhead
s 72 (start-up relief by carry-back) & First 4 tax years of trading &
Total income* & 3 tax years preceding tax year of loss. \\
s 63 (carry-across/ carry-back one year relief) & Any accounting year of
trading & Total income* and thereafter chargeable gains & Loss-making
tax year/ preceding tax year. \\
s 83 (carry-forward relief) & Any accounting year of trading &
Subsequent profits of the same trade & Any subsequent tax year until
loss absorbed. \\
s 89 (terminal relief by carry-back) & Final 12 months of trading &
Previous profits of the same trade & Final tax year and then 3 years
preceding. \\
s 86 (carry-forward relief on incorporation) & Up to incorporation &
Subsequent income received from company & Any subsequent tax year until
loss absorbed. \\
\bottomrule()
\end{longtable}

*subject to cap.

\end{document}
