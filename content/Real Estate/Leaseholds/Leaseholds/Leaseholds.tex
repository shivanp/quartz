% Options for packages loaded elsewhere
\PassOptionsToPackage{unicode}{hyperref}
\PassOptionsToPackage{hyphens}{url}
%
\documentclass[
]{article}
\usepackage{amsmath,amssymb}
\usepackage{lmodern}
\usepackage{iftex}
\ifPDFTeX
  \usepackage[T1]{fontenc}
  \usepackage[utf8]{inputenc}
  \usepackage{textcomp} % provide euro and other symbols
\else % if luatex or xetex
  \usepackage{unicode-math}
  \defaultfontfeatures{Scale=MatchLowercase}
  \defaultfontfeatures[\rmfamily]{Ligatures=TeX,Scale=1}
\fi
% Use upquote if available, for straight quotes in verbatim environments
\IfFileExists{upquote.sty}{\usepackage{upquote}}{}
\IfFileExists{microtype.sty}{% use microtype if available
  \usepackage[]{microtype}
  \UseMicrotypeSet[protrusion]{basicmath} % disable protrusion for tt fonts
}{}
\makeatletter
\@ifundefined{KOMAClassName}{% if non-KOMA class
  \IfFileExists{parskip.sty}{%
    \usepackage{parskip}
  }{% else
    \setlength{\parindent}{0pt}
    \setlength{\parskip}{6pt plus 2pt minus 1pt}}
}{% if KOMA class
  \KOMAoptions{parskip=half}}
\makeatother
\usepackage{xcolor}
\usepackage[margin=1in]{geometry}
\usepackage{longtable,booktabs,array}
\usepackage{calc} % for calculating minipage widths
% Correct order of tables after \paragraph or \subparagraph
\usepackage{etoolbox}
\makeatletter
\patchcmd\longtable{\par}{\if@noskipsec\mbox{}\fi\par}{}{}
\makeatother
% Allow footnotes in longtable head/foot
\IfFileExists{footnotehyper.sty}{\usepackage{footnotehyper}}{\usepackage{footnote}}
\makesavenoteenv{longtable}
\usepackage{graphicx}
\makeatletter
\def\maxwidth{\ifdim\Gin@nat@width>\linewidth\linewidth\else\Gin@nat@width\fi}
\def\maxheight{\ifdim\Gin@nat@height>\textheight\textheight\else\Gin@nat@height\fi}
\makeatother
% Scale images if necessary, so that they will not overflow the page
% margins by default, and it is still possible to overwrite the defaults
% using explicit options in \includegraphics[width, height, ...]{}
\setkeys{Gin}{width=\maxwidth,height=\maxheight,keepaspectratio}
% Set default figure placement to htbp
\makeatletter
\def\fps@figure{htbp}
\makeatother
\setlength{\emergencystretch}{3em} % prevent overfull lines
\providecommand{\tightlist}{%
  \setlength{\itemsep}{0pt}\setlength{\parskip}{0pt}}
\setcounter{secnumdepth}{-\maxdimen} % remove section numbering
\usepackage{xcolor}
\definecolor{aliceblue}{HTML}{F0F8FF}
\definecolor{antiquewhite}{HTML}{FAEBD7}
\definecolor{aqua}{HTML}{00FFFF}
\definecolor{aquamarine}{HTML}{7FFFD4}
\definecolor{azure}{HTML}{F0FFFF}
\definecolor{beige}{HTML}{F5F5DC}
\definecolor{bisque}{HTML}{FFE4C4}
\definecolor{black}{HTML}{000000}
\definecolor{blanchedalmond}{HTML}{FFEBCD}
\definecolor{blue}{HTML}{0000FF}
\definecolor{blueviolet}{HTML}{8A2BE2}
\definecolor{brown}{HTML}{A52A2A}
\definecolor{burlywood}{HTML}{DEB887}
\definecolor{cadetblue}{HTML}{5F9EA0}
\definecolor{chartreuse}{HTML}{7FFF00}
\definecolor{chocolate}{HTML}{D2691E}
\definecolor{coral}{HTML}{FF7F50}
\definecolor{cornflowerblue}{HTML}{6495ED}
\definecolor{cornsilk}{HTML}{FFF8DC}
\definecolor{crimson}{HTML}{DC143C}
\definecolor{cyan}{HTML}{00FFFF}
\definecolor{darkblue}{HTML}{00008B}
\definecolor{darkcyan}{HTML}{008B8B}
\definecolor{darkgoldenrod}{HTML}{B8860B}
\definecolor{darkgray}{HTML}{A9A9A9}
\definecolor{darkgreen}{HTML}{006400}
\definecolor{darkgrey}{HTML}{A9A9A9}
\definecolor{darkkhaki}{HTML}{BDB76B}
\definecolor{darkmagenta}{HTML}{8B008B}
\definecolor{darkolivegreen}{HTML}{556B2F}
\definecolor{darkorange}{HTML}{FF8C00}
\definecolor{darkorchid}{HTML}{9932CC}
\definecolor{darkred}{HTML}{8B0000}
\definecolor{darksalmon}{HTML}{E9967A}
\definecolor{darkseagreen}{HTML}{8FBC8F}
\definecolor{darkslateblue}{HTML}{483D8B}
\definecolor{darkslategray}{HTML}{2F4F4F}
\definecolor{darkslategrey}{HTML}{2F4F4F}
\definecolor{darkturquoise}{HTML}{00CED1}
\definecolor{darkviolet}{HTML}{9400D3}
\definecolor{deeppink}{HTML}{FF1493}
\definecolor{deepskyblue}{HTML}{00BFFF}
\definecolor{dimgray}{HTML}{696969}
\definecolor{dimgrey}{HTML}{696969}
\definecolor{dodgerblue}{HTML}{1E90FF}
\definecolor{firebrick}{HTML}{B22222}
\definecolor{floralwhite}{HTML}{FFFAF0}
\definecolor{forestgreen}{HTML}{228B22}
\definecolor{fuchsia}{HTML}{FF00FF}
\definecolor{gainsboro}{HTML}{DCDCDC}
\definecolor{ghostwhite}{HTML}{F8F8FF}
\definecolor{gold}{HTML}{FFD700}
\definecolor{goldenrod}{HTML}{DAA520}
\definecolor{gray}{HTML}{808080}
\definecolor{green}{HTML}{008000}
\definecolor{greenyellow}{HTML}{ADFF2F}
\definecolor{grey}{HTML}{808080}
\definecolor{honeydew}{HTML}{F0FFF0}
\definecolor{hotpink}{HTML}{FF69B4}
\definecolor{indianred}{HTML}{CD5C5C}
\definecolor{indigo}{HTML}{4B0082}
\definecolor{ivory}{HTML}{FFFFF0}
\definecolor{khaki}{HTML}{F0E68C}
\definecolor{lavender}{HTML}{E6E6FA}
\definecolor{lavenderblush}{HTML}{FFF0F5}
\definecolor{lawngreen}{HTML}{7CFC00}
\definecolor{lemonchiffon}{HTML}{FFFACD}
\definecolor{lightblue}{HTML}{ADD8E6}
\definecolor{lightcoral}{HTML}{F08080}
\definecolor{lightcyan}{HTML}{E0FFFF}
\definecolor{lightgoldenrodyellow}{HTML}{FAFAD2}
\definecolor{lightgray}{HTML}{D3D3D3}
\definecolor{lightgreen}{HTML}{90EE90}
\definecolor{lightgrey}{HTML}{D3D3D3}
\definecolor{lightpink}{HTML}{FFB6C1}
\definecolor{lightsalmon}{HTML}{FFA07A}
\definecolor{lightseagreen}{HTML}{20B2AA}
\definecolor{lightskyblue}{HTML}{87CEFA}
\definecolor{lightslategray}{HTML}{778899}
\definecolor{lightslategrey}{HTML}{778899}
\definecolor{lightsteelblue}{HTML}{B0C4DE}
\definecolor{lightyellow}{HTML}{FFFFE0}
\definecolor{lime}{HTML}{00FF00}
\definecolor{limegreen}{HTML}{32CD32}
\definecolor{linen}{HTML}{FAF0E6}
\definecolor{magenta}{HTML}{FF00FF}
\definecolor{maroon}{HTML}{800000}
\definecolor{mediumaquamarine}{HTML}{66CDAA}
\definecolor{mediumblue}{HTML}{0000CD}
\definecolor{mediumorchid}{HTML}{BA55D3}
\definecolor{mediumpurple}{HTML}{9370DB}
\definecolor{mediumseagreen}{HTML}{3CB371}
\definecolor{mediumslateblue}{HTML}{7B68EE}
\definecolor{mediumspringgreen}{HTML}{00FA9A}
\definecolor{mediumturquoise}{HTML}{48D1CC}
\definecolor{mediumvioletred}{HTML}{C71585}
\definecolor{midnightblue}{HTML}{191970}
\definecolor{mintcream}{HTML}{F5FFFA}
\definecolor{mistyrose}{HTML}{FFE4E1}
\definecolor{moccasin}{HTML}{FFE4B5}
\definecolor{navajowhite}{HTML}{FFDEAD}
\definecolor{navy}{HTML}{000080}
\definecolor{oldlace}{HTML}{FDF5E6}
\definecolor{olive}{HTML}{808000}
\definecolor{olivedrab}{HTML}{6B8E23}
\definecolor{orange}{HTML}{FFA500}
\definecolor{orangered}{HTML}{FF4500}
\definecolor{orchid}{HTML}{DA70D6}
\definecolor{palegoldenrod}{HTML}{EEE8AA}
\definecolor{palegreen}{HTML}{98FB98}
\definecolor{paleturquoise}{HTML}{AFEEEE}
\definecolor{palevioletred}{HTML}{DB7093}
\definecolor{papayawhip}{HTML}{FFEFD5}
\definecolor{peachpuff}{HTML}{FFDAB9}
\definecolor{peru}{HTML}{CD853F}
\definecolor{pink}{HTML}{FFC0CB}
\definecolor{plum}{HTML}{DDA0DD}
\definecolor{powderblue}{HTML}{B0E0E6}
\definecolor{purple}{HTML}{800080}
\definecolor{red}{HTML}{FF0000}
\definecolor{rosybrown}{HTML}{BC8F8F}
\definecolor{royalblue}{HTML}{4169E1}
\definecolor{saddlebrown}{HTML}{8B4513}
\definecolor{salmon}{HTML}{FA8072}
\definecolor{sandybrown}{HTML}{F4A460}
\definecolor{seagreen}{HTML}{2E8B57}
\definecolor{seashell}{HTML}{FFF5EE}
\definecolor{sienna}{HTML}{A0522D}
\definecolor{silver}{HTML}{C0C0C0}
\definecolor{skyblue}{HTML}{87CEEB}
\definecolor{slateblue}{HTML}{6A5ACD}
\definecolor{slategray}{HTML}{708090}
\definecolor{slategrey}{HTML}{708090}
\definecolor{snow}{HTML}{FFFAFA}
\definecolor{springgreen}{HTML}{00FF7F}
\definecolor{steelblue}{HTML}{4682B4}
\definecolor{tan}{HTML}{D2B48C}
\definecolor{teal}{HTML}{008080}
\definecolor{thistle}{HTML}{D8BFD8}
\definecolor{tomato}{HTML}{FF6347}
\definecolor{turquoise}{HTML}{40E0D0}
\definecolor{violet}{HTML}{EE82EE}
\definecolor{wheat}{HTML}{F5DEB3}
\definecolor{white}{HTML}{FFFFFF}
\definecolor{whitesmoke}{HTML}{F5F5F5}
\definecolor{yellow}{HTML}{FFFF00}
\definecolor{yellowgreen}{HTML}{9ACD32}
\usepackage[most]{tcolorbox}

\usepackage{ifthen}
\provideboolean{admonitiontwoside}
\makeatletter%
\if@twoside%
\setboolean{admonitiontwoside}{true}
\else%
\setboolean{admonitiontwoside}{false}
\fi%
\makeatother%

\newenvironment{env-d4f41a6d-5a9b-47fd-8da1-45ecdb177909}
{
    \savenotes\tcolorbox[blanker,breakable,left=5pt,borderline west={2pt}{-4pt}{firebrick}]
}
{
    \endtcolorbox\spewnotes
}
                

\newenvironment{env-ab94a4ec-2f9e-4d2f-94d1-74521d1a190e}
{
    \savenotes\tcolorbox[blanker,breakable,left=5pt,borderline west={2pt}{-4pt}{blue}]
}
{
    \endtcolorbox\spewnotes
}
                

\newenvironment{env-2aad614f-6fd6-4025-876c-fcdbeae766fb}
{
    \savenotes\tcolorbox[blanker,breakable,left=5pt,borderline west={2pt}{-4pt}{green}]
}
{
    \endtcolorbox\spewnotes
}
                

\newenvironment{env-da05fb3b-c564-4905-b729-fa3c9eadbf7c}
{
    \savenotes\tcolorbox[blanker,breakable,left=5pt,borderline west={2pt}{-4pt}{aquamarine}]
}
{
    \endtcolorbox\spewnotes
}
                

\newenvironment{env-1183acc6-7dc8-44d4-a985-6dec306e67fd}
{
    \savenotes\tcolorbox[blanker,breakable,left=5pt,borderline west={2pt}{-4pt}{orange}]
}
{
    \endtcolorbox\spewnotes
}
                

\newenvironment{env-ab392b14-a05f-4d9a-80eb-a63d00eb59ea}
{
    \savenotes\tcolorbox[blanker,breakable,left=5pt,borderline west={2pt}{-4pt}{gold}]
}
{
    \endtcolorbox\spewnotes
}
                

\newenvironment{env-a25021bf-49ba-4f13-8675-0bd41c1478b1}
{
    \savenotes\tcolorbox[blanker,breakable,left=5pt,borderline west={2pt}{-4pt}{darkred}]
}
{
    \endtcolorbox\spewnotes
}
                

\newenvironment{env-9740cc87-aba0-4946-8031-026058f06ad8}
{
    \savenotes\tcolorbox[blanker,breakable,left=5pt,borderline west={2pt}{-4pt}{pink}]
}
{
    \endtcolorbox\spewnotes
}
                

\newenvironment{env-b23dd66a-8a6c-447d-a163-3614469799d6}
{
    \savenotes\tcolorbox[blanker,breakable,left=5pt,borderline west={2pt}{-4pt}{cyan}]
}
{
    \endtcolorbox\spewnotes
}
                

\newenvironment{env-4361be9f-9a08-465c-86d1-6b024e2d00e2}
{
    \savenotes\tcolorbox[blanker,breakable,left=5pt,borderline west={2pt}{-4pt}{cyan}]
}
{
    \endtcolorbox\spewnotes
}
                

\newenvironment{env-8fa37bb4-768d-4893-9e93-b3dc64db7fb6}
{
    \savenotes\tcolorbox[blanker,breakable,left=5pt,borderline west={2pt}{-4pt}{purple}]
}
{
    \endtcolorbox\spewnotes
}
                

\newenvironment{env-b131cf96-9ac6-47cd-9742-4e5ff0f9bf3e}
{
    \savenotes\tcolorbox[blanker,breakable,left=5pt,borderline west={2pt}{-4pt}{darksalmon}]
}
{
    \endtcolorbox\spewnotes
}
                

\newenvironment{env-fdd3b853-72b2-4779-a214-b5da0ccdb5e7}
{
    \savenotes\tcolorbox[blanker,breakable,left=5pt,borderline west={2pt}{-4pt}{gray}]
}
{
    \endtcolorbox\spewnotes
}
                
\ifLuaTeX
  \usepackage{selnolig}  % disable illegal ligatures
\fi
\IfFileExists{bookmark.sty}{\usepackage{bookmark}}{\usepackage{hyperref}}
\IfFileExists{xurl.sty}{\usepackage{xurl}}{} % add URL line breaks if available
\urlstyle{same} % disable monospaced font for URLs
\hypersetup{
  pdftitle={Leaseholds},
  hidelinks,
  pdfcreator={LaTeX via pandoc}}

\title{Leaseholds}
\author{}
\date{}

\begin{document}
\maketitle

{
\setcounter{tocdepth}{3}
\tableofcontents
}
\hypertarget{leasehold-property}{%
\section{Leasehold Property}\label{leasehold-property}}

\hypertarget{advantages-and-disadvantages}{%
\subsection{Advantages and
Disadvantages}\label{advantages-and-disadvantages}}

Advantages:

\begin{itemize}
\tightlist
\item
  Flexibility over length of term.
\item
  Landlord gets a steady income from the property
\item
  Landlord retains an interest, which is a saleable capital asset.
\item
  Parties able to easily enforce covenants; almost all covenants
  enforceable against successors in title.

  \begin{itemize}
  \tightlist
  \item
    So leaseholds common where a property forms part of a larger
    building.
  \item
    Service charge payable by tenants for maintenance of common areas.
  \end{itemize}
\end{itemize}

Disadvantages:

\begin{itemize}
\tightlist
\item
  Lease will eventually expire; tenant may not be able to stay
  indefinitely.
\item
  Lease is a wasting asset (if it has a capital value).
\item
  Burdens on tenants (covenants and restrictions).
\item
  Landlord can forfeit lease as a remedy for breach of obligations.
\item
  Landlords also take on burdens.
\end{itemize}

\hypertarget{types}{%
\subsection{Types}\label{types}}

\hypertarget{residential-short-term}{%
\subsubsection{Residential Short-term}\label{residential-short-term}}

Usually granted at full open market rent (``rack rent''). The lease has
no capital value in the hands of the tenant. Lease cannot be assigned.
Tenants have certain statutory rights and protections.

\hypertarget{residential-long-term}{%
\subsubsection{Residential Long-term}\label{residential-long-term}}

\begin{itemize}
\tightlist
\item
  Standard where properties form part of a larger building.
\item
  Leaseholds allow greater ability to enforce positive and negative
  covenants.
\item
  Lease granted for a long term in return for a lump sum payment and low
  ``ground rent''.
\item
  Leases have a capital value and there are few controls on the tenant's
  ability to dispose of the property by assignment.
\item
  Very similar conveyancing process to freeholds.
\item
  The developer will usually dispose of the freehold of the common parts
  to a management company.
\end{itemize}

\hypertarget{commercial-property-market}{%
\subsubsection{Commercial Property
Market}\label{commercial-property-market}}

Most commercial property is occupied under leases because of the
flexibility they offer, and the undesirability of tying up capital in a
freehold.

\begin{env-ab392b14-a05f-4d9a-80eb-a63d00eb59ea}

Institutional lease

Where the landlord passes the day-to-day operational cost for occupying
the property to the teants, leaving rent as "pure profit".

\end{env-ab392b14-a05f-4d9a-80eb-a63d00eb59ea}

\begin{itemize}
\tightlist
\item
  Leases are usually 5-15 years long.
\item
  Possible for a capital sum to be payable, but this is rare. A market
  rent is payable and this is subject to rent review every few years.
\item
  Leases of commercial property usually contain detailed controls on the
  tenant's freedom to assign/ deal with the property.
\item
  Terms subject to detailed negotiation.
\item
  The Royal Institution of Chartered Surveyors (RICS) has produced a
  Code for Leasing Business Premises, which is a ``professional
  standard'' (surveyors must stick to it).
\item
  Less usual for parties to exchange contracts (unlikely to need to
  synchronise a chain of transactions).
\end{itemize}

\hypertarget{terminology}{%
\subsection{Terminology}\label{terminology}}

\begin{itemize}
\tightlist
\item
  A lease is ``granted'' when first created.
\item
  The first lease is the head-lease (the only superior title is the
  freehold reversion).
\item
  Sublease can be granted out of the head-lease.
\item
  The transfer of a leasehold is an ``assignment''
\item
  When granting a sublease, the period must be shorter than that of the
  head-lease (else it will be treated as an assignment)
\item
  There are commonly restrictions on assignment and subletting.
\end{itemize}

\hypertarget{licences}{%
\subsection{Licences}\label{licences}}

\begin{enumerate}
\tightlist
\item
  Licences are traditionally regarded as personal permissions, as
  opposed to being proprietary in nature.

  \begin{itemize}
  \tightlist
  \item
    Licences are regarded as capable of binding the licensor and
    licensee only.
  \item
    So licences are interests that operate over land rather than being
    interests in land.
  \item
    Licences are not proprietary rights capable of binding third parties
    (National Provincial Bank v Ainsworth (1965)). Licences can be
    revoked by the licensor.
  \item
    But this characterisation of licences as purely personal,
    non-proprietary interests has been subject to criticism and comment.
  \end{itemize}
\item
  Licences also operate to render lawful conduct which might otherwise
  constitute a trespass on someone else's land.
\item
  There are no general formalities for the creation of licences, though
  some types of licence do require a degree of compliance with
  formality.

  \begin{itemize}
  \tightlist
  \item
    Licences can be created orally or in writing, can be express or
    implied and can be contained within deeds or as part of the
    conveyancing process
  \item
    May also arise where the formalities for creating proprietary
    interests fail to be satisfied.
  \end{itemize}
\item
  The nature and extent of licences is determined by having regard to
  the specific terms of the licence.
\end{enumerate}

\hypertarget{characteristics}{%
\subsection{Characteristics}\label{characteristics}}

If it is found that the characteristics of a lease are not present,
looking at the transaction objectively, then the arrangement can only be
a licence.

\begin{env-b23dd66a-8a6c-447d-a163-3614469799d6}

Requirements

For a lease to exist, rather than a licence, there needs to be:

\begin{enumerate}
\tightlist
\item
  Certainty of term
\item
  Exclusive possession
\item
  The correct formalities used to create the lease
\end{enumerate}

\end{env-b23dd66a-8a6c-447d-a163-3614469799d6}

\begin{itemize}
\tightlist
\item
  Certainty of term means that the tenancy must be granted for a certain
  duration. This means you need to know when the arrangement will end.

  \begin{itemize}
  \tightlist
  \item
    If certainty of term is not present in the arrangement, no lease
    will be found.
  \item
    In the case of Lace v Chantler {[}1944{]} KB 368 a lease `for the
    duration of the war' failed for lack of certainty.
  \item
    The usual way of satisfying this requirement is a fixed term lease,
    e.g., five-year term.
  \end{itemize}
\item
  Exclusive possession means the right to exclude all others from the
  property, including the landlord.
\item
  Formalities are the rules that must be followed to formalise the
  arrangement, i.e., the steps the parties need to take.

  \begin{itemize}
  \tightlist
  \item
    Given the proprietary status of a lease, there is a high degree of
    formality that the parties must adhere to in order to create the
    legal estate.
  \end{itemize}
\end{itemize}

\hypertarget{certainty-of-term}{%
\subsubsection{Certainty of Term}\label{certainty-of-term}}

This can be shown either with a fixed or periodic term.

\begin{itemize}
\tightlist
\item
  A \textbf{fixed term} exists where the maximum duration of the
  arrangement is known from the outset.

  \begin{itemize}
  \tightlist
  \item
    Once a fixed term lease is created, neither party can unilaterally
    bring the lease to an end earlier, unless there is a break clause
  \end{itemize}
\item
  A \textbf{periodic tenancy} is technically a lease for one period.
  This is generally weekly/monthly/quarterly/yearly.

  \begin{itemize}
  \tightlist
  \item
    The tenancy extends itself automatically unless either the landlord
    or tenant give notice to terminate the tenancy.
  \item
    An \textbf{express} periodic tenancy is where there is a written
    agreement documenting the agreement.
  \item
    An \textbf{implied} periodic tenancy is where there is nothing set
    out in writing, but the certain term arises by looking objectively
    at all relevant circumstances, including payment and acceptance of
    rent on a periodic basis.
  \end{itemize}
\end{itemize}

\begin{env-8fa37bb4-768d-4893-9e93-b3dc64db7fb6}

Example

Monthly payments of rent (without agreement on a fixed term from the
outset) may create an implied monthly periodic tenancy.

\end{env-8fa37bb4-768d-4893-9e93-b3dc64db7fb6}

See also Prudential Assurance Co Ltd v London Residuary Body {[}1992{]}
2 A.C. 386.

The 'term' of the periodic tenancy depends upon the period by reference
to which the rent is calculated, rather than the intervals at which it
is payable.

e.g., if the tenant agrees to pay £10,000 a year by four quarterly
payments, the tenancy is a yearly tenancy not a quarterly tenancy,
because the rent is calculated annually.

\hypertarget{exclusive-possession}{%
\subsubsection{Exclusive Possession}\label{exclusive-possession}}

\begin{env-ab392b14-a05f-4d9a-80eb-a63d00eb59ea}

Definition

Exclusive possession means the right to exclude all others from the
property, including the landlord.

\end{env-ab392b14-a05f-4d9a-80eb-a63d00eb59ea}

Demonstrating that exclusive possession has been granted is often
difficult. Historically, tenants of residential leases were granted
greater protection and landlords were keen to avoid granting leases.
There are many cases where the court considered the nature of an
arrangement, looking beyond the label given to the agreement by the
parties (headline case: Street v Mountford {[}1985{]} AC 809).

\begin{env-fdd3b853-72b2-4779-a214-b5da0ccdb5e7}

Info

Since the introduction of Assured Shorthold Tenancies (ASTs), the
distinction between a lease and licence is of far less importance in a
residential context.

\end{env-fdd3b853-72b2-4779-a214-b5da0ccdb5e7}

Whether exclusive possession exists is a question of fact in each case:
the substance of the agreement has to be examined. The courts will look
at the reality of a situation. So even if a clause appears to defeat
exclusion possession but has been inserted into a lease only to make
what would otherwise be a lease appear like a licence, it will be thrown
out as a sham.

N.B. do not confuse exclusive possession with exclusive occupation.

\hypertarget{retention-of-a-key}{%
\paragraph{Retention of a Key}\label{retention-of-a-key}}

The fact that a landlord retains a key to the premises may make it
appear as if the occupant does not have exclusive possession. However,
it is the \textbf{purpose for which the key is retained} that matters.

For example, if the key is used only in an emergency or by arrangement,
then exclusive possession may still exist.

The courts will look at whether any right of access the landlord has is
\textbf{restricted} or \textbf{unrestricted}.

If the access is restricted e.g., 'to carry out repairs' then this is
seen as more of an acknowledgement of exclusive possession by the
landlord, rather than something that will defeat it. This was expressed
by the court in Street v Mountford {[}1985{]} AC 809

In Aslan v Murphy {[}1990{]} 1 WLR 766 the court held ``there is no
magic in the retention of a key'' -- it will not determine the nature of
arrangement either way.

\hypertarget{landlord-provides-services}{%
\paragraph{Landlord Provides
Services}\label{landlord-provides-services}}

If the landlord provides attendance or services there is a licence not a
tenancy (Marchant v Charters {[}1977{]} 1 WLR 1181.)

Services would include cleaning, changing linen etc.

The occupier is simply a \textbf{lodger} provided the services are
actually carried out, and a \textbf{lodger will never enjoy exclusive
possession} of the premises.

\hypertarget{sharing-clauses}{%
\paragraph{Sharing Clauses}\label{sharing-clauses}}

If a landlord reserves the right to share the property with the
occupiers or reserves the right to introduce others to share, that may
mean that there is no exclusive possession, as the occupier
\textbf{cannot exclude} whoever the landlord is able to introduce.

All of the circumstances must be looked at to see whether this is a
\textbf{genuine clause} or simply a \textbf{sham} to defeat exclusive
possession, as was made clear in A G Securities v Vaughan {[}1990{]} 1
AC 417 and Antoniades v Villiers {[}1990{]} 1 AC 417.

The following circumstances should therefore be considered when
determining if a sharing clause is genuine or a sham:

\begin{itemize}
\tightlist
\item
  The size and nature of the accommodation -- would it be realistic to
  introduce others into the accommodation given its size?
\item
  The relationship between the occupiers (if there is more than one) --
  would it be appropriate to introduce another to share given the
  relationship between the occupiers?
\item
  The wording of the clause (i.e., how widely it is drafted, as the
  wider it is drafted, the more likely it is a sham clause.)
\item
  Whether the clause has ever been exercised -- if it has not been
  exercised then this may indicate it is a sham clause.
\end{itemize}

\hypertarget{quiet-enjoyment}{%
\paragraph{Quiet Enjoyment}\label{quiet-enjoyment}}

If there is a clause for a tenant's 'quiet enjoyment of the land', this
points towards a lease (Addiscombe Garden Estates v Crabbe {[}1958{]} 1
QB 513).

\hypertarget{business-tenancies}{%
\subsubsection{Business Tenancies}\label{business-tenancies}}

Street v Mountford {[}1985{]} AC 809 is equally applicable to
non-residential tenancies.

A business tenant must also show it has a certain term and exclusive
possession of the premises in order to establish it is a tenant, rather
than licensee.

In the context of business arrangements, the result affects security of
tenure, as business tenants (but not licensees) are protected by the
Landlord and Tenant Act 1954, entitling them to remain in the premises
at the end of the lease term and request a new lease.

In the business context, the court tends to construe the document as a
whole to see if the landlord retains control over the property. In this
setting, the courts are more prepared to accept the reality of the label
'licence' than they are in the residential context as there tends to be
more equality in bargaining power, with commercial leases often
negotiated and parties legally represented.

\hypertarget{control}{%
\paragraph{Control}\label{control}}

The issue of control in a business context came before the court in Esso
Petroleum Co Ltd v Fumegrange Ltd and others {[}1994{]} 2 EGLR 90.

\begin{env-ab94a4ec-2f9e-4d2f-94d1-74521d1a190e}

\emph{Esso Petroleum Co Ltd v Fumegrange Ltd and others {[}1994{]} 2
EGLR 90}

The court considered whether exclusive possession of two service
stations was granted to Fumegrange.

The court held that the degree of control exercised by Esso over the
premises and the way in which it was conducted was inconsistent with an
exclusive right to possession of the service stations being vested in
Fumegrange. Esso could make alterations on the premises; it could
install a car wash (as was in fact done); and it could change the layout
of the shop. This degree of physical control over the premises and
conduct of the business at the service station was held to be very
significant.

\end{env-ab94a4ec-2f9e-4d2f-94d1-74521d1a190e}

\hypertarget{right-to-relocate}{%
\paragraph{Right to Relocate}\label{right-to-relocate}}

If the occupation agreement contains a right for the landlord to
relocate and move the tenant to alternative premises, it will not be a
lease.

\begin{quote}
\ldots you cannot have a tenancy granting exclusive possession of
particular premises, subject to a provision that the landlord can
require the tenant to move to somewhere else.\\
Dresden Estates v Collinson {[}1988{]} 55 P \& CR 47
\end{quote}

Of course, if such a clause is a sham clause, it will not defeat
exclusive possession in the circumstances and will be disregarded by the
court. The courts will always look at the substance of an arrangement.

\hypertarget{commencement-date}{%
\subsubsection{Commencement Date}\label{commencement-date}}

\begin{env-ab392b14-a05f-4d9a-80eb-a63d00eb59ea}

Commencement date

The date when the term of the lease begins.

\end{env-ab392b14-a05f-4d9a-80eb-a63d00eb59ea}

A lease does not have to take effect in possession immediately and can
be granted to take effect at some future time, provided this is within
21 years of its grant (s 149(3) LPA 1925). The commencement date can be
backdated to a date before the date of grant of the lease (useful if a
landlord wants all the leases to expire at the same time).

\hypertarget{formalities}{%
\subsection{Formalities}\label{formalities}}

The formalities required to create a legal lease will depend on the
length of the term of the lease.

\begin{longtable}[]{@{}ll@{}}
\toprule()
Rules & Explanation \\
\midrule()
\endhead
General rule & To create a legal lease, a deed must be used
(\href{https://www.legislation.gov.uk/ukpga/Geo5/15-16/20/section/52}{LPA
1925, s 52}). The requirements of a valid deed are set out in
\href{https://www.legislation.gov.uk/ukpga/1989/34/section/1}{LP(MP)A
1989, s 1}. \\
Leases over 7 years & If the term of the lease is over 7 years, the
lease must also be with registered
(\href{https://www.legislation.gov.uk/ukpga/2002/9/section/27}{LRA 2002,
s 27(2)(b)(i)}). This is a compulsory registration requirement. If not
done a legal leasehold estate will not have been created (LRA 2002, s 27
(1)). \\
Leases of 7 years or less & If the term of the lease is 7 years or less,
the lease does not need to be registered. Such leases still take effect
as legal leases and will be binding on a new freehold estate owner as an
overriding interest
(\href{https://www.legislation.gov.uk/ukpga/2002/9/schedule/3}{LRA 2002,
sch 3 para 1}). \\
\bottomrule()
\end{longtable}

\hypertarget{short-lease-exception}{%
\subsubsection{Short Lease Exception}\label{short-lease-exception}}

Certain short leases, which fulfil certain conditions, have no formal
requirements, yet they will still exist as legal leases. They need not
even be in writing.

\href{https://www.legislation.gov.uk/ukpga/Geo5/15-16/20/section/54}{LPA
1925, s 54 (2)} states that a lease with a term of three years or less
need not be created by deed provided the following three conditions are
all met:

\begin{enumerate}
\tightlist
\item
  The lease takes effect in possession (i.e., the tenant takes the lease
  immediately).
\item
  The lease is granted at 'best rent' (which has been interpreted as
  meaning `market rent').
\item
  The lease is not subject to a fine or premium (meaning there is no
  upfront payment for the grant of the lease, which you could commonly
  expect to see with very long leases).
\end{enumerate}

These short leases, also called \textbf{parol leases}, whether created
by deed or less formally under s 54(2), do not need to be registered to
exist as legal leases because only leases of over 7 years must be
registered. Leases less than 3 years can even be orally agreed, they
don't even need to be in writing.

The types of arrangement which fall within the ambit of this exception
are:

\begin{itemize}
\tightlist
\item
  \textbf{Short fixed term leases} (those with a maximum term of three
  years or less.)
\item
  \textbf{Express periodic tenancies} (where there is a tenancy
  agreement.)
\item
  \textbf{Implied periodic tenancies} (where an occupier is in
  possession and paying a rent at regular intervals.)
\end{itemize}

Note that periodic tenancies, whether express or implied, will only fall
within the ambit of
\href{https://www.legislation.gov.uk/ukpga/Geo5/15-16/20/section/54}{LPA
1925, s 54(2)} if each individual period of the tenancy is for
\textbf{three years or less}, which is likely to be the case.

\begin{env-8fa37bb4-768d-4893-9e93-b3dc64db7fb6}

Example

\begin{itemize}
\tightlist
\item
  \textbf{Short fixed term leases:} A tenant rents a flat for an agreed
  fixed term of 2 years. This arrangement does not need to be in
  writing.
\item
  \textbf{Express periodic tenancies:} A tenant rents a flat on a
  rolling monthly basis paying an agreed monthly rent. There is a
  written agreement documenting the agreement. In this situation, the
  written agreement does not need to comply with any formalities.
\item
  \textbf{Implied periodic tenancies:} A tenant is in occupation of
  premises paying an agreed rent on a monthly basis, but nothing
  documents this arrangement. In this situation, a periodic tenancy may
  be implied. It will still be a legal lease because no formalities are
  required to document the arrangement.
\end{itemize}

\end{env-8fa37bb4-768d-4893-9e93-b3dc64db7fb6}

\hypertarget{equitable-leases}{%
\subsubsection{Equitable Leases}\label{equitable-leases}}

An equitable lease may be created deliberately, if the parties choose to
enter into a contract for lease, or it may be that the parties try to
create a legal lease but fail, by either not creating a valid deed or
not registering the lease, if required.

The courts will recognise the tenant has having an equitable interest in
the land (an equitable lease) on the same terms as the defective legal
lease providing:

\begin{enumerate}
\tightlist
\item
  There is a document that complies with LP(MP)A 1989, s 2 -- signed
  contract in writing and incorporating all express terms.
\item
  The remedy of specific performance is available.
\end{enumerate}

\begin{env-4361be9f-9a08-465c-86d1-6b024e2d00e2}

Important

Where there is a conflict between the common law and equity, equity will
prevail (Walsh v Lonsdale (1882))

\end{env-4361be9f-9a08-465c-86d1-6b024e2d00e2}

Equitable leases usually require some form of registration to make them
binding on a purchaser of the reversion. If the reversionary title is
registered, the lease must be registered as a minor interest (i.e.,
notice) on the charges register of the reversionary title. But can be
binding without registration if there is an overriding interest (Sch 3
LRA 2002).

\hypertarget{summary}{%
\subsubsection{Summary}\label{summary}}

\begin{longtable}[]{@{}ll@{}}
\toprule()
{\(L\)}, length of lease (years) & Formalities \\
\midrule()
\endhead
{\(L > 7\)} & Deed + registration \\
{\(3 < L \leq 7\)} & Deed (lease binding as overriding interest) \\
{\(L \leq 3\)} & No formalities required, provided lease takes effect in
possession, is granted at market rate and no premium is payable. Lease
will be binding as an overriding interest. \\
Parties entering contract for lease or try to grant a legal lease which
fails to comply with formality requirements. & Tenant will have an
equitable lease (estate contract) if the agreement is in writing,
contains all terms and signed by both parties. \\
\bottomrule()
\end{longtable}

\hypertarget{examples}{%
\subsubsection{Examples}\label{examples}}

\begin{longtable}[]{@{}ll@{}}
\toprule()
Scenario & Lease type \\
\midrule()
\endhead
A landlord and tenant orally agree that the tenant will take a monthly
tenancy of a storage unit, paying a monthly market rent. The tenant
takes occupation of the storage unit. & Express monthly periodic tenancy
(legal lease) \\
A landlord and tenant enter into a contract to grant a lease for 10
years. The contract is in writing, contains the agreed terms of the
lease and is signed by both the landlord and tenant. & Fixed term
equitable lease \\
A landlord and tenant enter into a 8 year lease by a document labelled
as a deed. Only the landlord executes the deed, but does not get their
signature witnessed. The tenant does not try to register the deed, but
moves in and starts paying the annual rent due. & Implied annual
periodic tenancy (legal lease) \\
A landlord and tenant enter into a 5 year written agreement, which is
labelled as a deed and executed by the landlord in the presence of a
witness. The document is dated. The tenant does not register the deed. &
Fixed term legal lease \\
A landlord and tenant entered into a 10 year lease by valid deed. The
lease is validly executed by the landlord, but the tenant does not sign.
The tenant does not register the deed. & No valid lease has been
created \\
\bottomrule()
\end{longtable}

\hypertarget{liability-on-covenants-in-leases}{%
\subsection{Liability on Covenants in
Leases}\label{liability-on-covenants-in-leases}}

\hypertarget{leases-granted-on-or-after-010196}{%
\subsubsection{Leases Granted on or After
01/01/96}\label{leases-granted-on-or-after-010196}}

LTCA 1995 applies to leases granted on or after 01/01/96. It is the date
of grant of the lease, not of subsequent assignment, which determines
this.

When the tenant lawfully assigns the lease, he is automatically released
from future liability under the lease covenants, unless he has agreed to
enter into an authorised guarantee agreement (AGA). Under such an
agreement, the outgoing tenant guarantees the performance of the lease
covenants by his immediate successor in title.

The tenant is only obliged to provide an AGA if:

\begin{enumerate}
\tightlist
\item
  In leases of commercial property, where the parties have agreed in the
  lease that an AGA is to be provided on assignment.
\item
  If the landlord requires the provision of an AGA as a condition of
  giving consent to an assignment.
\end{enumerate}

\begin{itemize}
\tightlist
\item
  The guarantee in the AGA extends only the the performance of covenants
  by the immediate successor in title only.
\item
  An assignee of the lease gets the benefit of landlord covenants when
  he takes over an assignment of the lease (s 3(2)(b) LTCA 1995).
\item
  He is bound by all the tenant's covenants except those expressly
  stated to be personal to the original tenant (s 3(2)(a) LTCA 1995).
\item
  The assignee remains liable for any breaches committed before the date
  of assignment but is released from liability for any future breaches,
  unless also required to give an AGA to guarantee the performance of
  lease covenants.
\item
  The automatic release provisions do not apply if the assignment is in
  breach of a covenant in the lease/ if it occurs by operation of law.
\item
  Where a former tenant remains liable under an AGA, the former tenant
  should seek an indemnity covenant from successor in title

  \begin{itemize}
  \tightlist
  \item
    In the absence of such an indemnity, an indemnity under common law
    could be claimed under the rule in Moule v Garrett (1872) LR 7 Exch
    101.
  \end{itemize}
\item
  Any subsequent assignee of the landlord's interest gets the benefit of
  the tenant's covenants when taking an assignment of the lease (s
  3(3)(b) LTCA 1995).
\item
  Each landlord is \textbf{not} automatically released from covenants
  when selling the reversion, but can apply to the tenant to be released
  from future liability before/ within 4 weeks of the date of assignment
  of the reversion.
\end{itemize}

\hypertarget{leases-granted-before-010196}{%
\subsubsection{Leases Granted Before
01/01/96}\label{leases-granted-before-010196}}

\hypertarget{tenant-liability}{%
\paragraph{Tenant Liability}\label{tenant-liability}}

\begin{itemize}
\tightlist
\item
  The original tenant \textbf{remains bound by lease covenants under
  privity of contract}.

  \begin{itemize}
  \tightlist
  \item
    Continues even after assignment unless landlord expressly releases
    the tenant.
  \end{itemize}
\item
  Any assignee also liable for breaches committed whilst ownership of
  the land is vested in that assignee in respect of covenants which
  "touch and concern" the land (those entered into parties as
  landowners, rather than as individuals) {\(\rightarrow\)} privity of
  estate.
\item
  Original landlord has the benefit of tenant covenants by privity of
  contract.
\item
  Landlord's successor in title takes the benefit of tenant covenants
  under s 141(1) LPA 1925.
\end{itemize}

So the landlord could seek redress against either the original tenant or
any assignee who committed the breach. Where the original landlord
transfers the reversion, the right to sue passes to the transferee of
that reversion (all rights of action pass to the transferee -- s 141).

So, an assignee of a lease will usually be required to indemnify the
assignor in respect of any breach of covenant committed after the date
of assignment. Registered lease; such an indemnity covenant is implied
by Sch 12 para 20 LRA 2002, whether or not the assignment is for value.

\begin{env-2aad614f-6fd6-4025-876c-fcdbeae766fb}

LRA 2002 Sch 12 para 20 - Implied indemnity covenants on transfers of
pre-1996 leases

(1) On a disposition of a registered leasehold estate by way of
transfer, the following covenants are implied in the instrument
effecting the disposition, unless the contrary intention is expressed---

\begin{itemize}
\tightlist
\item
  (a) in the case of a transfer of the whole of the land comprised in
  the registered lease, the covenant in sub-paragraph (2), and
\item
  (b) in the case of a transfer of part of the land comprised in the
  lease---

  \begin{itemize}
  \tightlist
  \item
    (i) the covenant in sub-paragraph (3), and
  \item
    (ii) where the transferor continues to hold land under the lease,
    the covenant in sub-paragraph (4).
  \end{itemize}
\end{itemize}

(2) The transferee covenants with the transferor that during the residue
of the term granted by the registered lease the transferee and the
persons deriving title under him will---

\begin{itemize}
\tightlist
\item
  (a) pay the rent reserved by the lease,
\item
  (b) comply with the covenants and conditions contained in the lease,
  and
\item
  (c) keep the transferor and the persons deriving title under him
  indemnified against all actions, expenses and claims on account of any
  failure to comply with paragraphs (a) and (b).
\end{itemize}

(3) The transferee covenants with the transferor that during the residue
of the term granted by the registered lease the transferee and the
persons deriving title under him will---

\begin{itemize}
\tightlist
\item
  (a) where the rent reserved by the lease is apportioned, pay the rent
  apportioned to the part transferred,
\item
  (b) comply with the covenants and conditions contained in the lease so
  far as affecting the part transferred, and
\item
  (c) keep the transferor and the persons deriving title under him
  indemnified against all actions, expenses and claims on account of any
  failure to comply with paragraphs (a) and (b).
\end{itemize}

(4) The transferor covenants with the transferee that during the residue
of the term granted by the registered lease the transferor and the
persons deriving title under him will---

\begin{itemize}
\tightlist
\item
  (a) where the rent reserved by the lease is apportioned, pay the rent
  apportioned to the part retained,
\item
  (b) comply with the covenants and conditions contained in the lease so
  far as affecting the part retained, and
\item
  (c) keep the transferee and the persons deriving title under him
  indemnified against all actions, expenses and claims on account of any
  failure to comply with paragraphs (a) and (b).
\end{itemize}

(5) This paragraph does not apply to a lease which is a new tenancy for
the purposes of section 1 of the Landlord and Tenant (Covenants) Act
1995 (c. 30).

\end{env-2aad614f-6fd6-4025-876c-fcdbeae766fb}

SCPC 7.6.5 requires the transfer deed to contain an express indemnity
covenant, except where one is implied by law.

\begin{env-4361be9f-9a08-465c-86d1-6b024e2d00e2}

Important

If the lease contains a requirement to obtain the landlord's consent
before any assignment takes place, the landlord may, in appropriate
circumstances, insist that any assignee enter into a direct contractual
relationship with him to observe the tenant's covenants under the lease
for the remainder of the term. This imposes on any assignee the same
liability under privity of contract as faced by the original tenant.

\end{env-4361be9f-9a08-465c-86d1-6b024e2d00e2}

\hypertarget{landlord-liability}{%
\paragraph{Landlord Liability}\label{landlord-liability}}

If an original landlord is unable through his own act or default (e.g.,
by transferring the reversion to a third party) to carry out an
obligation imposed on him by the lease, the landlord may be liable in
damages to the tenant (Eagon v Dent {[}1965{]} 2 All ER 335).

Any transferee of the reversion for the time being will also be liable
in respect of breaches of covenant which touch and concern the land,
which are committed whilst the reversion is vested in them (privity of
estate).

\hypertarget{ltca-1995-retrospective-provisions}{%
\subsubsection{LT(C)A 1995 Retrospective
Provisions}\label{ltca-1995-retrospective-provisions}}

\includegraphics{C:/Users/shiva/Filen/MEGA/LegalPracticeCourse/LT(C)A retrospective_1.jpg}

There are some sections of the Landlord and Tenant (Covenants) Act 1995
which apply to all leases, regardless of when they were created. These
are:

Landlord and Tenant (Covenants) Act 1995

\begin{itemize}
\tightlist
\item
  s 17, Tenant Default Notice
\item
  s 18, Liability for Variations
\item
  s 19, Overriding Leases
\end{itemize}

These provisions apply to both old and new leases.

These provisions apply to former tenants who remain liable either under
an old lease (existing tenant liability) or because of their liability
under a new lease under an authorised guarantee agreement.

\hypertarget{tenant-default-notice}{%
\paragraph{Tenant Default Notice}\label{tenant-default-notice}}

The
\href{https://www.legislation.gov.uk/ukpga/1995/30/section/17}{Landlord
and Tenant (Covenants) Act 1995, s 17} states that where a landlord
wishes to pursue a former tenant who remains liable under the terms of
the lease for a fixed charge, the landlord must serve notice of the
potential claim on such tenants within \textbf{6 months} of the charge
becoming due.

\begin{env-b23dd66a-8a6c-447d-a163-3614469799d6}

Tip

So practically, cannot recover more than 6 months rent at a time in this
way.

\end{env-b23dd66a-8a6c-447d-a163-3614469799d6}

A fixed charge is defined in the
\href{https://www.legislation.gov.uk/ukpga/1995/30/section/17}{Landlord
and Tenant (Covenants) Act 1995, s 17} as including arrears of rent,
service charge or insurance premiums but not unascertained liabilities
or damages which become ascertained only after a court order has been
obtained.

If the landlord has not notified the former tenant of this claim and the
amount due within this period, the landlord will be precluded from
making a claim against him.

\hypertarget{liability-for-variations}{%
\paragraph{Liability for Variations}\label{liability-for-variations}}

The
\href{https://www.legislation.gov.uk/ukpga/1995/30/section/18}{Landlord
and Tenant (Covenants) Act 1995, s 18} states that former tenants and
guarantors are not liable to pay any additional amounts owing in respect
of variations which have been made to the lease subsequent to assignment
which they could not have anticipated at the time when the lease was
entered into.

\begin{env-8fa37bb4-768d-4893-9e93-b3dc64db7fb6}

Example

The former tenant who is still liable for the rent under an old lease or
under an authorised guarantee agreement will still be liable for rent
which is reviewed after assignment pursuant to the rent review clause in
the original lease. Such a variation was anticipated at the time when
the lease was entered into.

\end{env-8fa37bb4-768d-4893-9e93-b3dc64db7fb6}

\hypertarget{overriding-leases}{%
\paragraph{Overriding Leases}\label{overriding-leases}}

The
\href{https://www.legislation.gov.uk/ukpga/1995/30/section/19}{Landlord
and Tenant (Covenants) Act 1995, s 19} provides for the creation of an
overriding lease in certain circumstances.

A former tenant who has been served with a s 17 notice and has made
payment in full is entitled to call on the landlord to grant it an
overriding lease (LTCA 1995, s 19(1)). If granted, the former tenant
becomes the immediate landlord of the defaulting party.

\begin{env-ab392b14-a05f-4d9a-80eb-a63d00eb59ea}

Overriding lease

An overriding lease is a concurrent lease of the premises demised by the
first lease and makes the former tenant the immediate landlord of the
defaulting tenant. It therefore gives the former tenant the right to
forfeit the first lease to the defaulting tenant and/or exercise the
landlord's other remedies for non-payment of rent.

\end{env-ab392b14-a05f-4d9a-80eb-a63d00eb59ea}

The overriding lease is granted for a term equal to the remaining term
of the lease in question plus three days. It contains the same covenants
as the lease in question (other than those covenants which are expressed
to be personal). If the former tenant requests an overriding lease, the
landlord is obliged to grant it within a reasonable time.

The former tenant is then better able to ensure the assignee's
compliance under the terms of the lease, or terminate the assignee's
lease and re-let the property, or assign the overriding lease to a more
reliable tenant.

Summary

\begin{itemize}
\tightlist
\item
  The \textbf{LT(C)A 1995, ss 17 to 19} apply to all leases regardless
  of when they were created.
\item
  Under the \textbf{LT(C)A 1995, s 17}, if a landlord wishes to pursue a
  former tenant for a fixed charge, the landlord must serve notice of
  the potential claim on such a tenant within six months of the charge
  becoming due.
\item
  Under the \textbf{LT(C)A 1995, s 18,} former tenants are not liable in
  respect of variations which have been made to the lease which they
  could not have anticipated at the time when the lease was entered
  into.
\item
  Under the \textbf{LT(C)A 1995, s 19}, if a former tenant pays rent or
  other fixed charges pursuant to \textbf{s 17}, that former tenant may
  request from the landlord an overriding lease.
\item
  When he is granted an overriding lease, the former tenant is better
  able to ensure the assignee's compliance under the terms of the lease,
  or terminate the assignee's lease.
\end{itemize}

\hypertarget{liability-between-head-landlord-and-sub-tenant}{%
\subsubsection{Liability Between Head-landlord and
Sub-tenant}\label{liability-between-head-landlord-and-sub-tenant}}

No privity of estate exists between a head-landlord and a sub-tenant,
although a contractual relationship will exist between them if the
sub-tenant has entered into direct covenants with the head-landlord.

Irrespective of direct contractual liability, if the sub-tenant breaches
a covenant in the head-lease, the head-landlord will have the right to
forfeit the head-lease, and this will mean that the sub-lease which is
derived out of the head-lease will also terminate.

\hypertarget{determination-of-leases}{%
\subsection{Determination of Leases}\label{determination-of-leases}}

A lease is created when one person with an estate in land (the Landlord)
grants the temporary right to another person (the Tenant) to use and
enjoy that land exclusively.

The different ways a lease can end:

\begin{itemize}
\tightlist
\item
  Effluxion of time
\item
  Notice to quit
\item
  Merger
\item
  Forfeiture
\item
  Break Clause
\item
  Surrender
\end{itemize}

\hypertarget{by-effluxion-of-time}{%
\subsubsection{By Effluxion of Time}\label{by-effluxion-of-time}}

This is the usual common law way in which a lease for a fixed term comes
to an end. No notice is needed to end a fixed term. Note, however, that
at the end of the lease, the tenant may have security of tenure.
Security of tenure is a right for a tenant to remain in the premises at
the end of the lease term and to request the grant of a new lease.

\begin{env-8fa37bb4-768d-4893-9e93-b3dc64db7fb6}

Example

Many \textbf{business tenancies} are protected by the Landlord and
Tenant Act 1954, Part II. If the tenant follows certain procedures and
time limits, he may be entitled to require the landlord to grant him a
new lease.

A \textbf{residential tenant} may have rights to remain in occupation
under the Rent Act 1977 (for private tenancies granted before 15 January
1989) or the Housing Acts 1985 (public sector tenancies) and 1988
(private tenancies granted after 15 January 1989). Since the
introduction of assured shorthold tenancies (ASTs) it is now very rare
for a residential tenant to have security of tenure.

\end{env-8fa37bb4-768d-4893-9e93-b3dc64db7fb6}

\hypertarget{by-a-break-clause}{%
\subsubsection{By a `break' Clause}\label{by-a-break-clause}}

A lease for a fixed term may contain a provision allowing either party
(landlord or tenant or both) to serve notice during its currency to
bring it to a premature end. The terms of a break clause must be
strictly complied with if the exercise of it is to be valid.

\begin{env-8fa37bb4-768d-4893-9e93-b3dc64db7fb6}

Example

``Either party may end this lease on the fifth anniversary of the term
start date by serving 6 months written notice on the other.''

\end{env-8fa37bb4-768d-4893-9e93-b3dc64db7fb6}

\hypertarget{by-surrender}{%
\subsubsection{By Surrender}\label{by-surrender}}

A surrender is the handing back of the lease by the tenant to the
(immediate) landlord with the landlord's consent. This results in
premature termination of the lease. The lease is said to merge into the
landlord's reversion and is extinguished. To be legal, surrender must be
by deed (s 52 LPA 1925).

\hypertarget{merger}{%
\subsubsection{Merger}\label{merger}}

This is the converse of surrender; the tenant acquires the (immediate)
landlord's reversion interest, thus becoming his own landlord, and the
lease is absorbed by the reversion and destroyed. Can also happen where
a third party acquires both the lease and the reversion.

A deed of surrender is often entered into in practice to document the
agreement. Sometimes a surrender premium will be paid by the tenant to
the landlord too.

\hypertarget{notice-to-quit}{%
\subsubsection{Notice to Quit}\label{notice-to-quit}}

This is the method for determining a periodic tenancy.

For a yearly tenancy, either side must give at least half a year's
notice to quit to determine the tenancy. Other periodic tenancies, for
example quarterly, monthly or weekly, must be determined by a full
period's notice to expire at the end of a complete period.

\begin{env-ab392b14-a05f-4d9a-80eb-a63d00eb59ea}

Definition

A periodic tenancy is a lease that runs from one period of time to
another. Each period is seen as an individual lease.

\end{env-ab392b14-a05f-4d9a-80eb-a63d00eb59ea}

\hypertarget{forfeiture}{%
\subsubsection{Forfeiture}\label{forfeiture}}

This is a right for the landlord to terminate the lease prematurely for
breach of covenant by the tenant. Also known as a `legal right of
re-entry' and a recognised legal interest in the land:
\href{https://www.legislation.gov.uk/ukpga/Geo5/15-16/20/section/1}{LPA
1925, s 1(2)(e)}.

\hypertarget{landlords-remedies-for-breach}{%
\subsection{Landlord's Remedies for
Breach}\label{landlords-remedies-for-breach}}

\hypertarget{non-payment-of-rent}{%
\subsubsection{Non-payment of Rent}\label{non-payment-of-rent}}

Landlord may consider the following remedies:

\begin{itemize}
\tightlist
\item
  Debt action

  \begin{itemize}
  \tightlist
  \item
    The tenant can be sued on his covenant to pay rent.
  \item
    A landlord can only recover six years' arrears:
    \href{https://www.legislation.gov.uk/ukpga/1980/58/section/19}{Limitation
    Act 1980, s 19}.
  \item
    However, if a tenant is unable to pay the rent, it is unlikely to be
    able to pay any damages awarded by the court.
  \end{itemize}
\item
  Forfeiture
\item
  Commercial rent arrears recovery procedure
\item
  Collecting rent from a subtenant.

  \begin{itemize}
  \tightlist
  \item
    Possible under s 81 Tribunals, Courts and Enforcement Act 2007.
  \item
    If CRAR is available to a landlord, landlord may serve notice on any
    sub-tenant setting out prescribed information including the amount
    due
  \item
    Once 14 clear days have expired, the right to collect rent under the
    sublease transfers to the landlord, until any amount owing by the
    immediate tenant is paid off.
  \end{itemize}
\item
  Bankruptcy and winding up

  \begin{itemize}
  \tightlist
  \item
    If debt exceeds £5,000 (individual) or £750 (company), statutory
    demand can be served.
  \end{itemize}
\item
  Pursuing former tenants (if they remain liable)
\item
  Pursuing guarantors of the tenant.
\end{itemize}

Commercial Rent (Coronavirus) Act 2022 binding arbitration process put
in place to resolve claims for rent arrears accrued during periods of
enforced closure (``protected rent debt'') supported by a voluntary
code.

\hypertarget{commercial-rent-arrears-recovery-procedure}{%
\paragraph{Commercial Rent Arrears Recovery
Procedure}\label{commercial-rent-arrears-recovery-procedure}}

Distress was an ancient common law self-help remedy, entitling a
landlord to enter the premises as soon as the rent was due and unpaid
and to take possession of goods to the value of the rent owed.

If the tenant did not pay within five days, the goods could be sold.

Since the Human Rights Act 1998, concerns had been raised in cases and
academic commentary as to whether the self-help remedy of distress was
in breach of:

\begin{itemize}
\tightlist
\item
  Article 6: right to a fair trial
\item
  Article 8: respect for private and family life and the home.
\item
  Article 1 (Protocol 1): peaceful enjoyment of possessions.
\end{itemize}

From 6 April 2014, a landlord's right to levy distress for rent arrears
was therefore abolished and replaced with a new regime known as
\textbf{Commercial Rent Arrears Recovery.}

This requires a landlord to serve an enforcement notice on the tenant,
giving \textbf{seven clear days' notice} that he will seize goods. Once
notice has been served, the tenant can apply for it to be set aside or
for its execution to be delayed.

\begin{env-1183acc6-7dc8-44d4-a985-6dec306e67fd}

Warning

There must be a minimum of seven days' rent arrears in order to use this
procedure, and the remedy is not available in respect of mixed use or
residential premises.

\end{env-1183acc6-7dc8-44d4-a985-6dec306e67fd}

The notice requirements clearly reduce the effectiveness of the remedy
from a landlord's perspective because tenants have the opportunity to
remove goods from the premises and put them out of the landlord's reach.

Restrictions on using CRAR procedure:

\begin{itemize}
\tightlist
\item
  Applies to leases of commercial premises only.

  \begin{itemize}
  \tightlist
  \item
    Cannot be used if any part of the premises is let or occupied as a
    welling (except where this occurs in breach of a term of the lease).
  \item
    Cannot be used where the lease is oral.
  \end{itemize}
\item
  Only available in relation to \textbf{pure rent arrears} (excluding
  service charge/ insurance rent) and at least 7 days' rent must be
  outstanding.
\item
  Landlord required to serve an enforcement notice on the defaulting
  tenant. On the expiry of that notice, only an enforcement agent will
  be able to enter the premises to remove goods. Part 2 Taking Control
  of Goods Regulations 2013: information the enforcement notice must
  contain and how it must be given.
\item
  s 78 TCEA 2007: tenant who receives an enforcement notice may apply to
  court for an order that the notice be set aside.
\item
  Certain good exempt from CRAR and cannot be seized. Include items up
  to the value of £1,350 which are necessary for the tenant's business.
\item
  Enforcement agent must give the tenant at least 7 days' clear notice
  of the sale of any seized goods.
\item
  Any provision in a lease purporting to allow the seizure of goods for
  non-payment of rent, outside the CRAR procedure, will be void.
\end{itemize}

\hypertarget{other-covenants}{%
\subsubsection{Other Covenants}\label{other-covenants}}

\begin{itemize}
\tightlist
\item
  Forfeiture
\item
  Injunction

  \begin{itemize}
  \tightlist
  \item
    For breach of negative covenant or to prevent anticipated breach of
    covenant.
  \end{itemize}
\item
  Specific performance

  \begin{itemize}
  \tightlist
  \item
    Where the obligation to be enforced is sufficiently precise
  \item
    Will not be ordered where performance or supervision is required
    over a period of time, or where damages are adequate.
  \end{itemize}
\item
  Damages

  \begin{itemize}
  \tightlist
  \item
    Recoverable under Hadley v Baxendale (1854) 9 Ex 341 principles.
  \end{itemize}
\item
  Pursue former tenants
\item
  Pursue guarantors.
\end{itemize}

\hypertarget{forfeiture-1}{%
\subsubsection{Forfeiture}\label{forfeiture-1}}

\includegraphics{C:/Users/shiva/Filen/MEGA/LegalPracticeCourse/forfeiture_1.jpg}

A lease is a proprietary right to possess the land. One way in which a
lease can end early is if a landlord exercises a forfeiture right. This
is a right for the landlord to terminate the lease prematurely for
breach of covenant by the tenant. This is a very powerful right.

\hypertarget{legal-leases}{%
\paragraph{Legal Leases}\label{legal-leases}}

There must be an express forfeiture clause in the lease allowing the
landlord to forfeit the lease in the event of breach of covenant (or
insolvency) by the tenant. Every well-drawn lease should contain a
forfeiture clause. The right to forfeit is never implied into a legal
lease.

\begin{env-ab94a4ec-2f9e-4d2f-94d1-74521d1a190e}

Note

In a legal lease, the forfeiture clause creates a legal right of
re-entry (one of the five legal rights listed in LPA 1925, s 1(2)). A
forfeiture clause is therefore also known as a re-entry clause.

\end{env-ab94a4ec-2f9e-4d2f-94d1-74521d1a190e}

\hypertarget{equitable-leases-1}{%
\paragraph{Equitable Leases}\label{equitable-leases-1}}

There does not need to be an express forfeiture clause, because a right
to forfeit for non-payment of rent is implied into equitable leases as
one of the implied usual covenants.

\hypertarget{exercising-the-right}{%
\paragraph{Exercising the Right}\label{exercising-the-right}}

Forfeiture is exercised by the landlord by either:

\begin{itemize}
\tightlist
\item
  peaceably re-entering the property; or
\item
  by obtaining a court order.
\end{itemize}

There are then different rules, depending on whether the landlord is
forfeiting for non-payment of rent or for breach of covenant other than
non-payment of rent. However, the methods of forfeiture described above
(court order or peaceable re-entry) apply to both types of breach.

In the case of pure business premises the landlord may forfeit by
peaceable re-entry i.e., by physically entering, changing the locks and
putting up an unequivocal notice that he is forfeiting the lease.

Where the premises are residential, the landlord cannot forfeit without
a court order: Protection from Eviction Act 1977, s 2. This includes
where the premises are mixed residential and commercial: Patel v
Pirabakaran {[}2006{]} 1 WLR 3112.

The ability to forfeit residential tenancies is restricted by Commonhold
and Leasehold Reform Act 2002. Whether the property is commercial or
residential, it is an offence to use or threaten violence (s 6 Criminal
Law Act 1977) to achieve re-entry.

\hypertarget{forfeiture-procedure}{%
\paragraph{Forfeiture Procedure}\label{forfeiture-procedure}}

\begin{enumerate}
\tightlist
\item
  Has there been a breach?
\item
  Is there a forfeiture clause?
\item
  Has there been a waiver?
\end{enumerate}

The rules relating to waiver apply to breach of rent covenants and
breach of other types of covenant.

\hypertarget{waiver}{%
\paragraph{3. Waiver}\label{waiver}}

If the landlord wishes to forfeit, he must not have waived his right to
forfeit. The landlord may waive the right to forfeit if:

\begin{itemize}
\tightlist
\item
  he is aware of the acts or omissions giving rise to the right to
  forfeit; and
\item
  he does some unequivocal act recognising the continued existence of
  the lease.

  \begin{itemize}
  \tightlist
  \item
    Example: Demanding, accepting or suing for rent due after the
    breach; or distraining for rent due, despite knowing about the
    breach.
  \end{itemize}
\end{itemize}

Waiver can take place \textbf{inadvertently}.

For example, the landlord's agent sends out a rent demand where the
landlord (though not the agent) is aware of the breach (Central Estates
(Belgravia) Ltd v Woolgar {[}1972{]} 1 WLR 1048).

Where there is a \textbf{continuing} breach (e.g., failure to repair),
waiver only lasts until the next day the breach continues, at which
point the landlord can then choose to reject the rent and forfeit the
lease.

If the breach is \textbf{non-continuing} ('once and for all'), such as
sub-letting without consent, waiver is permanent, i.e., once rent is
accepted the landlord can never again forfeit for that specific breach.

Waiver operates only on the landlord's right to forfeit. The landlord
retains other remedies, such as an action for debt or damages.

Note that waiver can operate in relation to a breach of covenant to pay
the rent; the landlord could for example waive the right to forfeit for
failure to pay the previous month's rent, by demanding the next month's
rent.

Non-payment of rent is classified as a non-continuing breach (London and
County (A\&D) Ltd v Wilfred Sportsman Ltd {[}1971{]} Ch 764).

This means that each individual non-payment of rent (i.e., each missed
payment) is deemed to be a separate breach, creating a separate right to
forfeit.

So, even if the landlord permanently waives his right to forfeit for one
incidence of non-payment of rent, this does not preclude the possibility
that the landlord could forfeit for another non-payment of rent in the
future (which is a likely possibility in such circumstances).

\hypertarget{breach-of-a-rent-covenant}{%
\subsection{Breach of a Rent Covenant}\label{breach-of-a-rent-covenant}}

A lease can end early if a landlord exercises a forfeiture right. This
is a right for the landlord to terminate the lease prematurely for
breach of covenant by the tenant. This is a very powerful right. There
are then different rules depending on whether the landlord is forfeiting
for non-payment of rent or for breach of covenant other than non-payment
of rent.

If other payments due to the landlord are expressed in the lease to be
paid as additional rent (e.g. service charges) the landlord can forfeit
for non-payment in the same way as for rent.

Suppose there has been a breach of rent covenant, there is a forfeiture
clause (either express or implied), and there is no relevant waiver.
Step 4 is making a formal demand for the rent.

\hypertarget{formal-demand}{%
\subsubsection{4. Formal Demand}\label{formal-demand}}

The Landlord must first make a formal demand for the exact amount of
rent due on the day when it becomes payable, upon the premises, between
the hours of sunrise and sunset unless:

\begin{itemize}
\tightlist
\item
  The lease expressly waives this requirement. It usually does so by
  including in the forfeiture clause words such as `whether formally
  demanded or not'. Or;
\item
  At least six months' rent is in arrears and there are insufficient
  distrainable (i.e., seizable) goods on the premises to satisfy all the
  arrears due (Common Law Procedure Act 1852, s 210).
\end{itemize}

\hypertarget{exercise-the-right}{%
\subsubsection{5: Exercise the Right}\label{exercise-the-right}}

Once the landlord has made a formal demand, or if a formal demand is not
necessary, the landlord can proceed to forfeit by court order or
peaceable re-entry as described above.

\hypertarget{relief}{%
\subsubsection{6: Relief}\label{relief}}

The tenant may then apply for relief from forfeiture. 'Relief' means the
court's discretion to allow the lease to continue and thereby end the
forfeiture process. This discretion to grant relief is an ancient
equitable jurisdiction, but is now enshrined in various statutes, as
explained below.

\hypertarget{before-the-court-order}{%
\paragraph{Before the Court Order}\label{before-the-court-order}}

If, on the landlord's suing for possession, the tenant pays into court
all arrears and costs before the trial, all further proceedings are
stayed.

The legal authority for this depends upon whether the proceedings are in
the County Court or High Court:

\begin{itemize}
\tightlist
\item
  County Courts Act 1984, s 138, in the county court;
\item
  Common Law Procedure Act 1852, s 212 in the High Court, provided in
  the High Court that at least six months' rent is in arrears (Standard
  Pattern Co Ltd v Ivey {[}1962{]} Ch 432).
\end{itemize}

\hypertarget{at-or-after-the-court-order}{%
\paragraph{At or After the Court
Order}\label{at-or-after-the-court-order}}

The court has a discretion to grant relief, i.e., to allow the lease to
continue on condition that the arrears are paid.

Relief is usually given unless the circumstances are exceptional, for
example:

\begin{itemize}
\tightlist
\item
  Where the property has already been lawfully let to a new tenant after
  the forfeiture (Stanhope v Haworth (1886) 3 TLR 34.)
\item
  Where the non-payment of rent is exceptionally bad. No relief was
  given in Public Trustee v Westbrook {[}1965{]} 1 WLR 1160 where no
  rent had been paid for 22 years, as the site was bombed out and
  everyone thought that the lease had ended.
\end{itemize}

When the landlord has re-entered under a court order, the application
for relief must be made within \textbf{six months} of re-entry (see
Common Law Procedure Act 1852, s 210 in the High Court. Similar but
slightly different provisions apply in the county court under the County
Courts Act 1984.)

After peaceable re-entry if the landlord forfeits a non-residential
lease without a court order (or if less than six months' rent is in
arrears), the court has an inherent equitable jurisdiction to grant
relief if:

\begin{itemize}
\tightlist
\item
  the rent and landlord's costs are paid; and
\item
  it is just and equitable to grant relief.
\end{itemize}

Thatcher v Pearce {[}1968{]} 1 WLR 748 -- A lease of a scrapyard was
determined by peaceable re-entry by the landlord while the tenant was in
prison. The tenant applied more than six months after forfeiture for
relief and was successful.

\hypertarget{sub-tenant}{%
\paragraph{Sub-tenant}\label{sub-tenant}}

If a head-lease is forfeited, any sub-lease will also be destroyed. A
sub-tenant in the premises has the right to apply for relief from
forfeiture.

The authority:

\begin{itemize}
\tightlist
\item
  Under
  \href{https://www.legislation.gov.uk/ukpga/1984/28/section/138}{CCA
  1984, s 138} (county court action);
\item
  \href{https://www.legislation.gov.uk/ukpga/1981/54/section/38}{SCA
  1981, s 38} (High Court action);
\item
  \href{https://www.legislation.gov.uk/ukpga/Geo5/15-16/20/section/146}{LPA
  1925, s 146(4)} (peaceable re-entry).
\end{itemize}

This is the case even where the head-tenant cannot get relief himself.
The court has the power to vest the head-lease in the sub-tenant on such
terms as it sees fit, including rent, but cannot grant the sub-tenant a
lease for any longer than the term of the original sub-lease.

In summary, under the above provisions, the court usually allows the
lease to continue on condition that the tenant pays off the arrears.

\begin{env-fdd3b853-72b2-4779-a214-b5da0ccdb5e7}

Summary

\textbf{Step 1:} There must be a breach of covenant.

\textbf{Step 2:} There must be a forfeiture clause in the lease.

\textbf{Step 3:} The landlord must not have waived the right to forfeit:
distinguish here between a continuing and non-continuing breach.

\textbf{Breach of rent covenant:}

\textbf{Step 4:} The landlord must have served a demand for payment, or
the lease must waive this right.

\textbf{Step 5:} The landlord can then forfeit the lease for the breach
of covenant by peaceable re-entry or court order. However, if the
tenancy is residential, the landlord must obtain a court order.

\textbf{Step 6:} The tenant, and any sub-tenant, can apply for relief
from forfeiture. Relief is the court's discretion to allow the lease to
continue. Relief can be applied for regardless of whether the right has
been exercised by court order or peaceable re-entry.

\end{env-fdd3b853-72b2-4779-a214-b5da0ccdb5e7}

forfeiture-rent.png

\hypertarget{breach-of-other-covenants}{%
\subsection{Breach of Other Covenants}\label{breach-of-other-covenants}}

\hypertarget{step-4-s-146-notice}{%
\subsubsection{Step 4: S 146 Notice}\label{step-4-s-146-notice}}

In cases other than non-payment of rent, before the landlord can forfeit
they must first serve notice upon the tenant under
\href{https://www.legislation.gov.uk/ukpga/Geo5/15-16/20/section/146}{LPA
1925, s 146}.

This notice must:

\begin{itemize}
\tightlist
\item
  Specify the breach complained of.
\item
  If capable of remedy, require it to be remedied within a reasonable
  time.

  \begin{itemize}
  \tightlist
  \item
    Breaches which are capable of remedy include continuing breaches,
    for example failure to repair or unauthorised use.
  \item
    What is a reasonable time? This depends upon the nature of the
    breach. Some textbooks suggest three months as a rule of thumb.
  \item
    If capable of remedy and the tenant does not remedy the breach
    within a reasonable time, the landlord may proceed to forfeit either
    by peaceable re-entry or by obtaining a court order (this is
    compulsory in a residential case).
  \item
    If the breach is not capable of remedy, the landlord must merely
    give the tenant enough time to consider his position, for example 14
    days, before proceeding to forfeit (Scala House and District
    Property Co Ltd v Forbes.)
  \end{itemize}
\item
  Require the lessee to make compensation in money for the breach if the
  landlord requires such compensation.

  \begin{itemize}
  \tightlist
  \item
    The landlord does not have to request compensation if he does not
    want it (Rugby School (Governors) v Tannahill {[}1935{]} 1 KB 87).
  \end{itemize}
\end{itemize}

\begin{env-ab94a4ec-2f9e-4d2f-94d1-74521d1a190e}

Note

The law is unclear as to whether a s 146 notice needs to be served for
non-payment of sums like service charge and insurance rent, where these
sums are reserved as rent in the lease.

\end{env-ab94a4ec-2f9e-4d2f-94d1-74521d1a190e}

forfeiture-non-rent.png

\begin{longtable}[]{@{}ll@{}}
\toprule()
Case & Ratio \\
\midrule()
\endhead
Billson v Residential Apartments Ltd {[}1992{]} 1 AC 494 & Unauthorised
alterations may be capable of remedy. \\
Rugby School (Governors) v Tannahill {[}1935{]} 1 KB 87 & Immoral use
was held to be incapable of remedy because of the stigma attached to the
premises. \\
London Scottish Properties v Mehmet 1970 214 EG 837 & The court
construed a covenant against immoral use as one prohibiting the kind of
conduct that the great majority of people in this country would condemn
as being immoral. \\
Hoffman v Fineberg {[}1949{]} Ch 245 & Illegal use (unlawful gaming) was
held to be incapable of remedy. \\
Van Haarlam v Kasner {[}1992{]} 64 P \& CR 214 & Illegal use (spying)
was held to be incapable of remedy but in that case it was held that the
landlord had actually waived the right to forfeit. \\
\bottomrule()
\end{longtable}

\hypertarget{illegal-or-immoral-use}{%
\paragraph{Illegal or Immoral Use}\label{illegal-or-immoral-use}}

Although Greer LJ did not think that the breach of an illegal or immoral
use covenant was capable of remedy in the 1930s, more recent case law
has been more flexible in its approach, and it now seems that it depends
on the facts of the case as to whether an illegal or immoral use
covenant is capable of remedy, as per Harman J in Van Haarlam v Kasner
{[}1992{]} 64 P \& CR 214.

It was also held in Glass v Kencakes {[}1966{]} 1 QB 611 that such
breaches are capable of remedy if the lessee did not know of the illegal
or immoral use by a subtenant, as long as the lessee takes immediate
steps to stop the use, including forfeiting the sublease within a
reasonable time.

Further, even if the breach is incapable of remedy, relief from
forfeiture is still possible in the case of immoral use (see Ropemaker
Properties v Noonhaven {[}1989{]} 2 EGLR 50, where the immoral use had
ceased).

In the case of illegal use (in Van Haarlam), the judge said, obiter,
that he would have given relief from forfeiture if there had not been
waiver, because forfeiture of the lease was out of proportion to the
offence.

In Expert Clothing (Services and Sales) Ltd v Hillgate House Ltd
{[}1986{]} Ch 340, the House of Lords established the test for assessing
whether a breach of covenant is capable or incapable of remedy:

\begin{quote}
the ultimate question for the court was this: if the section 146 notice
had required the lessee to remedy the breach and the lessors had then
allowed a reasonable time to elapse to enable the lessee fully to comply
with the relevant covenant, would such compliance, coupled with the
payment of any appropriate monetary compensation, have effectively
remedied the harm which the lessors had suffered or were likely to
suffer from the breach? If, but only if, the answer to this question was
"No," would the failure of the section 146 notice to require remedy of
the breach have been justifiable.
\end{quote}

This approach was approved by the Court of Appeal in Savva v Hussein
(1997) 73 P\&CR 150, stating that the same approach was also relevant to
negative covenants.

\hypertarget{breach-of-repair-covenants}{%
\paragraph{Breach of Repair
Covenants}\label{breach-of-repair-covenants}}

\href{https://www.legislation.gov.uk/ukpga/Geo6/1-2/34/contents}{Leasehold
Property (Repairs) Act 1938} (LP(R)A 1938) applies to a covenant to
repair in any lease (except an agricultural holding) when the lease is
granted for at least seven years and there are \textbf{at least three
years still to run}.

Where the Act applies and the landlord seeks either forfeiture or
damages for breach of a repairing covenant, the s 146 notice (in
addition to the usual requirements) must inform the tenant of his right
under the 1938 Act to \textbf{serve a counter notice within 28 days}. If
the tenant serves a counter-notice, the landlord cannot proceed to claim
forfeiture or damages without first obtaining the leave of the court.

The effect of the Act is to limit severely the landlord's ability to
enforce a repairing obligation until the last three years of the lease.

\hypertarget{damages}{%
\subparagraph{Damages}\label{damages}}

\textbf{s 18 LTA 1927}: damages for the breach of a tenant's repairing
covenant are capped at the reduction in the value of the landlord's
reversion caused by the breach. This weeds out a lot of claims.

If the provisions of LP(R)A 1938 apply to the lease, the landlord must
serve a s 146 notice on the tenant \textbf{even if the landlord is not
seeking to forfeit} and this must contain a statement of the tenant's
right to serve a counter-notice within 28 days. If the tenant serves
this notice, the landlord can proceed with the claim only with the leave
of the court.

\hypertarget{internal-decorative-repair}{%
\subparagraph{Internal Decorative
Repair}\label{internal-decorative-repair}}

Where the landlord serves notice under LPA 1925, s 146 and the notice
relates to internal decorative repairs, the tenant may apply to the
court and be wholly or partially relieved of liability to the extent
that the court thinks the notice unreasonable (LPA 1925, s 147). Section
147 does not apply to certain covenants, such as to put into decorative
repair at the start of a lease, to yield up in a specified state of
repair at the end of a lease or to any provision for the maintenance of
the structure, amongst others. --- \textbf{don't care.}

\hypertarget{self-help}{%
\subparagraph{Self-help}\label{self-help}}

The limitations can be circumvented with a self-help clause in the
lease. This allows the landlord to enter the premises and carry out the
repairs if the tenant has failed to comply, and the tenant must then pay
the landlord's costs for the repairs.

\begin{env-4361be9f-9a08-465c-86d1-6b024e2d00e2}

Important

These costs are recoverable as a \textbf{debt}, not \textbf{damages}
(Jervis v Harris {[}1996{]} Ch 195 or lease clause), though best to
specify explicitly in the lease too) {\(\;\Longrightarrow\;\)} L will be
pursuing for non-payment of a debt, not T's failure to repair
{\(\;\Longrightarrow\;\)} LP(R)A 1938 \& s 18 LTA 19927 limitations can
be avoided.

\end{env-4361be9f-9a08-465c-86d1-6b024e2d00e2}

\hypertarget{specific-performance}{%
\subparagraph{Specific Performance?}\label{specific-performance}}

In Rainbow Estates Ltd v Tokenhold Ltd {[}1999{]} Ch 64, the court held
that in principle there is no reason why the equitable remedy of
specific performance should not be used to enforce compliance by tenant
with its repair covenant. However, other remedies (e.g. damages and
self-help) are likely to be more appropriate and the court stressed that
specific performance will only be awarded in exceptional circumstances.

forfeiture-repair.png

\hypertarget{conclusion}{%
\subparagraph{Conclusion}\label{conclusion}}

Most breaches are now technically capable of remedy provided the
mischief can be put right by making full recompense to the landlord,
leaving the landlord with no lasting damage. However, it is still
probably the case that breach of a covenant against assignment or
sub-letting, and possibly immoral or illegal use, is technically not
capable of remedy. This was all confirmed in an obiter discussion by
Neuberger LJ in Akici v LR Butlin {[}2006{]} 1 WLR 201.

\hypertarget{step-5-exercise-the-right}{%
\paragraph{Step 5: Exercise the Right}\label{step-5-exercise-the-right}}

Once the landlord has correctly served a LPA 1925, s 146 notice, the
landlord can proceed to forfeit by court order or peaceable re-entry as
described above.

If the tenant does not remedy the breach within a reasonable time, the
landlord may proceed to forfeit either by peaceable re-entry or by
obtaining a court order, which is compulsory in a residential case.

If the breach is not capable of remedy, the landlord must merely give
the tenant enough time to consider his position, for example 14 days,
before proceeding to forfeit (Scala House and District Property Co Ltd v
Forbes).

\hypertarget{step-6-relief}{%
\paragraph{Step 6: Relief}\label{step-6-relief}}

The tenant may then apply for relief from forfeiture. 'Relief' means the
court's discretion to allow the lease to continue and thereby end the
forfeiture process. This discretion to grant relief is an ancient
equitable jurisdiction, but is now enshrined in various statutes, as
explained below.

\begin{env-2aad614f-6fd6-4025-876c-fcdbeae766fb}

s 146(2) LPA 1925

Where a lessor is proceeding, by action or otherwise, to enforce such a
right of re-entry or forfeiture, the lessee may, in the lessor's action,
if any, or in any action brought by himself, apply to the court for
relief; and the court may grant or refuse relief, as the court, having
regard to the proceedings and conduct of the parties under the foregoing
provisions of this section, and to all the other circumstances, thinks
fit; and in case of relief may grant it on such terms, if any, as to
costs, expenses, damages, compensation, penalty, or otherwise, including
the granting of an injunction to restrain any like breach in the future,
as the court, in the circumstances of each case, thinks fit.

\end{env-2aad614f-6fd6-4025-876c-fcdbeae766fb}

LPA 1925, s 146(2) provides that, where the landlord is proceeding by
action or otherwise to enforce forfeiture, the tenant may in the
landlord's action, or in any action brought by himself, apply to the
court for relief.

The court may grant or refuse relief as the court, having regard to all
the circumstances, thinks fit.

Therefore, in cases of breach of covenants other than rent, the tenant
applies for relief under LPA 1925, s 146(2).

It generally depends on how wilful and blatant the breach was, the
gravity of the breach, the landlord's motives for wanting forfeiture,
the damage to the premises and whether the breach can be put right.
Relief is usually given on condition that the breach is remedied and
that the tenant undertakes not to breach the covenant again.

\begin{longtable}[]{@{}ll@{}}
\toprule()
Case & Ratio \\
\midrule()
\endhead
Hyman v Rose {[}1912{]} AC 623 & The House of Lords granted relief
despite the tenant failing to comply with the repair and alterations
covenants, because the tenant gave a deposit of money to ensure the
premises were restored to their original condition at the end of the
lease. \\
St Marylebone Property Co v Tesco Stores Ltd {[}1988{]} 2 EGLR 40 & The
court refused to grant relief to a head-tenant or sub-tenant in the case
of a blatant breach of the user covenant by an unlawful sub-tenant. \\
Ropemaker Properties v Noonhaven {[}1989{]} 2 EGLR 50 & Relief was
granted, despite the use of a nightclub for immoral purposes, because
the immoral use had ceased, there was no more stigma attaching to the
premises, the tenant was excellent in all other respects, the tenant had
offered to sell the lease, and the managing director of the tenant was
ill. \\
Billson v Residential Apartments Ltd {[}1992{]} 1 AC 494 1 & The House
of Lords focused on the granting of relief from forfeiture to a tenant,
where the forfeiture is for breach of a covenant other than a covenant
to pay rent. \\
\bottomrule()
\end{longtable}

The judicial dislike of peaceable re-entry as a method of forfeiture is
apparent from Lord Templeman's judgment in Billson v Residential
Apartments. If a landlord is going to forfeit a lease, following the
issue of a s146 notice, using this method rather than obtaining a court
order, the tenant may still apply for relief within a reasonable time
after the landlord's peaceable re-entry. This is in contrast to the
finality of an executed court order, where \textbf{no further relief
application is available}.

\hypertarget{subtenants-right-to-relief-for-non-rent-breaches}{%
\subparagraph{Subtenant's Right to Relief for Non-rent
Breaches}\label{subtenants-right-to-relief-for-non-rent-breaches}}

LPA 1925, s 146(4) allows a subtenant to apply to the court for relief
against forfeiture for breach of other covenants. It is a matter of
discretion whether a subtenant will be granted relief. An unlawful
subtenant (where no consent was given for the subletting) is unlikely to
get relief (St Marylebone Property Co v Tesco Stores Ltd {[}1988{]} 2
EGLR 40). The House of Lords' interpretation of s 146(2) in Billson must
surely apply equally to s 146(4), so that a subtenant must apply for
relief within a reasonable time after peaceable re-entry.

\begin{env-fdd3b853-72b2-4779-a214-b5da0ccdb5e7}

Summary

\textbf{Step 1:} There must be a breach of covenant.

\textbf{Step 2:} There must be a forfeiture clause in the lease.

\textbf{Step 3:} The landlord must not have waived the right to forfeit:
distinguish here between a continuing and non-continuing breach.

\textbf{Breach of a non rent covenant:}

\textbf{Step 4:} The landlord must have served a LPA 1925, s 146 Notice.

(Does the breach relate to a repair covenant? Consider does the
additional protection under \textbf{LP(R)A 1938 or LPA 1925, s 147}
apply?)

\textbf{Step 5:} The landlord can then forfeit the lease for the breach
of covenant by peaceable re-entry or court order. However, if the
tenancy is residential, the landlord must obtain a court order.

\textbf{Step 6:} The tenant, and any sub-tenant, can apply for relief
from forfeiture. Relief is the court's discretion to allow the lease to
continue. Relief can be applied for regardless of whether the right has
been exercised by court order or peaceable re-entry.

\end{env-fdd3b853-72b2-4779-a214-b5da0ccdb5e7}

\hypertarget{grant-of-a-lease}{%
\section{Grant of a Lease}\label{grant-of-a-lease}}

\hypertarget{lease-types}{%
\subsection{Lease Types}\label{lease-types}}

\begin{itemize}
\tightlist
\item
  Short residential lettings

  \begin{itemize}
  \tightlist
  \item
    Tenancy usually presented as ``take it or leave it''
  \item
    Governed by statute (Housing Act 1988)
  \item
    For residential leases under 7 years, landlords may be required to
    check the immigration status of prospective tenants.
  \end{itemize}
\item
  Long residential lease

  \begin{itemize}
  \tightlist
  \item
    Lease granted on terms stipulated by the landlord.
  \end{itemize}
\item
  Commercial lease

  \begin{itemize}
  \tightlist
  \item
    Parties usually agree terms of the lease and proceed directly to
    grant (rather than having an exchange of contracts stage).
  \end{itemize}
\end{itemize}

\hypertarget{taking-instructions-from-the-landlord}{%
\subsection{Taking Instructions From the
Landlord}\label{taking-instructions-from-the-landlord}}

Before drafting the lease, the landlord's solicitor should investigate
their own client's title:

\begin{itemize}
\tightlist
\item
  Ensures the client is entitled to grant the lease and to anticipate
  any problems with the title (e.g., covenants preventing a particular
  use).
\item
  Consider if any mortgage over the property restricts the landlord's
  ability to grant a lease over the property.
\end{itemize}

The draft lease is sent to the tenant's solicitor for negotiation and
approval.

\hypertarget{drafting-contract}{%
\subsection{Drafting Contract}\label{drafting-contract}}

Where a contract for the grant of lease is to be entered into, this is
also drafted by the landlord's solicitor.

\begin{itemize}
\tightlist
\item
  Particulars of sale should give details of the term vested in the
  tenant.
\item
  Incumbrances affecting freehold title must be disclosed.;
\item
  Contract should provide for an indemnity to be given in respect of
  future breaches of any covenants affecting the title.
\item
  Except where the lease is for a term {\(\leq 3\)} years, taking effect
  in possession and with no premium payable for its grant, the contract
  for the lease must satisfy s 2 LP(MP)A 1989.
\item
  SCPC 11.2.3: the lease will be in the form annexed to the draft
  contract. The landlord will engross the lease and supply the tenant
  with the engrossment {\(\geq 5\)} working days before completion.
\end{itemize}

\hypertarget{pre-contract-package}{%
\subsection{Pre-contract Package}\label{pre-contract-package}}

The landlord's solicitor should provide the tenant's solicitor with:

\begin{itemize}
\tightlist
\item
  Draft contract
\item
  Draft lease
\item
  Evidence of freehold title
\item
  Copies of relevant planning consents
\item
  Evidence of lender's consent to the grant of the lease.
\end{itemize}

Where the Protocol is used, the landlord's solicitor will adopt the
terms of that scheme in the\\
same way as would be the case for the sale of freehold land. So a
Property Information Form (TA06) and Fittings and Contents Form (TA10)
would be included as part of the pre-contract package.

In the case of the grant of a flat lease where there is going to be a
management company providing the services, the company will need
incorporating and share certificates (if relevant) and company documents
will need to be prepared before the completion of the sale of the first
flat.

\hypertarget{energy-and-efficiency}{%
\subsubsection{Energy and Efficiency}\label{energy-and-efficiency}}

From 1 October 2008, subject to certain limited exceptions (for example
temporary buildings which are intended to be used for no more than two
years), Energy Performance Certificates (EPCs) are required whenever a
building is constructed, sold or rented out. The general rule is that an
EPC is valid for 10 years unless a new EPC is issued to replace it.

From April 2018, landlords of certain domestic or non-domestic buildings
will not be able to grant new tenancies (or renew existing ones) unless
the building being let has an EPC rating of at least E.

\hypertarget{acting-for-the-tenant}{%
\subsection{Acting for the Tenant}\label{acting-for-the-tenant}}

\hypertarget{draft-contract-and-lease}{%
\subsubsection{Draft Contract and
Lease}\label{draft-contract-and-lease}}

When parties enter into a contract before grant, the contract will
normally require the tenant to accept the draft lease in the form
annexed to the contract (SCPC 11.2.3).

\hypertarget{searches}{%
\subsubsection{Searches}\label{searches}}

Searches usually done by the tenant's solicitor. The landlord's
solicitor may do some searches and supply these as part of the
pre-contract package.

\hypertarget{lenders-requirements}{%
\subsubsection{Lender's Requirements}\label{lenders-requirements}}

Where the tenant is acquiring the lease with the aid of a mortgage, the
Lenders' Handbook requires compliance with the following conditions:

\begin{enumerate}
\tightlist
\item
  the consent of the landlord's lender has been obtained to the
  transaction (where relevant);
\item
  the length of the term to be granted provides adequate security for
  the loan (terms of less than 60 years are often unacceptable for
  mortgage purposes in the case of residential leases);
\item
  the lease contains adequate insurance provisions relating both to the
  premises themselves and (where relevant) to common parts of the
  building, and the insurance provisions coincide with the lender's own
  requirements for insurance;
\item
  title to the freehold reversion is deduced, enabling the lease itself
  to be registered with an absolute title at Land Registry;
\item
  the lease contains proper repairing covenants in respect both of the
  property itself and (where relevant) the common parts of the building;
\item
  in the case of residential leases, that there is no provision for
  forfeiture on the insolvency of the tenant.
\end{enumerate}

\hypertarget{engrossment-and-execution}{%
\subsection{Engrossment and Execution}\label{engrossment-and-execution}}

The lease is normally prepared in two identical parts, the lease and
counterpart. The lease is executed by the landlord and the counterpart
by the tenant. On completion, these are exchanged so that each party has
a copy of the lease signed by the other in case of subsequent dispute.

\begin{itemize}
\tightlist
\item
  SCPC 11.2.5 provides that it is for the landlord's solicitor to
  prepare the engrossments.
\item
  If the landlord requires the tenant to pay a fee for the preparation
  of the engrossment, this must be dealt with by special condition in
  the contract.
\item
  The landlord will sign the lease itself in readiness for completion,
  and the counterpart should be sent to the tenant's solicitor
  \textbf{{\(\geq 5\)} working days} before contractual completion date
  (SCPC 11.2.5) for execution by the tenant.
\end{itemize}

\hypertarget{apportionment-of-rent}{%
\subsection{Apportionment of Rent}\label{apportionment-of-rent}}

Rent is usually paid in advance, not in arrears. Unless completion takes
place on a day when rent under the lease falls due, a proportionate
amount of rent from completion to the rent payment date is payable by
the tenant on completion. Either deal with this in the lease or by
express special condition.

\hypertarget{completion}{%
\subsection{Completion}\label{completion}}

The landlord will receive:

\begin{itemize}
\tightlist
\item
  The counterpart lease executed by the tenant.
\item
  Any premium payable for the grant (less deposit paid on exchange of
  contracts)
\item
  Apportioned sum representing rent payable in advance under the lease.
\end{itemize}

The landlord should give:

\begin{itemize}
\tightlist
\item
  Lease executed by landlord
\item
  Certified copy of the consent of the landlord's lender to the
  transaction (where applicable).
\end{itemize}

\hypertarget{post-completion}{%
\subsection{Post-completion}\label{post-completion}}

\hypertarget{registration}{%
\subsubsection{Registration}\label{registration}}

\begin{itemize}
\tightlist
\item
  Short lease

  \begin{itemize}
  \tightlist
  \item
    Legal lease {\(\leq 7\)} years takes effect as an overriding
    interest under LRA 2002.
  \item
    Voluntary notice of a lease {\(3 < L \leq 7\)} can be placed in the
    charges register.
  \end{itemize}
\item
  Registrable lease

  \begin{itemize}
  \tightlist
  \item
    {\(L > 7\)}: lease is registrable. Registered with its own separate
    title and title number.
  \end{itemize}
\end{itemize}

The lease must be registered at Land Registry within the relevant
priority period or, on first registration, within two months of
completion of the grant of the lease. If the freehold is itself
registered, the tenant's application is for registration of a dealing.
Where the landlord's title is registered, the lease will be noted
against the reversionary title.

\hypertarget{sdlt}{%
\subsubsection{SDLT}\label{sdlt}}

A land transaction return must be submitted to HMRC on the grant of a
lease. SDLT is potentially chargeable on any capital sum paid
(``premium'' of a lease) and the amount of the rent. SDLT on any rental
element is based on the Net Present Value of the rent (use a 3.5\%/pa
discount rate on future cash flows).

\hypertarget{residential-property}{%
\paragraph{Residential Property}\label{residential-property}}

The SDLT payable on any premium is calculated on the same basis as for
the consideration on the sale of freehold land. SDLT is charged at 1\%
on any element of the NPV that exceeds £125,000.

\hypertarget{non-residential-property}{%
\paragraph{Non-residential Property}\label{non-residential-property}}

SDLT payable on any premium is calculated on the same basis as for the
consideration on the sale of freehold land. SDLT is chargeable on the
NPV as follows:

\begin{longtable}[]{@{}ll@{}}
\toprule()
Net present value of rent & SDLT rate \\
\midrule()
\endhead
{\(\pounds 0 \leq P \leq \pounds 150,000\)} & {\(0\)} \\
{\(\pounds 150,000 < P \leq \pounds 5,000,000\)} & {\(1\%\)} \\
{\(P > \pounds 5,000,000\)} & {\(2\%\)} \\
\bottomrule()
\end{longtable}

\hypertarget{notice-to-landlord}{%
\subsubsection{Notice to Landlord}\label{notice-to-landlord}}

Para 5.13.13 of the Lender's Handbook requires notice of a mortgage to
be served on the landlord and the management company (if any). Notice
should be given by sending two copies of the notice, together with a
cheque for the appropriate fee, to the landlord's solicitor or other
person named in the covenant. One copy of the notice should be placed
with the landlord's title deeds, the other should be receipted on behalf
of the landlord and returned to the tenant's solicitor.

\hypertarget{rights-reservations-and-exceptions}{%
\section{Rights, Reservations and
Exceptions}\label{rights-reservations-and-exceptions}}

The rights granted to a tenant and the matters reserved to the landlord/
excepted from the grant are part of the description of the property.
These are very important and can have an impact at rent review.

\hypertarget{ancillary-rights-granted-to-the-tenant}{%
\subsection{Ancillary Rights Granted to the
Tenant}\label{ancillary-rights-granted-to-the-tenant}}

\hypertarget{new-rights-and-easements}{%
\subsubsection{New Rights and
Easements}\label{new-rights-and-easements}}

\begin{itemize}
\tightlist
\item
  On the grant of a lease, new easements benefitting the demised
  property may be created without express grant:

  \begin{itemize}
  \tightlist
  \item
    Wheeldon v Burrows (1879) 12 Ch. D. 31
  \item
    Common intention of the parties
  \item
    s 62 LPA 1925
  \end{itemize}
\item
  This could unexpectedly limit the landlord's use of the retained land.
\item
  So it is sensible to contract to prevent the creation of new
  easements, and provide that any new rights and easements must be
  expressly granted.

  \begin{itemize}
  \tightlist
  \item
    Carefully consider right to light
  \item
    Give the tenant rights over common parts (e.g., use of communal
    toilets, parking in designated spaces).
  \end{itemize}
\item
  Extent to which these rights are needed depends on the nature of the
  property demised.
\end{itemize}

Important to both landlord and tenant that all relevant rights are
granted:

\begin{longtable}[]{@{}ll@{}}
\toprule()
Party & Importance \\
\midrule()
\endhead
Landlord & Not including rights could have an adverse effect at rent
review. Failure to seek contribution to maintenance costs would leave
the landlord with a cash shortfall. \\
Tenant & Wants to be sure it has rights necessary to use the property
for its business and comply with its covenants under the lease. \\
\bottomrule()
\end{longtable}

\hypertarget{existing-rights-and-easements}{%
\subsubsection{Existing Rights and
Easements}\label{existing-rights-and-easements}}

An easement which is already in existence over 3rd party land passes
automatically on a disposition of the estate, without express mention.
This position is not altered by s 62.

The benefit of a right of way appurtenant to land passes to the tenant
of that land without express mention (Skull v Glenister (1864) 16 CB
(NS) 81). However, it is possible to include provisions in the lease to
prevent the passing of the benefit of some or all existing easements or
other appurtenant rights.

\hypertarget{investigating-title-to-rights}{%
\subsubsection{Investigating Title to
Rights}\label{investigating-title-to-rights}}

If existing rights are to be enjoyed by the tenant, the tenant should
investigate title to ensure they are capable of passing and sufficient
for the tenant's needs. The tenant should also investigate the
landlord's title to grant new rights in the same manner.

\hypertarget{registration-1}{%
\subsubsection{Registration}\label{registration-1}}

Easements expressly granted in a lease after 13/10/03 must be completed
by registration in the case of a registered estate. If the easements are
not registered, they only take effect in equity.

\begin{env-1183acc6-7dc8-44d4-a985-6dec306e67fd}

Warning

Easements expressly granted in a lease must be registered to be
effective, even ifi the lease itself is incapable of registration.

\end{env-1183acc6-7dc8-44d4-a985-6dec306e67fd}

\hypertarget{rights-excepted-and-reserved}{%
\subsection{Rights Excepted and
Reserved}\label{rights-excepted-and-reserved}}

Exceptions and reservations deal with rights benefitting the landlord:

\begin{itemize}
\tightlist
\item
  Easements benefitting any neighbouring property belonging to the
  landlord (must have a dominant and servient tenement to create an
  easement: London \& Blenheim Estates Ltd v Ladbroke Retail Parks Ltd
  {[}1992{]}).
\item
  Rights enabling the landlord to develop a neighbouring property
  without interference
\item
  Rights assisting the landlord in ensuring the tenant complies with the
  terms of the lease.
\end{itemize}

\begin{env-ab392b14-a05f-4d9a-80eb-a63d00eb59ea}

Exception

Something that exists and is (or would be) part of the thing granted,
but which is excluded from the grant (for example, mineral rights).

\end{env-ab392b14-a05f-4d9a-80eb-a63d00eb59ea}

\begin{env-ab392b14-a05f-4d9a-80eb-a63d00eb59ea}

Reservation

Something that is not in existence at the date of the grant but which
arises out of the demise.

\end{env-ab392b14-a05f-4d9a-80eb-a63d00eb59ea}

Sometimes a right over the property in favour of the landlord, such as a
right of way, is technically in the nature of a new grant by the tenant
to the landlord. But in practice no one cares.

\begin{env-fdd3b853-72b2-4779-a214-b5da0ccdb5e7}

How widely should reservations be interpreted?

\emph{\href{https://uk.westlaw.com/D-105-0126?originationContext=document\&transitionType=PLDocumentLink\&contextData=(sc.Default)\&ppcid=8480f1faa9a046d89d64dd6a4fb84a19}{Rees
v Windsor-Clive {[}2020{]} EWCA Civ 816}}: reservations are not always
to be given their narrowest possible interpretation. In every case, it
is a question of interpreting the clause in question in its context. If
rights of entry are reserved for reasonable purposes, which is a very
common formulation, the derogation from grant principle does not require
the court to give a right of entry the narrowest possible
interpretation.

\end{env-fdd3b853-72b2-4779-a214-b5da0ccdb5e7}

\hypertarget{common-exceptions-and-reservations}{%
\subsubsection{Common Exceptions and
Reservations}\label{common-exceptions-and-reservations}}

\begin{itemize}
\tightlist
\item
  Right to light

  \begin{itemize}
  \tightlist
  \item
    Landlords often reserve a right to light for the benefit of land
    owned by the landlord.
  \item
    A right of light can be acquired by prescription against a freehold
    while the dominant land is let. It is for this reason that landlords
    usually except the right to build on land adjoining the let
    property.
  \item
    Effect: the flow of light to the property is enjoyed with the
    consent of the landlord, so a prescriptive right to light cannot be
    acquired by the landlord.
  \end{itemize}
\item
  Right of support and protection

  \begin{itemize}
  \tightlist
  \item
    Rights of support and protection are often reserved, depending on
    the nature of the let property.
  \end{itemize}
\item
  Reservations to correspond to rights granted

  \begin{itemize}
  \tightlist
  \item
    Where the leases of other parts of the building grant rights over
    the let property, corresponding reservations are needed.
  \end{itemize}
\item
  Rights in relation to works and development.

  \begin{itemize}
  \tightlist
  \item
    Rights in relation to the landlord's ability to carry out works on
    neighbouring property are reserved in appropriate cases.
  \item
    Tenant will be concerned for the potential of disruption

    \begin{itemize}
    \tightlist
    \item
      Tenant will try to have the term qualified that the landlord
      cannot carry out the works in a way which materially adversely
      affects the tenant's use of the property.
    \item
      Courts hold that redevelopment rights are construed restrictively,
      with reference to the landlord's covenant for quiet enjoyment.
    \end{itemize}
  \end{itemize}
\item
  Rights of entry to the property

  \begin{itemize}
  \tightlist
  \item
    The landlord will usually reserve a right to enter onto the property
    for a broad range of purposes at any reasonable time and after
    reasonable notice.
  \item
    This reservation usually excludes liability for any damage the
    landlord causes to the property.
  \item
    Tenant may request reasonable timescale provision/ due care
    exercised when entering/ broadening of landlord's liability through
    any damage to its belongings.
  \item
    Provisions of UCTA 1977 relating to exclusion clauses do not apply
    to contracts for the creation, transfer or termination of an
    interest in land (Sch 1, UCTA).
  \end{itemize}
\end{itemize}

\hypertarget{third-party-rights}{%
\subsection{Third Party Rights}\label{third-party-rights}}

Encumbrances which affect the property at the date of grant of the
lease.

Where a right in favour of a third party already exists over the
property for the benefit of neighbouring land not owned by the landlord,
the lease will be granted subject to that right.

\hypertarget{tenant-concerns}{%
\subsubsection{Tenant Concerns}\label{tenant-concerns}}

The tenant may be concerned by the possible existence of undisclosed
encumbrances, especially undisclosed overriding interests and
restrictive covenants.

Whilst the tenant will, as a matter of law, generally take subject to
these, there may be costs or other obligations associated with them, and
leases often include a covenant by the tenant to comply with all
obligations relating to the third party rights. They may also affect the
ability of the tenant to use the property for the permitted use.

The tenant must usually rely on the information provided by the landlord
and its inspection of the property.

The disclosure of third party rights also limits any title guarantee
given by the landlord.

\hypertarget{code-for-leasing-business-premises-ew-2020}{%
\subsection{Code for Leasing Business Premises, E\&W
2020}\label{code-for-leasing-business-premises-ew-2020}}

The Royal Institution of Chartered Surveyors' (RICS) Professional
statement,
\emph{\href{https://www.rics.org/globalassets/code-for-leasing_ps-version_feb-2020.pdf}{Code
for Leasing Business Premises, England and Wales (1st edition, February
2020)}} (Lease Code 2020) (effective from 1 September 2020) contains a
mixture of mandatory requirements and best practice.

It provides that:

\begin{enumerate}
\tightlist
\item
  The agreement as to terms of the lease on a vacant possession letting
  must be recorded in written heads of terms.
\item
  The tenant should be granted all rights necessary for the intended use
  of the premises.
\end{enumerate}

\hypertarget{lease-assignment}{%
\section{Lease Assignment}\label{lease-assignment}}

An assignment arises in 2 contexts:

\begin{enumerate}
\tightlist
\item
  Residential

  \begin{enumerate}
  \tightlist
  \item
    Usually prohibited for short-term residential lettings.
  \item
    Assignment common for a long lease.
  \end{enumerate}
\item
  Commercial.

  \begin{enumerate}
  \tightlist
  \item
    Usually leases are at market value and have no capital value.
  \item
    So no purchase price will be paid by the assignee to the assignor on
    the assignment.
  \end{enumerate}
\end{enumerate}

Seller used to denote the assignor and buyer the assignee. The terms of
the lease are not open to negotiation by the buyer -- the lease is
already in existence. An alteration to the lease can only be made by
negotiating a deed of variation of the lease with the landlord.

\hypertarget{pre-contract-matters---seller}{%
\subsection{Pre-contract Matters -
Seller}\label{pre-contract-matters---seller}}

\begin{itemize}
\tightlist
\item
  Seller's solicitor investigates title

  \begin{itemize}
  \tightlist
  \item
    Check the superior freehold title as part of this process.
  \item
    Any covenants/ easements binding the freehold also bind the
    leasehold.
  \end{itemize}
\item
  Investigation of title to check terms of the lease and anticipate
  problems.

  \begin{itemize}
  \tightlist
  \item
    Check whether landlord consent to the transaction is required.
  \item
    If a mortgage is taken out to aid the purchase, the lender will
    usually require a minimum stated length of the term remains
    unexpired at the date of the acquisition of the buyer's interest.
  \end{itemize}
\end{itemize}

\hypertarget{pre-contract-package-1}{%
\subsection{Pre-contract Package}\label{pre-contract-package-1}}

Seller's solicitor should provide the buyer's solicitor with:

\begin{itemize}
\tightlist
\item
  Draft lease
\item
  Copy lease
\item
  Evidence of title
\item
  Any licence permitting assignment to current/ previous tenants
\item
  Insurance policy relating to the property \& receipt of last payment
\item
  Receipt for last payments of rent and service charge on the property
\item
  Details of management company, if applicable.
\end{itemize}

\hypertarget{energy-performance-certificates}{%
\subsubsection{Energy Performance
Certificates}\label{energy-performance-certificates}}

EPCs generally required when a building is constructed, sold or rented
out. So required on assignments. EPC valid for 10 years unless a new EPC
is issued.

\hypertarget{pre-contract-matters-buyer}{%
\subsection{Pre-contract Matters --
Buyer}\label{pre-contract-matters-buyer}}

\begin{itemize}
\tightlist
\item
  Buyer must investigate title and consider the terms of the draft
  contract.
\item
  Seller's solicitor will have supplied a copy of the lease, which
  should be checked carefully and problems identified.
\item
  Buyer's solicitor should make the same searches as on the purchase of
  a freehold.
\end{itemize}

\hypertarget{landlords-consent}{%
\subsection{Landlord's Consent}\label{landlords-consent}}

Commercial leases usually provide for the landlord's consent to be
obtained before assignment. Consent to assign, usually embodied in a
licence to assign. Covenants controlling the tenant's freedom to assign
also control alienation (sub-letting/ charging the property).

\hypertarget{references}{%
\subsubsection{References}\label{references}}

The landlord will want to take up references on the prospective buyer to
ensure the buyer is solvent and trustworthy. Commonly required from:

\begin{itemize}
\tightlist
\item
  A current landlord
\item
  Buyer's bankers
\item
  Buyer's employer
\item
  Professional person
\item
  Person or company with whom the buyer regularly trades
\item
  3 years' audited accounts for a company/ self-employed person.
\end{itemize}

\hypertarget{surety}{%
\subsubsection{Surety}\label{surety}}

The landlord may require a surety (guarantor) to the lease as a
condition of the grant of consent. The landlord may require the seller
to enter an AGA to guarantee the performance of the tenant's covenants
by the buyer.

\begin{itemize}
\tightlist
\item
  Often inserted in the lease as a pre-condition to giving consent (in
  which case it can be insisted on, regardless of reasonableness).
\item
  If not, can be requested if reasonable.
\end{itemize}

\hypertarget{absolute-covenants}{%
\subsubsection{Absolute Covenants}\label{absolute-covenants}}

If the covenant against alienation is absolute:

\begin{itemize}
\tightlist
\item
  Any assignment will be effective
\item
  But will be a breach of covenant by the tenant and may leased to the
  forfeiture of the lease.
\end{itemize}

T may ask L to grant a variation of the lease to permit assignment, but
no obligation for L to consent/ give reasons for refusal.

\hypertarget{qualified-covenants}{%
\subsubsection{Qualified Covenants}\label{qualified-covenants}}

Permits the tenant to assign, provided the seller tenant obtains the
prior consent of the landlord to the dealing. \textbf{s 19 LTA 1927}
converts this to fully qualified -- such consent must not be withheld
unreasonably by the landlord.

LTA 1998: the landlord must, upon being served with a written request
for consent, give consent within a reasonable time unless it is
reasonable to withhold consent. Written notice of decision must be
served on the tenant within a reasonable time, stating any conditions
attached to consent/ reasons for withholding. Breach = breach of
statutory duty (in tort) {\(\rightarrow\)} tenant gets damages.

\begin{env-1183acc6-7dc8-44d4-a985-6dec306e67fd}

Warning

The obligations under the 1988 Act will not be triggered unless the
written application for consent has been correctly served on the
landlord. In EON UK plc v Gilespor ts Ltd {[}2012{]} EWHC 2172 (Ch), the
landlord's statutory duties\\
under the 1988 Act were not triggered by an application for consent to
assign made by the tenant via email - since the lease did not allow
service by email.

\end{env-1183acc6-7dc8-44d4-a985-6dec306e67fd}

In leases of commercial property granted on or after 01/01/96, the lease
may provide for the circumstances in which the landlord would withhold
his consent to an assignment and any conditions subject to which such
consent will be granted (Landlord and Tenant (Covenants) Act 1995). A
landlord will not be withholding his consent unreasonably if he insists
on compliance with these conditions.

In an old commercial lease, it is common to have a requirement that the
buyer enter into a direct covenant with the landlord to comply with the
covenants of the lease. {\(\;\Longrightarrow\;\)} buyer remains liable
on the covenants even after a subsequent disposition.

\hypertarget{demanding-premium}{%
\subsubsection{Demanding Premium}\label{demanding-premium}}

Landlord cannot demand a premium as a condition for consent -- unless
the lease specifically allows this.

\hypertarget{costs-undertaking}{%
\subsubsection{Costs Undertaking}\label{costs-undertaking}}

The landlord is entitled to ask the tenant to pay the landlord's
solicitor's reasonable charges in connection with the preparation of the
deed of consent (licence to assign). On the assignment of a commercial
lease, it is usual for the landlord's solicitors to require an
undertaking from the seller's solicitors for the payment of the costs,
although it is not clear whether it is reasonable to insist on such an
undertaking.

\begin{env-da05fb3b-c564-4905-b729-fa3c9eadbf7c}

Action

The seller's solicitor should first seek their client's authority to
give the undertaking, the undertaking should be limited to the
reasonable costs incurred, and a cap on such costs could also be sought.

\end{env-da05fb3b-c564-4905-b729-fa3c9eadbf7c}

\hypertarget{scpc}{%
\subsubsection{SCPC}\label{scpc}}

\begin{itemize}
\tightlist
\item
  SCPC 11.3.3 requires the seller to enter into an AGA, if it is
  lawfully required.
\item
  SCPC 11.3.5 further provides that if the landlord's consent has not
  been obtained by the completion date, completion is postponed until
  five working days after the seller notifies the buyer that consent has
  been given.
\item
  The contract may not be rescinded until six months have passed since
  the original completion date (SCPC 11.3.6).
\end{itemize}

\hypertarget{protocol-for-applications-for-consent-to-assign-or-sublet}{%
\subsubsection{Protocol for Applications for Consent to Assign or
Sublet}\label{protocol-for-applications-for-consent-to-assign-or-sublet}}

Protocol created by law firms, intended to establish good practice.
Non-binding.

\hypertarget{title}{%
\subsection{Title}\label{title}}

\hypertarget{absolute-title}{%
\subsubsection{Absolute Title}\label{absolute-title}}

Seller must provide the buyer with:

\begin{itemize}
\tightlist
\item
  Official copies of the register and title plan (SCPC 7.1.2)
\item
  A copy of the lease (SCPC 11.1.2)
\end{itemize}

The buyer is then to be treated as entering into the contract knowing
and fully accepting the lease terms (SCPC 11.1.2). In any event, due to
the Open Registry rules, the buyer can always inspect the seller's
title. Since the title to the lease is guaranteed by Land Registry,
there is no need for the buyer to investigate the title to the freehold
or superior leases.

\hypertarget{good-leasehold-title}{%
\subsubsection{Good Leasehold Title}\label{good-leasehold-title}}

Registration with a good leasehold title provides no guarantee of the
soundness of the title to the freehold reversion and thus, although not
entitled under the general law to do so, the buyer should insist on
deduction of the superior title to him. The provision for deduction of
the reversionary title must be dealt with by a special condition in the
contract, because neither set of standard conditions deals with this
point. Without deduction of the reversionary title, the lease may be
unacceptable to the buyer and/or his lender.

\hypertarget{transfer-deed}{%
\subsection{Transfer Deed}\label{transfer-deed}}

\begin{itemize}
\tightlist
\item
  Transfer legal title to an estate in land by deed (s 52(1) LPA 1925).
\item
  Transfer deed for assignment often called a ``deed of assignment''.
\item
  Assignment of an existing registered lease: \textbf{use Form TR1}.
\item
  For an unregistered lease, use deed of assignment for the assignment
  of a lease of 7 years or less.
\end{itemize}

\hypertarget{covenants-for-title}{%
\subsubsection{Covenants for Title}\label{covenants-for-title}}

If a seller is in breach of a repairing covenant in the lease, the lack
of repair could involve him in liability to the buyer after completion
under the covenants for title, which will be implied in the transfer
deed. Where the seller sells with full or limited title guarantee, the
covenants for title include a promise that the seller has complied with
the tenant's covenants in the lease, including repair.

\begin{env-ab94a4ec-2f9e-4d2f-94d1-74521d1a190e}

Note

Seller should \textbf{not} make any promises about repiar, because
\emph{caveat emptor} applies.

\end{env-ab94a4ec-2f9e-4d2f-94d1-74521d1a190e}

There is a conflict here between the promise implied by the covenants
for title and caveat emptor. It is resolved by modifying the covenants
for title to bring them into line with caveat emptor (SCPC 4.2.2 and
7.6.4) by excluding references to repair. This type of contractual
condition must be reflected by an express modification of the implied
covenants for title in the transfer deed itself (usually box 9 or 11 of
TR1).

\begin{quote}
The covenants set out in section 4 of the Law of Property (Miscellaneous
Provisions) Act 1994 will not extend to any breach of the tenant's
covenants in the lease relating to the physical state of the property.
\end{quote}

\hypertarget{indemnity}{%
\subsubsection{Indemnity}\label{indemnity}}

\begin{itemize}
\tightlist
\item
  Assignment of leases granted before 1 January 1996

  \begin{itemize}
  \tightlist
  \item
    Indemnity covenant from the buyer to the seller is implied except
    where, in unregistered land, value is not given by the buyer for the
    transaction (LPA 1925, s 77).
  \end{itemize}
\item
  Assignment of leases granted on or after 1 January 1996

  \begin{itemize}
  \tightlist
  \item
    The seller will usually automatically be released for future
    liability on the assignment and so will not require indemnity.
  \item
    If, however, the seller is to remain liable (e.g., under the terms
    of an AGA), an express indemnity covenant between the buyer and the
    seller should be included in the transfer deed.
  \item
    SCPC 7.6.5 entitles the seller to insert an indemnity in such
    circumstances.
  \end{itemize}
\end{itemize}

\hypertarget{preparing-for-completion}{%
\subsection{Preparing for Completion}\label{preparing-for-completion}}

\begin{itemize}
\tightlist
\item
  Buyer's solicitor prepared transfer deed
\item
  Pre-completion searches

  \begin{itemize}
  \tightlist
  \item
    Company search (if applicable)
  \item
    Title searches

    \begin{itemize}
    \tightlist
    \item
      Registered lease

      \begin{itemize}
      \tightlist
      \item
        Official searches of leasehold title to check for any new
        entries and to gain a priority period.
      \item
        If lease has good leasehold title:

        \begin{itemize}
        \tightlist
        \item
          If freehold is registered, official search against the
          freehold to check for any new entries
        \item
          If freehold unregistered, land charges department search
          against estate owners.
        \end{itemize}
      \end{itemize}
    \item
      Unregistered lease

      \begin{itemize}
      \tightlist
      \item
        Land charges department search against estate owners of
        leasehold.
      \item
        Land charges search against freehold estate owners (if freehold
        unregistered)
      \item
        Official search of freehold (if registered).
      \end{itemize}
    \end{itemize}
  \end{itemize}
\item
  Landlord's consent

  \begin{itemize}
  \tightlist
  \item
    Landlord's solicitor supplies the engrossment of the licence
    (usually by deed)
  \item
    Usually drawn up in counterparts.
  \end{itemize}
\item
  Apportionments

  \begin{itemize}
  \tightlist
  \item
    Seller supplies buyer with completion statement showing the amounts
    due, together with receipts/ statements.
  \item
    Often a provisional apportionment is made on a best estimates basis
    (SCPC 9.3.5).
  \end{itemize}
\end{itemize}

\hypertarget{completion-1}{%
\subsection{Completion}\label{completion-1}}

After payment:

\hypertarget{seller-buyer}{%
\subsubsection{Seller → Buyer}\label{seller-buyer}}

\begin{enumerate}
\tightlist
\item
  the lease/sub-lease;
\item
  the transfer deed (TR1 or deed of assignment, as appropriate);
\item
  the landlord's licence to assign;
\item
  marked abstract or other evidence of superior titles in accordance
  with the contract (lease not registered or not registered with
  absolute title);
\item
  evidence of discharge of the seller's mortgage;
\item
  copies of duplicate notices served by the seller and his predecessors
  on the landlord in accordance with a covenant in the lease requiring
  the landlord to be notified of any dispositions;
\item
  insurance policy (or copy if insurance is effected by the landlord)
  and receipt (or copy) relating to the last premium due;
\item
  receipts for rent and other outgoings; and
\item
  share certificate/stock transfer form for management company.
\end{enumerate}

\hypertarget{buyer-seller}{%
\subsubsection{Buyer → Seller}\label{buyer-seller}}

\begin{enumerate}
\tightlist
\item
  Money due in accordance with the completion statement;
\item
  duly executed counterpart licence to assign; and
\item
  a release of deposit.
\end{enumerate}

\hypertarget{rent-receipts}{%
\subsubsection{Rent Receipts}\label{rent-receipts}}

s 45(2) LPA 1925: on production of the receipt for the last rent due
under the lease/ sublease, a buyer must assume that rent has been paid
and covenants performed under that and all superior leases. SCPC 9.6
requires a buyer to assume that the correct person gave the receipt.

\hypertarget{post-completion-1}{%
\subsection{Post-completion}\label{post-completion-1}}

\begin{itemize}
\tightlist
\item
  SDLT

  \begin{itemize}
  \tightlist
  \item
    On grant of a lease, SDLT potentially payable on any premium and on
    the rent.
  \item
    Assignment:

    \begin{itemize}
    \tightlist
    \item
      SDLT only payable on any purchase price charged by the seller.
    \item
      Due at the same rates as for the sale of freehold land.
    \item
      No SDLT charged on rent.
    \end{itemize}
  \end{itemize}
\item
  Notice of assignment

  \begin{itemize}
  \tightlist
  \item
    Where, following completion, notice has to be given to a landlord of
    an assignment or mortgage, such notice should be given in duplicate
    accompanied by the appropriate fee set out in the lease.
  \item
    The Lenders' Handbook requires notice of a mortgage to be given to
    the landlord, whether or not this is required by the lease.
  \item
    The recipient of the notice should be asked to sign one copy of the
    notice in acknowledgement of its receipt, and to return the
    receipted copy to the sender (as evidence).
  \end{itemize}
\item
  Registered lease

  \begin{itemize}
  \tightlist
  \item
    Application for registration of the transfer to the buyer should be
    made within the priority period.
  \item
    Irrespective of time left on the lease.
  \end{itemize}
\item
  Unregistered lease

  \begin{itemize}
  \tightlist
  \item
    If {\(> 7\)} years outstanding, the lease must be registered at the
    Land Registry within 2 months, or it will be void in respect of the
    legal estate.
  \item
    Application for registration with absolute title can be made where
    the buyer can evidence superior title. Else, only good leasehold
    title can be obtained.
  \item
    If applicable, an application for first registration of title should
    be made within this time limit. If the title to the reversion is
    already registered, the lease will be noted against the superior
    title.
  \end{itemize}
\item
  Outstanding apportioned sums

  \begin{itemize}
  \tightlist
  \item
    The parties' solicitors should make an adjustment of the provisional
    apportionments which were made on completion.
  \item
    SCPC 9.3.5: payment must be made \textbf{within 10 working days of
    notification} by one party to the other of the adjusted figures.
  \end{itemize}
\end{itemize}

\hypertarget{drafting-leases}{%
\section{Drafting Leases}\label{drafting-leases}}

\hypertarget{drafting-and-approval}{%
\subsection{Drafting and Approval}\label{drafting-and-approval}}

Typically, a lease will include provisions relating to:

\begin{enumerate}
\tightlist
\item
  the description of the premises to be let;
\item
  the easements to be granted and reserved;
\item
  the arrangements for repair, maintenance and other services;
\item
  the provisions for payment of the service charge (if any);
\item
  the insurance arrangements;
\item
  the restrictions on the use to which the premises may be put;
\item
  any restrictions on assignment and sub-letting;
\item
  any restrictions on the making of alterations and improvements;
\item
  enforcement of covenants; and
\item
  the provisions for rent and rent review.
\end{enumerate}

The draft lease is produced by the landlord's solicitor, usually by
reference to a precedent. The lease should favour the client but still
be fair. The tenant's solicitor then approves the draft lease, following
negotiation.

\hypertarget{land-registry-prescribed-clauses}{%
\subsubsection{Land Registry Prescribed
Clauses}\label{land-registry-prescribed-clauses}}

The LRA 2002 empowered Land Registry to prescribe a form of lease which
would have to be\\
used in all cases where the lease was registrable, which it decided
against. The Land Registration (Amendment) (No 2) Rules 2005 (SI
2005/1982) provide that certain leases must contain prescribed clauses
-- namely those dated on/ after 19/06/06 which are granted out of
registered land and are compulsorily registrable.

A lease will not, however, be a prescribed clauses lease if it arises
out of a variation of a lease\\
which is a deemed surrender and re-grant, or if it is granted in a form
expressly required by\\
any of the following:

\begin{enumerate}
\tightlist
\item
  an agreement entered into before 19 June 2006;
\item
  a court order;
\item
  an enactment; and
\item
  a necessary consent or licence for the grant of the lease given before
  19 June 2006.
\end{enumerate}

\begin{env-ab392b14-a05f-4d9a-80eb-a63d00eb59ea}

Prescribed clauses leases

A lease for which use of the prescribed clauses is compulsory.

\end{env-ab392b14-a05f-4d9a-80eb-a63d00eb59ea}

\begin{env-da05fb3b-c564-4905-b729-fa3c9eadbf7c}

Action

If an applicant claims that a lease is not a prescribed clauses lease
due to one of these exceptions, a conveyancer's certificate or other
evidence must be supplied with the application for registration.

\end{env-da05fb3b-c564-4905-b729-fa3c9eadbf7c}

Prescribed clauses must appear at the beginning of a lease. Sch 1A Land
Registration Rules 2003 sets out the required wording of prescribed
clauses and gives instructions as to how the prescribed clauses must be
completed.

\hypertarget{contents}{%
\subsection{Contents}\label{contents}}

\hypertarget{commencement}{%
\subsubsection{Commencement}\label{commencement}}

The lease starts with the words `This lease', followed by the date of
its grant (date of completion), and the names and addresses of the
parties.

\hypertarget{payment-of-premium-and-receipt}{%
\subsubsection{Payment of Premium and
Receipt}\label{payment-of-premium-and-receipt}}

For a long lease of residential property, a premium is charged for the
grant of the lease. Consideration and receipt clauses must be included
in the lease. The tenant's consideration for the grant consists of:

\begin{itemize}
\tightlist
\item
  Payment of the premium
\item
  Payment of the rent
\item
  Promise to perform lease covenants.
\end{itemize}

\hypertarget{operative-words}{%
\subsubsection{Operative Words}\label{operative-words}}

``The Landlord hereby demises/ grants/ leases/ lets''

\hypertarget{title-guarantee}{%
\subsubsection{Title Guarantee}\label{title-guarantee}}

After the operative words: ``with full/ limited title guarantee''. Any
express modifications of the covenants also included.

\hypertarget{term}{%
\subsubsection{Term}\label{term}}

Take care when setting the commencement date of the term. This can be
set at an earlier date than the actual start of the lease.

\begin{env-1183acc6-7dc8-44d4-a985-6dec306e67fd}

Warning

\begin{itemize}
\tightlist
\item
  "from 1st November" means the lease starts on 2nd November.
\item
  "from and including 1st November" means the lease starts on 1st
  November.
\end{itemize}

\end{env-1183acc6-7dc8-44d4-a985-6dec306e67fd}

\hypertarget{parcels-clause}{%
\subsubsection{Parcels Clause}\label{parcels-clause}}

\begin{itemize}
\tightlist
\item
  Must be possible to say with certainty what is and is not being let.
\item
  Particularly consider walls/ floors/ ceilings in a shared building --
  who owns the wall could determine who must fix it.
\item
  For a top-floor flat, is the roof space and area above it included in
  the letting? If yes, tenant may be able to extend upwards, but may
  have to pay for roof repairs.
\item
  Car parking

  \begin{itemize}
  \tightlist
  \item
    If a specific car parking space is allocated, this should be
    included in the property let to the tenant.
  \item
    If there is just the right to park somewhere in a car park, this
    will be an easement.
  \end{itemize}
\end{itemize}

\hypertarget{easements-granted}{%
\paragraph{Easements Granted}\label{easements-granted}}

\begin{itemize}
\tightlist
\item
  Access and services

  \begin{itemize}
  \tightlist
  \item
    Consider any ancillary rights benefitting the property.
  \item
    Rights of way for utilities and cables.
  \end{itemize}
\item
  Access for repair

  \begin{itemize}
  \tightlist
  \item
    Right of access to inspect and repair/ replace pipes and cables.
  \end{itemize}
\item
  New rights

  \begin{itemize}
  \tightlist
  \item
    Make provision for the possible need to install new cables/
    facilities.
  \end{itemize}
\item
  Use of toilets etc.

  \begin{itemize}
  \tightlist
  \item
    Use of any communal toilets in the building.
  \end{itemize}
\item
  Rubbish

  \begin{itemize}
  \tightlist
  \item
    Is there a communal bin? Right to use it?
  \end{itemize}
\end{itemize}

Also ensure that reservations are made in favour of the other tenants
and landlord.

\hypertarget{key-covenants-in-leases}{%
\subsection{Key Covenants in Leases}\label{key-covenants-in-leases}}

leases 3\_1.jpg

Four of the main covenants found in leases are introduced: repair,
alterations, user, and alienation.

The basic rule is that a tenant may do all the things that an owner of
an estate can do unless the lease prohibits such actions. For this
reason, leases are drafted in a prohibitory or negative manner, setting
out what the tenant cannot do by way of a number of tenant covenants. A
lease will also contain covenants given by the landlord.

\begin{env-ab392b14-a05f-4d9a-80eb-a63d00eb59ea}

Leasehold covenant

A promise contained in a lease given by a landlord or a tenant.

\end{env-ab392b14-a05f-4d9a-80eb-a63d00eb59ea}

\hypertarget{types-of-covenant}{%
\subsection{Types of Covenant}\label{types-of-covenant}}

In leasehold work you need to be able to distinguish between different
types of covenant;

\begin{itemize}
\tightlist
\item
  Absolute
\item
  Qualified
\item
  Fully qualified
\end{itemize}

\begin{longtable}[]{@{}ll@{}}
\toprule()
Type of covenant & Description \\
\midrule()
\endhead
Absolute covenant & If there is an absolute covenant the tenant is
completely prohibited from doing something (eg 'The Tenant shall not
underlet part of the Premises'); the tenant is absolutely not able to
underlet one of its rooms in its office block and will be at the mercy
of the landlord, who will be able to consider or ignore any request to
underlet part. \\
Qualified covenant & If there is a qualified covenant (e.g. the Tenant
shall not make any non-structural alterations to the Premises without
the consent of the Landlord) then the tenant can go and ask the landlord
for his consent, although the landlord does not have to give it! There
are statutes relating to user, alterations and alienation intervene with
regard to such qualified covenants. \\
Fully qualified covenant & If there is a fully qualified covenant (e.g.,
the Tenant shall not make any internal, non-structural alterations to
the Premises without the consent of the Landlord, such consent not to be
unreasonably withheld') the landlord has to be reasonable if he is going
to withhold his consent. \\
\bottomrule()
\end{longtable}

\hypertarget{reasonableness}{%
\subsubsection{Reasonableness}\label{reasonableness}}

The courts have considered the test of 'reasonableness', especially in
respect of cases on assignment, underletting and carrying out
alterations.

International Drilling Fluids Ltd v Louisville Investments (Uxbridge)
Ltd {[}1986{]} Ch 513 stated the basic principles to be applied in
determining the reasonableness of a landlord's decision and made it
clear that a landlord is not entitled to refuse his consent on grounds
which have nothing to do with the landlord and tenant relationship.

\begin{env-8fa37bb4-768d-4893-9e93-b3dc64db7fb6}

Example

It would not be reasonable for a landlord to refuse consent on the basis
that he did not like the proposed assignee, or he supported a different
football team. It must be something to do with, for example, the
proposed assignee's ability to pay the rent or bad references.

\end{env-8fa37bb4-768d-4893-9e93-b3dc64db7fb6}

\hypertarget{alterations-and-improvements}{%
\subsection{Alterations and
Improvements}\label{alterations-and-improvements}}

Unless the lease stipulates otherwise, the tenant is free to carry out
any alterations to the premises, subject to the legal doctrine of
'waste' which prevents alterations which would devalue the premises.

However, the landlord will usually wish to control this by:

\begin{enumerate}
\tightlist
\item
  Limiting the type of alterations permitted (e.g., only allowing
  non-structural alterations and prohibiting structural alterations.)
\item
  Requiring the landlord's approval or consent (licence) in order to do
  the alteration works.
\item
  Requiring reinstatement/removal of alterations at the end of the lease
  term.
\end{enumerate}

Best to set out expressly in the lease. The landlord should consider the
impact on property value of alterations and structural safety.

\hypertarget{absolute-prohibitions}{%
\subsubsection{Absolute Prohibitions}\label{absolute-prohibitions}}

In a short-term lease, a landlord is likely to prohibit all alterations
and impose an absolute covenant.

In a multi-occupied building, the landlord may have covenanted in the
leases of the building that it will enforce the tenant's covenants at
the request of another tenant. In these circumstances, the landlord
would have to seek the permission of all tenants before giving
permission to Tenant X to carry out alterations prohibited by the lease
(Duval v 11-13 Randolph Crescent Ltd {[}2020{]} UKSC 18).

If the tenant agrees to an absolute covenant against alterations, the
tenant will be at the mercy of the landlord, unless it can rely on an
exception:

\begin{enumerate}
\tightlist
\item
  s 3 LTA 1927

  \begin{itemize}
  \tightlist
  \item
    If the tenant wants to carry out improvements, it can serve a notice
    on the landlord detailing its proposals
  \item
    The landlord has 3 months within which to object, and if it does
    then the tenant has the right to apply to the court for
    authorisation to carry out the improvements.
  \item
    The court can authorise the improvements if they add to the letting
    value of the property, are reasonable, and are suitable to the
    character of the property and do not diminish the value of any other
    property of the landlord.
  \item
    Instead of objecting or consenting to the works, a landlord can
    offer to carry out the works itself in return for a reasonable
    increase in the rent (Landlord and Tenant Act 1927, s 3(1)).
  \item
    A tenant is under no obligation to accept an offer by the landlord
    to carry out the works, and may withdraw its notice.
  \item
    If the tenant rejects the landlord's offer, the court cannot give
    the tenant authority to do the works itself (Norfolk Capital Group
    Ltd v Cadogan Estates Ltd {[}2004{]} EWHC 384 (Ch)).
  \item
    If the landlord does not offer to carry out the works itself or
    object to the improvements within three months (or if the court
    authorises the work), then the tenant may lawfully carry them out,
    even if the lease prohibits the works.
  \end{itemize}
\item
  Equality Act 2010

  \begin{itemize}
  \tightlist
  \item
    Provides for a regime of implied consents in relation to `reasonable
    adjustments' required to meet the needs of those with disabilities.
  \end{itemize}
\end{enumerate}

\hypertarget{qualified-prohibitions}{%
\subsubsection{Qualified Prohibitions}\label{qualified-prohibitions}}

A qualified covenant against alterations prohibits alterations without
the landlord's prior consent. The
\href{https://www.legislation.gov.uk/ukpga/Geo5/17-18/36/section/19}{Landlord
and Tenant Act (`LTA') 1927, s 19(2)} applies to 'qualified'
\textbf{alterations} covenants.

\begin{env-2aad614f-6fd6-4025-876c-fcdbeae766fb}

s 19(2) LTA 1927

In all leases whether made before or after the commencement of this Act
containing a covenant condition or agreement against the making of
improvements without a licence or consent, such covenant condition or
agreement shall be deemed, notwithstanding any express provision to the
contrary, to be \textbf{subject to a proviso that such licence or
consent is not to be unreasonably withheld}; but this proviso does not
preclude the right to require as a condition of such licence or consent
the payment of a reasonable sum in respect of any damage to or
diminution in the value of the premises or any neighbouring premises
belonging to the landlord, and of any legal or other expenses properly
incurred in connection with such licence or consent nor, in the case of
an improvement which does not add to the letting value of the holding,
does it preclude the right to require as a condition of such licence or
consent, where such a requirement would be reasonable, an undertaking on
the part of the tenant to reinstate the premises in the condition in
which they were before the improvement was executed.

\end{env-2aad614f-6fd6-4025-876c-fcdbeae766fb}

Where there is a qualified covenant against alteration then LTA 1927, s
19(2) implies into a qualified covenant against improvements a proviso
that the landlord's consent is not to be unreasonably withheld. It
therefore \textbf{converts a qualified covenant against alterations that
amount to improvements into a fully qualified one}.

\hypertarget{improvements}{%
\paragraph{Improvements}\label{improvements}}

LTA 1927, s 19(2) only applies to 'improvements'. What then constitutes
an `improvement?'

The leading case is Lambert v FW Woolworth \& Co Limited {[}1938{]} Ch
833 which held that 'improvements' are to be construed widely as works
which improve the premises from the \textbf{tenant's perspective}.

\begin{env-b23dd66a-8a6c-447d-a163-3614469799d6}

Tip

If the works will increase the value or usefulness of the premises to
the tenant then they will constitute an improement, even if they result
in a reduction in the value of the landlord's reversionary interest.

\end{env-b23dd66a-8a6c-447d-a163-3614469799d6}

LTA 1927, s 19(2) allows the landlord to require as a condition of
giving consent:

\begin{itemize}
\tightlist
\item
  payment of compensation for loss in value to the reversion caused by
  the alterations;
\item
  reinstatement of the premises if reasonable (at the end of the lease
  term);
\item
  payment of the landlord's expenses in giving consent.
\end{itemize}

The above conditions would usually be set out in the \textbf{Licence for
Alterations} but, even if they are not, the landlord is still permitted
to ask for them.

\hypertarget{practicalities}{%
\paragraph{Practicalities}\label{practicalities}}

In practice, most commercial leases will allow certain alterations with
the landlord's consent not to be unreasonably withheld and expressly
prohibit others (e.g., structural alterations). LTA 1927, s 19(2) does
not apply to these covenants, which are already fully qualified.

LTA 1927, s 19(2) has no effect on absolute covenants. However, even
though there may be an absolute covenant in the lease against, e.g.,
structural alterations, this does not preclude the landlord from
granting a one-off consent to a particular structural alteration. This
one-off consent would be documented in a Licence for Alterations. The
landlord would be free to stipulate any conditions to such consent he
wanted, and these would not be subject to any test of reasonableness.

\hypertarget{compensation}{%
\paragraph{Compensation}\label{compensation}}

Part 1 LTA 1927 entitles a tenant of business premises to claim
compensation for improvements at the end of the term that "add to the
letting value of the holding". The tenant must have obtained prior
authorisation to make the improvements using the s 3 statutory
procedure. Rarely relevant since most leases contain a tenant's covenant
to remove all alterations and reinstate the premises at the end of the
term.

\hypertarget{insurance}{%
\subsection{Insurance}\label{insurance}}

Could be the responsibility of the tenant or landlord to keep the
property insured. For a letting of part, the management company usually
insures the whole block.

Consider the following:

\begin{enumerate}
\tightlist
\item
  The risks insured against

  \begin{itemize}
  \tightlist
  \item
    The risks insured against should be expressly stated.
  \item
    Common for the list of risks to conclude with "such other risks as
    the landlord may reasonably require/ the tenant may reasonably
    request".
  \end{itemize}
\item
  Amount of cover

  \begin{itemize}
  \tightlist
  \item
    The property should be insured to its full reinstatement value,
    including:

    \begin{itemize}
    \tightlist
    \item
      Cost of demolition and site clearance
    \item
      Professional fees
    \item
      Allowance for inflation.
    \end{itemize}
  \end{itemize}
\item
  Application of policy monies.

  \begin{itemize}
  \tightlist
  \item
    Covenant by the landlord to use the proceeds to reinstate the
    premises.
  \item
    Ideally for tenant, the landlord should covenant to make up any
    shortfall themselves.
  \item
    What if reinstatement is impossible? Probably fairest to have a term
    splitting the insurance pay-out between the tenant and landlord.
  \end{itemize}
\end{enumerate}

\hypertarget{rent-suspension}{%
\subsubsection{Rent Suspension}\label{rent-suspension}}

\begin{env-1183acc6-7dc8-44d4-a985-6dec306e67fd}

Warning

In the absence of an express term to the contrary, rent will continue to
be payable even if the property is rendered unusable.

\end{env-1183acc6-7dc8-44d4-a985-6dec306e67fd}

The tenant should ensure that the lease provides for the payment of rent
to be suspended during any period that the premises cannot be occupied
following damage by an insured risk.

\hypertarget{termination}{%
\subsubsection{Termination}\label{termination}}

If the building is totally destroyed, it will only be in exceptional
circumstances that the doctrine of frustration will apply and the lease
will be terminated (National Carriers Ltd v Panalpina (Northern) Ltd
{[}1981{]} AC 675). The lease will usually give the landlord the right
to determine the lease on notice if reinstatement is impossible. The
tenant should try to give itself a similar right.

\hypertarget{permitted-user}{%
\subsection{Permitted User}\label{permitted-user}}

A lease will often contain a tenant covenant relating to the use of the
premises. A landlord will want control over what the tenant is to use
the premises for, e.g., for residential use or a specific business
purpose. A tenant may be able to change the use of the premises
depending upon the type of covenant.

\begin{env-da05fb3b-c564-4905-b729-fa3c9eadbf7c}

Action

When you are acting for a tenant, you should consider the user
provisions carefully, to ensure that they are not going to cause your
client any problems in his occupation of the unit, or if he should wish
to assign the lease.

\end{env-da05fb3b-c564-4905-b729-fa3c9eadbf7c}

Both parties should consider any impact of competition legislation
(Martin Retail Group Ltd v Crawley Borough Council (24 December 2013,
Central London County Court)).

\hypertarget{premium-for-consent}{%
\subsubsection{Premium for Consent}\label{premium-for-consent}}

\href{https://www.legislation.gov.uk/ukpga/Geo5/17-18/36/section/19}{LTA
1927, s 19(3)} applies to 'qualified' \textbf{user} covenants.

\begin{env-2aad614f-6fd6-4025-876c-fcdbeae766fb}

s 19(3) LTA 1927

In all leases whether made before or after the commencement of this Act
containing a covenant condition or agreement against the alteration of
the user of the demised premises, without licence or consent, such
covenant condition or agreement shall, if the alteration \textbf{does
not involve any structural alteration} of the premises, be deemed,
notwithstanding any express provision to the contrary, to be subject to
a proviso that \textbf{no fine or sum of money in the nature of a fine,
whether by way of increase of rent or otherwise, shall be payable for or
in respect of such licence or consent}; but this proviso does not
preclude the right of the landlord to require payment of a reasonable
sum in respect of any damage to or diminution in the value of the
premises or any neighbouring premises belonging to him and of any legal
or other expenses incurred in connection with such licence or consent.

\end{env-2aad614f-6fd6-4025-876c-fcdbeae766fb}

If the change of use does involve a change to the structure, the
landlord can increase the rent or charge the tenant a lump sum (in the
statute referred to as a 'fine' or 'premium') in return for the consent.

The landlord will be entitled to recover its costs and expenses involved
in the application for consent, e.g., surveyor's fees and legal fees.

\begin{env-1183acc6-7dc8-44d4-a985-6dec306e67fd}

Warning

There is no statutory implication that the landlord's consent cannot be
unreasonably withheld. So a tenant would much prefer the covenant to be
fully qualified.

\end{env-1183acc6-7dc8-44d4-a985-6dec306e67fd}

\hypertarget{flats}{%
\subsection{Flats}\label{flats}}

In a lease of a flat, the landlord will usually want to ensure the block
is solely for residential use (to preserve the value of the reversion).
Such a covenant is usually acceptable to both landlords and tenants.

\hypertarget{commercial-leases}{%
\subsubsection{Commercial Leases}\label{commercial-leases}}

Considerations:

\begin{itemize}
\tightlist
\item
  Having a good mix of retail units in a retail park
\item
  Having only offices in an office block.
\item
  The broadness of the user clause usually impacts the rent.
\item
  Use Classes Order 1987 sets out different categories of uses, which
  can be used to specify the permitted uses.
\end{itemize}

\hypertarget{repair}{%
\subsection{Repair}\label{repair}}

\hypertarget{who-does-what}{%
\subsubsection{Who Does What}\label{who-does-what}}

\begin{env-ab392b14-a05f-4d9a-80eb-a63d00eb59ea}

"Full repairing lease"

The tenant has full responsibility for the repair of the whole property
- respondible for carrying out the repairs and bearing the cost.

\end{env-ab392b14-a05f-4d9a-80eb-a63d00eb59ea}

If the demised premises form part of a larger property, a "full
repairing lease" places direct responsibility for repairing the demised
premises on the tenant. It also makes the tenant indirectly responsible
for the cost (or a proportion of the cost) of repairs to the structure,
exterior and common parts of the property through the service charge.

\begin{itemize}
\tightlist
\item
  For a lease of a whole building, repairing obligations are usually
  imposed on the tenant
\item
  Where a tenant has a lease of part, responsibility will be divided.
\item
  Repairing obligations implied into short residential tenancies under
  Housing Act 1985, but this is not normally the case for longer
  residential leases/ leases of commercial property.
\end{itemize}

\hypertarget{obligations-by-3rd-parties}{%
\subsubsection{Obligations by 3rd
Parties}\label{obligations-by-3rd-parties}}

For flat leases, often a management company is used. Ensure the tenant
is able to enforce these obligations (e.g., privity of contract,
landlord and tenant grounds).

If repairing obligations are imposed on other tenants, check there is a
covenant from the landlord that the landlord will enforce them against
other tenants on request/ provisions of Contracts (Rights of Third
Parties) Act 1999 apply.

\hypertarget{who-pays}{%
\subsubsection{Who Pays}\label{who-pays}}

Usually the tenant, either directly or through a service charge.

\hypertarget{extent-of-obligation}{%
\subsubsection{Extent of Obligation}\label{extent-of-obligation}}

Under a general repairing covenant, the tenant must, according to the
Court of Appeal in Proudfoot v Hart (1890) LR 25 QBD 42, keep the
premises in the condition in which they would be kept by a reasonably
minded owner, having regard to:

\begin{enumerate}
\tightlist
\item
  The character and type of premises at the beginning of the lease. The
  obligation is neither diminished nor increased by a change in the
  character of the neighbourhood.
\item
  The age of the premises.
\item
  The express words of the covenant.
\end{enumerate}

Many leases require the tenant to keep the demised premises in "good
repair", "good and tenantable repair" or "substantial repair". It is
uncertain whether the additional words add anything to the obligation to
"repair" and there is case law to suggest that generally they do not
(\emph{Proudfoot v Hart (1890) 25 QBD 42}). A covenant that requires the
tenant to keep the property "in good repair and condition" is more
onerous than one that specifies "good repair" alone.

\begin{env-1183acc6-7dc8-44d4-a985-6dec306e67fd}

Warning

\textbf{A covenant to keep the premises in repair also entails an
obligation to put them in repair first}, if at the time of the letting
they were out of repair (Proudfoot v Hart (1890) LR 25 QBD 42).
Rationale: if the tenant covenants to keep property in repair, the
tenant cannot perform this covenant unless the tenant first puts the
property into repair
(\emph{\href{https://uk.practicallaw.thomsonreuters.com/D-009-7105?originationContext=document\&transitionType=PLDocumentLink\&contextData=(sc.Default)\&ppcid=47d8e69b58514ba1976c981efd5911b3}{Payne
v Haine {[}1847{]} 16 M\&W 41}}). This can be a very onerous obligation
if the premises are in a state of disrepair at the beginning of the
term.

\end{env-1183acc6-7dc8-44d4-a985-6dec306e67fd}

\begin{env-da05fb3b-c564-4905-b729-fa3c9eadbf7c}

Action

Before taking the lease, the tenant should inspect the property (and the
building of which it forms part) for disrepair and then assess the
potential repair costs.

\end{env-da05fb3b-c564-4905-b729-fa3c9eadbf7c}

\hypertarget{schedule-of-condition}{%
\paragraph{Schedule of Condition}\label{schedule-of-condition}}

If the tenant simply wants to maintain the premises in the condition
they were in at the outset of the lease, the tenant can employ a
surveyor to survey the premises and report on the state of repair prior
to taking the lease. Then a schedule of condition (photographs and
verbal description of the premises prepared by a surveyor) can be
annexed to the lease prior to the grant and the repairing covenant can
then refer to it as evidence of the state of repair. The repair
obligation would refer to the schedule of condition by stating that the
tenant is 'under no obligation to put the premises in any better state
of repair than as evidenced by the schedule of condition annexed to the
lease'. Beyond this, the tenant then has no obligation to the landlord.

A schedule of condition may be particularly important to an undertenant
who is taking an underlease. The undertenant may argue that a full
repairing obligation is inappropriate in these circumstances and that it
is unreasonable to expect it to hand back the property in a better
condition than it was in when the underlease was granted. This may
require the superior landlord's consent if the superior lease stipulates
that any underlease is to be on the same terms.

\hypertarget{inherent-defect}{%
\paragraph{Inherent Defect}\label{inherent-defect}}

A defect in the design/ construction of the building, which existed on
completion but was not apparent on inspection.

If the inherent defect does not cause damage, the tenant will not be
required to remedy
(\emph{\href{https://uk.practicallaw.thomsonreuters.com/D-009-7107?originationContext=document\&transitionType=PLDocumentLink\&contextData=(sc.Default)\&ppcid=966adc813d44458ba2271293efe92ed6}{Post
Office v Aquarius Properties Limited {[}1987{]} 1 All ER 1055}}).

Particularly where the property is newly built, the tenant may wish to
exclude liability for:

\begin{itemize}
\tightlist
\item
  Disrepair caused by inherent defects.
\item
  Remedying the inherent defect itself.
\end{itemize}

``Inherent defect'' should be defined specifically in the lease. The
tenant's exclusion should be coupled with an obligation on the landlord
to be responsible for such repairs (at its own cost, rather than through
the service charge).

Alternatively, rather than agreeing to a limitation of the tenant's
repairing liability, a landlord may offer the tenant direct warranties
from the professional team, or third party rights.

\hypertarget{insured-risks}{%
\paragraph{Insured Risks}\label{insured-risks}}

These are usually excluded from the tenant's repairing obligation.

But the tenant will normally be responsible for repairing damage caused
by an insured risk if the insurance monies cannot be recovered because
of an act or omission of the tenant.

\hypertarget{approval-of-works}{%
\paragraph{Approval of Works}\label{approval-of-works}}

If the tenant's covenant requires repairs to be carried out to the
satisfaction of the landlord's surveyor, the surveyor is entitled to
prescribe the work to be carried out as well as the method of repair
(\emph{\href{https://uk.practicallaw.thomsonreuters.com/D-009-7111?originationContext=document\&transitionType=PLDocumentLink\&contextData=(sc.Default)\&ppcid=966adc813d44458ba2271293efe92ed6}{Mason
v Totalfinaelf UK {[}2003{]} 3 EGLR 91}}). Should usually be resisted by
the tenant.

\hypertarget{repair-vs-renewal}{%
\paragraph{Repair Vs Renewal}\label{repair-vs-renewal}}

A covenant to repair does not require renewal of the whole or
substantially the whole of the property. It is a question of degree
whether the work involves repair or renewal, but if the works constitute
'renewal' rather than 'repair', they will not fall within the tenant's
repair obligation.

\begin{longtable}[]{@{}ll@{}}
\toprule()
Case & Ratio \\
\midrule()
\endhead
Lurcott v Wakely {[}1911{]} 1 KB 905 & Whether works are classed as
'repair' or 'renewal' depends upon whether the whole or substantially
the whole needs to be replaced. \\
Brew Brothers Ltd v Snax (Ross) Ltd {[}1970{]} 1 All ER 587 & The
question of whether works fall within the scope of a repairing covenant
is one of degree. \\
\emph{\href{https://uk.practicallaw.thomsonreuters.com/D-009-7108?originationContext=document\&transitionType=PLDocumentLink\&contextData=(sc.Default)\&ppcid=47d8e69b58514ba1976c981efd5911b3\&comp=pluk}{Lister
v Lane {[}1893{]} 2 QB 212}} & The standard and nature of work which the
tenant has to carry out depends on the age and nature of the property at
the date of the grant of the lease. \\
\emph{\href{https://uk.practicallaw.thomsonreuters.com/D-009-7109?originationContext=document\&transitionType=PLDocumentLink\&contextData=(sc.Default)\&ppcid=47d8e69b58514ba1976c981efd5911b3\&comp=pluk}{Minja
Properties v Cussins Property Group {[}1988{]} 2 EGLR 52}} & The
replacement of corroded metal-framed single-glazed windows with
double-glazed windows \textbf{was} within the tenant's repairing
covenant. The change in the design was ancillary to rectifying the
damage. \\
\bottomrule()
\end{longtable}

Equally, ``repair'' does not extend as far as improvements. There is a
risk that the landlord/ management company may decide to carry out
large-scale improvements which tenants do not want/ need, at the
tenant's expense.

Fair compromise: a clause to "improve and renew to the extent such
renewals or improvements are necessary or desirable to keep the Building
in good and substantial repair".

Before an obligation to repair can "bite", the property must be in
disrepair. This means that the physical condition of the property must
have deteriorated (\emph{Post Office v Aquarius Properties Ltd
{[}1987{]} 1 All ER 1055}).

Where the tenant is responsible for carrying out repairs, it will
normally be for the tenant to choose the method of repair.

\hypertarget{occupiers-liability}{%
\paragraph{Occupiers' Liability}\label{occupiers-liability}}

look out for clauses which require the tenant to indemnify the landlord
against any such claim under the Occupiers' Liability Act 1957 by the
tenant's visitors.

\hypertarget{other-phrases}{%
\paragraph{Other Phrases}\label{other-phrases}}

\hypertarget{good-condition}{%
\subparagraph{Good Condition}\label{good-condition}}

An obligation to keep the property in \textbf{good condition} can
require works to be carried out even if there is no disrepair (e.g.,
installing insulation to a property suffering condensation, even where
the property was not in disrepair -
\emph{\href{https://uk.practicallaw.thomsonreuters.com/D-008-7715?originationContext=document\&transitionType=PLDocumentLink\&contextData=(sc.Default)\&ppcid=47d8e69b58514ba1976c981efd5911b3\&comp=pluk}{Welsh
v Greenwich LBC (2001) 33 H.L.R. 40}}).

It is also worth noting that sometimes an obligation to keep in good
condition imposes a liability less than repair. In
\emph{\href{https://uk.practicallaw.thomsonreuters.com/D-023-2581?originationContext=document\&transitionType=PLDocumentLink\&contextData=(sc.Default)\&ppcid=47d8e69b58514ba1976c981efd5911b3}{Firstcross
Ltd v Teasdale {[}1983{]} 8 HLR 112}}, the High Court held that a tenant
of a flat, who had covenanted to keep it in a good and a tenantable
condition, was required to do no more than use the flat in a tenant-like
manner.

\hypertarget{to-maintain}{%
\subparagraph{To Maintain}\label{to-maintain}}

Ambiguous:

\begin{itemize}
\tightlist
\item
  May require only maintenance of premises in the state they were in
  when demised.
\item
  Or can be construed by reference to the purpose of the object
  requiring repair.
\end{itemize}

\hypertarget{physical-state-of-obligation}{%
\paragraph{Physical State of
Obligation}\label{physical-state-of-obligation}}

The scope of repairing obligation will coincide with the physical extent
of the demised premises. Where the demised premises are part of a larger
property, each tenant will be responsible for repairing its demised
property and the landlord will maintain the exterior, structure and
common parts.

\begin{env-da05fb3b-c564-4905-b729-fa3c9eadbf7c}

Action

Ensure there is no overlap/ gap in repairing obligations, e.g., windows,
ceilings.

\end{env-da05fb3b-c564-4905-b729-fa3c9eadbf7c}

\hypertarget{mechanical-electrical-services-and-plant}{%
\paragraph{Mechanical, Electrical Services and
Plant}\label{mechanical-electrical-services-and-plant}}

If plant and machinery are annexed to the land and have become
\emph{\href{https://uk.practicallaw.thomsonreuters.com/8-202-2732?originationContext=document\&transitionType=DocumentItem\&contextData=(sc.Default)\&ppcid=966adc813d44458ba2271293efe92ed6}{fixtures}},
the repairing obligations will apply to them.

Whether an object has become part of the land depends on the degree of
annexation to the land
(\emph{\href{https://uk.practicallaw.thomsonreuters.com/D-000-1799?originationContext=document\&transitionType=PLDocumentLink\&contextData=(sc.Default)\&ppcid=966adc813d44458ba2271293efe92ed6}{Elitestone
Ltd v Morris and Another {[}1997{]} UKHL 15}}) - fixtures vs fittings.

\begin{env-fdd3b853-72b2-4779-a214-b5da0ccdb5e7}

Maxim

"Quidquid Plantatur Solo, Solo Cedit" - whatever is attached to the land
becomes part of the land.

\end{env-fdd3b853-72b2-4779-a214-b5da0ccdb5e7}

This principle is encapsulated in modern land law in the LPA 1925's
definition of `land' which includes ``other corporeal hereditaments''
i.e., things attached/fixed to the land, what we `fixtures' in practice.

Plant and machinery:

\begin{itemize}
\tightlist
\item
  It may be sensible for the tenant to commission a mechanical and
  engineering consultant to carry out proper tests and report.
\item
  Although a landlord should always seek to impose a covenant to repair
  the mechanical, electrical services and plant, under a conventional
  repair obligation, there is no requirement to carry out any works
  until there is damage or deterioration.
\end{itemize}

In addition to an obligation to repair, there is often:

\begin{itemize}
\tightlist
\item
  A covenant requiring maintenance
\item
  A covenant to keep in working order/ keep in operating condition
\end{itemize}

Emphasis on practical function, not physical condition.

To make the clause more acceptable to a covenantor, the covenantee may
be prepared to qualify the obligation by only requiring the covenantee
to keep the mechanical, electrical services and plant so that they are
capable of functioning in a manner to the covenantor's
\textbf{reasonable} satisfaction.

\hypertarget{breach-of-repair-obligations}{%
\subsubsection{Breach of Repair
Obligations}\label{breach-of-repair-obligations}}

When a landlord enters into a repairing covenant, there will be no
breach unless and until he has notice (no matter from what source) of
the lack of repair.

\begin{env-da05fb3b-c564-4905-b729-fa3c9eadbf7c}

Action

When acting for a landlord, you should include a tenant's covenant to
notify the landlord of the need to repair. If the tenant fails to do
this, he will also be in breach of covenant and liable to the landlord
for damages

\end{env-da05fb3b-c564-4905-b729-fa3c9eadbf7c}

If the part of the property in question is in the control of the
landlord, for example the common parts, then the landlord's liability
does not depend upon his having received notice of the disrepair.

A ``self-help'' right is usually included: landlord's right to enter and
effect any repairs not done by the tenant, and then recharge the cost to
the tenant.

\hypertarget{assignment-and-sub-letting}{%
\subsection{Assignment and
Sub-letting}\label{assignment-and-sub-letting}}

Common law: a tenant is free to dispose of his properties, whether by
outright assignment of the lease or by sublease.

\hypertarget{flat-leases}{%
\subsubsection{Flat Leases}\label{flat-leases}}

Long-term flat lease: T will have paid a premium for the grant of the
lease. Lease must be acceptable to any prospective mortgagee. Assignment
of the whole usually permitted freely in this context, though there may
be restrictions on assignment and subletting of part. Landlord will want
to know the identity of new tenants, so often a covenant to register new
dealings with the landlord and pay a fee.

\hypertarget{commercial-leases-1}{%
\subsubsection{Commercial Leases}\label{commercial-leases-1}}

Far more restrictions - identity and status of the occupier matters.
Parting with possession or sharing occupation is either prohibited or
strictly controlled. This is wider than assignment/ sub-letting, but
does not include the granting of a licence.

Usually assignment, sub-letting and/or sharing with group companies is
allowed in circumstances, after this blanket ban.

\hypertarget{qualified-prohibition}{%
\subsubsection{Qualified Prohibition}\label{qualified-prohibition}}

Absolute prohibition usually unacceptable to tenants. So there is
usually a qualified/ fully qualified prohibition.

\begin{env-4361be9f-9a08-465c-86d1-6b024e2d00e2}

Important

A qualified covenant can be converted into a fully qualified covenant by
the operation of s 19(1)(a) LTA 1927.

\end{env-4361be9f-9a08-465c-86d1-6b024e2d00e2}

The Landlord and Tenant Act 1988 further strengthened the position of a
tenant seeking\\
consent to assign, sub-let, charge or part with possession. The Act
applies where the lease\\
contains a fully qualified covenant against alienation (whether express
or implied by statute). The Act still applies even if the fully
qualified covenant is subject to s 19(1A) of the Landlord and Tenant Act
1927.

\begin{env-2aad614f-6fd6-4025-876c-fcdbeae766fb}

s 1 LTA 1988 - Qualified duty to consent to assigning, underletting etc.
of premises.

(1) This section applies in any case where---

\begin{itemize}
\tightlist
\item
  (a) a tenancy includes a covenant on the part of the tenant not to
  enter into one or more of the following transactions, that is---

  \begin{itemize}
  \tightlist
  \item
    (i) assigning,
  \item
    (ii) underletting,
  \item
    (iii) charging, or
  \item
    (iv) parting with the possession of,
  \end{itemize}
\item
  the premises comprised in the tenancy or any part of the premises
  without the consent of the landlord or some other person, but
\item
  (b) the covenant is subject to the qualification that the consent is
  not to be unreasonably withheld (whether or not it is also subject to
  any other qualification).
\end{itemize}

(2) In this section and section 2 of this Act---

\begin{itemize}
\tightlist
\item
  (a) references to a proposed transaction are to any assignment,
  underletting, charging or parting with possession to which the
  covenant relates, and
\item
  (b) references to the person who may consent to such a transaction are
  to the person who under the covenant may consent to the tenant
  entering into the proposed transaction.
\end{itemize}

(3) Where there is served on the person who may consent to a proposed
transaction a written application by the tenant for consent to the
transaction, he owes a duty to the tenant within a reasonable time---

\begin{itemize}
\tightlist
\item
  (a) to give consent, except in a case where it is reasonable not to
  give consent,
\item
  (b) to serve on the tenant written notice of his decision whether or
  not to give consent specifying in addition---

  \begin{itemize}
  \tightlist
  \item
    (i) if the consent is given subject to conditions, the conditions,
  \item
    (ii) if the consent is withheld, the reasons for withholding it.
  \end{itemize}
\end{itemize}

(4) Giving consent subject to any condition that is not a reasonable
condition does not satisfy the duty under subsection (3)(a) above.

(5) For the purposes of this Act it is reasonable for a person not to
give consent to a proposed transaction only in a case where, if he
withheld consent and the tenant completed the transaction, the tenant
would be in breach of a covenant.

(6) It is for the person who owed any duty under subsection (3) above---

\begin{itemize}
\tightlist
\item
  (a) if he gave consent and the question arises whether he gave it
  within a reasonable time, to show that he did,
\item
  (b) if he gave consent subject to any condition and the question
  arises whether the condition was a reasonable condition, to show that
  it was,
\item
  (c) if he did not give consent and the question arises whether it was
  reasonable for him not to do so, to show that it was reasonable,
\end{itemize}

and, if the question arises whether he served notice under that
subsection within a reasonable time, to show that he did.

\end{env-2aad614f-6fd6-4025-876c-fcdbeae766fb}

\begin{env-fdd3b853-72b2-4779-a214-b5da0ccdb5e7}

Summary

When the tenant has made written application for consent, the landlord
owes a duty, within a reasonable time, to give consent, unless it is
reasonable not to do so. In addition, the landlord must serve on the
tenant written notice of his decision whether or not to give consent,
specifying in addition the conditions he is imposing, or the reasons why
he is withholding consent. The burden of proving the reasonableness of
any refusal or any conditions imposed is on the landlord.

\end{env-fdd3b853-72b2-4779-a214-b5da0ccdb5e7}

\hypertarget{covenants-against-assigning-commercial-leases}{%
\subsubsection{Covenants Against Assigning Commercial
Leases}\label{covenants-against-assigning-commercial-leases}}

\begin{env-2aad614f-6fd6-4025-876c-fcdbeae766fb}

s 19(1A) LTA 1927

Where the landlord and the tenant under a qualifying lease have entered
into an agreement specifying for the purposes of this subsection---

\begin{itemize}
\tightlist
\item
  (a) any circumstances in which the landlord may withhold his licence
  or consent to an assignment of the demised premises or any part of
  them, or
\item
  (b) any conditions subject to which any such licence or consent may be
  granted,
\end{itemize}

then the landlord---

\begin{itemize}
\tightlist
\item
  (i) shall not be regarded as unreasonably withholding his licence or
  consent to any such assignment if he withholds it on the ground (and
  it is the case) that any such circumstances exist, and
\item
  (ii) if he gives any such licence or consent subject to any such
  conditions, shall not be regarded as giving it subject to unreasonable
  conditions;
\end{itemize}

and section 1 of the Landlord and Tenant Act 1988 (qualified duty to
consent to assignment etc.) shall have effect subject to the provisions
of this subsection.

\end{env-2aad614f-6fd6-4025-876c-fcdbeae766fb}

These rules enable the landlord and tenant to agree in advance (i.e., in
the covenant against assigning) specified circumstances in which the
landlord may withhold his consent to an assignment and specified
conditions subject to which consent to assignment may be given. If the
landlord withholds consent because any of those specified circumstances
exist, or imposes any of those specified conditions on his consent, he
will not be taken to be acting unreasonably.

\hypertarget{factual-vs-discretionary}{%
\paragraph{Factual Vs Discretionary}\label{factual-vs-discretionary}}

The provisions permitted under s 19(1A) may be either of a factual
nature (eg, whether the\\
assignee is a company quoted on the London Stock Exchange), or
discretionary (eg, whether\\
in the landlord's opinion the assignee is capable of performing the
tenant covenants of the\\
lease). Where the provision involves an exercise of discretion, s 19(1A)
requires either:

\begin{enumerate}
\tightlist
\item
  that the provision states that discretion is to be exercised
  reasonably; or
\item
  that the tenant is given an unrestricted right to have the exercise of
  the discretion reviewed by an independent third party whose identity
  is ascertainable from the provision.
\end{enumerate}

\hypertarget{sub-lettings}{%
\subsubsection{Sub-lettings}\label{sub-lettings}}

The landlord will wish to control sub-letting as well as assignment.
Sub-lettings of the whole of the premises are usually subject to the
same kinds of restrictions as assignments. However, in the case of
sub-lettings of part, the lease provisions are usually considerably
stricter. Sub-letting of part is often subject to an absolute
prohibition.

Section 19(1)(a) of the Landlord and Tenant Act 1927 and the Landlord
and Tenant Act 1988 apply to covenants against assignment, sub-letting
or parting with possession. However, s 19(1A) of the Landlord and Tenant
Act 1927 applies only to assignments.

A similar result to s 19(1A) can nevertheless be achieved by careful
drafting of the alienation provisions, by imposing conditions precedent
to the landlord giving consent. Thus, for example, a requirement that
the tenant first offer to surrender the lease without any consideration
before assignment was held not to contravene s 19(1)(a) (Bocardo SA v S
\& M Hotels {[}1980{]} 1 WLR 17).

\hypertarget{charging-or-mortgaging}{%
\subsubsection{Charging or Mortgaging}\label{charging-or-mortgaging}}

Most leases will contain restrictions on charging -- the landlord
doesn't want to deal with a lender taking possession/ exercising power
of sale.

\hypertarget{forfeiture-clause}{%
\subsection{Forfeiture Clause}\label{forfeiture-clause}}

\begin{itemize}
\tightlist
\item
  Without an express forfeiture clause, the landlord will be unable to
  remove a tenant until the end of a fixed term, even though the tenant
  was not complying with the covenants in the lease.
\item
  The clause should give the landlord the right to forfeit if the rent
  is a specified number of days in arrears and for breach of any other
  covenant.
\item
  A landlord will not be able to forfeit for covenants other than
  non-payment unless a s 146 LPA 1925 notice has previously been served
  on the tenant.
\item
  A court order is required if possession cannot be acquired peaceably
  and is always necessary for residential property.
\item
  Commonhold and Leasehold Reform Act 2002

  \begin{itemize}
  \tightlist
  \item
    No forfeiture is possible in the case of non-payment of sums
    {\(\leq \pounds 350\)}, unless the sum has been outstanding for
    {\(> 3\)} years.
  \item
    For other breaches, landlord cannot forfeit unless T has admitted
    the breach/ Valuation Tribunal has determined there is a breach of
    covenant.
  \end{itemize}
\item
  Commercial leases commonly stipulate that L can forfeit if T becomes
  insolvent.
\end{itemize}

\hypertarget{lta-1954-part-ii}{%
\section{LTA 1954 (Part II)}\label{lta-1954-part-ii}}

``Security of tenure'' refers to statutory rights granted to tenants to
protect their interest under leases. This may include the right to renew
or extend an existing tenancy.

\hypertarget{application-of-lta-1954-part-ii}{%
\subsection{Application of LTA 1954 (Part
II)}\label{application-of-lta-1954-part-ii}}

\includegraphics{C:/Users/shiva/Filen/MEGA/LegalPracticeCourse/LTA 1954_1.jpg}

Under the Landlord and Tenant Act 1927, the lettings market was weighted
heavily in favour of the landlord, meaning that only tenants who could
demonstrate sufficient goodwill, such that compensation at the end of
their tenancy would be inadequate, were entitled to renew their tenancy.

The 1954 Act was introduced to strike a fairer balance between the
landlord and tenant in business lettings. Now, if a business tenant
wants to remain in occupation of its rented premises at the end of its
lease term it can, if the lease qualifies for protection and the tenant
complies with specified procedures. Landlords are only able to end
commercial leases if they can prove that one or more of the grounds
specified by the 1954 Act exist.

\begin{env-2aad614f-6fd6-4025-876c-fcdbeae766fb}

\href{https://www.legislation.gov.uk/ukpga/Eliz2/2-3/56/section/23}{s
23(1) LTA 1954}

Subject to the provisions of this Act, this Part of this Act applies to
any tenancy where the property comprised in the tenancy is or includes
premises which are occupied by the tenant and are so occupied for the
purposes of a business carried on by him or for those and other
purposes.

\end{env-2aad614f-6fd6-4025-876c-fcdbeae766fb}

The key elements to be satisfied are: tenancy, occupation and business
purposes.

\hypertarget{tenancy}{%
\subsubsection{Tenancy}\label{tenancy}}

This must fulfil the requirements established by Street v Mountford
{[}1985{]} AC 809 (exclusive possession and for a term absolute).
`Tenancy' includes periodic as well as fixed term tenancies and oral as
well as written tenancies. `Tenancy' excludes licences and tenancies at
will (which are terminable at any time by either party). Some tenancies
are also specifically excluded from the protection of the 1954 Act by s
43 LTA 1954.

Excluded tenancies under
\href{https://www.legislation.gov.uk/ukpga/Eliz2/2-3/56/section/43}{s 43
LTA 1954}:

\begin{itemize}
\tightlist
\item
  Tenancies of agricultural holdings (they have their own statutory
  regime of protection)
\item
  Mining leases
\item
  Service tenancies (a lease granted as part of a tenant's employment
  e.g. a security guard's flat)
\item
  Fixed term tenancies not exceeding six months (although protection can
  arise if the same business or tenant has been in occupation of the
  premises for 12 or more months through successive tenancies or if the
  tenancy is renewable beyond six months).

  \begin{itemize}
  \tightlist
  \item
    So if the tenant has occupied a property for 12 months and a new
    tenancy is granted, the new tenancy is protected (but not a tenancy
    granted within the 12 months).
  \end{itemize}
\item
  Tenancies at will (Wheeler v Mercer {[}1957{]} AC 416)
\end{itemize}

\hypertarget{occupation}{%
\subsubsection{Occupation}\label{occupation}}

The tenant must be the occupier of at least part of the premises to
obtain protection (the tenant will only be entitled to apply for a new
lease of the part of the premises that it occupies). As a result, if a
tenant underlets all the premises, it will lose the protection of the
1954 Act. When deciding if a tenant is in occupation for the purposes of
the 1954 Act, the courts will consider the measure of control the tenant
exercises over anyone else using the premises.

If there is a sub-tenant in occupation of the whole of the premises
originally let to the head-tenant, the head-tenant will not benefit from
security of tenure (but the sub-tenant may). If only part of the
premises is sublet, the head-tenant will be protected in relation to the
part he still occupies for business purposes, and the sub-tenant will be
protected in respect of the part occupied.

If the tenant named in the lease has a controlling interest in the
business carried out at the premises, this will be sufficient to satisfy
"occupation" for business purposes.

\hypertarget{business}{%
\subsubsection{Business}\label{business}}

This is widely defined by the 1954 Act as `trade, profession or
employment'. Case law has refined this, e.g., a members' tennis club and
a charity are businesses for the purposes of the 1954 Act but not a
Sunday School. Incidental residential use is acceptable, so long as
operating a business is a significant purpose of the occupation (so a
shop with a flat above it is included).

A tenancy does not attract security of tenure where the business use is
in breach of a term of the tenancy. s 23 LTA 1954:even if the landlord
has agreed to the breach, the tenant still does not obtain security of
tenure where the business use is solely for the purposes of a ``home
business''.

\hypertarget{protection}{%
\subsubsection{Protection}\label{protection}}

If the 1954 Act applies,
\href{https://www.legislation.gov.uk/ukpga/Eliz2/2-3/56/section/24}{s
24(1) LTA 1954} sets out two layers of protection given to the protected
tenant:

\begin{itemize}
\tightlist
\item
  The tenancy will continue after its contractual expiry date (the date,
  set out in the lease, on which the term of the lease is due to end)
  until terminated in one of the ways specified by the 1954 Act; and
\item
  The tenant will have the right to apply for a new tenancy on
  termination of the current tenancy.
\end{itemize}

\hypertarget{contracted-out-tenancies}{%
\subsubsection{'Contracted Out'
Tenancies}\label{contracted-out-tenancies}}

Under
\href{https://www.legislation.gov.uk/ukpga/Eliz2/2-3/56/section/38A}{s
38A LTA 1954}, before completing the lease, the landlord and tenant can
agree to exclude the lease from the protection of the 1954 Act.

\begin{env-1183acc6-7dc8-44d4-a985-6dec306e67fd}

Warning

Contracting out is not possible in respect of a periodic tenancy.

\end{env-1183acc6-7dc8-44d4-a985-6dec306e67fd}

\begin{itemize}
\tightlist
\item
  This is usual for short term lettings, where the landlord wants to be
  certain it can regain the premises at the end of the lease term. The
  decision as to whether a commercial lease is contracted out from the
  protection of the 1954 Act will be a part of the normal negotiation
  process prior to the grant of a lease. A protected lease could
  potentially command a higher rent than a contracted out lease.
\item
  Prior to June 2004, the contracting out procedure involved obtaining a
  court order. The lease also had to refer to the court order. From June
  2004, the procedure for contracting out changed. It no longer involves
  a court order, but the lease does still have to make reference to the
  fact that the parties have agreed to exclude the protection of the
  1954 Act (known as security of tenure).
\end{itemize}

The simplest way to check whether a lease has been contracted out of the
protection of the 1954 Act is to check the lease for the reference to
the court order or to the parties agreeing to contract out.

\hypertarget{contracting-out-procedure}{%
\subsubsection{Contracting Out
Procedure}\label{contracting-out-procedure}}

The agreement to contract out must be carried out in accordance with the
procedure set out in the Regulatory Reform (Business Tenancies) (England
and Wales) Order 2003 (the Reform Order). The procedure must be
completed before the lease is granted, or, if earlier, before the tenant
becomes contractually bound to take the lease.

\hypertarget{warning-notice}{%
\paragraph{Warning Notice}\label{warning-notice}}

The landlord must serve a warning notice on the tenant at least 14 days
before the tenant becomes bound to enter the lease (i.e. before the
tenant completes the lease or an agreement for lease). There is a
prescribed form of warning notice which must be used.

\hypertarget{tenants-declaration}{%
\paragraph{Tenant's Declaration}\label{tenants-declaration}}

The tenant must sign either a \textbf{simple declaration} or a
\textbf{statutory declaration} (both have prescribed forms). Which one
is needed depends on whether the 14 days of the landlord's warning
notice can be complied with or not. If there are 14 days between the
warning notice and the tenant becoming bound to enter the lease, a
simple declaration can be used by the tenant. If time is short, the
14-day period of the warning notice can be waived and the tenant must
sign a statutory declaration instead.

\begin{longtable}[]{@{}ll@{}}
\toprule()
Declaration & Details \\
\midrule()
\endhead
Simple declaration & In a prescribed form, states that the tenant has
received and accepted the consequences of the landlord's warning
notice. \\
Statutory declaration & In a prescribed form and made in front of an
independent solicitor, is to the effect that the warning notice has been
received and the tenant accepts the consequences. \\
\bottomrule()
\end{longtable}

\begin{env-1183acc6-7dc8-44d4-a985-6dec306e67fd}

Warning

The warning notice is served by the landlord on the tenant and the
tenant makes a declaration.

\end{env-1183acc6-7dc8-44d4-a985-6dec306e67fd}

\begin{env-8fa37bb4-768d-4893-9e93-b3dc64db7fb6}

Suggested wording

\begin{enumerate}
\tightlist
\item
  The Tenant confirms that before {[}the date of this Lease{]} {[}it
  became contractually bound to enter into the tenancy created by this
  Lease{]}:

  \begin{enumerate}
  \tightlist
  \item
    the Landlord served on the Tenant a notice dated {[} ~ ~ ~{]} in
    relation to the tenancy ~ ~ ~ created by this Lease ("the Notice")
    in a form complying with the requirements of Schedule 1 to the
    Regulatory Reform (Business Tenancies) (England and Wales Order 2003
    ("the Order")
  \item
    the Tenant or a person duly authorised by the Tenant in relation to
    the Notice made {[}a declaration{]} {[}a statutory declaration{]}
    ("the Declaration") dated {[} {]} in a form complying with the
    requirements of Schedule 2 of the Order
  \end{enumerate}
\item
  The Tenant further confirms that where the Declaration was made by a
  person other than the Tenant the declaration was duly authorised by
  the Tenant to make the Declaration on the Tenant's behalf.
\item
  {[}The Landlord and Tenant agree that there is no agreement for lease
  to which this Lease gives effect{]}
\item
  The Landlord and Tenant agree to exclude the provisions of sections 24
  to 28 (inclusive) of the 1954 Act in relation to the tenancy created
  by this Lease.
\end{enumerate}

\end{env-8fa37bb4-768d-4893-9e93-b3dc64db7fb6}

\hypertarget{lease-wording}{%
\paragraph{Lease Wording}\label{lease-wording}}

Whether the simple or statutory declaration procedure is used,
\textbf{the lease must contain}:

\begin{enumerate}
\tightlist
\item
  Wording that the parties have agreed to exclude security of tenure;
  and
\item
  Reference to both the warning notice and the tenant's declaration.
\end{enumerate}

The contracting out procedure is carried out by the landlord and tenant
of the proposed lease or underlease. If an underlease is to be
contracted out, it is the immediate landlord (tenant under the
head-lease) who serves the warning notice on their prospective tenant
(the undertenant) and not the superior landlord (owner of the freehold).

MERMAID2

LTA1954-protection.png

\hypertarget{termination-1}{%
\subsection{Termination}\label{termination-1}}

The Act provides for only seven methods of termination:

\begin{enumerate}
\tightlist
\item
  Section 25 notice by Landlord to terminate the tenancy (initiated by
  Landlord)
\item
  Section 26 request by Tenant for a new tenancy (initiated by Tenant)
\item
  Forfeiture (initiated by Landlord)
\item
  Surrender (initiated by Landlord and Tenant)
\item
  Notice to quit a periodic tenancy (initiated by Tenant)

  \begin{enumerate}
  \tightlist
  \item
    In the case of a periodic tenancy, the Act allows the tenant to
    serve a notice to quit which follows the common law method. However,
    you'll see this doesn't work both ways. The landlord can't serve
    notice to quit on the tenant.
  \end{enumerate}
\item
  Section 27 notice by tenant to end a fixed term tenancy (initiated by
  Tenant)

  \begin{enumerate}
  \tightlist
  \item
    Similarly, section 27 allows the tenant to serve a notice on the
    landlord to end a fixed term tenancy but doesn't afford the landlord
    the same right.
  \end{enumerate}
\item
  Section 27(1A) by Tenant ceasing to be in occupation for business
  purposes at end of the lease (initiated by Tenant)

  \begin{enumerate}
  \tightlist
  \item
    27 (1A) enables the tenancy to end by effluxion of time where the
    tenant ceases to be in occupation for business purposes at the end
    of the lease.
  \end{enumerate}
\end{enumerate}

\begin{env-fdd3b853-72b2-4779-a214-b5da0ccdb5e7}

If you are the tenant in a protected tenancy and do not wish to renew
once the term expires in 4 months, what should you do?

Serve notice in writing to the landlord at least 3 months before the end
of the tenancy. Or serve notice now to terminate in exactly 3 months.

\end{env-fdd3b853-72b2-4779-a214-b5da0ccdb5e7}

\begin{env-fdd3b853-72b2-4779-a214-b5da0ccdb5e7}

If you are the tenant in a protected tenancy and do not wish to renew
once the term expires tomorrow, what should you do?

Move out before the tenancy ends. Then the lease will come to the end as
you are not in occupation. This action was approved in Esselte AB v
Pearl Assurance plc (1997).

\end{env-fdd3b853-72b2-4779-a214-b5da0ccdb5e7}

\hypertarget{s-25-notice}{%
\subsection{S 25 Notice}\label{s-25-notice}}

Landlord starts the process of terminating the lease by serving a s 25
LTA 1954 notice on the tenant.

\begin{env-2aad614f-6fd6-4025-876c-fcdbeae766fb}

s 25 LTA 1954 - Termination of tenancy by the landlord.

(1) The landlord may terminate a tenancy to which this Part of this Act
applies by a notice given to the tenant in the prescribed form
specifying the date at which the tenancy is to come to an end
(hereinafter referred to as `` the date of termination '')\ldots{}

(2) Subject to the provisions of the next following subsection, a notice
under this section shall not have effect unless it is given not more
than twelve nor less than six months before the date of the termination
specified therein.

(3) In the case of a tenancy which apart from this Act could have been
brought to an end by notice to quit given by the landlord---

\begin{itemize}
\tightlist
\item
  (a) the date of termination specified in a notice under this section
  shall not be earlier than the earliest date on which apart from this
  Part of this Act the tenancy could have been brought to an end by
  notice to quit given by the landlord on the date of the giving of the
  notice under this section; and
\item
  (b) where apart from this Part of this Act more than six months'
  notice to quit would have been required to bring the tenancy to an
  end, the last foregoing subsection shall have effect with the
  substitution for twelve months of a period six months longer than the
  length of notice to quit which would have been required as aforesaid.
\end{itemize}

(4) In the case of any other tenancy, a notice under this section shall
not specify a date of termination earlier than the date on which apart
from this Part of this Act the tenancy would have come to an end by
effuxion of time.

(6) A notice under this section shall not have effect unless it states
whether the landlord is opposed to the grant of a new tenancy to the
tenant.

(7) A notice under this section which states that the landlord is
opposed to the grant of a new tenancy to the tenant shall not have
effect unless it also specifies one or more of the grounds specified in
section 30(1) of this Act as the ground or grounds for his opposition.

(8) A notice under this section which states that the landlord is not
opposed to the grant of a new tenancy to the tenant shall not have
effect unless it sets out the landlord's proposals as to---

\begin{itemize}
\tightlist
\item
  (a) the property to be comprised in the new tenancy (being either the
  whole or part of the property comprised in the current tenancy);
\item
  (b) the rent to be payable under the new tenancy; and
\item
  (c) the other terms of the new tenancy.
\end{itemize}

\end{env-2aad614f-6fd6-4025-876c-fcdbeae766fb}

\hypertarget{prescribed-form}{%
\subsubsection{Prescribed Form}\label{prescribed-form}}

\begin{itemize}
\tightlist
\item
  Notice must be in the prescribed form
\item
  Must be given {\(6 \leq T \leq 12\)} months prior to the date of
  termination specified in it.
\item
  The date of termination specified cannot be before the contractual
  termination date of the lease.
\item
  Notice must state whether the landlord will oppose an application by
  the tenant to the court for a new tenancy, and if so on what grounds.
\item
  Landlord can oppose the application on one of 7 grounds set out in s
  30 LTA 1954.
\item
  Often the landlord will be happy to grant a new lease -- on more
  favourable terms.
\item
  There are two versions of the prescribed form: Form 1 to be used when
  the landlord won't oppose a new lease being granted to the tenant and
  Form 2 for use when the landlord wants vacant possession and will
  oppose a new lease.
\end{itemize}

\hypertarget{application-to-court}{%
\subsubsection{Application to Court}\label{application-to-court}}

\begin{itemize}
\tightlist
\item
  Parties may enter into negotiations for a new lease.
\item
  But unless T applies to the court before the expiry of the s 25
  notice, will lose rights under the Act.
\item
  Parties can agree to extend the time limit between them.
\item
  L can apply to court for a new lease to be granted.
\end{itemize}

\begin{env-1183acc6-7dc8-44d4-a985-6dec306e67fd}

Warning

If a tenant wishes to preserve its security of tenure rights under LTA54
following the service of a~valid s. 25 notice by the landlord, it must
ensure that it applies to court for a new lease before the termination
date specified in the s. 25 notice (or any extension agreed in writing
before the end of that period).~ If~it fails to do this, the tenant will
lose its~protection under LTA54 and will be reliant on the landlord's
willingness to grant a lease to~the tenant in the~absence of this
protection. This is the case even if the landlord has~stated that~it
will not oppose~the grant of a new lease to the tenant in~the s. 25
notice served by it.

\end{env-1183acc6-7dc8-44d4-a985-6dec306e67fd}

s 25 LTA 1954.png

\hypertarget{s-26-request}{%
\subsection{S 26 Request}\label{s-26-request}}

\begin{env-2aad614f-6fd6-4025-876c-fcdbeae766fb}

s 26 LTA 1954 - Tenant's request for a new tenancy.

(1) A tenant's request for a new tenancy may be made where the current
tenancy is a tenancy granted for a term of years certain exceeding one
year, whether or not continued by section twenty-four of this Act, or
granted for a term of years certain and thereafter from year to year.

(2) A tenant's request for a new tenancy shall be for a tenancy
beginning with such date, not more than twelve nor less than six months
after the making of the request, as may be specified therein: Provided
that the said date shall not be earlier than the date on which apart
from this Act the current tenancy would come to an end by effluxion of
time or could be brought to an end by notice to quit given by the
tenant.

(3) A tenant's request for a new tenancy shall not have effect unless it
is made by notice in the prescribed form given to the landlord and sets
out the tenant's proposals as to the property to be comprised in the new
tenancy (being either the whole or part of the property comprised in the
current tenancy), as to the rent to be payable under the new tenancy and
as to the other terms of the new tenancy.

(4) A tenant's request for a new tenancy shall not be made if the
landlord has already given notice under the last foregoing section to
terminate the current tenancy, or if the tenant has already given notice
to quit or notice under the next following section; and no such notice
shall be given by the landlord or the tenant after the making by the
tenant of a request for a new tenancy.

(5) Where the tenant makes a request for a new tenancy in accordance
with the foregoing provisions of this section, the current tenancy
shall\ldots{} terminate immediately before the date specified in the
request for the beginning of the new tenancy.

(6) Within two months of the making of a tenant's request for a new
tenancy the landlord may give notice to the tenant that he will oppose
an application to the court for the grant of a new tenancy, and any such
notice shall state on which of the grounds mentioned in section thirty
of this Act the landlord will oppose the application.

\end{env-2aad614f-6fd6-4025-876c-fcdbeae766fb}

\begin{itemize}
\tightlist
\item
  The tenant can take the initiative and serve a request for a new
  tenancy under s 26.
\item
  The procedure is not available for periodic tenants or those for a
  fixed term {\(\leq 1\)} year.
\item
  Usually not advisable {\(\rightarrow\)} the rent will probably go up.

  \begin{itemize}
  \tightlist
  \item
    But may be if rent is currently payable is too high; or
  \item
    Lease is to be sold and may be more marketable with a new fixed
    term.
  \end{itemize}
\end{itemize}

\begin{env-ab94a4ec-2f9e-4d2f-94d1-74521d1a190e}

Reasons for serving s26 notice

\begin{enumerate}
\tightlist
\item
  Rents falling

  \begin{itemize}
  \tightlist
  \item
    If the tenant believes that rents have fallen and the current rent
    is more than what could be achieved under a new lease in the current
    market, then the tenant will want to terminate the current tenancy
    and obtain a new one at a reduced rent.
  \end{itemize}
\item
  Rents increased

  \begin{itemize}
  \tightlist
  \item
    Conversely, the tenant may be making a pre-emptive move to prolong
    the current tenancy. If rents have increased, the landlord will want
    to bring the current tenancy to an end and can only do that by
    service of a section 25 notice giving at least six months' notice.
  \item
    If however the tenant can step in first and serve their section 26
    request for a new tenancy the tenant could give up to 12 months'
    notice and prolong their current tenancy -- at the same rent -- for
    another six months.
  \end{itemize}
\item
  Tenants want to sell leasehold with new fixed term

  \begin{itemize}
  \tightlist
  \item
    The tenant may have plans to sell the lease and the tenancy would be
    more attractive to a buyer if a new fixed term has been granted.
  \end{itemize}
\end{enumerate}

\end{env-ab94a4ec-2f9e-4d2f-94d1-74521d1a190e}

\hypertarget{prescribed-form-1}{%
\subsubsection{Prescribed Form}\label{prescribed-form-1}}

\begin{itemize}
\tightlist
\item
  s 26 notice must be in prescribed form
\item
  Must be given {\(6 \leq T \leq 12\)} months prior to the start date
  specified in it.
\item
  Must state proposed terms of new tenancy.
\end{itemize}

\hypertarget{application-to-court-1}{%
\subsubsection{Application to Court}\label{application-to-court-1}}

\begin{itemize}
\tightlist
\item
  To oppose, the landlord must serve a counter-notice on the tenant
  within 2 months of the service of T's s 26 request.
\item
  Must state s 30 grounds of opposition.
\item
  If L does this, T must apply to court for a new lease, or rights under
  the Act will be lost.

  \begin{itemize}
  \tightlist
  \item
    Application must be made prior to the commencement date of the new
    tenancy specified in s 26 request.
  \end{itemize}
\item
  Parties may choose to negotiate and/or extend the time limit under the
  request.
\end{itemize}

s26-LTA-54.png

s25-26-LTA-timings.bmp

\hypertarget{s-30-landlords-grounds}{%
\subsection{S 30 Landlord's Grounds}\label{s-30-landlords-grounds}}

7 grounds for opposing the grant of a new lease:

\begin{itemize}
\tightlist
\item
  (a) T's failure to repair
\item
  (b) Persistent delay in paying rent
\item
  (c) Substantial breaches of other obligations
\item
  (d) Alternative accommodation
\item
  (e) Sub-letting of part where higher rent can be obtained by a single
  letting of the whole building.
\item
  (f) L intends to demolish/ reconstruct and could not reasonably do so
  without obtaining possession.

  \begin{itemize}
  \tightlist
  \item
    Must be a firm and settled intention to carry out the relevant work.
  \item
    Demolition/ reconstruction of the premises/ a substantial part of
    them substantial construction.
  \item
    Cannot reasonably\ldots{} is a question of fact.
  \item
    As long as a landlord has a genuine intention, if it subsequently
    and honestly changes its mind after having obtained possession, the
    ground will be made out (Fisher v Taylor's Fur nishings {[}1956{]} 2
    QB 78).
  \end{itemize}
\item
  (g) L's intention to occupy the holding for his own business/ as a
  residence.

  \begin{itemize}
  \tightlist
  \item
    Firm and settled intention.
  \item
    L must have a reasonable prospect of achieving intention.
  \item
    L \textbf{cannot} rely on this ground if his interest was purchased
    or created within 5 years before the end of the current tenancy.

    \begin{itemize}
    \tightlist
    \item
      Logic: to prevent L buying the reversion cheaply within 5 years of
      the end of the lease and acquiring vacant possession using this
      ground.
    \end{itemize}
  \item
    L \textbf{can} rely on the ground if she bought the property with
    vacant possession, let it, and then sought to possess within 5 years
    of buying.
  \item
    May be used by a competitor to drive out a business rival (Humber
    Oil Terminals Trustee Ltd v Associated British Ports {[}2011{]} EWHC
    2043 (Ch)).)
  \end{itemize}
\end{itemize}

\begin{env-4361be9f-9a08-465c-86d1-6b024e2d00e2}

Important

(1) to (3) are discretionary grounds - L must:

\begin{enumerate}
\tightlist
\item
  Establish the ground; and
\item
  Show T ought not to be granted a new tenancy in view of the facts
  giving rise to the ground.
\end{enumerate}

\end{env-4361be9f-9a08-465c-86d1-6b024e2d00e2}

\begin{env-2aad614f-6fd6-4025-876c-fcdbeae766fb}

s 30(1) LTA 1954 Opposition by landlord to application for new tenancy.

The grounds on which a landlord may oppose an application under section
24(1) of this Act, or make an application under section 29(2) of this
Act, are such of the following grounds as may be stated in the
landlord's notice under section 25 of this Act or, as the case may be,
under subsection (6) of section 26 thereof, that is to say:---

\begin{itemize}
\tightlist
\item
  (a) where under the current tenancy the tenant has any obligations as
  respects the repair and maintenance of the holding, that the tenant
  ought not to be granted a new tenancy in view of the state of repair
  of the holding, being a state resulting from the tenant's failure to
  comply with the said obligations;
\item
  (b) that the tenant ought not to be granted a new tenancy in view of
  his persistent delay in paying rent which has become due;
\item
  (c) that the tenant ought not to be granted a new tenancy in view of
  other substantial breaches by him of his obligations under the current
  tenancy, or for any other reason connected with the tenant's use or
  management of the holding;
\item
  (d) that the landlord has offered and is willing to provide or secure
  the provision of alternative accommodation for the tenant, that the
  terms on which the alternative accommodation is available are
  reasonable having regard to the terms of the current tenancy and to
  all other relevant circumstances, and that the accommodation and the
  time at which it will be available are suitable for the tenant's
  requirements (including the requirement to preserve goodwill) having
  regard to the nature and class of his business and to the situation
  and extent of, and facilities afforded by, the holding;
\item
  (e) where the current tenancy was created by the sub-letting of part
  only of the property comprised in a superior tenancy and the landlord
  is the owner of an interest in reversion expectant on the termination
  of that superior tenancy, that the aggregate of the rents reasonably
  obtainable on separate lettings of the holding and the remainder of
  that property would be substantially less than the rent reasonably
  obtainable on a letting of that property as a whole, that on the
  termination of the current tenancy the landlord requires possession of
  the holding for the purpose of letting or otherwise disposing of the
  said property as a whole, and that in view thereof the tenant ought
  not to be granted a new tenancy;
\item
  (f) that on the termination of the current tenancy the landlord
  intends to demolish or reconstruct the premises comprised in the
  holding or a substantial part of those premises or to carry out
  substantial work of construction on the holding or part thereof and
  that he could not reasonably do so without obtaining possession of the
  holding;
\item
  (g) subject as hereinafter provided, that on the termination of the
  current tenancy the landlord intends to occupy the holding for the
  purposes, or partly for the purposes, of a business to be carried on
  by him therein, or as his residence.
\end{itemize}

\end{env-2aad614f-6fd6-4025-876c-fcdbeae766fb}

\hypertarget{interim-rent}{%
\subsection{Interim Rent}\label{interim-rent}}

\begin{itemize}
\tightlist
\item
  Where T has applied to the court for a new tenancy, the current
  tenancy will not end on the expiry of the s 25 notice/ s 26 request.
\item
  Will continue at the same rent until \textbf{3 months} after the
  conclusion of the proceedings (s 64 LTA 1954).
\item
  L/ T can apply for an interim rent to apply pending the outcome of
  proceedings.

  \begin{itemize}
  \tightlist
  \item
    Typically, the rent payable under the new lease.
  \item
    But in certain cases, interim rent set at the market rent for a
    yearly tenancy of the premises, having regard to the rent under the
    old lease.
  \end{itemize}
\end{itemize}

\hypertarget{terms-of-new-lease}{%
\subsection{Terms of New Lease}\label{terms-of-new-lease}}

\hypertarget{premises}{%
\subsubsection{Premises}\label{premises}}

\begin{itemize}
\tightlist
\item
  Tenant only entitled to a tenancy of the ``holding'', meaning the
  property comprised of the current tenancy but excluding any part not
  occupied by the tenant (e.g., which has been sublet).
\item
  L has the right to insist that any new tenancy will be a tenancy of
  the whole.
\end{itemize}

\hypertarget{duration}{%
\subsubsection{Duration}\label{duration}}

\begin{itemize}
\tightlist
\item
  Such as reasonable in the circumstances
\item
  Cannot exceed 15 years.
\end{itemize}

\hypertarget{rent}{%
\subsubsection{Rent}\label{rent}}

\begin{itemize}
\tightlist
\item
  Open market rent having regard to the other tenancy terms.
\item
  Court must disregard;

  \begin{itemize}
  \tightlist
  \item
    Fact that T has been in occupation
  \item
    Any goodwill attached to the holding
  \item
    Effect on rent of voluntary improvements carried out by the tenant.
  \item
    Addition in value due to tenant's licence, for a licenced premises.
  \end{itemize}
\item
  Court can insert a rent review clause, even if there wasn't one
  before.
\end{itemize}

\hypertarget{other-terms}{%
\subsubsection{Other Terms}\label{other-terms}}

\begin{itemize}
\tightlist
\item
  If there is no agreement, these will be fixed by the court.
\item
  If the party wishes to change the current terms:

  \begin{itemize}
  \tightlist
  \item
    The onus is on them to justify that change (City of London Real
    Property Co Ltd v O'May {[}1983{]} 2 AC 726)
  \item
    Change must be fair and reasonable.
  \end{itemize}
\end{itemize}

\hypertarget{commencement-1}{%
\subsubsection{Commencement}\label{commencement-1}}

Any new lease will not commence until 3 months after the proceedings are
``finally disposed of'' {\(\;\Longrightarrow\;\)} when the time limit
for appeal has elapsed (4 weeks after the order). So (3 months + 4
weeks) total.

If T finds the terms of a new lease unacceptable, may apply for the
order to be revoked.

\hypertarget{compensation-for-failure}{%
\subsection{Compensation for Failure}\label{compensation-for-failure}}

\hypertarget{availability}{%
\subsubsection{Availability}\label{availability}}

\begin{itemize}
\tightlist
\item
  If you cannot get a new tenancy solely because one or more of grounds
  (e), (f) and (g) applies, you may be entitled to compensation under
  section 37.
\item
  If your landlord has opposed your application on any of the other
  grounds as well as (e), (f) or (g) you can only get compensation if
  the court's refusal to grant a new tenancy is based solely on one or
  more of grounds (e), (f) and (g).
\item
  In other words, you cannot get compensation under section 37 if the
  court has refused your tenancy on other grounds, even if one or more
  of grounds (e), (f) and (g) also applies.
\end{itemize}

\hypertarget{amount-of-compensation}{%
\subsubsection{Amount of Compensation}\label{amount-of-compensation}}

MERMAID1

s37-LTA-compensation.bmp

\hypertarget{contracting-out-compensation}{%
\subsubsection{Contracting Out
Compensation}\label{contracting-out-compensation}}

s 38(2) LTA 1954 provides that any clause in a lease purporting to
exclude a tenant's right to compensation will be void if the tenant (or
their predecessors in the same business) have been in occupation for
{\(\geq 5\)} years prior to the date of termination.

s30-LTA-grounds.bmp

\end{document}
