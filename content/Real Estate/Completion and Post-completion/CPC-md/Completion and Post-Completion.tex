% Options for packages loaded elsewhere
\PassOptionsToPackage{unicode}{hyperref}
\PassOptionsToPackage{hyphens}{url}
%
\documentclass[
]{article}
\usepackage{amsmath,amssymb}
\usepackage{lmodern}
\usepackage{iftex}
\ifPDFTeX
  \usepackage[T1]{fontenc}
  \usepackage[utf8]{inputenc}
  \usepackage{textcomp} % provide euro and other symbols
\else % if luatex or xetex
  \usepackage{unicode-math}
  \defaultfontfeatures{Scale=MatchLowercase}
  \defaultfontfeatures[\rmfamily]{Ligatures=TeX,Scale=1}
\fi
% Use upquote if available, for straight quotes in verbatim environments
\IfFileExists{upquote.sty}{\usepackage{upquote}}{}
\IfFileExists{microtype.sty}{% use microtype if available
  \usepackage[]{microtype}
  \UseMicrotypeSet[protrusion]{basicmath} % disable protrusion for tt fonts
}{}
\makeatletter
\@ifundefined{KOMAClassName}{% if non-KOMA class
  \IfFileExists{parskip.sty}{%
    \usepackage{parskip}
  }{% else
    \setlength{\parindent}{0pt}
    \setlength{\parskip}{6pt plus 2pt minus 1pt}}
}{% if KOMA class
  \KOMAoptions{parskip=half}}
\makeatother
\usepackage{xcolor}
\usepackage[margin=1in]{geometry}
\usepackage{color}
\usepackage{fancyvrb}
\newcommand{\VerbBar}{|}
\newcommand{\VERB}{\Verb[commandchars=\\\{\}]}
\DefineVerbatimEnvironment{Highlighting}{Verbatim}{commandchars=\\\{\}}
% Add ',fontsize=\small' for more characters per line
\newenvironment{Shaded}{}{}
\newcommand{\AlertTok}[1]{\textcolor[rgb]{1.00,0.00,0.00}{\textbf{#1}}}
\newcommand{\AnnotationTok}[1]{\textcolor[rgb]{0.38,0.63,0.69}{\textbf{\textit{#1}}}}
\newcommand{\AttributeTok}[1]{\textcolor[rgb]{0.49,0.56,0.16}{#1}}
\newcommand{\BaseNTok}[1]{\textcolor[rgb]{0.25,0.63,0.44}{#1}}
\newcommand{\BuiltInTok}[1]{#1}
\newcommand{\CharTok}[1]{\textcolor[rgb]{0.25,0.44,0.63}{#1}}
\newcommand{\CommentTok}[1]{\textcolor[rgb]{0.38,0.63,0.69}{\textit{#1}}}
\newcommand{\CommentVarTok}[1]{\textcolor[rgb]{0.38,0.63,0.69}{\textbf{\textit{#1}}}}
\newcommand{\ConstantTok}[1]{\textcolor[rgb]{0.53,0.00,0.00}{#1}}
\newcommand{\ControlFlowTok}[1]{\textcolor[rgb]{0.00,0.44,0.13}{\textbf{#1}}}
\newcommand{\DataTypeTok}[1]{\textcolor[rgb]{0.56,0.13,0.00}{#1}}
\newcommand{\DecValTok}[1]{\textcolor[rgb]{0.25,0.63,0.44}{#1}}
\newcommand{\DocumentationTok}[1]{\textcolor[rgb]{0.73,0.13,0.13}{\textit{#1}}}
\newcommand{\ErrorTok}[1]{\textcolor[rgb]{1.00,0.00,0.00}{\textbf{#1}}}
\newcommand{\ExtensionTok}[1]{#1}
\newcommand{\FloatTok}[1]{\textcolor[rgb]{0.25,0.63,0.44}{#1}}
\newcommand{\FunctionTok}[1]{\textcolor[rgb]{0.02,0.16,0.49}{#1}}
\newcommand{\ImportTok}[1]{#1}
\newcommand{\InformationTok}[1]{\textcolor[rgb]{0.38,0.63,0.69}{\textbf{\textit{#1}}}}
\newcommand{\KeywordTok}[1]{\textcolor[rgb]{0.00,0.44,0.13}{\textbf{#1}}}
\newcommand{\NormalTok}[1]{#1}
\newcommand{\OperatorTok}[1]{\textcolor[rgb]{0.40,0.40,0.40}{#1}}
\newcommand{\OtherTok}[1]{\textcolor[rgb]{0.00,0.44,0.13}{#1}}
\newcommand{\PreprocessorTok}[1]{\textcolor[rgb]{0.74,0.48,0.00}{#1}}
\newcommand{\RegionMarkerTok}[1]{#1}
\newcommand{\SpecialCharTok}[1]{\textcolor[rgb]{0.25,0.44,0.63}{#1}}
\newcommand{\SpecialStringTok}[1]{\textcolor[rgb]{0.73,0.40,0.53}{#1}}
\newcommand{\StringTok}[1]{\textcolor[rgb]{0.25,0.44,0.63}{#1}}
\newcommand{\VariableTok}[1]{\textcolor[rgb]{0.10,0.09,0.49}{#1}}
\newcommand{\VerbatimStringTok}[1]{\textcolor[rgb]{0.25,0.44,0.63}{#1}}
\newcommand{\WarningTok}[1]{\textcolor[rgb]{0.38,0.63,0.69}{\textbf{\textit{#1}}}}
\usepackage{longtable,booktabs,array}
\usepackage{calc} % for calculating minipage widths
% Correct order of tables after \paragraph or \subparagraph
\usepackage{etoolbox}
\makeatletter
\patchcmd\longtable{\par}{\if@noskipsec\mbox{}\fi\par}{}{}
\makeatother
% Allow footnotes in longtable head/foot
\IfFileExists{footnotehyper.sty}{\usepackage{footnotehyper}}{\usepackage{footnote}}
\makesavenoteenv{longtable}
\setlength{\emergencystretch}{3em} % prevent overfull lines
\providecommand{\tightlist}{%
  \setlength{\itemsep}{0pt}\setlength{\parskip}{0pt}}
\setcounter{secnumdepth}{-\maxdimen} % remove section numbering
\usepackage{xcolor}
\definecolor{aliceblue}{HTML}{F0F8FF}
\definecolor{antiquewhite}{HTML}{FAEBD7}
\definecolor{aqua}{HTML}{00FFFF}
\definecolor{aquamarine}{HTML}{7FFFD4}
\definecolor{azure}{HTML}{F0FFFF}
\definecolor{beige}{HTML}{F5F5DC}
\definecolor{bisque}{HTML}{FFE4C4}
\definecolor{black}{HTML}{000000}
\definecolor{blanchedalmond}{HTML}{FFEBCD}
\definecolor{blue}{HTML}{0000FF}
\definecolor{blueviolet}{HTML}{8A2BE2}
\definecolor{brown}{HTML}{A52A2A}
\definecolor{burlywood}{HTML}{DEB887}
\definecolor{cadetblue}{HTML}{5F9EA0}
\definecolor{chartreuse}{HTML}{7FFF00}
\definecolor{chocolate}{HTML}{D2691E}
\definecolor{coral}{HTML}{FF7F50}
\definecolor{cornflowerblue}{HTML}{6495ED}
\definecolor{cornsilk}{HTML}{FFF8DC}
\definecolor{crimson}{HTML}{DC143C}
\definecolor{cyan}{HTML}{00FFFF}
\definecolor{darkblue}{HTML}{00008B}
\definecolor{darkcyan}{HTML}{008B8B}
\definecolor{darkgoldenrod}{HTML}{B8860B}
\definecolor{darkgray}{HTML}{A9A9A9}
\definecolor{darkgreen}{HTML}{006400}
\definecolor{darkgrey}{HTML}{A9A9A9}
\definecolor{darkkhaki}{HTML}{BDB76B}
\definecolor{darkmagenta}{HTML}{8B008B}
\definecolor{darkolivegreen}{HTML}{556B2F}
\definecolor{darkorange}{HTML}{FF8C00}
\definecolor{darkorchid}{HTML}{9932CC}
\definecolor{darkred}{HTML}{8B0000}
\definecolor{darksalmon}{HTML}{E9967A}
\definecolor{darkseagreen}{HTML}{8FBC8F}
\definecolor{darkslateblue}{HTML}{483D8B}
\definecolor{darkslategray}{HTML}{2F4F4F}
\definecolor{darkslategrey}{HTML}{2F4F4F}
\definecolor{darkturquoise}{HTML}{00CED1}
\definecolor{darkviolet}{HTML}{9400D3}
\definecolor{deeppink}{HTML}{FF1493}
\definecolor{deepskyblue}{HTML}{00BFFF}
\definecolor{dimgray}{HTML}{696969}
\definecolor{dimgrey}{HTML}{696969}
\definecolor{dodgerblue}{HTML}{1E90FF}
\definecolor{firebrick}{HTML}{B22222}
\definecolor{floralwhite}{HTML}{FFFAF0}
\definecolor{forestgreen}{HTML}{228B22}
\definecolor{fuchsia}{HTML}{FF00FF}
\definecolor{gainsboro}{HTML}{DCDCDC}
\definecolor{ghostwhite}{HTML}{F8F8FF}
\definecolor{gold}{HTML}{FFD700}
\definecolor{goldenrod}{HTML}{DAA520}
\definecolor{gray}{HTML}{808080}
\definecolor{green}{HTML}{008000}
\definecolor{greenyellow}{HTML}{ADFF2F}
\definecolor{grey}{HTML}{808080}
\definecolor{honeydew}{HTML}{F0FFF0}
\definecolor{hotpink}{HTML}{FF69B4}
\definecolor{indianred}{HTML}{CD5C5C}
\definecolor{indigo}{HTML}{4B0082}
\definecolor{ivory}{HTML}{FFFFF0}
\definecolor{khaki}{HTML}{F0E68C}
\definecolor{lavender}{HTML}{E6E6FA}
\definecolor{lavenderblush}{HTML}{FFF0F5}
\definecolor{lawngreen}{HTML}{7CFC00}
\definecolor{lemonchiffon}{HTML}{FFFACD}
\definecolor{lightblue}{HTML}{ADD8E6}
\definecolor{lightcoral}{HTML}{F08080}
\definecolor{lightcyan}{HTML}{E0FFFF}
\definecolor{lightgoldenrodyellow}{HTML}{FAFAD2}
\definecolor{lightgray}{HTML}{D3D3D3}
\definecolor{lightgreen}{HTML}{90EE90}
\definecolor{lightgrey}{HTML}{D3D3D3}
\definecolor{lightpink}{HTML}{FFB6C1}
\definecolor{lightsalmon}{HTML}{FFA07A}
\definecolor{lightseagreen}{HTML}{20B2AA}
\definecolor{lightskyblue}{HTML}{87CEFA}
\definecolor{lightslategray}{HTML}{778899}
\definecolor{lightslategrey}{HTML}{778899}
\definecolor{lightsteelblue}{HTML}{B0C4DE}
\definecolor{lightyellow}{HTML}{FFFFE0}
\definecolor{lime}{HTML}{00FF00}
\definecolor{limegreen}{HTML}{32CD32}
\definecolor{linen}{HTML}{FAF0E6}
\definecolor{magenta}{HTML}{FF00FF}
\definecolor{maroon}{HTML}{800000}
\definecolor{mediumaquamarine}{HTML}{66CDAA}
\definecolor{mediumblue}{HTML}{0000CD}
\definecolor{mediumorchid}{HTML}{BA55D3}
\definecolor{mediumpurple}{HTML}{9370DB}
\definecolor{mediumseagreen}{HTML}{3CB371}
\definecolor{mediumslateblue}{HTML}{7B68EE}
\definecolor{mediumspringgreen}{HTML}{00FA9A}
\definecolor{mediumturquoise}{HTML}{48D1CC}
\definecolor{mediumvioletred}{HTML}{C71585}
\definecolor{midnightblue}{HTML}{191970}
\definecolor{mintcream}{HTML}{F5FFFA}
\definecolor{mistyrose}{HTML}{FFE4E1}
\definecolor{moccasin}{HTML}{FFE4B5}
\definecolor{navajowhite}{HTML}{FFDEAD}
\definecolor{navy}{HTML}{000080}
\definecolor{oldlace}{HTML}{FDF5E6}
\definecolor{olive}{HTML}{808000}
\definecolor{olivedrab}{HTML}{6B8E23}
\definecolor{orange}{HTML}{FFA500}
\definecolor{orangered}{HTML}{FF4500}
\definecolor{orchid}{HTML}{DA70D6}
\definecolor{palegoldenrod}{HTML}{EEE8AA}
\definecolor{palegreen}{HTML}{98FB98}
\definecolor{paleturquoise}{HTML}{AFEEEE}
\definecolor{palevioletred}{HTML}{DB7093}
\definecolor{papayawhip}{HTML}{FFEFD5}
\definecolor{peachpuff}{HTML}{FFDAB9}
\definecolor{peru}{HTML}{CD853F}
\definecolor{pink}{HTML}{FFC0CB}
\definecolor{plum}{HTML}{DDA0DD}
\definecolor{powderblue}{HTML}{B0E0E6}
\definecolor{purple}{HTML}{800080}
\definecolor{red}{HTML}{FF0000}
\definecolor{rosybrown}{HTML}{BC8F8F}
\definecolor{royalblue}{HTML}{4169E1}
\definecolor{saddlebrown}{HTML}{8B4513}
\definecolor{salmon}{HTML}{FA8072}
\definecolor{sandybrown}{HTML}{F4A460}
\definecolor{seagreen}{HTML}{2E8B57}
\definecolor{seashell}{HTML}{FFF5EE}
\definecolor{sienna}{HTML}{A0522D}
\definecolor{silver}{HTML}{C0C0C0}
\definecolor{skyblue}{HTML}{87CEEB}
\definecolor{slateblue}{HTML}{6A5ACD}
\definecolor{slategray}{HTML}{708090}
\definecolor{slategrey}{HTML}{708090}
\definecolor{snow}{HTML}{FFFAFA}
\definecolor{springgreen}{HTML}{00FF7F}
\definecolor{steelblue}{HTML}{4682B4}
\definecolor{tan}{HTML}{D2B48C}
\definecolor{teal}{HTML}{008080}
\definecolor{thistle}{HTML}{D8BFD8}
\definecolor{tomato}{HTML}{FF6347}
\definecolor{turquoise}{HTML}{40E0D0}
\definecolor{violet}{HTML}{EE82EE}
\definecolor{wheat}{HTML}{F5DEB3}
\definecolor{white}{HTML}{FFFFFF}
\definecolor{whitesmoke}{HTML}{F5F5F5}
\definecolor{yellow}{HTML}{FFFF00}
\definecolor{yellowgreen}{HTML}{9ACD32}
\usepackage[most]{tcolorbox}

\usepackage{ifthen}
\provideboolean{admonitiontwoside}
\makeatletter%
\if@twoside%
\setboolean{admonitiontwoside}{true}
\else%
\setboolean{admonitiontwoside}{false}
\fi%
\makeatother%

\newenvironment{env-39e5dfc8-f024-4536-bd9d-ea4734521a6a}
{
    \savenotes\tcolorbox[blanker,breakable,left=5pt,borderline west={2pt}{-4pt}{firebrick}]
}
{
    \endtcolorbox\spewnotes
}
                

\newenvironment{env-3993b1ce-acc7-4fb6-8d68-4b3fa6ebf093}
{
    \savenotes\tcolorbox[blanker,breakable,left=5pt,borderline west={2pt}{-4pt}{blue}]
}
{
    \endtcolorbox\spewnotes
}
                

\newenvironment{env-f0ecac1e-6b61-44f6-9c26-f17b8c17158c}
{
    \savenotes\tcolorbox[blanker,breakable,left=5pt,borderline west={2pt}{-4pt}{green}]
}
{
    \endtcolorbox\spewnotes
}
                

\newenvironment{env-2d8756f7-d24d-466f-858f-f6b933c94c5a}
{
    \savenotes\tcolorbox[blanker,breakable,left=5pt,borderline west={2pt}{-4pt}{aquamarine}]
}
{
    \endtcolorbox\spewnotes
}
                

\newenvironment{env-1a7ddd1e-a3d6-4bff-aa6c-99288903289c}
{
    \savenotes\tcolorbox[blanker,breakable,left=5pt,borderline west={2pt}{-4pt}{orange}]
}
{
    \endtcolorbox\spewnotes
}
                

\newenvironment{env-aabff0e3-bb4a-4457-b2fa-75d5f90e2076}
{
    \savenotes\tcolorbox[blanker,breakable,left=5pt,borderline west={2pt}{-4pt}{gold}]
}
{
    \endtcolorbox\spewnotes
}
                

\newenvironment{env-cb9d11d1-d966-4e66-a76b-643fd8254216}
{
    \savenotes\tcolorbox[blanker,breakable,left=5pt,borderline west={2pt}{-4pt}{darkred}]
}
{
    \endtcolorbox\spewnotes
}
                

\newenvironment{env-3e26a22d-f577-4961-99af-29b2c72110db}
{
    \savenotes\tcolorbox[blanker,breakable,left=5pt,borderline west={2pt}{-4pt}{pink}]
}
{
    \endtcolorbox\spewnotes
}
                

\newenvironment{env-d544c8f2-55e1-4371-801b-61a3230b4066}
{
    \savenotes\tcolorbox[blanker,breakable,left=5pt,borderline west={2pt}{-4pt}{cyan}]
}
{
    \endtcolorbox\spewnotes
}
                

\newenvironment{env-8c7cc46b-68c9-4783-90ed-b77280436161}
{
    \savenotes\tcolorbox[blanker,breakable,left=5pt,borderline west={2pt}{-4pt}{cyan}]
}
{
    \endtcolorbox\spewnotes
}
                

\newenvironment{env-4827461a-7e8c-472b-a1d4-2da344959d1d}
{
    \savenotes\tcolorbox[blanker,breakable,left=5pt,borderline west={2pt}{-4pt}{purple}]
}
{
    \endtcolorbox\spewnotes
}
                

\newenvironment{env-5b3ea11a-f9cc-475d-845b-f977033981b5}
{
    \savenotes\tcolorbox[blanker,breakable,left=5pt,borderline west={2pt}{-4pt}{darksalmon}]
}
{
    \endtcolorbox\spewnotes
}
                

\newenvironment{env-4578ed73-f7f6-43c8-b294-1d16097c7d6a}
{
    \savenotes\tcolorbox[blanker,breakable,left=5pt,borderline west={2pt}{-4pt}{gray}]
}
{
    \endtcolorbox\spewnotes
}
                
\ifLuaTeX
  \usepackage{selnolig}  % disable illegal ligatures
\fi
\IfFileExists{bookmark.sty}{\usepackage{bookmark}}{\usepackage{hyperref}}
\IfFileExists{xurl.sty}{\usepackage{xurl}}{} % add URL line breaks if available
\urlstyle{same} % disable monospaced font for URLs
\hypersetup{
  hidelinks,
  pdfcreator={LaTeX via pandoc}}

\author{}
\date{}

\begin{document}

{
\setcounter{tocdepth}{3}
\tableofcontents
}
\begin{Shaded}
\begin{Highlighting}[]
\NormalTok{min\_depth: 1}
\end{Highlighting}
\end{Shaded}

\hypertarget{completion}{%
\section{Completion}\label{completion}}

\hypertarget{effect}{%
\subsection{Effect}\label{effect}}

For registered land, title does not pass to the buyer until the buyer
becomes registered at the Land Registry as proprietor of the land. On
completion, the contract merges with the transfer deed in so far as the
contract and transfer deed cover the same ground.

So generally it is not possible to bring a claim arising out of one of
the terms of the contract unless the provision has been expressly
preserved by a term of the contract. Contract usually contains a
non-merger clause (SCPC 10.4) preserving the right to sue on the
contract even though completion has taken place.

\hypertarget{date-of-completion}{%
\subsection{Date of Completion}\label{date-of-completion}}

Agreed between the parties' solicitors shortly before exchange of
contracts. May have to be synchronised in a chain.

\hypertarget{standard-conditions}{%
\subsubsection{Standard Conditions}\label{standard-conditions}}

SCPC 9.1: in the absence of express agreement, completion takes place on
the 20th working day after exchange. Time is not ``of the essence'' (so
the completion date is just a warranty, not a condition). Breach of the
completion date will activate the compensation provisions in SCPC 10.3.

Very unusual to make time of the essence (e.g., what if there is a
postal delay).

\hypertarget{time-of-completion}{%
\subsection{Time of Completion}\label{time-of-completion}}

In a linked sale and purchase, the sale must be completed before the
purchase (so there are sufficient funds), and there must be a sufficient
interval between the two to allow the money received from the sale to be
transferred to and used in the purchase transaction. Generally,
everything should happen between 10am and 3pm because of restrictions on
banking hours. Completion time later than 2.30pm inadvisable.

\hypertarget{standard-conditions-1}{%
\subsubsection{Standard Conditions}\label{standard-conditions-1}}

SCPC 9.12: if completion does not take place by 2pm on the day of
completion, interest for late completion becomes payable (completion
deemed to have taken place on the next working day). Noncompliance with
this condition is a deemed late completion which invokes the
compensation provisions of SCPC 10.3, requiring payment of compensation
at the contractual interest rate for the delay.

SCPC 9.1.2 does not apply where sale is with vacant possession and the
seller has not vacated the property by 2pm on the day of actual
completion.

\hypertarget{place-of-completion}{%
\subsection{Place of Completion}\label{place-of-completion}}

SCPC 9.2: completion is to take place in England and Wales, either at
the seller's solicitor's office or at some other place which the seller
reasonably specifies. If not at the seller's solicitor's office, give
buyer's solicitor sufficient notice. For a chain transaction, might be
helpful to have all the solicitors meet at one place.

Completion can be effected by post, which is especially common in
residential transactions, using Law Society's Code for Completion by
Post.

\hypertarget{money}{%
\subsection{Money}\label{money}}

\hypertarget{method-of-payment}{%
\subsubsection{Method of Payment}\label{method-of-payment}}

SCPC 9.7: payment is to be made only by a direct transfer of cleared
funds and an unconditional release of a deposit held by a stakeholder.

Electronic bank transfer known as ``telegraphic transfer''. The seller's
bank should be asked to phone the seller's solicitor to inform them of
the receipt of the funds immediately on arrival.

\begin{itemize}
\tightlist
\item
  Cleared funds

  \begin{itemize}
  \tightlist
  \item
    Payment of completion money should only be made from cleared funds
    in client account. Recall solicitor's accounts.
  \end{itemize}
\item
  Discharge of seller's mortgage

  \begin{itemize}
  \tightlist
  \item
    Often discharged immediately after completion by sending part of the
    proceeds of sale to the lender's solicitor/ having lender's
    solicitor receive the full sum, and transferring the balance.
  \end{itemize}
\item
  Release of deposit

  \begin{itemize}
  \tightlist
  \item
    If held as stakeholder, the buyer's solicitor should provide the
    seller's solicitor with a written release addressed to the
    stakeholder, authorising payment of the deposit to the seller.
  \item
    If held by the seller's solicitor as stakeholder, no written release
    needed.
  \end{itemize}
\end{itemize}

\hypertarget{completion-in-person}{%
\subsection{Completion in Person}\label{completion-in-person}}

Process:

\begin{itemize}
\tightlist
\item
  Appointment for completion fixed
\item
  Balance of purchase price transferred to seller's solicitor.
\item
  Take to completion:

  \begin{itemize}
  \tightlist
  \item
    Contract
  \item
    Evidence of title
  \item
    Copy of approved draft transfer deed and any other document to be
    executed by the seller
  \item
    Replies to the Request for Completion Information
  \item
    Completion checklist and completion statement.
  \item
    Any documents to be handed over to the seller's solicitor on
    completion.
  \end{itemize}
\item
  Title documents

  \begin{itemize}
  \tightlist
  \item
    Planning consents, building regulation approvals and indemnity
    insurance policies in relation to title defects
  \end{itemize}
\item
  Transfer deed

  \begin{itemize}
  \tightlist
  \item
    Should be dated at completion after being checked by the buyer's
    solicitor.
  \end{itemize}
\item
  Schedule of deeds and documents

  \begin{itemize}
  \tightlist
  \item
    Two copies prepared, one to be signed by the seller's solicitor.
  \end{itemize}
\item
  Inspection of receipts

  \begin{itemize}
  \tightlist
  \item
    May be necessary to inspect receipts, e.g., for utility bills to be
    apportioned.\\
  \end{itemize}
\item
  Chattels

  \begin{itemize}
  \tightlist
  \item
    Receipt of money paid for the items signed by seller's solicitor and
    handed to buyer's solicitor.
  \end{itemize}
\item
  Discharge of seller's mortgage

  \begin{itemize}
  \tightlist
  \item
    Will have been agreed between parties at the Request for Completion
    Information stage.
  \item
    If the mortgage is a first mortgage of a property, parties may have
    agreed to permit the seller to discharge the mortgage after actual
    completion.
  \item
    Seller's lender's solicitor should hand buyer's solicitor on
    completion an undertaking (Law Society wording) to discharge the
    mortgage and forward Form DS1 to the buyer's solicitor as soon as it
    is received from the lender.
  \item
    Undertaking should only be accepted from a solicitor or licenced
    conveyor.
  \end{itemize}
\end{itemize}

Law Society wording:

\begin{quote}
In consideration of you today completing the purchase of {[}insert
description of property{]} we hereby undertake forthwith to pay over to
{[}insert name of lender{]} the money required to discharge the
mortgage/legal charge dated {[}insert date of charge{]} and to forward
the receipted mortgage/Form DS1 to you as soon as it is received by us
from {[}insert name of lender{]}.
\end{quote}

\begin{Shaded}
\begin{Highlighting}[]
\NormalTok{Above wording should be amended if ED or e{-}DS1 system is used. }
\end{Highlighting}
\end{Shaded}

Although acceptance of such an undertaking is not entirely risk-free, it
is not negligent to rely on such undertakings except in ``exceptional
circumstances'' ({[}{[}Pat el v Daybells\\
{[}2001{]} EWCA Civ 1229{]}{]}).

\hypertarget{clls-protocol}{%
\subsubsection{CLLS Protocol}\label{clls-protocol}}

Intended for use in commercial property transactions and high value
residential transactions where the seller's lenders will be releasing
the property from its charge and the buyer has a lender taking a first
legal charge over the property.

Prior to completion,

\begin{enumerate}
\def\labelenumi{\arabic{enumi}.}
\tightlist
\item
  DS1 executed by the outgoing lender to be obtained and held undated by
  either the seller's solicitors or the outgoing lender's solicitors.
\item
  Undertaking to be given by the seller's solicitors to the seller's
  lender's solicitors to forward the funds needed to redeem the mortgage
  immediately after completion. The amount needed to redeem the mortgage
  will have been agreed with the seller's lender's solicitors.
\end{enumerate}

At completion, the buyer's solicitors would release the completion
monies to the seller, and the seller's solicitors (or seller's lender's
solicitors) would release the DS1 to the buyer. The seller's solicitors
can then redeem the seller's mortgage using the completion monies, and
the buyer's solicitors (if in personal attendance) are immediately in
receipt of the DS1.

BEFORE:

MERMAID1

AFTER:

MERMAID2

\hypertarget{copy-documents}{%
\subsubsection{Copy Documents}\label{copy-documents}}

There are cases in which the buyer is entitled to have only a copy of a
document relating to the seller's title (e.g., purchases from PRs, who
are entitled to retain the original grant). Buyer's solicitor should
check the copy is the same as the original, and then mark it as such. If
a certified copy is needed, ask the solicitor to write:

\begin{quote}
I certify this to be a true copy of the {[}insert type of document{]}
dated {[}insert date of document being certified{]} signed {[}signature
of solicitor{]} and dated {[}insert date of certification{]}.
\end{quote}

\hypertarget{postal-completion}{%
\subsection{Postal Completion}\label{postal-completion}}

\hypertarget{law-societys-code-for-completion-by-post}{%
\subsubsection{Law Society's Code for Completion by
Post}\label{law-societys-code-for-completion-by-post}}

Any variations should be agreed in writing before completion. Written
instructions should be sent from the buyer's solicitor to the seller's
solicitor. Money sent by electronic transfer to arrive in time for
completion at the agreed time.

Seller's solicitor will act as the buyer's solicitor's \textbf{agent}
for the purposes of carrying out the completion procedure. Seller's
solicitor must carry out the buyer's instructions and effect completion
on his behalf. Immediately phone or email the buyer's solicitor to tell
them completion has taken place, and post documents which the buyer is
entitled to receive on completion to the buyer (first-class/ DX).

\hypertarget{telephone-completion}{%
\subsection{Telephone Completion}\label{telephone-completion}}

CLLS Protocol can be adopted, or partly adopted by the parties.

\begin{itemize}
\tightlist
\item
  Executed DS1 will already be held by the seller's solicitors (or the
  solicitors of their lender) by the time of completion.
\item
  DS1 released to buyer in return for the release of completion monies
  (sent in advance to the seller's solicitor and held to the order of
  the buyer's solicitor).
\item
  Parties agree to date all relevant documentation during phone calls.
\item
  Seller's solicitor undertakes to send TR1, title deeds and DS1 to
  buyer's solicitor.
\end{itemize}

\hypertarget{lenders-requirements}{%
\subsection{Lender's Requirements}\label{lenders-requirements}}

If the buyer's solicitor is acting for the buyer's lender, they should
check the lender's requirements for completion when preparing checklist
and making arrangements for completion.

\hypertarget{after-completion}{%
\section{After Completion}\label{after-completion}}

Contact the client asap after completion to inform of the success of the
transaction.

\hypertarget{acting-for-seller}{%
\subsection{Acting for Seller}\label{acting-for-seller}}

\begin{itemize}
\tightlist
\item
  Contact buyer's solicitor to inform them that completion has taken
  place.
\item
  Contact estate agent and ask for release of property keys
\item
  Send transfer deed, title deeds and other documents to buyer's
  solicitor by first-class post or DX
\item
  Deal with proceeds of sale.
\item
  Discharge seller's mortgage.

  \begin{itemize}
  \tightlist
  \item
    Either discharge by electronic transfer or sending client account
    cheque for the amount required to the lender.
  \item
    If CLLS Protocol not used, engrossment of Form DS1 requesting the
    lender to discharge the mortgage and return receipted Form DS1 must
    be sent.
  \item
    Reassign the benefit of any life insurance policy which was assigned
    to the lender as collateral.
  \item
    Lender insuring the property should be told to cancel the property
    insurance cover.
  \item
    If ED or e-DS1 system used, notification of discharge will be sent
    directly by the lender to Land Registry.
  \end{itemize}
\item
  Bill client
\item
  Letter to client, reminding them to notify local authority and water
  company of change of ownership, to cancel insurance over the property
  and about CGT liability (if applicable).
\item
  Custody deeds, if any remain
\item
  Check file.
\end{itemize}

\hypertarget{acting-for-buyer}{%
\subsection{Acting for Buyer}\label{acting-for-buyer}}

\begin{itemize}
\tightlist
\item
  Complete mortgage deed by insertion of date and other info
\item
  Complete file copies of documents
\item
  Pay SDLT
\item
  Account for bridging finance
\item
  Send bill to client
\item
  Discharged mortgage -- check Form DS1, acknowledge receipt and release
  sender from undertaking given on completion.
\item
  Make copies of documents of which Land Registry requires copies.
\item
  Register title: Form AP1 within relevant priority period (land already
  registered) or Form FR1 within 2 months of completion (first
  registration).
\item
  Register company charges

  \begin{itemize}
  \tightlist
  \item
    At Companies House within 21 days of creation.
  \item
    Fee payable, attach certified copy of mortgage deed and s 859D
    statement of particulars in Form MR01.
  \item
    Companies Act registration of charges separate and additional to
    Land Registry registration.
  \end{itemize}
\item
  Diary entry for registration to be effected.
\item
  Discharge entries protecting the contract
\item
  Check register entries once the transaction has been registered.
\item
  Custody of deeds

  \begin{itemize}
  \tightlist
  \item
    e.g., planning consents, building regulation approvals, defective
    title insurance policies, guarantees for building work.
  \item
    If there is a mortgage, the lender is entitled to these.
  \end{itemize}
\end{itemize}

\hypertarget{lenders-solicitor}{%
\subsection{Lender's Solicitor}\label{lenders-solicitor}}

Where a separate solicitor has been instructed to act for the buyer's
lender, the lender's solicitor will normally have taken custody of the
transfer deed and other title deeds on completion, and they will deal
with the stamping and registration of the documents instead of the
buyer's solicitor.

\hypertarget{undertakings}{%
\subsection{Undertakings}\label{undertakings}}

Once performed, formally ask the recipient to release the giver from the
undertaking, so the giver has written evidence of the fulfilment of the
undertaking.

\hypertarget{sdlt-returns}{%
\subsection{SDLT Returns}\label{sdlt-returns}}

SDLT payable to HMRC within 14 days of completion. Payable on the value
of land, not chattels. Buyer personally responsible for completing the
tax return, but normally completed by the buyer's solicitor. SDLT return
must be completed for most transactions even if SDLT not actually
payable.

Form SDLT1 (if sent by post) must be signed personally by the buyer.
HMRC will issue Form SDLT 5 to prove the return has been submitted and
any tax paid. Without SDLT 5, transaction will not be accepted for
registration on the Land Register.

Most returns submitted online. Solicitor must certify that SDLT return
has been approved by the buyer. Sensible to get this in writing.

\hypertarget{registration-of-title}{%
\subsection{Registration of Title}\label{registration-of-title}}

\begin{itemize}
\tightlist
\item
  Failure to make an application for first registration within \textbf{2
  months} of completion results in transfer of the legal estate becoming
  void.
\item
  Failure to make an application for registration of a dealing within
  the priority period of \textbf{30 working days} given by a
  pre-completion Land Registry search may result in the client's
  interest losing priority.
\end{itemize}

\hypertarget{registration-of-dealings}{%
\subsubsection{Registration of
Dealings}\label{registration-of-dealings}}

Where registered land is transferred, an application for registration of
the dealing must be made on the appropriate application form,
accompanied by the correct documentation and fee, and must be received
by Land Registry within the priority period of 30 working days given by
the Land Registry search made before completion.

Land Registry operates an early completion policy. Applies to all
applications to discharge the whole of a registered charge where other
applications are made (such as an application to register the transfer
of the property to the buyer) but no evidence of satisfaction of the
charge is supplied (perhaps because the lender has been slow in
executing the DS1).

\begin{Shaded}
\begin{Highlighting}[]
\NormalTok{A buyer purchasing a property with a new mortgage will be registered as the proprietor, but the seller’s mortgage will remain on the title until evidence of discharge is provided, and the buyer’s mortgage will rank in priority behind the seller’s mortgage on the register.}
\end{Highlighting}
\end{Shaded}

Where there is a restriction on the register preventing a disposal or
the registration of a new charge without the consent of the existing
lender, proof of satisfaction of the charge or evidence of compliance
with the restriction must be provided to Land Registry within \textbf{20
working days} (which can be extended to 40 working days on application);
if this is not done, Land Registry will cancel the buyer's applications
for discharge, transfer and charge.

\hypertarget{transfer-of-whole}{%
\subsubsection{Transfer of Whole}\label{transfer-of-whole}}

Application for registration of the dealing on Form AP1, accompanied by
the relevant documents, should be lodged within \textbf{30 working days}
of the date of issue of the applicant's pre-completion official search
certificate. Originals not needed; certified copies can be sent. Fee
payable. Send:

\begin{enumerate}
\def\labelenumi{\arabic{enumi}.}
\tightlist
\item
  Certified copy of the transfer
\item
  Appropriate fee
\item
  Copy of grant of representation if seller was PR
\item
  Certified copy power of attorney
\item
  SDLT certificate
\item
  Form DI
\item
  Office copy of appropriate death certificate if a registered
  proprietor has died.
\end{enumerate}

\hypertarget{additional-applications}{%
\subsubsection{Additional Applications}\label{additional-applications}}

May be associated applications to be made to Land Registry:

\hypertarget{application-to-discharge-mortgage}{%
\paragraph{Application to Discharge
Mortgage}\label{application-to-discharge-mortgage}}

If lender used ED system, Form AP1 should be endorsed with the words:
``Charge discharged electronically''. If using a paper form, buyer's
solicitor applies to Land Registry for the charge to be discharged and a
certified copy of the paper form of discharge submitted.

\hypertarget{application-to-register-new-legal-charge}{%
\paragraph{Application to Register New Legal
Charge}\label{application-to-register-new-legal-charge}}

If there is a new mortgage over the property, this must be registered.
Submit:

\begin{enumerate}
\def\labelenumi{\arabic{enumi}.}
\tightlist
\item
  Certified copy of the mortgage deed; and
\item
  If borrower is a company:

  \begin{enumerate}
  \def\labelenumii{\arabic{enumii}.}
  \tightlist
  \item
    Certified copy of the certificate of registration
  \item
    Conveyancer's/ lender's written confirmation that the certified copy
    mortgage deed being lodged for registration is the same as that
    filed at Companies House.
  \end{enumerate}
\end{enumerate}

\hypertarget{disclosing-overriding-interests}{%
\subsubsection{Disclosing Overriding
Interests}\label{disclosing-overriding-interests}}

Applicant for registration must complete Form DI setting out any
overriding interests that burden the title. These will be entered on the
register and cease to be overriding.

\hypertarget{identity-requirements}{%
\subsubsection{Identity Requirements}\label{identity-requirements}}

Applicant must give details of the conveyancer acting for each party. If
a party is not represented, evidence of the party's identity must be
provided.

\hypertarget{late-completion}{%
\section{Late Completion}\label{late-completion}}

Completion may be delayed:

\begin{itemize}
\tightlist
\item
  Problem in the chain of transactions
\item
  Seller's solicitor not managing to get transfer deed signed by their
  client
\item
  Buyer not being in receipt of funds from lender.
\item
  Usually just a temporary hitch.
\end{itemize}

\hypertarget{breach-of-contract}{%
\subsection{Breach of Contract}\label{breach-of-contract}}

The innocent party will be entitled to damages for his loss, but cannot
terminate the contract unless time was of the essence of the completion
date ({[}{[}Raineri v Miles {[}1981{]} AC 1050{]}{]}). SCPC 9.1.1: time
is not of the essence unless a notice to complete has been served.

\hypertarget{related-transactions}{%
\subsection{Related Transactions}\label{related-transactions}}

A solicitor should try to ensure that no breach of contract occurs
(e.g., by arranging bridging finance to complete on time), but the
overriding duty is to act in their client's best interest. Consider
practicalities -- cost of bridging finance, risk completion may take a
while, client becoming ``homeless''.

\hypertarget{compensation-for-delay}{%
\subsection{Compensation for Delay}\label{compensation-for-delay}}

Damages payable under normal contractual principles for delayed
completion. SCPC 10.3: payment of compensation for delayed completion
irrespective of whether the relevant party has suffered any loss. Where
loss has been suffered in excess of the amount payable under SCPC 10.3,
can be recovered in a claim for breach of contract.

SCPC 10.3: Contractual entitlement to compensation given to the seller
where the buyer has defaulted in some way and completion is delayed. If
the seller defaults and completion is delayed, there is no contractual
right to compensation (so the buyer would have to bring a claim for
breach of contract to recover).

\hypertarget{deemed-late-completion}{%
\subsubsection{Deemed Late Completion}\label{deemed-late-completion}}

SCPC 9.1.2/ 9.1.3: where the sale is with vacant possession and the
money due on completion is not paid by 2pm on the day of actual
completion, for the purposes of compensation, completion is deemed to
have taken place the next working day (unless seller has not vacated by
2pm).

\begin{Shaded}
\begin{Highlighting}[]
\NormalTok{If buyer\textquotesingle{}s money did not arrive until 2.15pm on Friday,seller would be entitled to interest for 3 days. }
\end{Highlighting}
\end{Shaded}

\hypertarget{service-of-notice-to-complete}{%
\subsection{Service of Notice to
Complete}\label{service-of-notice-to-complete}}

When it appears a delay in completion is not likely to be resolved
quickly, consideration may be given to the service of a notice to
complete, with the effect of making time of the essence of the contract.
Then the aggrieved party has the option to terminate the contract
immediately.

\begin{Shaded}
\begin{Highlighting}[]
\NormalTok{Making time of the essence imposes a condition which binds both parties. So if previously aggrieved party $A$ makes time of the essence, and subsequently is unable to complete, $B$ could then terminate the contract at the specified new completion date.}
\end{Highlighting}
\end{Shaded}

SCPC 9.8: on service of a notice to complete, completion must take place
within 10 working days (exclusive of the date of service) and time is of
the essence.

\hypertarget{standard-conditions-2}{%
\subsubsection{Standard Conditions}\label{standard-conditions-2}}

\begin{itemize}
\tightlist
\item
  SCPC 9.8: on service of a notice to complete, completion must take
  place within 10 working days (excluding date of service) and makes
  time of the essence.
\item
  SCPC 9.8.3: a buyer who has paid \(<10\%\) deposit must pay the
  balance of the full 10\% on receipt of a notice to complete.
\item
  SCPC 10.5, 10.6: parties' rights and obligations where a valid notice
  has been served but not complied with. Once served, cannot be
  withdrawn.
\end{itemize}

Non-compliance with a notice to complete gives the aggrieved party the
right to terminate the contract.

SCPC 10.5: as well as rescinding the contract, the seller may:

\begin{enumerate}
\def\labelenumi{\arabic{enumi}.}
\tightlist
\item
  forfeit and keep the deposit and any accrued interest;
\item
  resell the property included in the contract; and
\item
  claim damages.
\end{enumerate}

SCPC 10.6: in addition to rescinding the contract, the buyer is entitled
to the return of the deposit and accrued interest.

\hypertarget{remedies}{%
\section{Remedies}\label{remedies}}

Sometimes completion is delayed/ cannot occur at all. Or the buyer may
discover an incumbrance which has not been disclosed.

\hypertarget{breach-of-contract-1}{%
\subsection{Breach of Contract}\label{breach-of-contract-1}}

Warranties, conditions, innominate terms. In conveyancing contracts, all
terms are usually called conditions, but in law some of them will only
be warranties. \textbf{The classification attached to a term by the
parties is not conclusive as to its status.}

\hypertarget{limitation-periods}{%
\subsubsection{Limitation Periods}\label{limitation-periods}}

6 years for breach of contract, 12 years where contract made by deed
(Limitation Act 1980).

\hypertarget{merger}{%
\subsubsection{Merger}\label{merger}}

On completion, the terms of the contract merge with the transfer deed in
so far as the two documents cover the same ground. SCPC 10.4 is a
non-merger clause to prevent this.

\hypertarget{exclusion-clauses}{%
\subsubsection{Exclusion Clauses}\label{exclusion-clauses}}

SCPC 10.1 restricts the remedies available for a breach of contract.
Buyer entitled to damages only if there is a material difference in the
tenure/ value of the property. Entitled to terminate contract only if
the error/ omission results from fraud/ recklessness or where otherwise
obliged to accept property differing substantially in quality, quantity
or tenure from expectations.

\begin{Shaded}
\begin{Highlighting}[]
\NormalTok{So the buyer is only entitled to damages for an undisclosed incumbrance if this caused a material difference in the value of the land. }
\end{Highlighting}
\end{Shaded}

Exclusion clauses contained in contracts for the sale of land (except
those relating to the exclusion of liability for misrepresentation) are
not subject to the reasonableness test in the Unfair Contract Terms Act
1977.

\hypertarget{damages}{%
\subsubsection{Damages}\label{damages}}

Damages for breach of a contract for the sale of land are assessed under
the normal\\
contractual principles established in {[}{[}Hadley v Baxendale (1854) 9
Ex 341{]}{]}.

{[}{[}Causation, remoteness, mitigation\#Remoteness{]}{]}

\hypertarget{quantum}{%
\subsubsection{Quantum}\label{quantum}}

Under consequential loss head, the quantum of damages limited to loss
which was within the contemplation of the parties in the light of the
facts known at the date when the contract was made (not at the date of
the breach of contract).

Loss of development profit can be claimed only if D was aware of the
claimant's proposals for the property at the time the contract was made.

\hypertarget{loss-of-development-profit}{%
\subsubsection{Loss of Development
Profit}\label{loss-of-development-profit}}

Claimable only if the defendant was aware of the claimant's proposals
for the property at the time the contract was made ({[}{[}Diamond v
Campbell Jones {[}1961{]} Ch 22{]}{]}).

\hypertarget{resale-by-seller}{%
\subsubsection{Resale by Seller}\label{resale-by-seller}}

Where the buyer defaults and the seller makes a loss on the resale, the
loss can be claimed as damages. If the seller makes a profit on the
resale, will have to give credit for the amount of the profit in his
claim. \textbf{Seller not entitled to benefit from the buyer's breach}.

\hypertarget{mental-distress}{%
\subsubsection{Mental Distress}\label{mental-distress}}

Lol, you can't recover for this (buying a house isn't a leisure activity
like {[}{[}Jarvis v Swan Tours {[}1973{]} QB 233{]}{]}).

\hypertarget{pre-contract-losses}{%
\subsubsection{Pre-contract Losses}\label{pre-contract-losses}}

Generally, no possibility of recovering expenses incurred at
pre-contract stages of the transaction.

\hypertarget{mitigation}{%
\subsubsection{Mitigation}\label{mitigation}}

{[}{[}Causation, remoteness, mitigation\#Mitigation{]}{]}

\hypertarget{rescission}{%
\subsection{Rescission}\label{rescission}}

Restoration of parties to their pre-contract position by undoing the
contract and balancing the position of the parties with the payment of
compensation. Conventional damages are not payable because there is no
breach of contract. Equitable remedy (clean hands, lapse of time etc.)

\hypertarget{contractual-right-to-rescind}{%
\subsubsection{Contractual Right to
Rescind}\label{contractual-right-to-rescind}}

Right to rescind is available:

\begin{enumerate}
\def\labelenumi{\arabic{enumi}.}
\tightlist
\item
  for misrepresentation (SCPC 10.1);
\item
  where a licence to assign is not forthcoming in leasehold transactions
  (SCPC 11.3);
\item
  where either the buyer or the seller has failed to comply with a
  notice to complete (SCPC 10.5 and 10.6).
\end{enumerate}

\hypertarget{limitation-periods-1}{%
\subsubsection{Limitation Periods}\label{limitation-periods-1}}

Where the right to rescind arises out of a contractual provision, must
be exercised within time limits/ within a reasonable time.

\hypertarget{misrepresentation}{%
\subsection{Misrepresentation}\label{misrepresentation}}

A misrepresentation is an untrue statement of fact made by one
contracting party which is relied on by the aggrieved party, which
induces him to enter the contract, and as a result of which he suffers
loss. The statement must be of fact, not law ({[}{[}Solle v Butcher
{[}1950{]} 1 KB 671{]}{]}). A statement of opinion is not actionable
unless it can be proved that the opinion was never genuinely held
({[}{[}Edgington v Fitzmaurice (1885) 29 Ch D 459{]}{]}).

\hypertarget{fraudulent-misrepresentation}{%
\subsubsection{Fraudulent
Misrepresentation}\label{fraudulent-misrepresentation}}

The aggrieved party may issue a claim in tort for deceit, which may
result in rescission of the contract and damages. V onerous to prove.

\hypertarget{claims-under-misrepresentation-act-1967}{%
\subsubsection{Claims Under Misrepresentation Act
1967}\label{claims-under-misrepresentation-act-1967}}

In relation to contracts between `consumers' and `traders' regulated by
the Consumer Rights Act 2015 (see 18.9), s 2 of the Misrepresentation
Act 1967 does not apply to the extent that the consumer has a right to
redress under Part 4A of the Consumer Protection from Unfair Trading
Regulations 2008. These Regulations prohibit traders from engaging in
unfair commercial practices when dealing with consumers.

For consumer-to-consumer or business-to-business contracts, claimant
must show an actionable misrepresentation, after which the burden of
proof shifts to D must disprove negligence. Misrepresentation is
negligent if D cannot prove (1) reasonable grounds for believing and (2)
did believe the statement he made was true up to the time the contract
was made.

Remedies: rescission of contract and damages.

If D successfully establishes the defence, showing the misrepresentation
was innocent, rescission is available but not damages.

\hypertarget{rescission-1}{%
\subsubsection{Rescission}\label{rescission-1}}

Award of rescission lies within the equitable jurisdiction of the
courts. Discretionary remedy subject to equitable bars. The court may
award damages in lieu of rescission to the Claimant (s 2(2)
Misrepresentation Act 1967). Likely to be awarded only where the result
of misrepresentation is substantially to deprive claimant of his
bargain. Available after completion, though may not be possible where a
third party (e.g., lender) has acquired an interest.

\hypertarget{damages-1}{%
\subsubsection{Damages}\label{damages-1}}

Damages under the Misrepresentation Act 1967 are awarded on a tortious
basis ({[}{[}Chesneau v Interhome Ltd (1983) The Times, 9 June{]}{]}).
An award of damages can be made both as an award in lieu of rescission
and as an award to compensate the claimant for his loss, subject to the
overriding principle that the claimant cannot recover more than his
actual loss.

\hypertarget{limitation-period}{%
\subsubsection{Limitation Period}\label{limitation-period}}

Claim for misrepresentation does not arise out of contract or tort.
Limitation period subject to judicial debate, but in {[}{[}Green v Eadie
and others {[}2011{]} EWHC B24 (Ch){]}{]}, the High Court held that a
claim for damages under s 2(1) of the Misrepresentation Act 1967 should
become statute barred six years from when the cause of action accrued.

\hypertarget{incorporation-as-a-term}{%
\subsubsection{Incorporation as a Term}\label{incorporation-as-a-term}}

s 1 Misrepresentation Act 1967: where a misrepresentation has become
incorporated as a term of the contract, it is possible to treat the
statement as a representation and to pursue a Misrepresentation Act
remedy. Can allow claimant the right to ask for rescission as well as
damages.

\hypertarget{imputed-knowledge}{%
\subsubsection{Imputed Knowledge}\label{imputed-knowledge}}

Knowledge gained by a solicitor in the course of a transaction is deemed
to be known by the solicitor's client, whether or not this is in fact
the case.

\begin{Shaded}
\begin{Highlighting}[]
\NormalTok{Where a solicitor gives an incorrect reply to pre{-}contract enquiries, the solicitor’s knowledge, and also their misstatement, is attributable to the client who w ill be liable to the buyer in misrepresentation ([[CEMP Properties (UK) Ltd v Dentsply Research and Development Corporation (No 1) (1989) 2 EGLR 192]]). In such a situation, the solicitor would be liable to their own client in negligence.}
\end{Highlighting}
\end{Shaded}

\hypertarget{exclusion-clauses-1}{%
\subsubsection{Exclusion Clauses}\label{exclusion-clauses-1}}

For consumer-to-consumer and business-to-business contracts, s 3
Misrepresentation Act 1967 provides that any clause which purports to
limit or exclude liability for misrepresentation is valid only in so far
as it satisfies the reasonableness test laid down in s 11 and Sch 2 UCTA
1977. For contracts between traders and consumers, the relevant test is
the fairness test in s 62 CRA 2015. If a clause is unfair, it will not
be binding on the consumer.

SCPC 10.1 purport to limit the seller's liability for misrepresentation.
Damages only payable if there is a material difference between the
property as represented and in reality.

\textbf{Special condition 2 SCPCs}: intended to prevent reliance on any
misrepresentation which is not in writing and made by the other party/
solicitor, while preserving liability for fraud or recklessness.

\hypertarget{mis-description-and-non-disclosure}{%
\subsection{Mis-description and
Non-disclosure}\label{mis-description-and-non-disclosure}}

Where an error is made in the particulars of sale of the contract/
seller fails to comply with duty of disclosure, \textbf{SCPC 10.1} deals
with remedies.

\hypertarget{specific-performance}{%
\subsection{Specific Performance}\label{specific-performance}}

Equitable remedy granted at the discretion of the court. Order for
specific performance not uncommon in sale of land cases where an award
of damages would be inadequate. Equitable bars apply, so will not be
awarded where:

\begin{enumerate}
\def\labelenumi{\arabic{enumi}.}
\tightlist
\item
  Damages adequate compensation
\item
  Vitiating element like mistake, fraud or illegality
\item
  3rd party has acquired an interest for value in the property
\item
  Seller cannot make good title.
\end{enumerate}

The doctrine of laches (lapse of time) applies to equitable remedies.

If, in a situation where specific performance would otherwise be
available to the injured party, the court decides not to make such an
order, it can award damages in lieu of specific performance under the
\textbf{s 50 SCA 1981}. Damages assessed on normal contractual
principles. Where an award of specific performance has been made but has
not been complied with, the injured party may return to the court asking
the court to withdraw the order and to substitute the decree of specific
performance with an award of damages ({[}{[}Johnson v Agnew {[}1980{]}
AC 367{]}{]}).

\hypertarget{return-of-deposit}{%
\subsection{Return of Deposit}\label{return-of-deposit}}

Where the buyer defaults on completion, the seller will want to forfeit
the deposit, but \textbf{s 49(2) LPA 1925} gives the court an absolute
discretion to order the return of the deposit to the buyer. Where the
seller defaults on completion, the buyer will have a right to the return
of the deposit under \textbf{SCPC 10.6}.

\hypertarget{rectification}{%
\subsection{Rectification}\label{rectification}}

An application for rectification of the contract to correct the error
can be made. s 2(4) LP(MP)A 1989: court has discretion to determine the
date on which the contract comes into operation.

Where a term of the contract is either omitted from or inaccurately
represented in the transfer deed, an application for rectification of
the deed may be made to the court.

\hypertarget{covenants-for-title}{%
\subsection{Covenants for Title}\label{covenants-for-title}}

Certain covenants for title will be implied into the transfer deed. The
State guarantee of a registered title means that, in practice, claims on
the covenants are unlikely to occur with the transfer of registered
land.

\end{document}
