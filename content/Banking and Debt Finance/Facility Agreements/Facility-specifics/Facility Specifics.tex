% Options for packages loaded elsewhere
\PassOptionsToPackage{unicode}{hyperref}
\PassOptionsToPackage{hyphens}{url}
%
\documentclass[
]{article}
\usepackage{amsmath,amssymb}
\usepackage{lmodern}
\usepackage{iftex}
\ifPDFTeX
  \usepackage[T1]{fontenc}
  \usepackage[utf8]{inputenc}
  \usepackage{textcomp} % provide euro and other symbols
\else % if luatex or xetex
  \usepackage{unicode-math}
  \defaultfontfeatures{Scale=MatchLowercase}
  \defaultfontfeatures[\rmfamily]{Ligatures=TeX,Scale=1}
\fi
% Use upquote if available, for straight quotes in verbatim environments
\IfFileExists{upquote.sty}{\usepackage{upquote}}{}
\IfFileExists{microtype.sty}{% use microtype if available
  \usepackage[]{microtype}
  \UseMicrotypeSet[protrusion]{basicmath} % disable protrusion for tt fonts
}{}
\makeatletter
\@ifundefined{KOMAClassName}{% if non-KOMA class
  \IfFileExists{parskip.sty}{%
    \usepackage{parskip}
  }{% else
    \setlength{\parindent}{0pt}
    \setlength{\parskip}{6pt plus 2pt minus 1pt}}
}{% if KOMA class
  \KOMAoptions{parskip=half}}
\makeatother
\usepackage{xcolor}
\usepackage[margin=1in]{geometry}
\usepackage{longtable,booktabs,array}
\usepackage{calc} % for calculating minipage widths
% Correct order of tables after \paragraph or \subparagraph
\usepackage{etoolbox}
\makeatletter
\patchcmd\longtable{\par}{\if@noskipsec\mbox{}\fi\par}{}{}
\makeatother
% Allow footnotes in longtable head/foot
\IfFileExists{footnotehyper.sty}{\usepackage{footnotehyper}}{\usepackage{footnote}}
\makesavenoteenv{longtable}
\setlength{\emergencystretch}{3em} % prevent overfull lines
\providecommand{\tightlist}{%
  \setlength{\itemsep}{0pt}\setlength{\parskip}{0pt}}
\setcounter{secnumdepth}{-\maxdimen} % remove section numbering
\usepackage{xcolor}
\definecolor{aliceblue}{HTML}{F0F8FF}
\definecolor{antiquewhite}{HTML}{FAEBD7}
\definecolor{aqua}{HTML}{00FFFF}
\definecolor{aquamarine}{HTML}{7FFFD4}
\definecolor{azure}{HTML}{F0FFFF}
\definecolor{beige}{HTML}{F5F5DC}
\definecolor{bisque}{HTML}{FFE4C4}
\definecolor{black}{HTML}{000000}
\definecolor{blanchedalmond}{HTML}{FFEBCD}
\definecolor{blue}{HTML}{0000FF}
\definecolor{blueviolet}{HTML}{8A2BE2}
\definecolor{brown}{HTML}{A52A2A}
\definecolor{burlywood}{HTML}{DEB887}
\definecolor{cadetblue}{HTML}{5F9EA0}
\definecolor{chartreuse}{HTML}{7FFF00}
\definecolor{chocolate}{HTML}{D2691E}
\definecolor{coral}{HTML}{FF7F50}
\definecolor{cornflowerblue}{HTML}{6495ED}
\definecolor{cornsilk}{HTML}{FFF8DC}
\definecolor{crimson}{HTML}{DC143C}
\definecolor{cyan}{HTML}{00FFFF}
\definecolor{darkblue}{HTML}{00008B}
\definecolor{darkcyan}{HTML}{008B8B}
\definecolor{darkgoldenrod}{HTML}{B8860B}
\definecolor{darkgray}{HTML}{A9A9A9}
\definecolor{darkgreen}{HTML}{006400}
\definecolor{darkgrey}{HTML}{A9A9A9}
\definecolor{darkkhaki}{HTML}{BDB76B}
\definecolor{darkmagenta}{HTML}{8B008B}
\definecolor{darkolivegreen}{HTML}{556B2F}
\definecolor{darkorange}{HTML}{FF8C00}
\definecolor{darkorchid}{HTML}{9932CC}
\definecolor{darkred}{HTML}{8B0000}
\definecolor{darksalmon}{HTML}{E9967A}
\definecolor{darkseagreen}{HTML}{8FBC8F}
\definecolor{darkslateblue}{HTML}{483D8B}
\definecolor{darkslategray}{HTML}{2F4F4F}
\definecolor{darkslategrey}{HTML}{2F4F4F}
\definecolor{darkturquoise}{HTML}{00CED1}
\definecolor{darkviolet}{HTML}{9400D3}
\definecolor{deeppink}{HTML}{FF1493}
\definecolor{deepskyblue}{HTML}{00BFFF}
\definecolor{dimgray}{HTML}{696969}
\definecolor{dimgrey}{HTML}{696969}
\definecolor{dodgerblue}{HTML}{1E90FF}
\definecolor{firebrick}{HTML}{B22222}
\definecolor{floralwhite}{HTML}{FFFAF0}
\definecolor{forestgreen}{HTML}{228B22}
\definecolor{fuchsia}{HTML}{FF00FF}
\definecolor{gainsboro}{HTML}{DCDCDC}
\definecolor{ghostwhite}{HTML}{F8F8FF}
\definecolor{gold}{HTML}{FFD700}
\definecolor{goldenrod}{HTML}{DAA520}
\definecolor{gray}{HTML}{808080}
\definecolor{green}{HTML}{008000}
\definecolor{greenyellow}{HTML}{ADFF2F}
\definecolor{grey}{HTML}{808080}
\definecolor{honeydew}{HTML}{F0FFF0}
\definecolor{hotpink}{HTML}{FF69B4}
\definecolor{indianred}{HTML}{CD5C5C}
\definecolor{indigo}{HTML}{4B0082}
\definecolor{ivory}{HTML}{FFFFF0}
\definecolor{khaki}{HTML}{F0E68C}
\definecolor{lavender}{HTML}{E6E6FA}
\definecolor{lavenderblush}{HTML}{FFF0F5}
\definecolor{lawngreen}{HTML}{7CFC00}
\definecolor{lemonchiffon}{HTML}{FFFACD}
\definecolor{lightblue}{HTML}{ADD8E6}
\definecolor{lightcoral}{HTML}{F08080}
\definecolor{lightcyan}{HTML}{E0FFFF}
\definecolor{lightgoldenrodyellow}{HTML}{FAFAD2}
\definecolor{lightgray}{HTML}{D3D3D3}
\definecolor{lightgreen}{HTML}{90EE90}
\definecolor{lightgrey}{HTML}{D3D3D3}
\definecolor{lightpink}{HTML}{FFB6C1}
\definecolor{lightsalmon}{HTML}{FFA07A}
\definecolor{lightseagreen}{HTML}{20B2AA}
\definecolor{lightskyblue}{HTML}{87CEFA}
\definecolor{lightslategray}{HTML}{778899}
\definecolor{lightslategrey}{HTML}{778899}
\definecolor{lightsteelblue}{HTML}{B0C4DE}
\definecolor{lightyellow}{HTML}{FFFFE0}
\definecolor{lime}{HTML}{00FF00}
\definecolor{limegreen}{HTML}{32CD32}
\definecolor{linen}{HTML}{FAF0E6}
\definecolor{magenta}{HTML}{FF00FF}
\definecolor{maroon}{HTML}{800000}
\definecolor{mediumaquamarine}{HTML}{66CDAA}
\definecolor{mediumblue}{HTML}{0000CD}
\definecolor{mediumorchid}{HTML}{BA55D3}
\definecolor{mediumpurple}{HTML}{9370DB}
\definecolor{mediumseagreen}{HTML}{3CB371}
\definecolor{mediumslateblue}{HTML}{7B68EE}
\definecolor{mediumspringgreen}{HTML}{00FA9A}
\definecolor{mediumturquoise}{HTML}{48D1CC}
\definecolor{mediumvioletred}{HTML}{C71585}
\definecolor{midnightblue}{HTML}{191970}
\definecolor{mintcream}{HTML}{F5FFFA}
\definecolor{mistyrose}{HTML}{FFE4E1}
\definecolor{moccasin}{HTML}{FFE4B5}
\definecolor{navajowhite}{HTML}{FFDEAD}
\definecolor{navy}{HTML}{000080}
\definecolor{oldlace}{HTML}{FDF5E6}
\definecolor{olive}{HTML}{808000}
\definecolor{olivedrab}{HTML}{6B8E23}
\definecolor{orange}{HTML}{FFA500}
\definecolor{orangered}{HTML}{FF4500}
\definecolor{orchid}{HTML}{DA70D6}
\definecolor{palegoldenrod}{HTML}{EEE8AA}
\definecolor{palegreen}{HTML}{98FB98}
\definecolor{paleturquoise}{HTML}{AFEEEE}
\definecolor{palevioletred}{HTML}{DB7093}
\definecolor{papayawhip}{HTML}{FFEFD5}
\definecolor{peachpuff}{HTML}{FFDAB9}
\definecolor{peru}{HTML}{CD853F}
\definecolor{pink}{HTML}{FFC0CB}
\definecolor{plum}{HTML}{DDA0DD}
\definecolor{powderblue}{HTML}{B0E0E6}
\definecolor{purple}{HTML}{800080}
\definecolor{red}{HTML}{FF0000}
\definecolor{rosybrown}{HTML}{BC8F8F}
\definecolor{royalblue}{HTML}{4169E1}
\definecolor{saddlebrown}{HTML}{8B4513}
\definecolor{salmon}{HTML}{FA8072}
\definecolor{sandybrown}{HTML}{F4A460}
\definecolor{seagreen}{HTML}{2E8B57}
\definecolor{seashell}{HTML}{FFF5EE}
\definecolor{sienna}{HTML}{A0522D}
\definecolor{silver}{HTML}{C0C0C0}
\definecolor{skyblue}{HTML}{87CEEB}
\definecolor{slateblue}{HTML}{6A5ACD}
\definecolor{slategray}{HTML}{708090}
\definecolor{slategrey}{HTML}{708090}
\definecolor{snow}{HTML}{FFFAFA}
\definecolor{springgreen}{HTML}{00FF7F}
\definecolor{steelblue}{HTML}{4682B4}
\definecolor{tan}{HTML}{D2B48C}
\definecolor{teal}{HTML}{008080}
\definecolor{thistle}{HTML}{D8BFD8}
\definecolor{tomato}{HTML}{FF6347}
\definecolor{turquoise}{HTML}{40E0D0}
\definecolor{violet}{HTML}{EE82EE}
\definecolor{wheat}{HTML}{F5DEB3}
\definecolor{white}{HTML}{FFFFFF}
\definecolor{whitesmoke}{HTML}{F5F5F5}
\definecolor{yellow}{HTML}{FFFF00}
\definecolor{yellowgreen}{HTML}{9ACD32}
\usepackage[most]{tcolorbox}

\usepackage{ifthen}
\provideboolean{admonitiontwoside}
\makeatletter%
\if@twoside%
\setboolean{admonitiontwoside}{true}
\else%
\setboolean{admonitiontwoside}{false}
\fi%
\makeatother%

\newenvironment{env-5da6e2cb-3d74-4c5e-b7d7-7055e14f6fdf}
{
    \savenotes\tcolorbox[blanker,breakable,left=5pt,borderline west={2pt}{-4pt}{firebrick}]
}
{
    \endtcolorbox\spewnotes
}
                

\newenvironment{env-2910d6b1-dcce-4560-bf30-b36f12b78587}
{
    \savenotes\tcolorbox[blanker,breakable,left=5pt,borderline west={2pt}{-4pt}{blue}]
}
{
    \endtcolorbox\spewnotes
}
                

\newenvironment{env-a3e9d811-f68a-44b4-a14b-e49bc126ec2b}
{
    \savenotes\tcolorbox[blanker,breakable,left=5pt,borderline west={2pt}{-4pt}{green}]
}
{
    \endtcolorbox\spewnotes
}
                

\newenvironment{env-4eea08fd-40ea-4873-96f0-acafbcfcec5e}
{
    \savenotes\tcolorbox[blanker,breakable,left=5pt,borderline west={2pt}{-4pt}{aquamarine}]
}
{
    \endtcolorbox\spewnotes
}
                

\newenvironment{env-7d209900-6332-42f9-8e1d-8e1841cec73a}
{
    \savenotes\tcolorbox[blanker,breakable,left=5pt,borderline west={2pt}{-4pt}{orange}]
}
{
    \endtcolorbox\spewnotes
}
                

\newenvironment{env-8fea4a55-4c19-42d0-8c4c-c6e2f388c6f2}
{
    \savenotes\tcolorbox[blanker,breakable,left=5pt,borderline west={2pt}{-4pt}{blue}]
}
{
    \endtcolorbox\spewnotes
}
                

\newenvironment{env-7c6b66cc-f079-42aa-b2e4-4e0400858143}
{
    \savenotes\tcolorbox[blanker,breakable,left=5pt,borderline west={2pt}{-4pt}{gold}]
}
{
    \endtcolorbox\spewnotes
}
                

\newenvironment{env-59531b83-2b90-4f4b-b28a-0d0d14c17c03}
{
    \savenotes\tcolorbox[blanker,breakable,left=5pt,borderline west={2pt}{-4pt}{darkred}]
}
{
    \endtcolorbox\spewnotes
}
                

\newenvironment{env-3cae73f9-a778-418d-94ce-585e2ebbe725}
{
    \savenotes\tcolorbox[blanker,breakable,left=5pt,borderline west={2pt}{-4pt}{pink}]
}
{
    \endtcolorbox\spewnotes
}
                

\newenvironment{env-2b801cbb-a953-4b0a-aa1f-4f4d37e46523}
{
    \savenotes\tcolorbox[blanker,breakable,left=5pt,borderline west={2pt}{-4pt}{cyan}]
}
{
    \endtcolorbox\spewnotes
}
                

\newenvironment{env-d797b130-4ec5-4d02-9c44-4f2a86698248}
{
    \savenotes\tcolorbox[blanker,breakable,left=5pt,borderline west={2pt}{-4pt}{cyan}]
}
{
    \endtcolorbox\spewnotes
}
                

\newenvironment{env-694b7873-5083-4609-b999-a03e18c94516}
{
    \savenotes\tcolorbox[blanker,breakable,left=5pt,borderline west={2pt}{-4pt}{purple}]
}
{
    \endtcolorbox\spewnotes
}
                

\newenvironment{env-d292529f-d8e7-4540-b0cb-665a7e01a1a9}
{
    \savenotes\tcolorbox[blanker,breakable,left=5pt,borderline west={2pt}{-4pt}{darksalmon}]
}
{
    \endtcolorbox\spewnotes
}
                

\newenvironment{env-06f4cd75-5590-49f1-9dac-bb1e27002edf}
{
    \savenotes\tcolorbox[blanker,breakable,left=5pt,borderline west={2pt}{-4pt}{gray}]
}
{
    \endtcolorbox\spewnotes
}
                
\ifLuaTeX
  \usepackage{selnolig}  % disable illegal ligatures
\fi
\IfFileExists{bookmark.sty}{\usepackage{bookmark}}{\usepackage{hyperref}}
\IfFileExists{xurl.sty}{\usepackage{xurl}}{} % add URL line breaks if available
\urlstyle{same} % disable monospaced font for URLs
\hypersetup{
  pdftitle={Facility Specifics},
  hidelinks,
  pdfcreator={LaTeX via pandoc}}

\title{Facility Specifics}
\author{}
\date{}

\begin{document}
\maketitle

{
\setcounter{tocdepth}{3}
\tableofcontents
}
\hypertarget{facility-specifics}{%
\section{Facility Specifics}\label{facility-specifics}}

\hypertarget{intro-clauses}{%
\subsection{Intro Clauses}\label{intro-clauses}}

\begin{itemize}
\tightlist
\item
  Date clause

  \begin{itemize}
  \tightlist
  \item
    First page
  \item
    Completed simultaneously with singing
  \item
    If several parties, date at which the last party signs.
  \end{itemize}
\item
  Parties

  \begin{itemize}
  \tightlist
  \item
    Mechanism for new borrowers (``Acceding Borrowers'').
  \item
    Guarantors may be required.
  \item
    Obligors = borrowers + guarantors.
  \item
    Names of all the banks with their different functions.
  \item
    Contract (Rights of Third Parties) Act 1999 gives 3rd parties rights
    to enforce a contractual term if the contract expressly provides, or
    if the term confers a benefit to the party. Most facility agreements
    exclude the effect of the 1999 Act other than its protective
    implications.
  \end{itemize}
\end{itemize}

\hypertarget{interpretation-definitions}{%
\subsection{Interpretation/
Definitions}\label{interpretation-definitions}}

Defined terms appear a the front. Many documents have an interpretation
section, defining common words such as month, immediately after the
definitions section.

\hypertarget{facility}{%
\subsection{Facility}\label{facility}}

Facility clause outlines the type of facility provided by the bank and
the amount to be made available. Many contain both a term loan and an
RCF. Each bank in the syndicated loan agrees to provide a proportion of
the total facility, known as the bank's commitment. This will specify
that obligations are several - not joint and several.

\hypertarget{why-facility}{%
\subsubsection{Why Facility}\label{why-facility}}

The term loan facility may be more appropriate than loan agreement:

\begin{enumerate}
\tightlist
\item
  Term agreement suggests the borrower has agreed to borrow money and is
  obliged to draw down.
\item
  If a document acknowledges or creates debt, it may be classified as a
  `debenture' (Levy v Abercorris Slate and Slab Co (1887) 37 Ch D 260).
  If facility agreements are debentures, they may then be `specified
  investments' under the FSMA 2000 (Regulated Activities) Order 2001
  (RAO). This would trigger licensing requirements for banks carrying
  out regulated activities with respect to specified investments. So
  loan facilities often deliberately drafted to create a commitment to
  lend but not to create or acknowledge a debt. UK regulators have
  accepted that such facilities fall outside of FSMA 2000. Court of
  Appeal screwed up in Fons Hf v Corporal Ltd and Another {[}2014{]}
  EWCA Civ 304 by suggesting that a loan agreement is an instrument
  creating and acknowledging debt and therefore a debenture, but the FCA
  has indicated that this decision does not change how it views loan
  agreements.
\end{enumerate}

\begin{env-06f4cd75-5590-49f1-9dac-bb1e27002edf}

Question

How can a bank prove that a borrower owes it any money\\
The accounting records kept by a bank showing the total amount
outstanding and certified by an appropriate oficer are prima facie
evidence of the borrower's obligations. Facility agreement will usually
require the borrower to accept the account recordds as conclusive
evidence. N.B. in the US, bank account records are insufficient
evidence. A "promissory note" (certificate acknowledging utilisation)
would need to be issued.

\end{env-06f4cd75-5590-49f1-9dac-bb1e27002edf}

\hypertarget{facility-office}{%
\subsubsection{Facility Office}\label{facility-office}}

Facility office/ lending office is the particular branch of the bank
through which a loan is booked. Will be specified in the loan agreement.
Important for tax reasons.

\hypertarget{purpose-clause}{%
\subsection{Purpose Clause}\label{purpose-clause}}

\hypertarget{control}{%
\subsubsection{Control}\label{control}}

The bank will want the facility agreement to explicitly state how the
loan monies can be used. Wording may be quite wide ("general corporate
purposes").

Connected matters: capacity and authority.

\hypertarget{capacity}{%
\subsubsection{Capacity}\label{capacity}}

Banks will want to ensure that the borrower has unrestricted capacity to
borrow under the facility, and to give guarantee and or security if
applicable.

\begin{env-a3e9d811-f68a-44b4-a14b-e49bc126ec2b}

s 39(1) - A company's capacity

The validity of an act done by a company shall not be called into
question on the ground of lack of capacity by reason of anything in the
company's constitution.

\end{env-a3e9d811-f68a-44b4-a14b-e49bc126ec2b}

Gives protection to third parties. But banks will usually check articles
anyway since the stakes are high.

\hypertarget{authority}{%
\subsubsection{Authority}\label{authority}}

The bank must be sure that whoever executes the facility agreement on
behalf of the borrower has authority to do so.

\begin{env-a3e9d811-f68a-44b4-a14b-e49bc126ec2b}

s 40 - Power of directors to bind the company

(1) In favour of a person dealing with a company in good faith, the
power of the directors to bind the company, or authorise others to do
so, is deemed to be free of any limitation under the company's
constitution.

(2) For this purpose---

\begin{itemize}
\tightlist
\item
  (a) a person ``deals with'' a company if he is a party to any
  transaction or other act to which the company is a party,
\item
  (b) a person dealing with a company---

  \begin{itemize}
  \tightlist
  \item
    (i) is not bound to enquire as to any limitation on the powers of
    the directors to bind the company or authorise others to do so,
  \item
    (ii) is presumed to have acted in good faith unless the contrary is
    proved, and
  \item
    (iii) is not to be regarded as acting in bad faith by reason only of
    his knowing that an act is beyond the powers of the directors under
    the company's constitution.
  \end{itemize}
\end{itemize}

(3) The references above to limitations on the directors' powers under
the company's constitution include limitations deriving---\\
(a) from a resolution of the company or of any class of shareholders,
or\\
(b) from any agreement between the members of the company or of any
class of shareholders.

\end{env-a3e9d811-f68a-44b4-a14b-e49bc126ec2b}

In practice: the bank checks the director has authority. Doesn't want to
be relying on ``good faith''. The bank will insist on a board meeting to
approve and execute the transaction.

\hypertarget{resulting-trust}{%
\subsubsection{Resulting Trust}\label{resulting-trust}}

A purpose clause is also included in an attempt to create a
\textbf{resulting trust} if the purpose fails. This was first recognised
in Barclays Bank Ltd v Quistclose Investments Ltd {[}1970{]} AC 567.

\hypertarget{bank-deposits}{%
\subsection{Bank Deposits}\label{bank-deposits}}

\begin{env-d797b130-4ec5-4d02-9c44-4f2a86698248}

Important

A bank is not a trustee of money deposited by its customers. Rather it
is a debtor in respect of those deposits.

\end{env-d797b130-4ec5-4d02-9c44-4f2a86698248}

Key case: Foley v Hill (1848) 11 HLC 2

Lord Cottenham LC:

\begin{quote}
Money, when paid into a bank, ceases altogether to be the money of the
principal. {[}\ldots{]} The money placed in the custody of a banker is,
to all intents and purposes, the money of the banker, to do with it as
he pleases; he is guilty of no breach of trust in employing it; he is
not answerable to the principal if he puts it into jeopardy, if he
engages in a hazardous speculation; he is not bound to keep it or deal
with it as the property of his principal, but he is of course answerable
for the amount, because he has contracted, having received that money.
\end{quote}

Thus, a debt is a different legal concept from a trust. But they are not
mutually exclusive. They\\
can be combined in a single transaction.

\hypertarget{quistclose-trust}{%
\subsubsection{Quistclose Trust}\label{quistclose-trust}}

The best example of a device which combines a debt and a trust is a
``Quistclose trust''. The name derives from the case in which such
arrangements were first recognised: Barclays Bank Ltd v Quistclose
Investments Ltd {[}1970{]} AC 567.

This was applied in Twinsectra Ltd v Yardley {[}2002{]} UKHL 12.
Twinsectra is the most important case on Quistclose trusts as it makes
clear that it is not sufficient to demonstrate that money was advanced
to the borrower for a particular purpose. Rather, it is necessary to
demonstrate the parties' mutual intention that the money could only be
applied for the purpose and was not at the free disposal of the
borrower. It is this feature of the transaction which generates the
trust in favour of the lender.

\begin{env-2b801cbb-a953-4b0a-aa1f-4f4d37e46523}

Tip

A Quistclose trust does not arise where money is paid for a purpose but
is at the recipient's free disposal.

\end{env-2b801cbb-a953-4b0a-aa1f-4f4d37e46523}

Twinsectra also provides helpful guidance on the requirement for
certainty of purpose:

\begin{enumerate}
\tightlist
\item
  The borrower's power to apply the money is valid only if the purpose
  is sufficiently certain.
\item
  A purpose is certain if it is possible to determine whether any given
  application the money does or does not fall within it.
\item
  If the purpose is uncertain, the borrower cannot make any use of the
  money and simply holds it on trust for the lender.
\end{enumerate}

\hypertarget{operation-of-quistclose-trust}{%
\paragraph{Operation of Quistclose
Trust}\label{operation-of-quistclose-trust}}

\begin{itemize}
\tightlist
\item
  When the lender advances the money to the borrower, the borrower holds
  the money on trust for the lender, with a power to use it for a
  specified purpose.
\item
  To the extent that the borrower uses the money for the purpose, the
  lender's equitable interest is extinguished. The relationship changes
  from trustee-beneficiary to debtor-creditor.
\item
  To the extent that the borrower applies the money for any other
  purpose, the borrower commits a breach of trust. The lender can assert
  their equitable proprietary interest in the misapplied money (or its
  traceable proceeds).
\item
  If it becomes impossible to apply the money for the purpose, the
  borrower must return the money to the lender.
\end{itemize}

See also Re Farepak Food and Gifts Ltd (in administration) {[}2006{]}
EWHC 3272 (Ch).

\hypertarget{other-contexts}{%
\paragraph{Other Contexts}\label{other-contexts}}

It is worth noting that although most of the case law involves loans or
transactions involving money, Quistclose trusts are not restricted to
such circumstances. They arise in any situation where property is
transferred to a person whose use of the property is restricted to a
specified purpose: to any case where the property is not at the free
disposal of the transferee. (Ali v Dinc {[}2020{]} EWHC 3055 (Ch), paras
234, 238).

\begin{env-d797b130-4ec5-4d02-9c44-4f2a86698248}

Important

On the insolvency of the borrower B, any money held by B on trust for
the bank must be paid to the bank, irrespective of any other claims.

\end{env-d797b130-4ec5-4d02-9c44-4f2a86698248}

\hypertarget{condition-precedent}{%
\subsection{Condition Precedent}\label{condition-precedent}}

\hypertarget{purpose}{%
\subsubsection{Purpose}\label{purpose}}

Specific conditions which a bank requires a borrower to fulfil before
facility agreement takes place. Tangible evidence that the
representations and warranties have been met.

Borrower will not be able to utilise unless and until the CPs are
satisfied. Advantages for the bank:

\begin{enumerate}
\tightlist
\item
  Borrower is locked into the min provisions of the agreement
  regardless. Bank fees are payable and remedies for default apply.
\item
  Representations, warranties, undertakings, events of default and
  boiler plate all become operative as soon as the document is executed.
  So bank can begin monitoring.
\end{enumerate}

\hypertarget{waivers}{%
\subsubsection{Waivers}\label{waivers}}

If the borrower cannot satisfy one or more conditions precedent before
closing (date on which facility is meant to become available), will have
to ask the banks for temporary/ permanent waiver. In practice, the
bank's solicitor takes responsibility for ensuring satisfied/ waived.

\hypertarget{uncertainty}{%
\subsubsection{Uncertainty}\label{uncertainty}}

CP could be so vague as to make the agreement void for uncertainty. For
example, a condition in a contract which stipulated that the sale was
`subject to the purchaser obtaining a satisfactory mortgage' was held to
make the entire contract void in Lee-Parker v Izett (No 2) {[}1972{]} 2
All ER 800. It is, however, common to see wording requiring a CP
document to be `in form and substance satisfactory to the banks'. Likely
to be certain enough in legal terms.

The bank may ask for original documents/ certified copies. Most CPs
involve providing documentary evidence. May also be a CP that no event
of default is continuing.

\hypertarget{conditions-subsequent}{%
\subsubsection{Conditions Subsequent}\label{conditions-subsequent}}

If a deal is fast (e.g., acquisition finance), some CPs may be delayed
to happen within a period after utilisation, such as the granting of
full security.

\hypertarget{borrower-perspective}{%
\subsubsection{Borrower Perspective}\label{borrower-perspective}}

\begin{itemize}
\tightlist
\item
  Be wary of CPs requiring third party actions.
\item
  Talk to the bank -- likely to be flexible if problems addressed in
  good time.
\item
  Try to agree forms of certain documents which are to be "in form and
  substance satisfactory to the bank" in advance, like legal opinions.
  Ask to add ``reasonable''.
\item
  If there's a syndicate, try to push for acceptable of documents by the
  agent bank/ by a simple majority in the syndicate.
\end{itemize}

\hypertarget{availability}{%
\subsection{Availability}\label{availability}}

\hypertarget{single-utilisation-facility}{%
\subsubsection{Single Utilisation
Facility}\label{single-utilisation-facility}}

Simple term loan facility might allow the money to be utilised in one
amount on a specified day. Utilisation clause will say that if the
borrower is not default and complies with all its obligations, the bank
will comply with the utilisation request.

\hypertarget{multiple-utilisation-loan}{%
\subsubsection{Multiple Utilisation
Loan}\label{multiple-utilisation-loan}}

B allowed to utilise in a number of tranches as and when required, up to
the available amount. Tranches must be utilised within a specified
commitment period. RCFs even more flexible. Usually there is a
commitment fee.

\hypertarget{notice-of-utilisation}{%
\subsubsection{Notice of Utilisation}\label{notice-of-utilisation}}

\begin{env-d797b130-4ec5-4d02-9c44-4f2a86698248}

Important

Fundamental premise of syndicated facility mechanics is that banks will
use the interbank market to fund large advances ("matched funding").

\end{env-d797b130-4ec5-4d02-9c44-4f2a86698248}

Borrower will therefore be required to give notice of utilisation, with
the notice period depending on the currency (on the London market,
Sterling could be funded on the day of request). Common for a syndicated
loan to require B to give 3 business days notice for Eurocurrencies, or
1 business day (day before) notice for Sterling. Notice periods may be
shorter with a revolving credit facility.

Once utilisation notice given, B is usually committed to taking the
money. In each utilisation request, the borrower must confirm there is
no current default and that utilisation will not result in a default.

\hypertarget{floating-rate-interest}{%
\subsection{Floating Rate Interest}\label{floating-rate-interest}}

Most loans are at floating rate interest.

\hypertarget{base-rates}{%
\subsubsection{Base Rates}\label{base-rates}}

Commercial banks raise loan capital through attracting depositors and
issuing debt securities. Incur regulatory costs. Bank's base rate
amalgamates these costs to provide a general cost of raising funds.
Typically used to charge interest on bilateral loans to individuals and
small/ medium-sized businesses.

Bank of England publishes the official rate, decided by the MPC each
month. Commercial banks' base rates will usually follow the movements of
the official rate.

\hypertarget{libor}{%
\subsubsection{LIBOR}\label{libor}}

Commercial facilities, typically syndicated, will not usually use base
rates. Floating rate used, made up of:

\begin{enumerate}
\tightlist
\item
  LIBOR, or equivalent
\item
  Margin: bank's profit and covering regulatory capital costs.
\end{enumerate}

LIBOR is the London Interbank Offered Rate; the interest rate charged
for loans between banks in the London interbank market. Intended to
provide banks with liquidity. LIBOR rates depend on :

\begin{enumerate}
\tightlist
\item
  Currency
\item
  Duration
\item
  Credit standing of borrowing bank
\item
  Liquidity of interbank market.
\end{enumerate}

The definition should specify the date and time at which the rate is to
be set.

\hypertarget{screen-rate}{%
\subsubsection{Screen Rate}\label{screen-rate}}

British Bankers Association publish an average LIBOR rate: bbalibor or
screen rate. Widely used across products and markets. But found to have
been manipulated -- to create a healthier picture of banks' credit
quality during credit crunch. Govt. took the measure away from BBA and
awarded it to ICE. ICE uses the same methodology. Similar rates in other
financial markets.

\hypertarget{sonia}{%
\subsubsection{SONIA}\label{sonia}}

Majority of LIBOR rates ceased to be published immediately after
31/12/21. In relation to sterling, LIBOR is largely replaced by the
risk-free rate Sterling Overnight Index Average (SONIA). Rates reflect
the average interest that banks pay to borrow overnight from other
financial institutions. Based on actual transactions. Features:

\begin{enumerate}
\tightlist
\item
  Considered ``risk-free'' rates because they do not include any premium
  for term or credit risk;
\item
  Not published as forward-looking term rates, so parties cannot know in
  advance how much interest will be due at the end of those periods; and
\item
  Each currency will have its own distinct risk-free rate and related
  administrator.
\end{enumerate}

A term and implicit credit premium was always incorporated into LIBOR.
As participants switch, an adjustment is required to adjust for the
differences that exist. Known as a Credit Adjustment Spread (``CAS'').
Aim: the amount of interest aid by a borrower after transition is as
close as possible to under LIBOR.

So under SONIA:

\[\text{Interest\ payable} = \text{SONIA} + \text{CAS} + \text{Margin}\]

CAS calculated as the historic 5-year median average difference between
SONIA and LIBOR.

Regulators made "synthetic LIBOR" to prevent market disruption. Any
remaining references to LIBOR deemed to rely to synthetic LIBOR. Not
really relevant any more.

\hypertarget{screen-rate-unavailability}{%
\subsubsection{Screen Rate
Unavailability}\label{screen-rate-unavailability}}

If a screen rate is used, it might be unavailable/ insufficient to cover
the banks' actual funding costs. Provisions may be included for:

\begin{itemize}
\tightlist
\item
  Interpolation between shorter and longer rates
\item
  Shortening interest period to a tenor that it available
\item
  Using historic interest rates
\item
  Using reference banks.
\item
  Cost of funds -- requires each bank in the syndicate to provide the
  agent with its actual cost of funding the loan, to be added to the
  margin. Usually use as a last resort (v burdensome).
\end{itemize}

\hypertarget{fixed-rate-interest}{%
\subsection{Fixed Rate Interest}\label{fixed-rate-interest}}

Tend to be unattractively high. Better to use hedging.

\hypertarget{interest-periods}{%
\subsection{Interest Periods}\label{interest-periods}}

The borrower may be required to pay interest at regular intervals
throughout the term of the loan, or tranches may be divided into
interest periods of varying duration. Interest periods primarily chosen
to support matched funding:

\begin{enumerate}
\tightlist
\item
  Interest period mirrored by interbank loan (1/3/6 months)
\item
  LIBOR/ SONIA recalculated at the start of each interest period and
  applies for the duration of that period (sometimes). So can really be
  a succession of different fixed rates.
\item
  Interest payable at the end of each interest period.
\item
  Borrower can select the duration of each interest period, giving it
  more control over the rate and timing of payment.
\item
  Repayments/ prepayments required to coincide with the end of an
  interest period
\item
  Some representations will be repeated on the first day of each
  interest period.
\end{enumerate}

\hypertarget{apportionment}{%
\subsection{Apportionment}\label{apportionment}}

Interest rates quoted at an annual rate. Bank and borrower must agree
the basis on which the annual interest will be apportioned between each
interest period (``day count fraction'').

\begin{itemize}
\tightlist
\item
  For domestic sterling facility agreements, traditional basis is
  dividing by 365
\item
  For others, it is 360 (check this).
\end{itemize}

\hypertarget{ratchets}{%
\subsection{Ratchets}\label{ratchets}}

The facility agreement may provide for the margin to vary in accordance
with the health of the borrower, measured through monitoring its
financial ratios (cross-over/ leveraged) or credit rating (investment
grade).

\hypertarget{hedging}{%
\subsection{Hedging}\label{hedging}}

Leveraged borrowers typically have large loans and high margins. Many
hedge their exposure to interest rate fluctuations by entering an
interest rate swap. Borrower contracts to periodically pay a fixed
amount to the counterparty in return for the floating rate. Often a
syndicate bank provides the hedge.

\hypertarget{default-interest}{%
\subsection{Default Interest}\label{default-interest}}

If a borrower fails to pay a sum due under a facility agreement, default
interest charged on any overdue amount (a fixed rate above the usual
rate payable). But cant be a penalty clause because then it would be
unenforceable! Cavendish Square Holdings BV v Talal El Makdessi
{[}2015{]} UKSC 67.

\hypertarget{liquidated-damages}{%
\subsubsection{Liquidated Damages}\label{liquidated-damages}}

Contracts arise from the agreement of the parties. The starting point of
contract law is to support parties' agreements on as many matters as
possible.

The parties can agree not only to the terms of the contract, but also
the nature and scope of the consequences of a breach of contract.

\begin{quote}
A liquidated damages clause stipulates a certain sum which is to be
payable on a particular breach of contract.
\end{quote}

This can be commercially advantageous for a company, since it fixes the
amount that will be due for a breach as a debt arising under the
contract (reduced uncertainty). Means a party can take risks into
account when determining the price for a contract. Very common clauses
in construction and technology industries.

\hypertarget{court-intervention}{%
\paragraph{Court Intervention}\label{court-intervention}}

There are instances where a court will intervene. Courts have
jurisdiction to intervene in a contract to strike down a liquidated
damages' clause which requires the party in breach to pay an excessive
sum such that it becomes a 'penalty'. If a clause is regarded as a
penalty, it will be struck down by the Court, and the claimant will only
be entitled to 'unliquidated damages' (those assessed in the normal
way).

\begin{quote}
A \textbf{penalty clause} is a liquidated damages' clause which requires
the party in breach to pay an excessive sum, such that it becomes a
penalty, and therefore the clause will not be upheld.
\end{quote}

Test for determining whether a clause is a valid liquidated damages
clause or a penalty: ParkingEye Limited v Beavis {[}2015{]} UKSC 67 and
Cavendish Square Holdings BV v Talal El Makdessi {[}2015{]} UKSC 67
(these two cases were heard together).

\hypertarget{makdessi-test}{%
\paragraph{\texorpdfstring{\emph{Makdessi}
Test}{Makdessi Test}}\label{makdessi-test}}

Is the clause a primary or secondary obligation?

\begin{itemize}
\tightlist
\item
  A clause will be primary if it is part of the primary obligations in
  the commercial context of the contract, i.e., furthers the commercial
  objective of the contract.
\item
  A clause will be secondary if it is an obligation triggered by a
  breach of contract to compensate the innocent party.
\end{itemize}

If primary, the clause will not engage the penalty rule at all, so will
be valid.

If secondary, the clause will be a penalty if it imposes a detriment out
of all proportion to any legitimate interest of the innocent party in
the performance of the primary obligation. Test for this:

\begin{itemize}
\tightlist
\item
  What (if any) legitimate business interest is served and protected by
  the clause?
\item
  Is the detriment imposed to protect that interest extravagant,
  exorbitant or unconscionable?
\end{itemize}

The burden of proof is on the person alleging that the clause is a
penalty to prove this.

The law on penalties is a clear interference with freedom of contract,
so will not be invoked lightly to strike down a clause in a contract
freely negotiated between parties of equal bargaining power. A party can
sometimes have a legitimate interest in enforcing performance which goes
beyond simply being compensated for losses. A clause which is not
disproportionate to that protection of legitimate business interest will
be upheld.

Where the parties have negotiated a contract, on a level playing field
and with the assistance of professional advisors, it will be hard for
the party paying liquidated damages to challenge the validity of those
provisions on the basis that they are a penalty.

Careful drafting to frame provisions as primary rather than secondary
obligations can reduce the chance of a party falling foul of the law on
penalties (Holyoake v Candy {[}2017{]} EWHC 3397 (Ch)). Key: to prevent
an allegation that the clause is a secondary liability payable on breach
in either substance or form.

For example, taking a scenario where Brett agrees to build a house for
Damon at a price of £100,000 by 15 May 2019:

(a) a clause requiring payment of a sum of £1000 per day for late
completion might be regarded as secondary and subject to the law on
penalties; alternatively

(b) the obligation could be redrafted so that the price payable is
lower, but Brett receives a 'bonus' for delivery on time. This provision
might be regarded as a primary obligation intended to set the price and
so not subject to the law on penalties.

MERMAID3

Practically, most facility agreements impose default interest of 1-2\%.

\begin{env-7c6b66cc-f079-42aa-b2e4-4e0400858143}

Basis point

=0.01\%. Also referred to as a tick. 1\% can be referred to as a point.

\end{env-7c6b66cc-f079-42aa-b2e4-4e0400858143}

\end{document}
