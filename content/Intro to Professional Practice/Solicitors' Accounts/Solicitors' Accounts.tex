% Options for packages loaded elsewhere
\PassOptionsToPackage{unicode}{hyperref}
\PassOptionsToPackage{hyphens}{url}
%
\documentclass[
]{article}
\usepackage{amsmath,amssymb}
\usepackage{lmodern}
\usepackage{iftex}
\ifPDFTeX
  \usepackage[T1]{fontenc}
  \usepackage[utf8]{inputenc}
  \usepackage{textcomp} % provide euro and other symbols
\else % if luatex or xetex
  \usepackage{unicode-math}
  \defaultfontfeatures{Scale=MatchLowercase}
  \defaultfontfeatures[\rmfamily]{Ligatures=TeX,Scale=1}
\fi
% Use upquote if available, for straight quotes in verbatim environments
\IfFileExists{upquote.sty}{\usepackage{upquote}}{}
\IfFileExists{microtype.sty}{% use microtype if available
  \usepackage[]{microtype}
  \UseMicrotypeSet[protrusion]{basicmath} % disable protrusion for tt fonts
}{}
\makeatletter
\@ifundefined{KOMAClassName}{% if non-KOMA class
  \IfFileExists{parskip.sty}{%
    \usepackage{parskip}
  }{% else
    \setlength{\parindent}{0pt}
    \setlength{\parskip}{6pt plus 2pt minus 1pt}}
}{% if KOMA class
  \KOMAoptions{parskip=half}}
\makeatother
\usepackage{xcolor}
\usepackage[margin=1in]{geometry}
\usepackage{color}
\usepackage{fancyvrb}
\newcommand{\VerbBar}{|}
\newcommand{\VERB}{\Verb[commandchars=\\\{\}]}
\DefineVerbatimEnvironment{Highlighting}{Verbatim}{commandchars=\\\{\}}
% Add ',fontsize=\small' for more characters per line
\newenvironment{Shaded}{}{}
\newcommand{\AlertTok}[1]{\textcolor[rgb]{1.00,0.00,0.00}{\textbf{#1}}}
\newcommand{\AnnotationTok}[1]{\textcolor[rgb]{0.38,0.63,0.69}{\textbf{\textit{#1}}}}
\newcommand{\AttributeTok}[1]{\textcolor[rgb]{0.49,0.56,0.16}{#1}}
\newcommand{\BaseNTok}[1]{\textcolor[rgb]{0.25,0.63,0.44}{#1}}
\newcommand{\BuiltInTok}[1]{#1}
\newcommand{\CharTok}[1]{\textcolor[rgb]{0.25,0.44,0.63}{#1}}
\newcommand{\CommentTok}[1]{\textcolor[rgb]{0.38,0.63,0.69}{\textit{#1}}}
\newcommand{\CommentVarTok}[1]{\textcolor[rgb]{0.38,0.63,0.69}{\textbf{\textit{#1}}}}
\newcommand{\ConstantTok}[1]{\textcolor[rgb]{0.53,0.00,0.00}{#1}}
\newcommand{\ControlFlowTok}[1]{\textcolor[rgb]{0.00,0.44,0.13}{\textbf{#1}}}
\newcommand{\DataTypeTok}[1]{\textcolor[rgb]{0.56,0.13,0.00}{#1}}
\newcommand{\DecValTok}[1]{\textcolor[rgb]{0.25,0.63,0.44}{#1}}
\newcommand{\DocumentationTok}[1]{\textcolor[rgb]{0.73,0.13,0.13}{\textit{#1}}}
\newcommand{\ErrorTok}[1]{\textcolor[rgb]{1.00,0.00,0.00}{\textbf{#1}}}
\newcommand{\ExtensionTok}[1]{#1}
\newcommand{\FloatTok}[1]{\textcolor[rgb]{0.25,0.63,0.44}{#1}}
\newcommand{\FunctionTok}[1]{\textcolor[rgb]{0.02,0.16,0.49}{#1}}
\newcommand{\ImportTok}[1]{#1}
\newcommand{\InformationTok}[1]{\textcolor[rgb]{0.38,0.63,0.69}{\textbf{\textit{#1}}}}
\newcommand{\KeywordTok}[1]{\textcolor[rgb]{0.00,0.44,0.13}{\textbf{#1}}}
\newcommand{\NormalTok}[1]{#1}
\newcommand{\OperatorTok}[1]{\textcolor[rgb]{0.40,0.40,0.40}{#1}}
\newcommand{\OtherTok}[1]{\textcolor[rgb]{0.00,0.44,0.13}{#1}}
\newcommand{\PreprocessorTok}[1]{\textcolor[rgb]{0.74,0.48,0.00}{#1}}
\newcommand{\RegionMarkerTok}[1]{#1}
\newcommand{\SpecialCharTok}[1]{\textcolor[rgb]{0.25,0.44,0.63}{#1}}
\newcommand{\SpecialStringTok}[1]{\textcolor[rgb]{0.73,0.40,0.53}{#1}}
\newcommand{\StringTok}[1]{\textcolor[rgb]{0.25,0.44,0.63}{#1}}
\newcommand{\VariableTok}[1]{\textcolor[rgb]{0.10,0.09,0.49}{#1}}
\newcommand{\VerbatimStringTok}[1]{\textcolor[rgb]{0.25,0.44,0.63}{#1}}
\newcommand{\WarningTok}[1]{\textcolor[rgb]{0.38,0.63,0.69}{\textbf{\textit{#1}}}}
\usepackage{longtable,booktabs,array}
\usepackage{calc} % for calculating minipage widths
% Correct order of tables after \paragraph or \subparagraph
\usepackage{etoolbox}
\makeatletter
\patchcmd\longtable{\par}{\if@noskipsec\mbox{}\fi\par}{}{}
\makeatother
% Allow footnotes in longtable head/foot
\IfFileExists{footnotehyper.sty}{\usepackage{footnotehyper}}{\usepackage{footnote}}
\makesavenoteenv{longtable}
\setlength{\emergencystretch}{3em} % prevent overfull lines
\providecommand{\tightlist}{%
  \setlength{\itemsep}{0pt}\setlength{\parskip}{0pt}}
\setcounter{secnumdepth}{-\maxdimen} % remove section numbering
\usepackage{xcolor}
\definecolor{aliceblue}{HTML}{F0F8FF}
\definecolor{antiquewhite}{HTML}{FAEBD7}
\definecolor{aqua}{HTML}{00FFFF}
\definecolor{aquamarine}{HTML}{7FFFD4}
\definecolor{azure}{HTML}{F0FFFF}
\definecolor{beige}{HTML}{F5F5DC}
\definecolor{bisque}{HTML}{FFE4C4}
\definecolor{black}{HTML}{000000}
\definecolor{blanchedalmond}{HTML}{FFEBCD}
\definecolor{blue}{HTML}{0000FF}
\definecolor{blueviolet}{HTML}{8A2BE2}
\definecolor{brown}{HTML}{A52A2A}
\definecolor{burlywood}{HTML}{DEB887}
\definecolor{cadetblue}{HTML}{5F9EA0}
\definecolor{chartreuse}{HTML}{7FFF00}
\definecolor{chocolate}{HTML}{D2691E}
\definecolor{coral}{HTML}{FF7F50}
\definecolor{cornflowerblue}{HTML}{6495ED}
\definecolor{cornsilk}{HTML}{FFF8DC}
\definecolor{crimson}{HTML}{DC143C}
\definecolor{cyan}{HTML}{00FFFF}
\definecolor{darkblue}{HTML}{00008B}
\definecolor{darkcyan}{HTML}{008B8B}
\definecolor{darkgoldenrod}{HTML}{B8860B}
\definecolor{darkgray}{HTML}{A9A9A9}
\definecolor{darkgreen}{HTML}{006400}
\definecolor{darkgrey}{HTML}{A9A9A9}
\definecolor{darkkhaki}{HTML}{BDB76B}
\definecolor{darkmagenta}{HTML}{8B008B}
\definecolor{darkolivegreen}{HTML}{556B2F}
\definecolor{darkorange}{HTML}{FF8C00}
\definecolor{darkorchid}{HTML}{9932CC}
\definecolor{darkred}{HTML}{8B0000}
\definecolor{darksalmon}{HTML}{E9967A}
\definecolor{darkseagreen}{HTML}{8FBC8F}
\definecolor{darkslateblue}{HTML}{483D8B}
\definecolor{darkslategray}{HTML}{2F4F4F}
\definecolor{darkslategrey}{HTML}{2F4F4F}
\definecolor{darkturquoise}{HTML}{00CED1}
\definecolor{darkviolet}{HTML}{9400D3}
\definecolor{deeppink}{HTML}{FF1493}
\definecolor{deepskyblue}{HTML}{00BFFF}
\definecolor{dimgray}{HTML}{696969}
\definecolor{dimgrey}{HTML}{696969}
\definecolor{dodgerblue}{HTML}{1E90FF}
\definecolor{firebrick}{HTML}{B22222}
\definecolor{floralwhite}{HTML}{FFFAF0}
\definecolor{forestgreen}{HTML}{228B22}
\definecolor{fuchsia}{HTML}{FF00FF}
\definecolor{gainsboro}{HTML}{DCDCDC}
\definecolor{ghostwhite}{HTML}{F8F8FF}
\definecolor{gold}{HTML}{FFD700}
\definecolor{goldenrod}{HTML}{DAA520}
\definecolor{gray}{HTML}{808080}
\definecolor{green}{HTML}{008000}
\definecolor{greenyellow}{HTML}{ADFF2F}
\definecolor{grey}{HTML}{808080}
\definecolor{honeydew}{HTML}{F0FFF0}
\definecolor{hotpink}{HTML}{FF69B4}
\definecolor{indianred}{HTML}{CD5C5C}
\definecolor{indigo}{HTML}{4B0082}
\definecolor{ivory}{HTML}{FFFFF0}
\definecolor{khaki}{HTML}{F0E68C}
\definecolor{lavender}{HTML}{E6E6FA}
\definecolor{lavenderblush}{HTML}{FFF0F5}
\definecolor{lawngreen}{HTML}{7CFC00}
\definecolor{lemonchiffon}{HTML}{FFFACD}
\definecolor{lightblue}{HTML}{ADD8E6}
\definecolor{lightcoral}{HTML}{F08080}
\definecolor{lightcyan}{HTML}{E0FFFF}
\definecolor{lightgoldenrodyellow}{HTML}{FAFAD2}
\definecolor{lightgray}{HTML}{D3D3D3}
\definecolor{lightgreen}{HTML}{90EE90}
\definecolor{lightgrey}{HTML}{D3D3D3}
\definecolor{lightpink}{HTML}{FFB6C1}
\definecolor{lightsalmon}{HTML}{FFA07A}
\definecolor{lightseagreen}{HTML}{20B2AA}
\definecolor{lightskyblue}{HTML}{87CEFA}
\definecolor{lightslategray}{HTML}{778899}
\definecolor{lightslategrey}{HTML}{778899}
\definecolor{lightsteelblue}{HTML}{B0C4DE}
\definecolor{lightyellow}{HTML}{FFFFE0}
\definecolor{lime}{HTML}{00FF00}
\definecolor{limegreen}{HTML}{32CD32}
\definecolor{linen}{HTML}{FAF0E6}
\definecolor{magenta}{HTML}{FF00FF}
\definecolor{maroon}{HTML}{800000}
\definecolor{mediumaquamarine}{HTML}{66CDAA}
\definecolor{mediumblue}{HTML}{0000CD}
\definecolor{mediumorchid}{HTML}{BA55D3}
\definecolor{mediumpurple}{HTML}{9370DB}
\definecolor{mediumseagreen}{HTML}{3CB371}
\definecolor{mediumslateblue}{HTML}{7B68EE}
\definecolor{mediumspringgreen}{HTML}{00FA9A}
\definecolor{mediumturquoise}{HTML}{48D1CC}
\definecolor{mediumvioletred}{HTML}{C71585}
\definecolor{midnightblue}{HTML}{191970}
\definecolor{mintcream}{HTML}{F5FFFA}
\definecolor{mistyrose}{HTML}{FFE4E1}
\definecolor{moccasin}{HTML}{FFE4B5}
\definecolor{navajowhite}{HTML}{FFDEAD}
\definecolor{navy}{HTML}{000080}
\definecolor{oldlace}{HTML}{FDF5E6}
\definecolor{olive}{HTML}{808000}
\definecolor{olivedrab}{HTML}{6B8E23}
\definecolor{orange}{HTML}{FFA500}
\definecolor{orangered}{HTML}{FF4500}
\definecolor{orchid}{HTML}{DA70D6}
\definecolor{palegoldenrod}{HTML}{EEE8AA}
\definecolor{palegreen}{HTML}{98FB98}
\definecolor{paleturquoise}{HTML}{AFEEEE}
\definecolor{palevioletred}{HTML}{DB7093}
\definecolor{papayawhip}{HTML}{FFEFD5}
\definecolor{peachpuff}{HTML}{FFDAB9}
\definecolor{peru}{HTML}{CD853F}
\definecolor{pink}{HTML}{FFC0CB}
\definecolor{plum}{HTML}{DDA0DD}
\definecolor{powderblue}{HTML}{B0E0E6}
\definecolor{purple}{HTML}{800080}
\definecolor{red}{HTML}{FF0000}
\definecolor{rosybrown}{HTML}{BC8F8F}
\definecolor{royalblue}{HTML}{4169E1}
\definecolor{saddlebrown}{HTML}{8B4513}
\definecolor{salmon}{HTML}{FA8072}
\definecolor{sandybrown}{HTML}{F4A460}
\definecolor{seagreen}{HTML}{2E8B57}
\definecolor{seashell}{HTML}{FFF5EE}
\definecolor{sienna}{HTML}{A0522D}
\definecolor{silver}{HTML}{C0C0C0}
\definecolor{skyblue}{HTML}{87CEEB}
\definecolor{slateblue}{HTML}{6A5ACD}
\definecolor{slategray}{HTML}{708090}
\definecolor{slategrey}{HTML}{708090}
\definecolor{snow}{HTML}{FFFAFA}
\definecolor{springgreen}{HTML}{00FF7F}
\definecolor{steelblue}{HTML}{4682B4}
\definecolor{tan}{HTML}{D2B48C}
\definecolor{teal}{HTML}{008080}
\definecolor{thistle}{HTML}{D8BFD8}
\definecolor{tomato}{HTML}{FF6347}
\definecolor{turquoise}{HTML}{40E0D0}
\definecolor{violet}{HTML}{EE82EE}
\definecolor{wheat}{HTML}{F5DEB3}
\definecolor{white}{HTML}{FFFFFF}
\definecolor{whitesmoke}{HTML}{F5F5F5}
\definecolor{yellow}{HTML}{FFFF00}
\definecolor{yellowgreen}{HTML}{9ACD32}
\usepackage[most]{tcolorbox}

\usepackage{ifthen}
\provideboolean{admonitiontwoside}
\makeatletter%
\if@twoside%
\setboolean{admonitiontwoside}{true}
\else%
\setboolean{admonitiontwoside}{false}
\fi%
\makeatother%

\newenvironment{env-be892a89-1e2f-456f-9a66-9360378d761d}
{
    \savenotes\tcolorbox[blanker,breakable,left=5pt,borderline west={2pt}{-4pt}{firebrick}]
}
{
    \endtcolorbox\spewnotes
}
                

\newenvironment{env-6be7f2da-e5f2-4236-8595-02b7f227e0d2}
{
    \savenotes\tcolorbox[blanker,breakable,left=5pt,borderline west={2pt}{-4pt}{blue}]
}
{
    \endtcolorbox\spewnotes
}
                

\newenvironment{env-01e69c07-38ea-4db5-88eb-cd8cdd37b768}
{
    \savenotes\tcolorbox[blanker,breakable,left=5pt,borderline west={2pt}{-4pt}{green}]
}
{
    \endtcolorbox\spewnotes
}
                

\newenvironment{env-5799b106-f807-41ad-8581-f0d0ea5d489d}
{
    \savenotes\tcolorbox[blanker,breakable,left=5pt,borderline west={2pt}{-4pt}{aquamarine}]
}
{
    \endtcolorbox\spewnotes
}
                

\newenvironment{env-b74e22d1-c582-4a42-8efd-d25679e7a543}
{
    \savenotes\tcolorbox[blanker,breakable,left=5pt,borderline west={2pt}{-4pt}{orange}]
}
{
    \endtcolorbox\spewnotes
}
                

\newenvironment{env-5d212a80-72cb-4ee2-ac58-8fb02828ff70}
{
    \savenotes\tcolorbox[blanker,breakable,left=5pt,borderline west={2pt}{-4pt}{blue}]
}
{
    \endtcolorbox\spewnotes
}
                

\newenvironment{env-208b147f-a811-4374-a76c-b246707771a9}
{
    \savenotes\tcolorbox[blanker,breakable,left=5pt,borderline west={2pt}{-4pt}{yellow}]
}
{
    \endtcolorbox\spewnotes
}
                

\newenvironment{env-d06c67a0-b583-4913-9035-211f19ac83c8}
{
    \savenotes\tcolorbox[blanker,breakable,left=5pt,borderline west={2pt}{-4pt}{darkred}]
}
{
    \endtcolorbox\spewnotes
}
                

\newenvironment{env-851b1153-27b1-45fe-a97e-4b230b0cf5a8}
{
    \savenotes\tcolorbox[blanker,breakable,left=5pt,borderline west={2pt}{-4pt}{pink}]
}
{
    \endtcolorbox\spewnotes
}
                

\newenvironment{env-b7496de0-00bb-4d3c-b91d-8cb844a2110e}
{
    \savenotes\tcolorbox[blanker,breakable,left=5pt,borderline west={2pt}{-4pt}{cyan}]
}
{
    \endtcolorbox\spewnotes
}
                

\newenvironment{env-424bca06-dee0-462e-8854-1eb52bfa5db2}
{
    \savenotes\tcolorbox[blanker,breakable,left=5pt,borderline west={2pt}{-4pt}{cyan}]
}
{
    \endtcolorbox\spewnotes
}
                

\newenvironment{env-9d9db5a0-46ac-4ac6-b1b5-e91346b125e0}
{
    \savenotes\tcolorbox[blanker,breakable,left=5pt,borderline west={2pt}{-4pt}{purple}]
}
{
    \endtcolorbox\spewnotes
}
                

\newenvironment{env-1f68d7d9-7190-48f9-99de-2b368357fe97}
{
    \savenotes\tcolorbox[blanker,breakable,left=5pt,borderline west={2pt}{-4pt}{darksalmon}]
}
{
    \endtcolorbox\spewnotes
}
                

\newenvironment{env-3974e6d8-2024-4088-ae4e-3b22d061cb48}
{
    \savenotes\tcolorbox[blanker,breakable,left=5pt,borderline west={2pt}{-4pt}{gray}]
}
{
    \endtcolorbox\spewnotes
}
                
\ifLuaTeX
  \usepackage{selnolig}  % disable illegal ligatures
\fi
\IfFileExists{bookmark.sty}{\usepackage{bookmark}}{\usepackage{hyperref}}
\IfFileExists{xurl.sty}{\usepackage{xurl}}{} % add URL line breaks if available
\urlstyle{same} % disable monospaced font for URLs
\hypersetup{
  hidelinks,
  pdfcreator={LaTeX via pandoc}}

\author{}
\date{}

\begin{document}

{
\setcounter{tocdepth}{3}
\tableofcontents
}
\begin{Shaded}
\begin{Highlighting}[]
\NormalTok{min\_depth: 1}
\end{Highlighting}
\end{Shaded}

\hypertarget{recording-client-transactions}{%
\section{Recording Client
Transactions}\label{recording-client-transactions}}

Records of client money dealings must be clearly separated from the
records of the ordinary business dealings of a firm.

\hypertarget{format}{%
\subsection{Format}\label{format}}

Books must be:

\begin{itemize}
\tightlist
\item
  Maintained on the double-entry principle
\item
  Legible, up-to-date and contain narratives alongside entries which
  provide adequate information about a transaction.
\item
  Ledger accounts should include the name of the client and a
  description
\item
  Business counts in relation to each client must be kept up-to-date
\item
  Inter-ledger transfers separately recorded.
\end{itemize}

\hypertarget{dual-cash-account}{%
\subsubsection{Dual Cash Account}\label{dual-cash-account}}

Normal to have two individual cash accounts displayed next to one
another, and two separate ledger accounts displayed together (though
usually these are combined into one big ledger).

\hypertarget{receipts-of-money}{%
\subsection{Receipts of Money}\label{receipts-of-money}}

\begin{itemize}
\tightlist
\item
  Receipts of client money held for the relevant client in the client
  bank account
\item
  Receipts of business money will reduce the indebtedness of the
  relevant client to the firm.
\end{itemize}

\hypertarget{payments-of-money}{%
\subsection{Payments of Money}\label{payments-of-money}}

\begin{itemize}
\tightlist
\item
  Decide which account the payment is being made from.
\item
  Record CR entry on cash account.
\item
  DR entry on the ledger account of the client on whose behalf the
  payment is being made.
\end{itemize}

\hypertarget{firms-professional-charges}{%
\subsection{Firm's Professional
Charges}\label{firms-professional-charges}}

A bill to clients will include professional charges and VAT on these
charges. Accounts should show:

\begin{itemize}
\tightlist
\item
  Client owes the firm for charges and VAT.
\item
  Firm has earned income in the form of professional charges and owes
  HMRC VAT.
\end{itemize}

The bill should include details of disbursements already paid on the
client's behalf and disbursements to be paid in the future.

No entries are made on the accounts in relation to disbursements when
the bill is sent.

Recall r 8.1: all inter-account transfers must be recorded on the client
account ledger.

\hypertarget{cash-transfers}{%
\subsection{Cash Transfers}\label{cash-transfers}}

May occur on:

\begin{itemize}
\tightlist
\item
  Payment of a firm's professional fees and disbursements

  \begin{itemize}
  \tightlist
  \item
    Once a bill or other written notification has been issued.
  \end{itemize}
\item
  To advance money where the solicitor needs to make a payment on behalf
  of the client/ trust where there is insufficient client money
  available.
\item
  To replace money withdrawn in breach of r 5.3 (withdrawn client money
  where insufficient funds held to make the payment).
\item
  To allow the client a sum in lieu of interest.
\end{itemize}

\hypertarget{client-business-bank-account}{%
\subsection{Client → Business Bank
Account}\label{client-business-bank-account}}

\begin{enumerate}
\def\labelenumi{\arabic{enumi}.}
\tightlist
\item
  Record the payment of money from the client bank account

  \begin{enumerate}
  \def\labelenumii{\arabic{enumii}.}
  \tightlist
  \item
    CR cash account
  \item
    DR client's ledger account
  \end{enumerate}
\item
  Record the receipt of money into the business bank account.

  \begin{enumerate}
  \def\labelenumii{\arabic{enumii}.}
  \tightlist
  \item
    DR cash account
  \item
    CR client's ledger account.
  \end{enumerate}
\end{enumerate}

\hypertarget{inter-client-transfers}{%
\subsection{Inter-client Transfers}\label{inter-client-transfers}}

A `paper' or `inter-client' transfer is when the money held in the
client bank account recorded as being held for client A, is changed to
be recorded as held for client B.

\hypertarget{recording}{%
\subsubsection{Recording}\label{recording}}

\begin{itemize}
\tightlist
\item
  DR client ledger account of first client
\item
  CR client ledger account of second client.
\end{itemize}

\hypertarget{split-cheques-mixed-receipts-and-bills}{%
\subsection{Split Cheques, Mixed Receipts and
Bills}\label{split-cheques-mixed-receipts-and-bills}}

\hypertarget{split-cheques}{%
\subsubsection{Split Cheques}\label{split-cheques}}

If you receive a cheque made up partly of business money and partly of
client money, may be possible to ``split'' the cheque (if allowed by the
bank).

\begin{itemize}
\tightlist
\item
  Business portion to cash account -- DR business section
\item
  Client portion to cash account -- DR client section
\item
  Business portion to client ledger -- CR business section

  \begin{itemize}
  \tightlist
  \item
    Client portion to client ledger -- CR client section.
  \end{itemize}
\end{itemize}

\hypertarget{unsplit-cheques-and-direct-transfers}{%
\subsubsection{Unsplit Cheques and Direct
Transfers}\label{unsplit-cheques-and-direct-transfers}}

When a mixed receipt is received, the firm can choose whether to pay it
initially into the client or business bank account. Money must be
transferred `promptly' to the correct account (14 days was specified
under the old rules).

When clients pay by bank transfer, it is usual to give the client
details of a client bank account, and then make transfers as necessary
to the business bank account.

Entries:

\begin{itemize}
\tightlist
\item
  DR whole amount to cash account (client section)
\item
  CR whole amount to client ledger (client section)
\end{itemize}

When later transferring business portion

\begin{itemize}
\tightlist
\item
  CR cash account (client section)
\item
  DR client ledger (client section)
\item
  DR cash account (business section)
\item
  CR client ledger (business section).
\end{itemize}

\hypertarget{dealing-with-bills}{%
\subsection{Dealing with Bills}\label{dealing-with-bills}}

\begin{itemize}
\tightlist
\item
  Payments for professional charges/ disbursements will be the firm's
  own money and must be paid into the business bank account.
\item
  Where the firm is holding client money and money is due to the firm
  for disbursements paid with the firm's own money, money cannot be
  transferred unless a bill has been issued, or the firm made it clear
  that money would be used this way.
\item
  Money received for a paid disbursement is the firm's money, not client
  money
\item
  Billing for unpaid disbursements can cause risks to client money.
\end{itemize}

\hypertarget{receipt-of-cheque-made-out-to-the-client-tp}{%
\subsection{Receipt of Cheque Made Out to the client/
TP}\label{receipt-of-cheque-made-out-to-the-client-tp}}

If you receive a cheque made out to the client, you cannot pay the
cheque into a firm bank account. The only obligation is to forward the
cheque to the payee without delay. The cheque is not client money, just
paper. Many firms have a policy of recording where all cheques are
received, irrespective of the payee.

\hypertarget{dishonouring-cheques}{%
\subsection{Dishonouring Cheques}\label{dishonouring-cheques}}

There is nothing preventing a firm drawing against a cheque which has
been paid into the client bank account but which has not yet been
cleared. But if the cheque is dishonoured, there will be a breach of r
5.3 (only withdrawing from client account if \(\exists\) sufficient
funds).

Any breaches of rules must be corrected promptly on discovery (r 6.1).

\hypertarget{abatements}{%
\subsection{Abatements}\label{abatements}}

A firm may decide to abate the costs to the client (e.g., if the client
complains). To record the abatement, reverse the entries made on the
profit costs and HMRC account when the bill was sent (to the extent of
the abatement). Also send the client a VAT credit note.

\begin{itemize}
\tightlist
\item
  DR profit costs account
\item
  DR HMRC account
\item
  CR client ledger account (business section).
\end{itemize}

If preferred, the firm may debit abatements to a separate abatements
account.

\hypertarget{bad-debts}{%
\subsection{Bad Debts}\label{bad-debts}}

If a client is not going to pay the amount it owes to a firm, the firm
will have to write off the amount owing for:

\begin{itemize}
\tightlist
\item
  Professional charges
\item
  VAT
\item
  Disbursements paid from the business bank account.
\end{itemize}

\begin{longtable}[]{@{}
  >{\raggedright\arraybackslash}p{(\columnwidth - 2\tabcolsep) * \real{0.1429}}
  >{\raggedright\arraybackslash}p{(\columnwidth - 2\tabcolsep) * \real{0.8571}}@{}}
\toprule()
\begin{minipage}[b]{\linewidth}\raggedright
VAT treatment
\end{minipage} & \begin{minipage}[b]{\linewidth}\raggedright
Description
\end{minipage} \\
\midrule()
\endhead
General rule & There is no VAT relief; VAT must be accounted for even if
debt is written off \\
Exception & Where the debt has been outstanding for at least 6 months
since the date payment was due, solicitor is entitled to a refund from
HMRC. \\
\bottomrule()
\end{longtable}

So generally, for the whole amount:

\begin{itemize}
\tightlist
\item
  CR client ledger, business section
\item
  DR bad debts account
\end{itemize}

And under the exception, when VAT relief becomes available, add:

\begin{itemize}
\tightlist
\item
  CR bad debts account VAT
\item
  DR HMRC VAT
\end{itemize}

\hypertarget{petty-cash}{%
\subsection{Petty Cash}\label{petty-cash}}

When cash is withdrawn from the bank for petty cash

\begin{itemize}
\tightlist
\item
  CR cash, business section
\item
  DR petty cash account.
\end{itemize}

If a payment is made for a client using petty cash, an election has been
made to use business money for the transaction.

\hypertarget{insurance-commission}{%
\subsection{Insurance Commission}\label{insurance-commission}}

A law firm may be offered commission from insurance companies/ financial
services companies. But it is rare for a firm to be entitled to keep
this commission.

\begin{Shaded}
\begin{Highlighting}[]
\NormalTok{title: SRA Code for Firms, r 5.1}
\NormalTok{You properly account to clients for any financial benefit you receive as a result of their instructions, except where they have agreed otherwise.}
\end{Highlighting}
\end{Shaded}

``Properly account for'' means

\begin{itemize}
\tightlist
\item
  Paying to the client
\item
  Offsetting against fees
\item
  Keeping only when the client has been informed and agreed
\end{itemize}

Firms that wish to take advantage of the exemption allowing professional
firms to avoid regulation by the Financial Reporting Council in relation
to investment business must account for all commission to clients.

\hypertarget{sra-accounts-rules-2019}{%
\section{SRA Accounts Rules 2019}\label{sra-accounts-rules-2019}}

Law firms frequently hold money belonging to others. SRA imposes strict
rules on dealing with clients' money. In June 2017 these were greatly
simplified from the complicated Accounts Rules 2011.

\hypertarget{principles}{%
\subsection{Principles}\label{principles}}

Designed to reduce the risk of accidental or deliberate misuse of client
money. The 7 pervasive Principles apply to the regulation of accounts.

\hypertarget{who-is-bound}{%
\subsection{Who is Bound}\label{who-is-bound}}

\begin{Shaded}
\begin{Highlighting}[]
\NormalTok{title: r 1 SRA Accounts Rules 2019}
\NormalTok{1. These rules apply to authorised bodies, their managers and employees and references to "you" in these rules should be read accordingly}
\NormalTok{2. The authorised body\textquotesingle{}s managers are jointly and severally responsible for compliance by the authorised body, its managers and employees with these rules.}
\end{Highlighting}
\end{Shaded}

\hypertarget{client-money}{%
\subsection{Client Money}\label{client-money}}

r 4.1: client money should be kept separate from the money belonging to
the firm.

\begin{Shaded}
\begin{Highlighting}[]
\NormalTok{title: r 2.1}
\NormalTok{"Client money" is money held or received by you:}
\NormalTok{{-} relating to regulated services delivered by you to a client;}
\NormalTok{{-} on behalf of a third party in relation to regulated services delivered by you (such as money held as agent, stakeholder or held to the sender\textquotesingle{}s order);}
\NormalTok{{-} as a trustee or as the holder of a specified office or appointment, such as donee of a power of attorney, Court of Protection deputy or trustee of an occupational pension scheme;}
\NormalTok{{-} in respect of your fees and any unpaid disbursements if held or received prior to delivery of a bill for the same.}
\end{Highlighting}
\end{Shaded}

\begin{longtable}[]{@{}
  >{\raggedright\arraybackslash}p{(\columnwidth - 2\tabcolsep) * \real{0.1781}}
  >{\raggedright\arraybackslash}p{(\columnwidth - 2\tabcolsep) * \real{0.8219}}@{}}
\toprule()
\begin{minipage}[b]{\linewidth}\raggedright
Term
\end{minipage} & \begin{minipage}[b]{\linewidth}\raggedright
Definition
\end{minipage} \\
\midrule()
\endhead
Costs & Fees and disbursements \\
Fees & Your own charges or profit costs (including any VAT element) \\
Disbursements & Any costs or expenses paid or to be paid to a third
party on behalf of the client or trust. \\
\bottomrule()
\end{longtable}

A solicitor has a duty to clarify any ambiguity as to whom client money
is held for ({[}{[}{[}Challinor and Others v Juliet Bellis \& Co and
Another {[}2013{]} EWHC 347 (Ch){]}{]}{]}).

The effect of Rule 2.1(d) is that money received for all fees and future
disbursements paid to the firm is considered client money unless and
until billed. Such money is often referred to as money received
`generally on account of costs'.

The definition of client money does not include money received for
disbursements which have already been paid, so where money is received
in reimbursement of such a payment, it is a receipt of business money by
the firm.

\begin{Shaded}
\begin{Highlighting}[]
\NormalTok{title: Workaround?}
\NormalTok{A firm might wish to send a bill to a client for its anticipated future fees and disbursements with a view to paying the money received in payment of that bill into the firm’s business account, thus escaping the requirement that would apply under Rule 2.1 (d) to pay money received for unbilled fees and unpaid disbursements into the client bank account. Allowed, but there are reasons to be cautious about doing this. }
\end{Highlighting}
\end{Shaded}

\hypertarget{client-bank-account}{%
\subsection{Client Bank Account}\label{client-bank-account}}

A client bank account is one opened by the firm in the name of the firm,
but used for client money. Rule 3.1 requires such an account to be at a
bank or building society in England and Wales, and Rule 3.2 requires the
account to include the word `client' in its title to distinguish it from
the firm's own business accounts.

Adding client in the title means section 85(2) of the Solicitors Act
1974 applies. Provides that a bank does not have any recourse or right
against money in a client bank account in respect of any liability of
the solicitor to the bank.

Rule 3.3 prohibits the use of a client bank account to provide banking
facilities to clients or third parties.

\hypertarget{prohibition-on-providing-banking-facilities}{%
\subsection{Prohibition on Providing Banking
Facilities}\label{prohibition-on-providing-banking-facilities}}

\begin{Shaded}
\begin{Highlighting}[]
\NormalTok{title: r 3.3}
\NormalTok{You must not use a client account to provide banking facilities to clients or third parties. Payments into, and transfers or withdrawals from a client account must be in respect of the delivery by you of regulated services.}
\end{Highlighting}
\end{Shaded}

\begin{Shaded}
\begin{Highlighting}[]
\NormalTok{title: Using a client account improperly}
\NormalTok{As well as sanctions for breaching r 3.3, a solicitor could be in breach of SRA Principles by failing to act }
\NormalTok{{-} in a way that upholds the constitutional principle of the rule of law, and the proper}
\NormalTok{administration of justice: Principle 1}
\NormalTok{{-} in a way that upholds public trust and confidence in the solicitors’ profession and in}
\NormalTok{legal services provided by authorised persons: Principle 2}
\NormalTok{{-} with independence: Principle 3}
\NormalTok{{-} with integrity: Principle 5.}
\end{Highlighting}
\end{Shaded}

\hypertarget{sra-warning}{%
\subsubsection{SRA Warning}\label{sra-warning}}

\begin{quote}
Law firms, their managers and employees should not allow the firm's
client account to be used to provide banking facilities to clients or
third parties. You must also actively consider whether there are any
risk factors suggesting that the transaction on which you are acting,
even if it appears to be the normal work of a solicitor, is not genuine
or is suspicious.\\
SRA warning notice 25/11/19
\end{quote}

Funds can only be received into the client bank account where there is a
proper connection between receipt of the funds and the delivery of
regulated services.

Simply having a retainer with a client is insufficient to allow for
processing funds freely through the client bank account.

Note that traditionally, solicitors held funds for clients to enable
them to pay routine outgoings. This is no longer acceptable practice.

\begin{Shaded}
\begin{Highlighting}[]
\NormalTok{You should always ask why you are being asked to make a payment or why the client cannot make or receive the payment directly themselves. The client’s convenience is not a legitimate reason, nor is not having access to a bank account in the UK.}
\end{Highlighting}
\end{Shaded}

See {[}{[}Fuglers \& Others v SRA {[}2014{]} EWHC 179 (Admin){]}{]};
{[}{[}Premji Naram Patel v SRA {[}2012{]} EWHC 3373 (Admin){]}{]}; and
{[}{[}Attorney General of Zambia v Meer Care \& Desai {[}2008{]} EWCA
Civ 1007{]}{]}.

\hypertarget{deputies}{%
\subsubsection{Deputies}\label{deputies}}

Solicitors acting as deputies who used client bank accounts for the
affairs of people without capacity to manage their own financial affairs
under the Mental Capacity Act 2005 do not infringe r 3.3. But this is
not recommended/ authorised by the Office of the Public Guardian (mixing
funds is never really a good idea, and having a separate bank account
means you can have safeguards on authorisation).

\hypertarget{paying-client-money-into-client-bank-account}{%
\subsection{Paying Client Money into Client Bank
Account}\label{paying-client-money-into-client-bank-account}}

\begin{Shaded}
\begin{Highlighting}[]
\NormalTok{title: r 2.3}
\NormalTok{You ensure that client money is paid promptly into a client account unless:}
\NormalTok{{-} in relation to money falling within 2.1(c), to do so would conflict with your obligations under rules or regulations relating to your specified office or appointment;}
\NormalTok{{-} the client money represents payments received from the Legal Aid Agency for your costs; or}
\NormalTok{{-} you agree in the individual circumstances an alternative arrangement in writing with the client, or the third party, for whom the money is held.}
\end{Highlighting}
\end{Shaded}

This is subject to an exception:

\begin{Shaded}
\begin{Highlighting}[]
\NormalTok{title: r 2.2}
\NormalTok{In circumstances where the only client money you hold or receive falls within rule 2.1(d) above, and:}
\NormalTok{{-} any money held for disbursements relates to costs or expenses incurred by you on behalf of your client and for which you are liable; and}
\NormalTok{{-} you do not for any other reason maintain a client account;}

\NormalTok{you are not required to hold this money in a client account if you have informed your client in advance of where and how the money will be held. Rules 2.3, 2.4, 4.1, 7, 8.1(b) and (c) and 12 do not apply to client money held outside of a client account in accordance with this rule.}
\end{Highlighting}
\end{Shaded}

The ability to dispense with a client bank account can lead to cost
savings in professional indemnity insurance, compliance and producing an
accountant's report.

\begin{itemize}
\tightlist
\item
  Client money must be available on demand (r 2.4)
\item
  Client money must be returned promptly once there is no reason to hold
  the funds (r 2.5)
\end{itemize}

\hypertarget{keeping-money-separate}{%
\subsection{Keeping Money Separate}\label{keeping-money-separate}}

\begin{Shaded}
\begin{Highlighting}[]
\NormalTok{title: r 4.1}
\NormalTok{You keep client money separate from money belonging to the authorised body.}
\end{Highlighting}
\end{Shaded}

\hypertarget{mixed-receipts}{%
\subsubsection{Mixed Receipts}\label{mixed-receipts}}

If a firm receives a mixed payment containing client money + money to
pay a firm's bill, the 2019 Rules allow the firm to choose whether to
pay the receipt into the client or business bank account. This is a
controversial change. Note that you still can't use client money to prop
up business accounts because this would be contrary to the Principles.

\hypertarget{client-business-account-transfers}{%
\subsubsection{Client → Business Account
Transfers}\label{client-business-account-transfers}}

\begin{Shaded}
\begin{Highlighting}[]
\NormalTok{title: r 4.3}
\NormalTok{Where you are holding client money and some or all of that money will be used to pay your costs:}
\NormalTok{{-} you must give a bill of costs, or other written notification of the costs incurred, to the client or the paying party;}
\NormalTok{{-} this must be done before you transfer any client money from a client account to make the payment; and}
\NormalTok{{-} any such payment must be for the specific sum identified in the bill of costs, or other written notification of the costs incurred, and covered by the amount held for the particular client or third party.}
\end{Highlighting}
\end{Shaded}

Exception: where a bill includes anticipated disbursements which have
not yet been incurred, not in breach of r 43 for leaving the money
associated with those billed anticipated disbursements in the client
bank account until they are paid.

\hypertarget{bill-for-anticipated-fees-disbursements}{%
\subsubsection{Bill for Anticipated fees/
Disbursements}\label{bill-for-anticipated-fees-disbursements}}

2019 Rules permit monies to be transferred from client bank account if a
bill has been given to the client for work to be undertaken in the
future.

But consider the risk of insolvency and immediacy with which money could
be paid back. You should not bill in advance for advance disbursements
that the client will remain liable to pay for such as SDLT.

\hypertarget{transferring-money-for-paid-disbursements}{%
\subsubsection{Transferring Money for Paid
Disbursements}\label{transferring-money-for-paid-disbursements}}

\begin{itemize}
\tightlist
\item
  The obligation to send or give a bill to the client prior to
  transferring sums from the client bank account applies to the firms
  ``costs''

  \begin{itemize}
  \tightlist
  \item
    This includes profit costs and disbursements
  \item
    Change from the previous regime, which was just profit costs
    (``fees'')
  \end{itemize}
\item
  What about disbursements paid using firm money on behalf of the
  client?

  \begin{itemize}
  \tightlist
  \item
    SRA guidance states that r 5 permits money to be withdrawn from the
    client bank account ``for the purpose for which it is being held'',
    provided that this has been made clear in writing.
  \item
    But r 5 does not permit a transfer where disbursements have not yet
    been incurred.
  \end{itemize}
\end{itemize}

\hypertarget{withdrawals-from-client-bank-account}{%
\subsection{Withdrawals From Client Bank
Account}\label{withdrawals-from-client-bank-account}}

\hypertarget{circumstances}{%
\subsubsection{Circumstances}\label{circumstances}}

\begin{Shaded}
\begin{Highlighting}[]
\NormalTok{title: r 5.1}
\NormalTok{You only withdraw client money from a client account:}
\NormalTok{{-} for the purpose for which it is being held;}
\NormalTok{{-} following receipt of instructions from the client, or the third party for whom the money is held; or}
\NormalTok{{-} on the SRA\textquotesingle{}s prior written authorisation or in prescribed circumstances.}
\end{Highlighting}
\end{Shaded}

\begin{Shaded}
\begin{Highlighting}[]
\NormalTok{title: r 5.2}
\NormalTok{You appropriately authorise and supervise all withdrawals made from a client account.}
\end{Highlighting}
\end{Shaded}

\begin{Shaded}
\begin{Highlighting}[]
\NormalTok{title: r 5.3}
\NormalTok{You only withdraw client money from a client account if sufficient funds are held on behalf of that specific client or third party to make the payment.}
\end{Highlighting}
\end{Shaded}

If there are insufficient funds:

\begin{itemize}
\tightlist
\item
  Pay from business bank account, or
\item
  Firm advances money to client bank account
\end{itemize}

\hypertarget{residual-client-account-balances}{%
\subsubsection{Residual Client Account
Balances}\label{residual-client-account-balances}}

Residual client account balance arises where money was not returned to
the client at the end of a retainer and cannot be returned as the client
cannot be identified/ traced.

\begin{itemize}
\tightlist
\item
  Residual client balances of \(<£500\) can be withdrawn provided the
  balance is paid to a charity and reasonable steps have been taken to
  return the money to the rightful owner.
\item
  For amounts over £500, contact the SRA for authority to remove the
  money from the client account.
\end{itemize}

\hypertarget{client-accounting-systems}{%
\subsubsection{Client Accounting
Systems}\label{client-accounting-systems}}

Firms must maintain accurate records:

\begin{Shaded}
\begin{Highlighting}[]
\NormalTok{title: r 8.1}
\NormalTok{You keep and maintain accurate, contemporaneous, and chronological records to:}
\NormalTok{{-} (a) record in client ledgers identified by the client\textquotesingle{}s name and an appropriate description of the matter to which they relate:}
\NormalTok{    {-} (i) all receipts and payments which are client money on the client side of the client ledger account;}
\NormalTok{    {-} (ii) all receipts and payments which are not client money and bills of costs including transactions through the authorised body\textquotesingle{}s accounts on the business side of the client ledger account;}
\NormalTok{{-} (b) maintain a list of all the balances shown by the client ledger accounts of the liabilities to clients (and third parties), with a running total of the balances; and}
\NormalTok{{-} (c) provide a cash book showing a running total of all transactions through client accounts held or operated by you.}
\end{Highlighting}
\end{Shaded}

\begin{longtable}[]{@{}
  >{\raggedright\arraybackslash}p{(\columnwidth - 2\tabcolsep) * \real{0.0625}}
  >{\raggedright\arraybackslash}p{(\columnwidth - 2\tabcolsep) * \real{0.9375}}@{}}
\toprule()
\begin{minipage}[b]{\linewidth}\raggedright
Rule
\end{minipage} & \begin{minipage}[b]{\linewidth}\raggedright
Summary
\end{minipage} \\
\midrule()
\endhead
r 8.2 & Firms obtain bank statements for all client accounts at least
every 5 weeks \\
r 8.3 & Bank reconciliation statements produced \\
r 8.4 & Central record of bills and costs kept in a readily accessible
form. \\
\bottomrule()
\end{longtable}

\hypertarget{paying-interest-to-clients}{%
\subsubsection{Paying Interest to
Clients}\label{paying-interest-to-clients}}

\begin{Shaded}
\begin{Highlighting}[]
\NormalTok{title: r 7.1}
\NormalTok{You account to clients or third parties for a fair sum of interest on any client money held by you on their behalf.}
\end{Highlighting}
\end{Shaded}

This can be varied by written agreement, provided sufficient information
is given for the client to give informed consent (r 7.2)

\hypertarget{rd-party-managed-accounts}{%
\subsubsection{3rd Party Managed
Accounts}\label{rd-party-managed-accounts}}

\begin{Shaded}
\begin{Highlighting}[]
\NormalTok{title: r 11.1}
\NormalTok{You may enter into arrangements with a client to use a third party managed account for the purpose of receiving payments from or on behalf of, or making payments to or on behalf of, the client in respect of regulated services delivered by you to the client, only if:}
\NormalTok{{-} (a) use of the account does not result in you receiving or holding the client\textquotesingle{}s money; and}
\NormalTok{{-} (b) you take reasonable steps to ensure, before accepting instructions, that the client is informed of and understands:}
\NormalTok{    {-} (i) the terms of the contractual arrangements relating to the use of the third party managed account, and in particular how any fees for use of the third party managed account will be paid and who will bear them; and}
\NormalTok{    {-} (ii) the client\textquotesingle{}s right to terminate the agreement and dispute payment requests made by you.}
\end{Highlighting}
\end{Shaded}

Obtain regular statements from the third party (r 11.2). Provisions in
the Accounts Rules relating to holding client money do not apply to
monies in a TPMA (third-party managed account). Regulation will be by
the FCA. Firms using a TPMA must notify the SRA using a form.

\hypertarget{other-client-tp-money}{%
\subsection{Other client/ TP Money}\label{other-client-tp-money}}

\begin{itemize}
\tightlist
\item
  Firms may open a joint account with client/ TP. The rules do not apply
  to such an account, other than r 8.2 (bank statements) and r 8.4
  (keeping a central record).
\item
  If the firm operates the client's bank account as a signatory, r 8.3
  applies, requiring reconciliations of the account at least every 5
  weeks.
\end{itemize}

\hypertarget{accountants-reports}{%
\subsection{Accountants' Reports}\label{accountants-reports}}

\begin{Shaded}
\begin{Highlighting}[]
\NormalTok{title: r 12.1}
\NormalTok{If you have, at any time during an accounting period, held or received client money, or operated a joint account or a client\textquotesingle{}s own account as signatory, you must:}

\NormalTok{{-} (a) obtain an accountant\textquotesingle{}s report for that accounting period within six months of the end of the period; and}
\NormalTok{{-} (b) deliver it to the SRA within six months of the end of the accounting period if the accountant\textquotesingle{}s report is qualified to show a failure to comply with these rules, such that money belonging to clients or thid parties is, or has been, or is likely to be placed, at risk.}
\end{Highlighting}
\end{Shaded}

These must be prepared and signed by a chartered account and is/ works
for a registered auditor.

\begin{Shaded}
\begin{Highlighting}[]
\NormalTok{title: r 12.2}
\NormalTok{You are not required to obtain an accountant\textquotesingle{}s report if:}

\NormalTok{{-} (a) all of the client money held or received during an accounting period is money received from the Legal Aid Agency; or}
\NormalTok{{-} (b) in the accounting period, the statement or passbook balance of client money you have held or received does not exceed:}
\NormalTok{    {-} (i) an average of £10,000; and}
\NormalTok{    {-} (ii) a maximum of £250,000,}
\NormalTok{{-} (c) or the equivalent in foreign currency.}
\end{Highlighting}
\end{Shaded}

\hypertarget{accounting-to-client-for-interest}{%
\section{Accounting to Client for
Interest}\label{accounting-to-client-for-interest}}

\hypertarget{obligation-to-account}{%
\subsection{Obligation to Account}\label{obligation-to-account}}

Relevant legislation:

\begin{longtable}[]{@{}
  >{\raggedright\arraybackslash}p{(\columnwidth - 2\tabcolsep) * \real{0.0875}}
  >{\raggedright\arraybackslash}p{(\columnwidth - 2\tabcolsep) * \real{0.9125}}@{}}
\toprule()
\begin{minipage}[b]{\linewidth}\raggedright
Statute
\end{minipage} & \begin{minipage}[b]{\linewidth}\raggedright
Description
\end{minipage} \\
\midrule()
\endhead
Principle 5 & You act with integrity \\
Principle 7 & You act in the best interests of each client. \\
para 1.2 Code for Firms & You do not abuse your position by taking
unfair advantage of clients or others. \\
para 5.1 Code for Firms & You properly account to clients for any
financial benefit you receive as a result of their instructions, except
where they have agreed otherwise. \\
r 7.1 Accounts Rules & You account to clients or third parties for a
fair sum of interest on any client money held by you on their behalf. \\
r 7.2 Accounts Rules & You may by a written agreement come to a
different arrangement with the client or the third party for whom the
money is held as to the payment of interest, but you must provide
sufficient information to enable them to give informed consent. \\
\bottomrule()
\end{longtable}

So it is up to firms to decide what is a fair way to calculate interest
and how often to pay it. Usually there is a minimum threshold set for
interest to be payable to the client.

\hypertarget{dealing-with-interest}{%
\subsubsection{Dealing with Interest}\label{dealing-with-interest}}

Firms can choose to pay client money into:

\begin{enumerate}
\def\labelenumi{\arabic{enumi}.}
\tightlist
\item
  A general client bank account, or
\item
  A separate deposit bank account for that particular client.
\end{enumerate}

Usually a separate account is used where a significant amount of client
money is to be held for some time. Generally, firms will account to
clients for all interest earned on money in designated accounts ⇾ acting
in the client's best interests.

When a general client account is used, the firm has a bit more
discretion. Should aim to earn interest on client money and have that
interest paid into the business bank account. A sum is then paid from
the business bank account as an expense.

\hypertarget{choice-of-method}{%
\subsubsection{Choice of Method}\label{choice-of-method}}

Under the Solicitors Act 1974, solicitors are allowed to keep any
interest earned over and above what is required to be paid under the
rules. By putting all client money into a single account, the firm can
earn a high rate of interest, and only pay out a lower `fair' rate of
interest to a particular client.

\hypertarget{separate-deposit-account}{%
\subsection{Separate Deposit Account}\label{separate-deposit-account}}

When opening a separate deposit account, ensure the recording
requirements of r 8.1 Accounts Rules is complied with (client ledger and
cash ledger records).

\hypertarget{general-client-account}{%
\subsection{General Client Account}\label{general-client-account}}

The firm has to calculate how much would have been earned in respect to
each individual client. A well-organised firm will put a proportion of
client money on deposit in a general deposit bank account. Then it
should be able to make money.

\hypertarget{vat}{%
\section{VAT}\label{vat}}

\hypertarget{principles-1}{%
\subsection{Principles}\label{principles-1}}

See also {[}{[}Property Taxation\#VAT{]}{]}.

\begin{itemize}
\tightlist
\item
  A business registered for VAT charges customers \textbf{output tax},
  paid to HMRC
\item
  Normally possible to deduct input tax charged to the business from the
  amount accounted for to HMRC.
\end{itemize}

\hypertarget{output-tax}{%
\subsubsection{Output Tax}\label{output-tax}}

\begin{Shaded}
\begin{Highlighting}[]
\NormalTok{VAT is chargeable on the supply of goods or services where the supply is (s 4(1) VAT Act 1994)}
\NormalTok{1. a taxable supply }
\NormalTok{2. made by a taxable person }
\NormalTok{3. in the course or furtherance of a business carried out by them. }
\end{Highlighting}
\end{Shaded}

\begin{longtable}[]{@{}
  >{\raggedright\arraybackslash}p{(\columnwidth - 2\tabcolsep) * \real{0.0451}}
  >{\raggedright\arraybackslash}p{(\columnwidth - 2\tabcolsep) * \real{0.9549}}@{}}
\toprule()
\begin{minipage}[b]{\linewidth}\raggedright
Element
\end{minipage} & \begin{minipage}[b]{\linewidth}\raggedright
Explanation
\end{minipage} \\
\midrule()
\endhead
Supply of goods & All forms of supply whereby the whole property in
goods is transferred, including a gift. \\
Supply of services & Anything which is not a supply of goods but is done
for consideration. The gratuitous supply of services is not a supply for
VAT purposes. \\
Taxable supply & Any supply of goods or services other than an exempt
supply. Exempt supplies listed in Sch 9 VATA 1994 (including most
supplies of land, insurance, postal services, finance health, burial,
cremation). Zero-rated supplies listed in Sch 8 VATA 1994 (food, water,
books, transport). A reduced rate of 5\% applies to some products, like
domestic fuel. Legal services are standard rated. \\
Taxable person & A person who is or is required to be registered under
the Act. Value of taxable supplies in 12 months must exceed £85,000.
Voluntary registration is always permitted. \\
Business & Includes any trade, profession, or vocation, as well as
provisions by some clubs/ associations. Where a person in the course of
business accepts any office, any service supplied by them as holder of
the office shall be treated as supplied in the course of business. \\
\bottomrule()
\end{longtable}

\begin{Shaded}
\begin{Highlighting}[]
\NormalTok{If a law firm is selling off old office equipment, the sales also attract VAT. }
\end{Highlighting}
\end{Shaded}

\hypertarget{input-tax}{%
\subsubsection{Input Tax}\label{input-tax}}

If a taxable person is charged VAT on the supply of goods services for
the purposes of their business, they may deduct the tax charged to them
from the amount of output tax which they account for to HMRC (s 25(2)
VATA 1994).

\begin{Shaded}
\begin{Highlighting}[]
\NormalTok{title: What is a taxable person makes both taxable and exempt supplies?}
\NormalTok{They are partly exempt and may recover a proportion of input tax charged to them. Exception: when exempt supplies fall within de miimus limits, they can be ignored. }
\end{Highlighting}
\end{Shaded}

\hypertarget{value-of-supply}{%
\subsubsection{Value of Supply}\label{value-of-supply}}

Where a supply is fully taxable, standard rate VAT (20\%) is payable on
the value of the supply.

\[\text{Value of supply} = \text{Consideration} - \text{Tax payable}\]

A quoted price is deemed to be inclusive of VAT unless otherwise
expressly stated.

\hypertarget{time-of-supply}{%
\subsubsection{Time of Supply}\label{time-of-supply}}

VAT is payable quarterly. So the time of supply (``tax point'')
determines the quarter in which a taxable person can claim input tax on
a taxable supply and are liable to output tax.

\begin{longtable}[]{@{}
  >{\raggedright\arraybackslash}p{(\columnwidth - 2\tabcolsep) * \real{0.1026}}
  >{\raggedright\arraybackslash}p{(\columnwidth - 2\tabcolsep) * \real{0.8974}}@{}}
\toprule()
\begin{minipage}[b]{\linewidth}\raggedright
Supply
\end{minipage} & \begin{minipage}[b]{\linewidth}\raggedright
Basic tax point
\end{minipage} \\
\midrule()
\endhead
Goods & When the goods are removed or made available to the purchaser (s
6(2)) \\
Services & When the services are completed (s 6(3)). \\
\bottomrule()
\end{longtable}

Variations:

\begin{itemize}
\tightlist
\item
  If within 14 days of the basic tax point, the supplier issues a tax
  invoice, the date of the invoice becomes the tax point, subject to any
  agreement with HMRC (ss 6(5) \& (6))
\item
  For solicitors, the 14 days is extended to 3 months
\item
  If, before a basic tax point arises, the supplier issues a tax invoice
  or receives payment, the supply will be treated as taking place at the
  date of the invoice or payment (s 6(4)).
\end{itemize}

\hypertarget{tax-invoices}{%
\subsubsection{Tax Invoices}\label{tax-invoices}}

A taxable person making a taxable supply to another taxable person must
within 30 days of the time of supply (subject to extensions agreed with
HMRC) provide them with a tax invoice. The tax invoice must contain
various details.

\hypertarget{collection-and-accounts}{%
\subsubsection{Collection and Accounts}\label{collection-and-accounts}}

Within a month of the end of each quarter, a taxable person must submit
a completed return form to HMRC together with a tax remittance. The
amount payable is obtained from a statutory VAT account.

\hypertarget{law-firms}{%
\subsection{Law Firms}\label{law-firms}}

Firms providing legal services must charge VAT.

\hypertarget{disbursements}{%
\subsubsection{Disbursements}\label{disbursements}}

Disbursements are not regarded by HMRC as part of the supply of legal
services, so the firm does \textbf{not} have to charge VAT on them. To
qualify as a disbursement, all the following conditions must be met
(para 25.1.1 VAT Notice 700):

\begin{enumerate}
\def\labelenumi{\arabic{enumi}.}
\tightlist
\item
  you acted as the agent of your client when you paid the third party;
\item
  your client actually received and used the goods or services provided
  by the third party (this condition usually prevents the agent's own
  travelling and subsistence expenses, telephone bills, postage, and
  other costs being treated as disbursements for VAT purposes);
\item
  your client was responsible for paying the third party (examples
  include estate duty and stamp duty payable by your client on a
  contract to be made by the client);
\item
  your client authorised you to make the payment on their behalf;
\item
  your client knew that the goods or services you paid for would be
  provided by a third party;
\item
  your outlay will be separately itemised when you invoice your client;
  you recover only the exact amount which you paid to the third party;
  and
\item
  the goods or services, which you paid for, are clearly additional to
  the supplies which you make to your client on your own account.
\end{enumerate}

The principle behind this is that there is a distinction between
{[}{[}Nell Gwynn House Maintenance Fund Trustees v C\&E Commissioners
{[}1999{]} STC 79{]}{]}:

\begin{itemize}
\tightlist
\item
  Expenses paid to a 3rd party incurred in the course of making your own
  supply of services to the client which are part of the whole of the
  services rendered to the client.
\item
  Expenses for specific services that have been supplied by the 3rd
  party to your client, and you have acted as a client's known and
  authorised representative in paying the third party.
\end{itemize}

\begin{Shaded}
\begin{Highlighting}[]
\NormalTok{title: Disbursements}
\NormalTok{{-} IHT}
\NormalTok{{-} CGT}
\NormalTok{{-} Stampt duty}
\NormalTok{{-} Estate agents\textquotesingle{} fees}
\NormalTok{{-} Counsel\textquotesingle{}s fees {-} even though the solicitor is responsible for ensuring payment. }
\NormalTok{{-} Land Registry fees for registration of title}
\end{Highlighting}
\end{Shaded}

Broadly, these can be divided into the categories of statutory charges
and charges for the professional services of a 3rd party.

\hypertarget{not-disbursements}{%
\subsubsection{Not Disbursements}\label{not-disbursements}}

Items which are a necessary part of the service supplied to a client.

\begin{Shaded}
\begin{Highlighting}[]
\NormalTok{title: Not disbursements}
\NormalTok{{-} Telephone charges}
\NormalTok{{-} Postage and photocopying charges}
\NormalTok{{-} Travelling expenses incurred by solicitor ([[Rowe \& Maw (A Firm) v Customs and Excise Commissioners [1975] 1 WLR 1291]])}
\NormalTok{{-} Electronic search fee ([[Brabners LLP v HMRC [2017] UKFTT 0666 (TC)]])}
\end{Highlighting}
\end{Shaded}

\hypertarget{search-fees}{%
\paragraph{Search Fees}\label{search-fees}}

Whether a fee for a search is to be treated as a disbursement will
depend on how information obtained in the search is used. If passed on
to the client without comment or analysis, may be treated as a
disbursement ({[}{[}Barratt, Goff and Tomlinson (a firm) v HMRC (Law
Society Intervening) {[}2011{]} UKFTT 71 (TC){]}{]}). But it is unusual
for this to be the case. More commonly, the solicitor will prepare
advice/ a report based on the search. Therefore, the search will be part
of the overall service and subject to VAT.

The same item may amount to a disbursement in one situation, but not in
another. Fees for Official Copy entries are unlikely to fulfil the
criteria for a disbursement.

\hypertarget{accounting-non-disbursements}{%
\subsubsection{Accounting
Non-disbursements}\label{accounting-non-disbursements}}

Firms usually do not make a separate charge for overheads such as
photocopying and phone calls. Possible to deal with search fees
similarly.

\begin{itemize}
\tightlist
\item
  CR cash -- business section
\item
  DR searches account.
\item
  DR HMRC account if applicable.
\end{itemize}

Firms will normally want a record of the searches incurred for each
client, so that they are shown separately from professional charges.
They should then record these on the client ledger.

\begin{itemize}
\tightlist
\item
  CR cash -- business section
\item
  DR client ledger -- business section with VAT exclusive amount. Make a
  note the payment is part of the firm's taxable supply.
\item
  DR HMRC account with the VAT input tax charged to the firm.
\end{itemize}

Where the original supplier charges VAT, the solicitor will treat the
VAT as input tax. The VAT exclusive amount will be added to the supply
and the solicitor will charge output tax.

\hypertarget{accounting-disbursements}{%
\subsubsection{Accounting
Disbursements}\label{accounting-disbursements}}

\hypertarget{non-taxable-disbursements}{%
\paragraph{Non-taxable Disbursements}\label{non-taxable-disbursements}}

Paid out of the client bank account if there is sufficient money, else
out of the business bank account. There are no VAT considerations.

\hypertarget{taxable-disbursements}{%
\paragraph{Taxable Disbursements}\label{taxable-disbursements}}

Payment made by the firm will include a VAT element, which is passed
onto the client. Clients who are registered for VAT will want to recover
this from HMRC, so need a VAT invoice addressed to them. But the
supplier may have addressed the invoice to the firm making the payment.

If addressed to the client, the agency method is used. If addressed to
the firm, the principal method is used, requiring the solicitor to
resupply the service to the client.

\hypertarget{agency-method}{%
\paragraph{Agency Method}\label{agency-method}}

The firm acts as the agent. Payment made using client bank account if
sufficient credit, else business bank account. Only the total amount
paid needs to be recorded. The firm then sends the supplier's tax
invoice to the client.

\hypertarget{principal-method}{%
\paragraph{Principal Method}\label{principal-method}}

If the invoice is addressed to the firm, the supply is treated as made
to the firm.

\begin{itemize}
\tightlist
\item
  Firm claims supply as input
\item
  Firm uses business money to pay supplier fees + input tax
\item
  Firm resupplies the item to the client at the same price.
\item
  Firm charges client output tax on the firm's professional charges, and
  the disbursement
\item
  Firm provides 1 invoice to cover the above.
\end{itemize}

\begin{Shaded}
\begin{Highlighting}[]
\NormalTok{A disbursement paid on the principal method must be paid out of the business bank account, even if there is client money available. Supply is treated as being made to the firm, not the client. }
\end{Highlighting}
\end{Shaded}

\hypertarget{counsel-fees}{%
\subsubsection{Counsel Fees}\label{counsel-fees}}

Counsel fees usually addressed to the firm and would have to be treated
on the principal basis. But HMRC has agreed a concession. The firm is
allowed to alter the fee note so that it is dressed to the client. It is
then treated on the agency basis. The firm must give the fee note to the
client. The firm should keep a copy of the amended receipted fee note
for assessment of costs purposes.

\hypertarget{accounting-problems}{%
\section{Accounting Problems}\label{accounting-problems}}

One problem is that a large number of client ledgers are required, with
frequent inter-client transactions.

\hypertarget{stakeholder-money}{%
\subsection{Stakeholder Money}\label{stakeholder-money}}

May receive a deposit to hold as stakeholder. This is a receipt of
client money, so must be held in the client bank account. This is held
jointly for the buyer and seller. It will not become the property of the
seller unless and until completion.

\begin{Shaded}
\begin{Highlighting}[]
\NormalTok{title: Law Society guidance}
\NormalTok{Stakeholder money may be shown on the seller\textquotesingle{}s ledger but must be clearly labelled as stakeholder money held for both buyer and seller. }
\end{Highlighting}
\end{Shaded}

Alternatively, use a separate stakeholder ledger. If a separate
stakeholder ledger is used, once monies received from the buyer:

\begin{itemize}
\tightlist
\item
  DR cash
\item
  CR Joint stakeholder ledger.
\end{itemize}

Then on completion:

\begin{itemize}
\tightlist
\item
  DR joint stakeholder ledger
\item
  CR seller's ledger
\end{itemize}

\hypertarget{bridging-finance}{%
\subsection{Bridging Finance}\label{bridging-finance}}

A deposit received as stakeholder is not available to the seller until
completion (subject to agreement to the contrary -- Standard Conditions
SC 2.2.5). Often a bridging loan is obtained from a bank to cover the
period from exchange of contracts to completion of the sale. This is a
personal loan to the borrower. So, when received, credited to the
borrower's ledger account.

\hypertarget{mortgages}{%
\subsection{Mortgages}\label{mortgages}}

A client who sells property subject to a mortgage will have to redeem
the mortgage after completion. If you act for a client buying a
property, may also act for the lender, provided there is no conflict of
interest.

\begin{itemize}
\tightlist
\item
  Mortgage advance received from a lender is usually held for the lender
  until the day of completion
\item
  r 8.1: receipts and payments of client money must be recorded on
  client ledgers identified by the client's name and description.
\item
  Firms acting for both the lender and borrower can choose to either:

  \begin{enumerate}
  \def\labelenumi{\arabic{enumi}.}
  \tightlist
  \item
    Credit the mortgage advance to the borrower's ledger account.
    Details column should include the name of the lender and ``mortgage
    advance''
  \item
    Mortgage advance credited to a separate ledger account in the name
    of the lender on receipt. On completion, perform an inter-client
    audit trail.
  \end{enumerate}
\end{itemize}

\hypertarget{professional-charges}{%
\subsubsection{Professional Charges}\label{professional-charges}}

Entitled to charge the lender for work done in connection with the
mortgage advance and charge the buyer for work done in connection with
the purchase. The buyer may have agreed with the lender to pay the costs
charged to the lender.

Normal rule: costs and VAT must be debited to the ledger account of the
person to whom legal services were supplied.

\hypertarget{mortgage-redemption}{%
\subsubsection{Mortgage Redemption}\label{mortgage-redemption}}

Many sellers have a balance left on their mortgage at the date of sale.
The balance must be paid off from the proceeds of sale.

If acting for both the lender and seller (rare but possible), some sale
proceeds will be received for the seller and some for the lender. The
whole receipt of client money will be paid into the client bank account.
Just make sure there is a clear audit trail.

Options:

\begin{enumerate}
\def\labelenumi{\arabic{enumi}.}
\tightlist
\item
  Credit the whole amount to the seller's ledger account initially, and
  then do an immediate inter-client transfer of the amount required to
  redeem the mortgage.
\item
  Split the credit entries at the time of receipt. Credit part of the
  proceeds to the seller's ledger account and part to the lender's
  ledger account. Debit entire amount to cash account client section.
\end{enumerate}

\hypertarget{legal-fees-on-mortgage-redemption}{%
\subsubsection{Legal Fees on Mortgage
Redemption}\label{legal-fees-on-mortgage-redemption}}

The seller may have agreed to pay the lender's legal fees, in which case
a transfer should be made.

\hypertarget{agency-transactions}{%
\subsection{Agency Transactions}\label{agency-transactions}}

A firm may decide to use another firm as its agent. This frequently
happens in litigation.

\hypertarget{agency-firm}{%
\subsubsection{Agency Firm}\label{agency-firm}}

Treats the instructing firm like any other client.

\hypertarget{instructing-firm}{%
\subsubsection{Instructing Firm}\label{instructing-firm}}

Professional fees charged by the agent are not a disbursement paid by
the instructing firm on behalf of the client. The professional fees of
the agent are an expense of the instructing firm. The firm will charge
enough for its legal services to cover the expense of using an agent.
Any true disbursements will be charged to the client as usual.

When paying an agent's bill, send one business bank account cheque for
the total amount. 3 elements:

\begin{enumerate}
\def\labelenumi{\arabic{enumi}.}
\tightlist
\item
  Agent's professional fees (DR agency expenses account, business
  section)
\item
  VAT on these fees (DR HMRC account, business section)
\item
  Any disbursements paid by the agent (DR client account ledger,
  business section)
\end{enumerate}

Record these on the separate relevant accounts.

\hypertarget{tp-managed-accounts}{%
\section{TP Managed Accounts}\label{tp-managed-accounts}}

\hypertarget{rules}{%
\subsection{Rules}\label{rules}}

\begin{Shaded}
\begin{Highlighting}[]
\NormalTok{title: r 11.1 Accounts\textquotesingle{} Rules}
\NormalTok{You may enter into arrangements with a client to use a third party managed account for the purpose of receiving payments from or on behalf of, or making payments to or on behalf of, the client in respect of regulated services delivered by you to the client, only if:}
\NormalTok{{-} (a) use of the account does not result in you receiving or holding the client\textquotesingle{}s money; and}
\NormalTok{{-} (b) you take reasonable steps to ensure, before accepting instructions, that the client is informed of and understands:}
\NormalTok{    {-} (i) the terms of the contractual arrangements relating to the use of the third party managed account, and in particular how any fees for use of the third party managed account will be paid and who will bear them; and}
\NormalTok{    {-} (ii) the client\textquotesingle{}s right to terminate the agreement and dispute payment requests made by you.}
\end{Highlighting}
\end{Shaded}

\begin{Shaded}
\begin{Highlighting}[]
\NormalTok{title: r 11.2}
\NormalTok{You obtain regular statements from the provider of the third party managed account and ensure that these accurately reflect all transactions on the account.}
\end{Highlighting}
\end{Shaded}

To protect client money and assets (para 4.2 SRA Code for Solicitors)
and to act in the best interests of each client (Principle 7), a
solicitor must ensure that a decision to use a TPMA is appropriate in
each individual case.

\hypertarget{checks}{%
\subsection{Checks}\label{checks}}

\begin{itemize}
\tightlist
\item
  The TPMA must be authorised and regulated by the FCA. This means it
  must be:

  \begin{itemize}
  \tightlist
  \item
    An authorised payment institution
  \item
    A small payment institution which has adopted voluntary safeguarding
    arrangements to the same level as an API
  \item
    An EEA authorised payment institution.
  \end{itemize}
\item
  TPMA must be an account held at a bank or building society operated as
  an escrow payment service.
\end{itemize}

\hypertarget{engaging-client}{%
\subsection{Engaging Client}\label{engaging-client}}

Firm must take appropriate steps to ensure that:

\begin{itemize}
\tightlist
\item
  Client is informed and understands their rights and obligations. Also
  whether they are required to authorise payments and any fees they may
  have to pay.
\item
  Obtains regular statements and ensure these reflect the transactions
  on the account correctly.
\item
  Makes sure funds in the TPMA are only used for the designated purpose.
\item
  Keep overview records of transactions.
\end{itemize}

Para 2.1 Code for Firms; the firm must have in place effective systems
and controls to ensure that the business and employees comply with SRA
regulatory arrangements. SRA should be notified that a firm is using a
TPMA, using a TPMA form.

\hypertarget{tell-clients}{%
\subsection{Tell Clients}\label{tell-clients}}

\begin{itemize}
\tightlist
\item
  How the money will be held and how the transaction will work
  (paragraph 8.6 of the SRA Code of Conduct for Solicitors, RELs and
  RFLs)
\item
  their right to terminate the agreement (rule 11.1(b)(ii) of the
  accounts rules)
\item
  their right to dispute payment requests made by you
\item
  who will be responsible for costs associated with the agreement (rule
  11.1(b)(i) of the Accounts Rules)
\item
  that the TPMA is regulated by the FCA and that complaints about the
  TPMA provider should be made to that provider in accordance with their
  complaints procedure, and
\item
  that the regulatory protections that apply to TPMAs are different to
  those that apply to client money held in a firm's client account
  (paragraph 8.11 of the SRA Code of Conduct for Solicitors, RELs and
  RFLs).
\end{itemize}

\hypertarget{compliance}{%
\section{Compliance}\label{compliance}}

\hypertarget{accountants-reports-1}{%
\subsection{Accountants' Reports}\label{accountants-reports-1}}

\begin{Shaded}
\begin{Highlighting}[]
\NormalTok{title: r 12.1}
\NormalTok{If you have, at any time during an accounting period, held or received client money, or operated a joint account or a client\textquotesingle{}s own account as signatory, you must:}

\NormalTok{{-} (a) obtain an accountant\textquotesingle{}s report for that accounting period within six months of the end of the period; and}
\NormalTok{{-} (b) deliver it to the SRA within six months of the end of the accounting period if the accountant\textquotesingle{}s report is qualified to show a failure to comply with these rules, such that money belonging to clients or third parties is, or has been, or is likely to be placed, at risk.}
\end{Highlighting}
\end{Shaded}

The obligation to send qualified reports rests with the firm and its
managers. ``Qualified'' is not defined, but some examples of serious and
moderate factors are given. Serious factors include a significant
shortfall on client account, systematic billing for fees that have not
occurred, disregard for the safety of client money,\ldots{}

\begin{Shaded}
\begin{Highlighting}[]
\NormalTok{title: r 12.2}
\NormalTok{You are not required to obtain an accountant\textquotesingle{}s report if:}

\NormalTok{{-} (a) all of the client money held or received during an accounting period is money received from the Legal Aid Agency; or}
\NormalTok{{-} (b) in the accounting period, the statement or passbook balance of client money you have held or received does not exceed:}
\NormalTok{    {-} (i) an average of £10,000; and}
\NormalTok{    {-} (ii) a maximum of £250,000,}
\NormalTok{{-} or the equivalent in foreign currency.}
\end{Highlighting}
\end{Shaded}

\begin{itemize}
\tightlist
\item
  When a firm shuts down, the SRA may require it to send the SRA an
  accountant's report on reasonable notice (r 12.4).
\item
  SRA may disqualify an accountant from preparing a report if they have
  been found negligent/ guilty of misconduct (r 12.6).
\item
  Accountants must be provided with relevant details (r 12.8)
\item
  All accounting records must be kept for at least 6 years.
\end{itemize}

\end{document}
