% Options for packages loaded elsewhere
\PassOptionsToPackage{unicode}{hyperref}
\PassOptionsToPackage{hyphens}{url}
%
\documentclass[
]{article}
\usepackage{amsmath,amssymb}
\usepackage{lmodern}
\usepackage{iftex}
\ifPDFTeX
  \usepackage[T1]{fontenc}
  \usepackage[utf8]{inputenc}
  \usepackage{textcomp} % provide euro and other symbols
\else % if luatex or xetex
  \usepackage{unicode-math}
  \defaultfontfeatures{Scale=MatchLowercase}
  \defaultfontfeatures[\rmfamily]{Ligatures=TeX,Scale=1}
\fi
% Use upquote if available, for straight quotes in verbatim environments
\IfFileExists{upquote.sty}{\usepackage{upquote}}{}
\IfFileExists{microtype.sty}{% use microtype if available
  \usepackage[]{microtype}
  \UseMicrotypeSet[protrusion]{basicmath} % disable protrusion for tt fonts
}{}
\makeatletter
\@ifundefined{KOMAClassName}{% if non-KOMA class
  \IfFileExists{parskip.sty}{%
    \usepackage{parskip}
  }{% else
    \setlength{\parindent}{0pt}
    \setlength{\parskip}{6pt plus 2pt minus 1pt}}
}{% if KOMA class
  \KOMAoptions{parskip=half}}
\makeatother
\usepackage{xcolor}
\usepackage[margin=1in]{geometry}
\usepackage{color}
\usepackage{fancyvrb}
\newcommand{\VerbBar}{|}
\newcommand{\VERB}{\Verb[commandchars=\\\{\}]}
\DefineVerbatimEnvironment{Highlighting}{Verbatim}{commandchars=\\\{\}}
% Add ',fontsize=\small' for more characters per line
\newenvironment{Shaded}{}{}
\newcommand{\AlertTok}[1]{\textcolor[rgb]{1.00,0.00,0.00}{\textbf{#1}}}
\newcommand{\AnnotationTok}[1]{\textcolor[rgb]{0.38,0.63,0.69}{\textbf{\textit{#1}}}}
\newcommand{\AttributeTok}[1]{\textcolor[rgb]{0.49,0.56,0.16}{#1}}
\newcommand{\BaseNTok}[1]{\textcolor[rgb]{0.25,0.63,0.44}{#1}}
\newcommand{\BuiltInTok}[1]{#1}
\newcommand{\CharTok}[1]{\textcolor[rgb]{0.25,0.44,0.63}{#1}}
\newcommand{\CommentTok}[1]{\textcolor[rgb]{0.38,0.63,0.69}{\textit{#1}}}
\newcommand{\CommentVarTok}[1]{\textcolor[rgb]{0.38,0.63,0.69}{\textbf{\textit{#1}}}}
\newcommand{\ConstantTok}[1]{\textcolor[rgb]{0.53,0.00,0.00}{#1}}
\newcommand{\ControlFlowTok}[1]{\textcolor[rgb]{0.00,0.44,0.13}{\textbf{#1}}}
\newcommand{\DataTypeTok}[1]{\textcolor[rgb]{0.56,0.13,0.00}{#1}}
\newcommand{\DecValTok}[1]{\textcolor[rgb]{0.25,0.63,0.44}{#1}}
\newcommand{\DocumentationTok}[1]{\textcolor[rgb]{0.73,0.13,0.13}{\textit{#1}}}
\newcommand{\ErrorTok}[1]{\textcolor[rgb]{1.00,0.00,0.00}{\textbf{#1}}}
\newcommand{\ExtensionTok}[1]{#1}
\newcommand{\FloatTok}[1]{\textcolor[rgb]{0.25,0.63,0.44}{#1}}
\newcommand{\FunctionTok}[1]{\textcolor[rgb]{0.02,0.16,0.49}{#1}}
\newcommand{\ImportTok}[1]{#1}
\newcommand{\InformationTok}[1]{\textcolor[rgb]{0.38,0.63,0.69}{\textbf{\textit{#1}}}}
\newcommand{\KeywordTok}[1]{\textcolor[rgb]{0.00,0.44,0.13}{\textbf{#1}}}
\newcommand{\NormalTok}[1]{#1}
\newcommand{\OperatorTok}[1]{\textcolor[rgb]{0.40,0.40,0.40}{#1}}
\newcommand{\OtherTok}[1]{\textcolor[rgb]{0.00,0.44,0.13}{#1}}
\newcommand{\PreprocessorTok}[1]{\textcolor[rgb]{0.74,0.48,0.00}{#1}}
\newcommand{\RegionMarkerTok}[1]{#1}
\newcommand{\SpecialCharTok}[1]{\textcolor[rgb]{0.25,0.44,0.63}{#1}}
\newcommand{\SpecialStringTok}[1]{\textcolor[rgb]{0.73,0.40,0.53}{#1}}
\newcommand{\StringTok}[1]{\textcolor[rgb]{0.25,0.44,0.63}{#1}}
\newcommand{\VariableTok}[1]{\textcolor[rgb]{0.10,0.09,0.49}{#1}}
\newcommand{\VerbatimStringTok}[1]{\textcolor[rgb]{0.25,0.44,0.63}{#1}}
\newcommand{\WarningTok}[1]{\textcolor[rgb]{0.38,0.63,0.69}{\textbf{\textit{#1}}}}
\usepackage{longtable,booktabs,array}
\usepackage{calc} % for calculating minipage widths
% Correct order of tables after \paragraph or \subparagraph
\usepackage{etoolbox}
\makeatletter
\patchcmd\longtable{\par}{\if@noskipsec\mbox{}\fi\par}{}{}
\makeatother
% Allow footnotes in longtable head/foot
\IfFileExists{footnotehyper.sty}{\usepackage{footnotehyper}}{\usepackage{footnote}}
\makesavenoteenv{longtable}
\setlength{\emergencystretch}{3em} % prevent overfull lines
\providecommand{\tightlist}{%
  \setlength{\itemsep}{0pt}\setlength{\parskip}{0pt}}
\setcounter{secnumdepth}{-\maxdimen} % remove section numbering
\usepackage{xcolor}
\definecolor{aliceblue}{HTML}{F0F8FF}
\definecolor{antiquewhite}{HTML}{FAEBD7}
\definecolor{aqua}{HTML}{00FFFF}
\definecolor{aquamarine}{HTML}{7FFFD4}
\definecolor{azure}{HTML}{F0FFFF}
\definecolor{beige}{HTML}{F5F5DC}
\definecolor{bisque}{HTML}{FFE4C4}
\definecolor{black}{HTML}{000000}
\definecolor{blanchedalmond}{HTML}{FFEBCD}
\definecolor{blue}{HTML}{0000FF}
\definecolor{blueviolet}{HTML}{8A2BE2}
\definecolor{brown}{HTML}{A52A2A}
\definecolor{burlywood}{HTML}{DEB887}
\definecolor{cadetblue}{HTML}{5F9EA0}
\definecolor{chartreuse}{HTML}{7FFF00}
\definecolor{chocolate}{HTML}{D2691E}
\definecolor{coral}{HTML}{FF7F50}
\definecolor{cornflowerblue}{HTML}{6495ED}
\definecolor{cornsilk}{HTML}{FFF8DC}
\definecolor{crimson}{HTML}{DC143C}
\definecolor{cyan}{HTML}{00FFFF}
\definecolor{darkblue}{HTML}{00008B}
\definecolor{darkcyan}{HTML}{008B8B}
\definecolor{darkgoldenrod}{HTML}{B8860B}
\definecolor{darkgray}{HTML}{A9A9A9}
\definecolor{darkgreen}{HTML}{006400}
\definecolor{darkgrey}{HTML}{A9A9A9}
\definecolor{darkkhaki}{HTML}{BDB76B}
\definecolor{darkmagenta}{HTML}{8B008B}
\definecolor{darkolivegreen}{HTML}{556B2F}
\definecolor{darkorange}{HTML}{FF8C00}
\definecolor{darkorchid}{HTML}{9932CC}
\definecolor{darkred}{HTML}{8B0000}
\definecolor{darksalmon}{HTML}{E9967A}
\definecolor{darkseagreen}{HTML}{8FBC8F}
\definecolor{darkslateblue}{HTML}{483D8B}
\definecolor{darkslategray}{HTML}{2F4F4F}
\definecolor{darkslategrey}{HTML}{2F4F4F}
\definecolor{darkturquoise}{HTML}{00CED1}
\definecolor{darkviolet}{HTML}{9400D3}
\definecolor{deeppink}{HTML}{FF1493}
\definecolor{deepskyblue}{HTML}{00BFFF}
\definecolor{dimgray}{HTML}{696969}
\definecolor{dimgrey}{HTML}{696969}
\definecolor{dodgerblue}{HTML}{1E90FF}
\definecolor{firebrick}{HTML}{B22222}
\definecolor{floralwhite}{HTML}{FFFAF0}
\definecolor{forestgreen}{HTML}{228B22}
\definecolor{fuchsia}{HTML}{FF00FF}
\definecolor{gainsboro}{HTML}{DCDCDC}
\definecolor{ghostwhite}{HTML}{F8F8FF}
\definecolor{gold}{HTML}{FFD700}
\definecolor{goldenrod}{HTML}{DAA520}
\definecolor{gray}{HTML}{808080}
\definecolor{green}{HTML}{008000}
\definecolor{greenyellow}{HTML}{ADFF2F}
\definecolor{grey}{HTML}{808080}
\definecolor{honeydew}{HTML}{F0FFF0}
\definecolor{hotpink}{HTML}{FF69B4}
\definecolor{indianred}{HTML}{CD5C5C}
\definecolor{indigo}{HTML}{4B0082}
\definecolor{ivory}{HTML}{FFFFF0}
\definecolor{khaki}{HTML}{F0E68C}
\definecolor{lavender}{HTML}{E6E6FA}
\definecolor{lavenderblush}{HTML}{FFF0F5}
\definecolor{lawngreen}{HTML}{7CFC00}
\definecolor{lemonchiffon}{HTML}{FFFACD}
\definecolor{lightblue}{HTML}{ADD8E6}
\definecolor{lightcoral}{HTML}{F08080}
\definecolor{lightcyan}{HTML}{E0FFFF}
\definecolor{lightgoldenrodyellow}{HTML}{FAFAD2}
\definecolor{lightgray}{HTML}{D3D3D3}
\definecolor{lightgreen}{HTML}{90EE90}
\definecolor{lightgrey}{HTML}{D3D3D3}
\definecolor{lightpink}{HTML}{FFB6C1}
\definecolor{lightsalmon}{HTML}{FFA07A}
\definecolor{lightseagreen}{HTML}{20B2AA}
\definecolor{lightskyblue}{HTML}{87CEFA}
\definecolor{lightslategray}{HTML}{778899}
\definecolor{lightslategrey}{HTML}{778899}
\definecolor{lightsteelblue}{HTML}{B0C4DE}
\definecolor{lightyellow}{HTML}{FFFFE0}
\definecolor{lime}{HTML}{00FF00}
\definecolor{limegreen}{HTML}{32CD32}
\definecolor{linen}{HTML}{FAF0E6}
\definecolor{magenta}{HTML}{FF00FF}
\definecolor{maroon}{HTML}{800000}
\definecolor{mediumaquamarine}{HTML}{66CDAA}
\definecolor{mediumblue}{HTML}{0000CD}
\definecolor{mediumorchid}{HTML}{BA55D3}
\definecolor{mediumpurple}{HTML}{9370DB}
\definecolor{mediumseagreen}{HTML}{3CB371}
\definecolor{mediumslateblue}{HTML}{7B68EE}
\definecolor{mediumspringgreen}{HTML}{00FA9A}
\definecolor{mediumturquoise}{HTML}{48D1CC}
\definecolor{mediumvioletred}{HTML}{C71585}
\definecolor{midnightblue}{HTML}{191970}
\definecolor{mintcream}{HTML}{F5FFFA}
\definecolor{mistyrose}{HTML}{FFE4E1}
\definecolor{moccasin}{HTML}{FFE4B5}
\definecolor{navajowhite}{HTML}{FFDEAD}
\definecolor{navy}{HTML}{000080}
\definecolor{oldlace}{HTML}{FDF5E6}
\definecolor{olive}{HTML}{808000}
\definecolor{olivedrab}{HTML}{6B8E23}
\definecolor{orange}{HTML}{FFA500}
\definecolor{orangered}{HTML}{FF4500}
\definecolor{orchid}{HTML}{DA70D6}
\definecolor{palegoldenrod}{HTML}{EEE8AA}
\definecolor{palegreen}{HTML}{98FB98}
\definecolor{paleturquoise}{HTML}{AFEEEE}
\definecolor{palevioletred}{HTML}{DB7093}
\definecolor{papayawhip}{HTML}{FFEFD5}
\definecolor{peachpuff}{HTML}{FFDAB9}
\definecolor{peru}{HTML}{CD853F}
\definecolor{pink}{HTML}{FFC0CB}
\definecolor{plum}{HTML}{DDA0DD}
\definecolor{powderblue}{HTML}{B0E0E6}
\definecolor{purple}{HTML}{800080}
\definecolor{red}{HTML}{FF0000}
\definecolor{rosybrown}{HTML}{BC8F8F}
\definecolor{royalblue}{HTML}{4169E1}
\definecolor{saddlebrown}{HTML}{8B4513}
\definecolor{salmon}{HTML}{FA8072}
\definecolor{sandybrown}{HTML}{F4A460}
\definecolor{seagreen}{HTML}{2E8B57}
\definecolor{seashell}{HTML}{FFF5EE}
\definecolor{sienna}{HTML}{A0522D}
\definecolor{silver}{HTML}{C0C0C0}
\definecolor{skyblue}{HTML}{87CEEB}
\definecolor{slateblue}{HTML}{6A5ACD}
\definecolor{slategray}{HTML}{708090}
\definecolor{slategrey}{HTML}{708090}
\definecolor{snow}{HTML}{FFFAFA}
\definecolor{springgreen}{HTML}{00FF7F}
\definecolor{steelblue}{HTML}{4682B4}
\definecolor{tan}{HTML}{D2B48C}
\definecolor{teal}{HTML}{008080}
\definecolor{thistle}{HTML}{D8BFD8}
\definecolor{tomato}{HTML}{FF6347}
\definecolor{turquoise}{HTML}{40E0D0}
\definecolor{violet}{HTML}{EE82EE}
\definecolor{wheat}{HTML}{F5DEB3}
\definecolor{white}{HTML}{FFFFFF}
\definecolor{whitesmoke}{HTML}{F5F5F5}
\definecolor{yellow}{HTML}{FFFF00}
\definecolor{yellowgreen}{HTML}{9ACD32}
\usepackage[most]{tcolorbox}

\usepackage{ifthen}
\provideboolean{admonitiontwoside}
\makeatletter%
\if@twoside%
\setboolean{admonitiontwoside}{true}
\else%
\setboolean{admonitiontwoside}{false}
\fi%
\makeatother%

\newenvironment{env-33779924-6684-4a9c-9dc9-f8e9c2323334}
{
    \savenotes\tcolorbox[blanker,breakable,left=5pt,borderline west={2pt}{-4pt}{firebrick}]
}
{
    \endtcolorbox\spewnotes
}
                

\newenvironment{env-c8468814-8649-4fc7-a16e-5197047efdbc}
{
    \savenotes\tcolorbox[blanker,breakable,left=5pt,borderline west={2pt}{-4pt}{blue}]
}
{
    \endtcolorbox\spewnotes
}
                

\newenvironment{env-a65b98aa-78eb-4d68-ac06-3fc6a0488511}
{
    \savenotes\tcolorbox[blanker,breakable,left=5pt,borderline west={2pt}{-4pt}{green}]
}
{
    \endtcolorbox\spewnotes
}
                

\newenvironment{env-a7e70e98-00de-4681-b3a1-4460d4b65dd3}
{
    \savenotes\tcolorbox[blanker,breakable,left=5pt,borderline west={2pt}{-4pt}{aquamarine}]
}
{
    \endtcolorbox\spewnotes
}
                

\newenvironment{env-845eaea8-4999-42c2-8d96-ee1fd34e27f0}
{
    \savenotes\tcolorbox[blanker,breakable,left=5pt,borderline west={2pt}{-4pt}{orange}]
}
{
    \endtcolorbox\spewnotes
}
                

\newenvironment{env-b00743e6-8201-4246-94b1-ad49ae50bab7}
{
    \savenotes\tcolorbox[blanker,breakable,left=5pt,borderline west={2pt}{-4pt}{blue}]
}
{
    \endtcolorbox\spewnotes
}
                

\newenvironment{env-993e03a2-732f-4909-97ab-9a278f6342b7}
{
    \savenotes\tcolorbox[blanker,breakable,left=5pt,borderline west={2pt}{-4pt}{yellow}]
}
{
    \endtcolorbox\spewnotes
}
                

\newenvironment{env-766384b5-2cd2-4ee1-a333-c1b98e5ac0ad}
{
    \savenotes\tcolorbox[blanker,breakable,left=5pt,borderline west={2pt}{-4pt}{darkred}]
}
{
    \endtcolorbox\spewnotes
}
                

\newenvironment{env-fd1d7e90-ca13-4fb5-a952-d69e451d8988}
{
    \savenotes\tcolorbox[blanker,breakable,left=5pt,borderline west={2pt}{-4pt}{pink}]
}
{
    \endtcolorbox\spewnotes
}
                

\newenvironment{env-64a6bb08-2482-4974-add1-474d2194397d}
{
    \savenotes\tcolorbox[blanker,breakable,left=5pt,borderline west={2pt}{-4pt}{cyan}]
}
{
    \endtcolorbox\spewnotes
}
                

\newenvironment{env-d69cdc54-8d21-4855-9678-be921046bb32}
{
    \savenotes\tcolorbox[blanker,breakable,left=5pt,borderline west={2pt}{-4pt}{cyan}]
}
{
    \endtcolorbox\spewnotes
}
                

\newenvironment{env-66bb670e-b1c5-451e-a6c9-096aff82c4c3}
{
    \savenotes\tcolorbox[blanker,breakable,left=5pt,borderline west={2pt}{-4pt}{purple}]
}
{
    \endtcolorbox\spewnotes
}
                

\newenvironment{env-f6b2f07f-e9a8-4c1a-a3a6-f3d331473a6c}
{
    \savenotes\tcolorbox[blanker,breakable,left=5pt,borderline west={2pt}{-4pt}{darksalmon}]
}
{
    \endtcolorbox\spewnotes
}
                

\newenvironment{env-92409375-3f8a-4734-b37a-0c56510a8c05}
{
    \savenotes\tcolorbox[blanker,breakable,left=5pt,borderline west={2pt}{-4pt}{gray}]
}
{
    \endtcolorbox\spewnotes
}
                
\ifLuaTeX
  \usepackage{selnolig}  % disable illegal ligatures
\fi
\IfFileExists{bookmark.sty}{\usepackage{bookmark}}{\usepackage{hyperref}}
\IfFileExists{xurl.sty}{\usepackage{xurl}}{} % add URL line breaks if available
\urlstyle{same} % disable monospaced font for URLs
\hypersetup{
  hidelinks,
  pdfcreator={LaTeX via pandoc}}

\author{}
\date{}

\begin{document}

{
\setcounter{tocdepth}{3}
\tableofcontents
}
\begin{Shaded}
\begin{Highlighting}[]
\NormalTok{min\_depth: 1}
\end{Highlighting}
\end{Shaded}

\hypertarget{inheritance-tax}{%
\section{Inheritance Tax}\label{inheritance-tax}}

\hypertarget{introduction}{%
\subsection{Introduction}\label{introduction}}

Usually a solicitor does the tax calculations for inheritance tax
(accountants do income and capital gains tax). It has not kept pace with
inflation, so has in practice become a tax on the middle classes. Levied
on wealth at death. Governed principally by \textbf{Inheritance Tax Act
1984 (IHTA 1984)}. Charged on 3 main occasions:

\begin{enumerate}
\def\labelenumi{\arabic{enumi}.}
\tightlist
\item
  Death
\item
  Lifetime gifts made by individuals within 7 years of death

  \begin{itemize}
  \tightlist
  \item
    Such gifts known as `potentially exempt transfers' (PETs)
  \item
    If transferor survives for 7 years, becomes exempt.
  \item
    If transferor dies within 7 years, becomes chargeable.
  \end{itemize}
\item
  Lifetime gifts to a company or into a trust.

  \begin{itemize}
  \tightlist
  \item
    Such lifetime gifts are immediately chargeable to IHT at the time
    when it is made, unless the trust is for a disabled person.
  \end{itemize}
\end{enumerate}

IHT subject to `general anti-abuse rule' (GAAR) introduced by Finance
Act 2013 designed to allow HMRC to counteract abusive tax avoidance
schemes.

\hypertarget{calculation}{%
\subsection{Calculation}\label{calculation}}

\begin{enumerate}
\def\labelenumi{\arabic{enumi}.}
\tightlist
\item
  Identify the transfer of value

  \begin{itemize}
  \tightlist
  \item
    Must have been a transfer of value. This is a disposition which
    reduces the value of the transferor's estate. Basically, there must
    have been a gift.
  \item
    You are deemed to have made a gift when you die. So death is a
    deemed transfer of value.
  \item
    Death estate: all the assets to which the deceased was beneficially
    entitled immediately before death.
  \end{itemize}
\item
  Find the value transferred.

  \begin{itemize}
  \tightlist
  \item
    Upon death, value transferred is market value less debts and funeral
    expenses.
  \item
    For a lifetime transfer, it is the amount of the reduction in the
    transferor's estate.
  \end{itemize}
\item
  Apply exemptions and reliefs.

  \begin{itemize}
  \tightlist
  \item
    Depends on status of person to whom assets are transferred, or
    nature of assets.
  \item
    Spouse exemption: full exemption between spouses.
  \item
    Charity exemption
  \item
    Business and agricultural property relief
  \item
    Annual exemption

    \begin{itemize}
    \tightlist
    \item
      £3,000
    \item
      Applies only to lifetime transfers.
    \end{itemize}
  \end{itemize}
\item
  Calculate tax at appropriate rate

  \begin{itemize}
  \tightlist
  \item
    Residence nil rate band

    \begin{itemize}
    \tightlist
    \item
      Must be a ``qualifying residential interest''
    \item
      ``Closely inherited''--lineal descendants
    \item
      First £175,000 taxed at 0\%
    \item
      Only available on death
    \end{itemize}
  \item
    Nil rate band

    \begin{itemize}
    \tightlist
    \item
      £325,000 taxed at 0\%
    \item
      Complicated by cumulation--relates to the nil rate band
    \item
      Can be passed onto spouse/ civil partner.
    \end{itemize}
  \item
    So potentially £1 million taxed at 0\%.
  \end{itemize}
\end{enumerate}

Chargeable transfers eat up the nil rate band.

\begin{Shaded}
\begin{Highlighting}[]
\NormalTok{It is not correct to say that the property falling into the nil rate band is "exempt". When something is exempt it has no IHT implications.}
\end{Highlighting}
\end{Shaded}

\hypertarget{transfers-on-death}{%
\subsection{Transfers on Death}\label{transfers-on-death}}

\hypertarget{transfer-of-value}{%
\subsubsection{1. Transfer of Value}\label{transfer-of-value}}

The value transferred is the value of the deceased's estate immediately
before death.

\hypertarget{estate}{%
\paragraph{Estate}\label{estate}}

Defined in s 5(1) IHTA 1984 to mean all the property to which a person
was beneficially entitled immediately before their death, with the
exception of `excluded property'. So it includes:

\begin{itemize}
\tightlist
\item
  Property passing under will/ intestacy

  \begin{itemize}
  \tightlist
  \item
    Deceased ``beneficially entitled'' to all such property immediately
    before death.
  \end{itemize}
\item
  Other property to which the deceased was beneficially entitled
  immediately before their death.

  \begin{itemize}
  \tightlist
  \item
    Interest in any joint property passing on death by survivorship to
    surviving joint tenants.
  \end{itemize}
\item
  Property included because of special statutory provisions.

  \begin{itemize}
  \tightlist
  \item
    Certain trust properties

    \begin{itemize}
    \tightlist
    \item
      In certain circumstances, a person entitled to the income from a
      trust is treated for IHT purposes as beneficially entitled to the
      capital which produces that income.
    \item
      Known as a ``qualifying interest in possession''.
    \item
      Interest in possession: fixed trust where beneficiary is entitled
      to income.
    \item
      An interest in possession arising \textbf{before} 22 March 2006
      will be a qualifying interest in possession.
    \item
      After this, more limited circumstances: e.g., being an `immediate
      post-death interest'
    \end{itemize}
  \item
    Property given away during lifetime `subject to a reservation'.

    \begin{itemize}
    \tightlist
    \item
      Applies where deceased gave away property during their lifetime
      but did not transfer possession and enjoyment of the property to
      the donee.
    \item
      Donor treated as being beneficially entitled to the property.
    \end{itemize}
  \end{itemize}
\end{itemize}

But does not include:

\begin{itemize}
\tightlist
\item
  Property outside the estate

  \begin{itemize}
  \tightlist
  \item
    Life assurance policy once it is written in trust for a named
    beneficiary.
  \item
    Discretionary lump sum payment made from a pension fund to the
    deceased's family.
  \end{itemize}
\item
  Excluded property

  \begin{itemize}
  \tightlist
  \item
    e.g., `reversionary interest': interest in remainder under a trust.
  \end{itemize}
\end{itemize}

\hypertarget{value-transferred}{%
\subsubsection{2. Value Transferred}\label{value-transferred}}

\begin{Shaded}
\begin{Highlighting}[]
\NormalTok{title: General principle {-} s 160 IHTA 1984}
\NormalTok{Assets are valued for IHT pruposes at \textquotesingle{}the price which the property might reasonably be expected to fetch if sold on open market\textquotesingle{} **immediately before** death.}
\end{Highlighting}
\end{Shaded}

With assets such as land, may need negotiations with the district valuer
to reach an agreed valuation.

\begin{Shaded}
\begin{Highlighting}[]
\NormalTok{The value of an asset agreed for IHT purposes is known as the **probate value**.}
\end{Highlighting}
\end{Shaded}

Where death causes the value of an asset to increase or decrease, this
will be taken into account (important for e.g., life insurance
policies).

\begin{longtable}[]{@{}
  >{\raggedright\arraybackslash}p{(\columnwidth - 2\tabcolsep) * \real{0.1184}}
  >{\raggedright\arraybackslash}p{(\columnwidth - 2\tabcolsep) * \real{0.8816}}@{}}
\toprule()
\begin{minipage}[b]{\linewidth}\raggedright
Asset
\end{minipage} & \begin{minipage}[b]{\linewidth}\raggedright
How to value
\end{minipage} \\
\midrule()
\endhead
Quoted {[}{[}Shares{]}{]} & Value taken from Stock Exchange Daily
Official List for date of death.
\(\text{Value} = \text{bid} + \frac{1}{4}\left( \text{bid-ask spread}\right)\) \\
Debts and expenses & Debts/ expenses at time of death deductible
provided incurred for money/ money's worth (s 505 IHTA 1984). Reasonable
funeral expenses deductible (s 162 IHTA 1984). \\
\bottomrule()
\end{longtable}

\hypertarget{exemptions-reliefs}{%
\subsubsection{3. Exemptions/ Reliefs}\label{exemptions-reliefs}}

\hypertarget{spouse-civil-partner-exemption}{%
\paragraph{Spouse/ Civil Partner
Exemption}\label{spouse-civil-partner-exemption}}

\begin{Shaded}
\begin{Highlighting}[]
\NormalTok{title: s 18 IHTA 1984}
\NormalTok{A transfer of value is an exempt transfer to the extent that the value transferred is attributable to property which becomes comprised in the estate of the transferor’s spouse or civil partner.}
\end{Highlighting}
\end{Shaded}

The spouse exemption can apply in conjunction with the rule applicable
to qualifying interest in possession trusts, that IHT is charged as if
the person with the right to income owned the capital. So if a trust is
validly created with income for life for a spouse, remainder to
children, the whole estate will be spouse exempt on the settlor's death.

\hypertarget{charity-exemption}{%
\paragraph{Charity Exemption}\label{charity-exemption}}

\begin{Shaded}
\begin{Highlighting}[]
\NormalTok{title: s 23(1) IHTA 1984}
\NormalTok{Transfers of value are exempt to the extent that the values transferred by them are attributable to property which is given to charities.}
\end{Highlighting}
\end{Shaded}

A similar exemption applies to gifts to certain bodies, bodies providing
a public benefit, and to political parties.

\hypertarget{business-property-relief}{%
\paragraph{Business Property Relief}\label{business-property-relief}}

Provided that the transferor owned the property/ a replacement property
for \textbf{2 years} immediately before the transfer, there is a 100\%
reduction for:

\begin{itemize}
\tightlist
\item
  A business or interest in a business (including {[}{[}Business Law and
  Practice/Company Law/Business models/Partnership{]}{]} share), and
\item
  Company shares not listed on a recognised stock exchange

  \begin{itemize}
  \tightlist
  \item
    AIM not included in definition of recognised stock exchange.
  \end{itemize}
\end{itemize}

And a 50\% reduction for:

\begin{itemize}
\tightlist
\item
  Quoted shares giving the transferor voting control of the company, and

  \begin{itemize}
  \tightlist
  \item
    Voting control means the ability to exercise \(>50\%\) votes on all
    resolutions. So this can be denied by \emph{Bushell v Faith}
    clauses.
  \item
    Separate shareholdings of spouses/ civil partners sometimes taken
    into account.
  \end{itemize}
\item
  Land, buildings and machinery owned by the transferor personally but
  used for business purposes by a partnership of which the transferor is
  a member, or by a company of which the transferor has voting control.

  \begin{itemize}
  \tightlist
  \item
    Note that partnerships treated more favourably than companies here.
  \end{itemize}
\end{itemize}

Where there is a lifetime transfer followed by the death of the
transferor within 7 years, BPR given at the time of the lifetime
transfer will be withdrawn unless the transferee still owns the business
property at the date of the transferor's death.

Note:

\begin{itemize}
\tightlist
\item
  Transfer need not be of the entire interest/ shareholding.
\item
  Where a person has entered into a contract for sale of their interest
  in a business, their interest is then taken to be in the proceeds of
  sale. Cash is not a relevant business property, so no relief
  available.
\end{itemize}

\hypertarget{agricultural-property-relief}{%
\paragraph{Agricultural Property
Relief}\label{agricultural-property-relief}}

The relief operates to reduce the agricultural value of agricultural
property by a certain percentage.

\begin{Shaded}
\begin{Highlighting}[]
\NormalTok{title: Agricultural value}
\NormalTok{The value of the property if it were subject to a perpetual covenant prohibiting its use other than for agriculture. }
\end{Highlighting}
\end{Shaded}

Reduction of 100\% is allowed where:

\begin{itemize}
\tightlist
\item
  The transferor had the right to vacant possession immediately before
  the transfer, or
\item
  Where the property was subject to a letting commencing on/ after
  01/09/1995.
\end{itemize}

Reduction of 50\% allowed in other cases.

Must also have been occupied by transferor for 2 years prior to
transfer/ owned by them for 7 years prior to transfer and occupied
throughout for the purpose of agriculture. Also applies when there is a
company structure in the way, of which the transferor had voting
control.

Agricultural property relief given in priority to business property
relief.

\hypertarget{calculate-tax}{%
\subsubsection{4. Calculate Tax}\label{calculate-tax}}

\begin{itemize}
\tightlist
\item
  0\% on first £325,000
\item
  40\% on anything higher
\item
  If \(\geq 10\%\) of a defined `component' of a person's net estate
  passes to charity, 36\% rate applies instead of 40\%.
\end{itemize}

\hypertarget{cumulation}{%
\paragraph{Cumulation}\label{cumulation}}

Any chargeable transfers made by the transferor over the 7 years before
the transfer must be taken into account to determine how much of the nil
rate band remains available. Any lifetime exemptions are taken into
account.

Not applicable to the residence nil rate band which should be available
in full on death, subject to any adjustments for estates over £2
million.

\hypertarget{residence-nil-rate-band}{%
\paragraph{Residence Nil Rate Band}\label{residence-nil-rate-band}}

\begin{longtable}[]{@{}ll@{}}
\toprule()
Tax year & Rate \\
\midrule()
\endhead
2017/18 & £100,000 \\
2018/19 & £125,000 \\
2019/20 & £150,000 \\
2020/21 onwards & £175,000 \\
\bottomrule()
\end{longtable}

\hypertarget{conditions}{%
\subparagraph{Conditions}\label{conditions}}

This is available in addition to the nil rate band if the deceased dies
owning a `qualifying residential interest' which is `closely inherited'.

\begin{Shaded}
\begin{Highlighting}[]
\NormalTok{title: Qualifying residential interest}
\NormalTok{An interest in a dwelling house which has at any time been the deceased’s residence and which forms part of the deceased’s estate.}
\end{Highlighting}
\end{Shaded}

For a property to be \textbf{`closely inherited'}, it must pass to:

\begin{enumerate}
\def\labelenumi{\arabic{enumi}.}
\tightlist
\item
  a child, grandchild or other \textbf{lineal descendant} of the
  deceased outright or on certain types of trust. Lineal descendants are
  defined as children, step-children, adopted children, foster children,
  or children where the deceased had been appointed as a guardian;
\item
  the current spouse or civil partner of the deceased's lineal
  descendants; or
\item
  the widow, widower or surviving civil partner of a lineal descendant
  who predeceased the deceased, unless such persons have remarried or
  formed a new civil partnership before the deceased's death.
\end{enumerate}

Only applies up to the value of the deceased's residence. If estate
worth \(\geq £2,000,000\), RNRB reduced by £1 for every £2 over £2
million threshold.

\hypertarget{downsizing}{%
\subparagraph{Downsizing}\label{downsizing}}

Finance Bill 2016 introduced a downsizing allowance; allowing PRs to
claim RNRB to which the deceased would have been entitled, had they held
onto their property, but instead moved to a care home etc.

\hypertarget{potentially-exempt-transfers-pets}{%
\subsection{Potentially Exempt Transfers
(PETs)}\label{potentially-exempt-transfers-pets}}

\begin{Shaded}
\begin{Highlighting}[]
\NormalTok{title: PET}
\NormalTok{Any gift made by an individual to another **individual or into a disabled trust** (IHTA 1984, s 3A(1A)), to the extent in either case that the gift would otherwise be chargeable.}
\end{Highlighting}
\end{Shaded}

\hypertarget{transfer-of-value-1}{%
\subsubsection{1. Transfer of Value}\label{transfer-of-value-1}}

\begin{Shaded}
\begin{Highlighting}[]
\NormalTok{title: Transfer of value}
\NormalTok{Any lifetime disposition made by the transferor which reduces the value of their estate (s 3(1) IHTA 1984)}
\end{Highlighting}
\end{Shaded}

Some dispositions are excluded:

\begin{itemize}
\tightlist
\item
  Maintenance, education or training of transferor's child under 18/
  adult in full time education
\item
  Maintenance of a dependent relative (s 11 IHTA 1984).
\end{itemize}

\hypertarget{value-transferred-1}{%
\subsubsection{2. Value Transferred}\label{value-transferred-1}}

\begin{Shaded}
\begin{Highlighting}[]
\NormalTok{title: Value transferred}
\NormalTok{The amount by which the value of the transferor\textquotesingle{}s estate is less than it would have been but for the transfer (s 3(1) IHTA 1984). }
\end{Highlighting}
\end{Shaded}

Recall, the transferor's \textbf{estate} is the aggregate of all the
property to which they are beneficially entitled.

\hypertarget{related-property}{%
\paragraph{Related Property}\label{related-property}}

Designed to prevent tax avoidance in relation to a group of assets. The
value of a single asset within a group of similar assets is the
appropriate fraction of the total collection of assets if higher, rather
than the value of the single item.

\hypertarget{exemptions-and-reliefs}{%
\subsubsection{3. Exemptions and Reliefs}\label{exemptions-and-reliefs}}

A gift is potentially exempt only to the extent that it would otherwise
be chargeable.

\hypertarget{standard-exemptions}{%
\paragraph{Standard Exemptions}\label{standard-exemptions}}

The usual exemptions/ reliefs apply:

\begin{itemize}
\tightlist
\item
  Spouse/ civil partner exemption
\item
  Charity exemption
\item
  Business and agricultural property relief.

  \begin{itemize}
  \tightlist
  \item
    Only available if transferee still owns the property or qualifying
    replacement when transferor dies.
  \end{itemize}
\end{itemize}

\begin{Shaded}
\begin{Highlighting}[]
\NormalTok{title: Order of application}
\NormalTok{Apply exemptions and reliefs in the following order:}
\NormalTok{1. Spouse/ civil partner/ charity exemptions}
\NormalTok{2. Business/ agricultural property reliefs}
\NormalTok{3. \textquotesingle{}Lifetime only\textquotesingle{} exemptions}
\end{Highlighting}
\end{Shaded}

\hypertarget{lifetime-exemptions}{%
\paragraph{Lifetime Exemptions}\label{lifetime-exemptions}}

\begin{longtable}[]{@{}
  >{\raggedright\arraybackslash}p{(\columnwidth - 4\tabcolsep) * \real{0.0262}}
  >{\raggedright\arraybackslash}p{(\columnwidth - 4\tabcolsep) * \real{0.0930}}
  >{\raggedright\arraybackslash}p{(\columnwidth - 4\tabcolsep) * \real{0.8808}}@{}}
\toprule()
\begin{minipage}[b]{\linewidth}\raggedright
IHTA 1984
\end{minipage} & \begin{minipage}[b]{\linewidth}\raggedright
Exemption
\end{minipage} & \begin{minipage}[b]{\linewidth}\raggedright
Details
\end{minipage} \\
\midrule()
\endhead
s 19 & Annual exemption & The annual exemption applies to the
\textbf{first} £3,000 transferred by lifetime transfers in each tax
year. This can be carried forward for one year--but current year's
exemption must be used before previous year's can be carried forward. So
there is a total annual exemption for any given year of £6,000. \\
s 20 & Small gifts & Lifetime gifts in any tax year of £250 or less to
any one person are exempt. Exemption cannot be set against a gift
exceeding £250. \\
s 21 & Normal expenditure out of income & Lifetime transfer exempt if:
(1) Made out of transferor's normal expenditure; (2) Made out of
transferor's income; and (3) After allowing for such payments,
transferor was left with sufficient income to maintain usual standard of
living. \\
s 21 & Marriage consideration & Lifetime gifts on marriage are exempt up
to: £5,000 by a parent of a party to the marriage; £2,500 by a remoter
ancestor of a party to the marriage (eg, a grandparent); and £1,000 in
any other case. \\
\bottomrule()
\end{longtable}

\hypertarget{potentially-exempt}{%
\paragraph{Potentially Exempt}\label{potentially-exempt}}

Any value remaining after exemptions and reliefs applied is potentially
exempt. Transfer becomes chargeable only if transferor dies within 7
years.

\hypertarget{lifetime-chargeable-transfers-lcts}{%
\subsection{Lifetime Chargeable Transfers
(LCTs)}\label{lifetime-chargeable-transfers-lcts}}

\begin{Shaded}
\begin{Highlighting}[]
\NormalTok{An LCT is made to a company or trust. }
\end{Highlighting}
\end{Shaded}

Any lifetime transfer which does not fall within the definition of a PET
is an LCT. The aim is to close loopholes involving companies and trusts.

Main examples:

\begin{itemize}
\tightlist
\item
  Lifetime transfer made on or after 22/03/06 into any trust (other than
  a disabled trust)
\item
  Lifetime trust into a discretionary trust or company.
\end{itemize}

\hypertarget{calculate-tax-1}{%
\subsubsection{4. Calculate Tax}\label{calculate-tax-1}}

First identify transfer of value, calculate value transferred and apply
exemptions and reliefs, as set out in (1) to (3) above. Then apply
rates:

\begin{itemize}
\tightlist
\item
  0\% on first £325,000 (nil rate band)
\item
  20\% on the balance of the chargeable transfer.
\end{itemize}

\hypertarget{cumulation-1}{%
\paragraph{Cumulation}\label{cumulation-1}}

\begin{Shaded}
\begin{Highlighting}[]
\NormalTok{Chargeable transfers made in the 7 years before the current chargeable transfer reduce the nil rate band available to that current transfer. }
\end{Highlighting}
\end{Shaded}

Note that while the transferor is alive, any PETs made by the transferor
are ignored for cumulation purposes because they may never become
chargeable.

\hypertarget{effect-of-death-on-lifetime-transfers}{%
\subsection{Effect of Death on Lifetime
Transfers}\label{effect-of-death-on-lifetime-transfers}}

The death of a transferor may result in a charge /additional charge to
IHT on any transfers of value which they have made in the 7 years
immediately preceding death, whether these were PETs or LCTs.

\begin{itemize}
\tightlist
\item
  PETs made become chargeable and the transferee becomes liable for any
  IHT payable.
\item
  IHT liability on LCTs is recalculated, and trustees will be liable for
  any extra tax payable.
\end{itemize}

Work chronologically forwards over 7 years, calculating appropriate tax.

\hypertarget{effect-on-pets}{%
\subsubsection{4. Effect on PETs}\label{effect-on-pets}}

Apply steps (1) to (3) to determine size of PET. Then establish the
transferor's cumulative total of transfers at the time of the PET. Made
up of:

\begin{enumerate}
\def\labelenumi{\arabic{enumi}.}
\tightlist
\item
  Any LCT's made in 7 years before the PET, and
\item
  Any other PETs made during the 7 years before the PET being assessed.
\end{enumerate}

Rate of tax applicable to \textbf{chargeable transfers} made within 7
years of death are:

\begin{itemize}
\tightlist
\item
  0\% on available nil rate band
\item
  40\% on balance (note 36\% charity rate not available).
\item
  But subject to tapering relief, see below.
\end{itemize}

\begin{Shaded}
\begin{Highlighting}[]
\NormalTok{Draw a timeline. }
\end{Highlighting}
\end{Shaded}

\hypertarget{tapering-relief}{%
\paragraph{Tapering Relief}\label{tapering-relief}}

When a PET becomes chargeable, tapering relief is available if
transferor survives \(>3\) years after transfer. Any tax payable on PET
is reduced.

\begin{longtable}[]{@{}ll@{}}
\toprule()
Years between PET and death & Tax rate \\
\midrule()
\endhead
3 to 4 years & 32\% \\
4 to 5 years & 24\% \\
5 to 6 years & 16\% \\
6 to 7 years & 8\% \\
7 or more & 0\% \\
\bottomrule()
\end{longtable}

\hypertarget{effect-on-lcts}{%
\subsubsection{4. Effect on LCTs}\label{effect-on-lcts}}

If transferor dies within 7 years of making an LCT, more IHT may be
payable, because:

\begin{enumerate}
\def\labelenumi{\arabic{enumi}.}
\tightlist
\item
  Full death rate of IHT now payable
\item
  PETs made before the LCT may have become chargeable.
\end{enumerate}

A rate of
\(\min \left(\text{rate at death}, \text{rate at transfer}\right)\) is
used to calculate IHT. Note this means the inheritance tax rate which
happens to be in force at the time of death/ transfer. So for our
purposes this will always be 40\%.

To determine how much of nil rate band is available, find the cumulative
total of:

\begin{enumerate}
\def\labelenumi{\arabic{enumi}.}
\tightlist
\item
  Any other LCTs made in 7 years before LCT
\item
  Any PETs made during 7 years prior to LCT, which have now become
  chargeable.
\end{enumerate}

So an LCT within 14 years of death may remain relevant for cumulation
purposes.

(TODO: consider adding timeline here)

\hypertarget{tapering-relief-1}{%
\paragraph{Tapering Relief}\label{tapering-relief-1}}

If \textgreater3 years have passed between transfer and death, tapering
relief applies to reduce the recalculated tax. Credit given for IHT paid
on LCT at the time, though if the recalculated bill is lower than the
original amount paid, no tax is refunded.

\hypertarget{liability-and-burden-of-payment}{%
\subsection{Liability and Burden of
Payment}\label{liability-and-burden-of-payment}}

Payment will usually be obtained by people holding property in a
representative capacity (PRs and trustees), though those who are
beneficially entitled to the property are concurrently liable with such
representatives.

\hypertarget{estate-rate}{%
\subsubsection{Estate Rate}\label{estate-rate}}

\begin{Shaded}
\begin{Highlighting}[]
\NormalTok{The average rate of tax applicable to each item of property in the estate.}
\end{Highlighting}
\end{Shaded}

Tax is divided between the various assets in the estate proportionately,
according to their value. So multiply the value of the asset by the
estate rate to calculate the tax liability of an asset.

\hypertarget{liability-on-death}{%
\subsubsection{Liability on Death}\label{liability-on-death}}

\hypertarget{prs}{%
\paragraph{PRs}\label{prs}}

PRs are liable for IHT attributable to the non-settled estate: any
property which `was not immediately before the death comprised in a
settlement' (s 200 IHTA 1984). Includes:

\begin{itemize}
\tightlist
\item
  Property vesting in PRs
\item
  Property to which the deceased was beneficially entitled immediately
  before death.
\end{itemize}

\begin{Shaded}
\begin{Highlighting}[]
\NormalTok{PR liability limited to the value of assets received/ would have received but for their own negligence or default. }
\end{Highlighting}
\end{Shaded}

Concurrently liable with the PRs is `any person in whom property is
vested\ldots at any time after\\
the death or who at any such time is beneficially entitled to an
interest in possession in the\\
property' (IHTA 1984, s 200(1)(c)). But unusual for HMRC to claim tax
from the recipient of property.

If a deceased gave away a property during their lifetime but retained a
benefit in the property, the donee of the gift is primarily liable to
pay tax attributable to the property. If tax remains unpaid 12 months
after the end of the month of death, PRs become liable.

\hypertarget{trustees}{%
\paragraph{Trustees}\label{trustees}}

Trustees of settlement liable for IHT attributable to the property.

\hypertarget{liability-on-pets}{%
\subsubsection{Liability on PETs}\label{liability-on-pets}}

Transferee primarily liable, but PR becomes liable if tax remains unpaid
for 12 months after the end of the month of death. PRs should ideally
delay distribution of the estate until IHT on lifetime gifts has been
paid.

\hypertarget{liability-on-lcts}{%
\subsubsection{Liability on LCTs}\label{liability-on-lcts}}

Transferor primarily liable, though HMRC may also claim tax from
trustees. In practice, trustees often pay.

\textbf{If transferor pays, the amount of tax will usually be more than
if trustees pay}. This is because IHT is charged on the value
transferred, i.e., the loss to the transferor's estate brought about by
the gift. This is just maths; no more `efficient' way of doing things.

\hypertarget{burden}{%
\subsubsection{Burden}\label{burden}}

The transferor can decide where the burden of tax falls.

\hypertarget{time-for-payment}{%
\subsection{Time for Payment}\label{time-for-payment}}

\hypertarget{death-estates}{%
\subsubsection{Death Estates}\label{death-estates}}

Basic rule: IHT due 6 months after the end of the month of death. If tax
not paid on time, interest runs on amount outstanding.

\hypertarget{instalments}{%
\paragraph{Instalments}\label{instalments}}

Instalment option is available

\begin{itemize}
\tightlist
\item
  10 yearly instalments, first due 6 months after the end of the month
  of death.
\item
  Tax apportioned using the estate rate
\item
  Applies to:

  \begin{itemize}
  \tightlist
  \item
    Land
  \item
    Business or interest in business
  \item
    Shares giving control of company to deceased
  \item
    Unquoted shares where holding is sufficiently large (10\% nominal
    value \& \textgreater£20,000)
  \item
    HMRC thinks lump payment would cause undue hardship
  \item
    IHT payable amounts to \(\geq 20\%\) IHT payable on estate.
  \end{itemize}
\item
  For anything except non-agricultural land, instalments carry interest
  only on late payments
\item
  For non-agricultural land, interest payable on the amount outstanding.
\item
  All outstanding tax and interest payable upon sale.
\end{itemize}

\hypertarget{pets}{%
\subsubsection{PETs}\label{pets}}

Due 6 months after the end of the month of death.

\hypertarget{lcts}{%
\subsubsection{LCTs}\label{lcts}}

\begin{longtable}[]{@{}
  >{\raggedright\arraybackslash}p{(\columnwidth - 2\tabcolsep) * \real{0.1724}}
  >{\raggedright\arraybackslash}p{(\columnwidth - 2\tabcolsep) * \real{0.8276}}@{}}
\toprule()
\begin{minipage}[b]{\linewidth}\raggedright
Timing
\end{minipage} & \begin{minipage}[b]{\linewidth}\raggedright
Due
\end{minipage} \\
\midrule()
\endhead
At the time of transfer & Generally due 6 months after the end of the
month in which LCT made. But for LCTs made 6th April - 30th September,
due 30th April the next year. \\
Following death within 7 years & 6 months after the end of the month of
death. \\
\bottomrule()
\end{longtable}

\hypertarget{succession-to-property-on-death}{%
\section{Succession to Property on
Death}\label{succession-to-property-on-death}}

Many valuable assets pass independently of the terms of the will; and
even if there is a will, the court may override its terms under the
Inheritance (Provision for Family and Dependants) Act 1975 if it
concludes that reasonable financial provision was not made for a close
relative or dependant.

\hypertarget{property}{%
\subsection{Property}\label{property}}

An individual can provide for the disposition of their property on death
by leaving a valid will. This includes property held in the sole name of
the testator at the time of death and any share of property owned as
beneficial tenants in common.

If an individual does not dispose of property by will, it can pass on
death according to intestacy rules.

\hypertarget{property-passing-independently-of-will-intestacy}{%
\subsubsection{Property Passing Independently of will/
Intestacy}\label{property-passing-independently-of-will-intestacy}}

\begin{longtable}[]{@{}
  >{\raggedright\arraybackslash}p{(\columnwidth - 2\tabcolsep) * \real{0.0638}}
  >{\raggedright\arraybackslash}p{(\columnwidth - 2\tabcolsep) * \real{0.9362}}@{}}
\toprule()
\begin{minipage}[b]{\linewidth}\raggedright
Property
\end{minipage} & \begin{minipage}[b]{\linewidth}\raggedright
Details
\end{minipage} \\
\midrule()
\endhead
Joint property & Where property held as joint tenants in equity, the
property passes by survivorship. \\
Nominated property & Allows individuals to nominate what is to happen to
certain types of funds after the nominator's death. Statutory provisions
apply to deposits \(\leq £5,000\) in certain savings banks/ societies.
Nomination takes precedence over any term of the will to the
contrary. \\
Insurance policies & A person can take out a life assurance policy
expressed to be for the benefit of specified individuals (effectively a
gift on trust). Can express the policy to be in favour of spouse/
children/ expressly write the policy in trust. \\
Pension benefits & Usually, a lump sum is paid to members of the family/
dependents chosen at the trustees' discretion. Common for employees to
be allowed to write a letter to trustees saying who they would like to
benefit. \\
\bottomrule()
\end{longtable}

\hypertarget{requirements-of-a-valid-will}{%
\subsection{Requirements of a Valid
Will}\label{requirements-of-a-valid-will}}

\begin{Shaded}
\begin{Highlighting}[]
\NormalTok{title: Valid will}
\NormalTok{To create a valid will, a testator must have}
\NormalTok{1. the necessary capacity and intention, and}
\NormalTok{2. must observe the formalities for execution of wills laid down in the Wills Act 1837. }
\end{Highlighting}
\end{Shaded}

\hypertarget{capacity}{%
\subsubsection{Capacity}\label{capacity}}

Individual must be:

\begin{itemize}
\tightlist
\item
  Aged 18 or over
\item
  Must have requisite mental capacity.

  \begin{itemize}
  \tightlist
  \item
    `soundness of mind, memory and understanding' ({[}{[}Banks v
    Goodfellow (1870) LR 5 QB 549{]}{]})
  \item
    Must understand

    \begin{itemize}
    \tightlist
    \item
      Nature of their act and its broad effects
    \item
      Extent of their property
    \item
      Moral claims they ought to consider.
    \end{itemize}
  \end{itemize}
\end{itemize}

\hypertarget{proof-and-presumptions}{%
\paragraph{Proof and Presumptions}\label{proof-and-presumptions}}

The person putting forward a will (e.g.., the sole beneficiary) must
prove all necessary elements are present. Generally, mental capacity
presumed unless there is any sign of mental confusion.

s 3 Mental Capacity Act (MCA) 2005 introduces a statutory test to
determine whether a person has capacity to take a decision. This applies
only to decisions taken under the Act by the Court of Protection on
behalf of another.

\hypertarget{intention}{%
\subsubsection{Intention}\label{intention}}

The testator must have both general intention (to make a will) and
specific intention (to make the particular will now being executed).

\hypertarget{proof-and-presumptions-1}{%
\paragraph{Proof and Presumptions}\label{proof-and-presumptions-1}}

The burden of proving the testator's knowledge and approval falls on the
person putting forward the will.

\begin{itemize}
\tightlist
\item
  General assumption

  \begin{itemize}
  \tightlist
  \item
    A testator who has capacity and has then read and executed the will
    is presumed to have the requisite knowledge and approval.
  \end{itemize}
\item
  Exceptions

  \begin{itemize}
  \tightlist
  \item
    Testator blind/ illiterate/ not signing personally

    \begin{itemize}
    \tightlist
    \item
      Necessary to provide evidence
    \item
      Usual to include a statement at the end of the will stating that
      the will was read to/ by the testator, who seemed to know and
      approve.
    \end{itemize}
  \item
    Suspicious circumstances

    \begin{itemize}
    \tightlist
    \item
      e.g., the will is prepared by someone who is a major beneficiary
      under its terms.
    \end{itemize}
  \end{itemize}
\end{itemize}

\hypertarget{conduct}{%
\paragraph{Conduct}\label{conduct}}

\begin{Shaded}
\begin{Highlighting}[]
\NormalTok{Those regulated by the SRA are required to act with honesty and integrity and in the best interests of clients (Principles 4, 5 and 7). They must not abuse their position by taking unfair advantage of clients, and they must not act if there is a significant risk that the duty to act in the best interests of the client conflicts with their own interests. There is such a risk where someone prepares a will which benefits themselves or someone close to them.}

\NormalTok{It is therefore sensible for firms to have a policy of refusing to act where a client proposes to make a gift of significant value to a fee earner or member of their family unless the client takes independent legal advice.}
\end{Highlighting}
\end{Shaded}

\hypertarget{challenging}{%
\paragraph{Challenging}\label{challenging}}

Where a testator with capacity appears to have known and approved the
contents of the will, any person who wishes to challenge the will (or
any part of it) must prove \textbf{force, fear, fraud, undue influence
or mistake}, to prevent some or all of the will from being admitted to
probate.

\hypertarget{undue-influence}{%
\subparagraph{Undue Influence}\label{undue-influence}}

\begin{itemize}
\tightlist
\item
  It is necessary to \emph{prove} undue influence in relation to a will
  (no presumptions).
\item
  For a lifetime gift made to a person in a position of trust and
  confidence, there is a presumption of undue influence to be discharge
  for the gift to be kept.
\end{itemize}

For a party to prove it has been the victim of undue influence, needs to
show:

\begin{enumerate}
\def\labelenumi{\arabic{enumi}.}
\tightlist
\item
  There is a relationship of trust and confidence, and
\item
  There is a `transaction which requires explanation' (suspicious/
  suspiciously high value).
\end{enumerate}

\hypertarget{mistake}{%
\subparagraph{Mistake}\label{mistake}}

\begin{itemize}
\tightlist
\item
  Any words included without the knowledge and approval of the testator
  will be omitted from probate.
\item
  If there is misunderstanding as to the legal meaning of words used in
  a will, mistaken words will not be omitted, but may be interpreted to
  give effect to the wishes of the testator.
\end{itemize}

\hypertarget{formalities}{%
\paragraph{Formalities}\label{formalities}}

\begin{Shaded}
\begin{Highlighting}[]
\NormalTok{title: s 9 Wills Act 1837}
\NormalTok{Signing and attestation of wills}

\NormalTok{(1) No will shall be valid unless—}

\NormalTok{{-} (a) it is in writing, and signed by the testator, or by some other person in his presence and by his direction; and}

\NormalTok{{-} (b) it appears that the testator intended by his signature to give effect to the will; and}

\NormalTok{{-} (c) the signature is made or acknowledged by the testator in the presence of two or more witnesses present at the same time; and}

\NormalTok{{-} (d) each witness either—}
\NormalTok{    {-} (i) attests and signs the will; or}
\NormalTok{    {-} (ii) acknowledges his signature,}

\NormalTok{    in the presence of the testator (but not necessarily in the presence of any other witness),}

\NormalTok{but no form of attestation shall be necessary.}

\NormalTok{(2) For the purposes of paragraphs (c) and (d) of subsection (1), in relation to wills made on or after 31 January 2020 and on or before 31 January, “presence” includes presence by means of videoconference or other visual transmission.}
\end{Highlighting}
\end{Shaded}

A will made on actual military service or by a mariner or seaman at sea
is valid and may be in any form, including a mere oral statement: Wills
Act 1837, s 11.

What is required for `presence' is a line of sight ({[}{[}Casson v Dade
(1781) 1 Bro C C 99{]}{]}) and a consciousness of what is going on
({[}{[}Brown v Skirrow {[}1903{]} P 3{]}{]}).

\hypertarget{remote-witnessing}{%
\paragraph{Remote Witnessing}\label{remote-witnessing}}

s 2 allows remote witnessing for a limited period.

Must be:

\begin{itemize}
\tightlist
\item
  Witnessed in real time
\item
  Witness and testator can be at different locations
\item
  Electronic signatures not permitted. Testator must date the signature.
\item
  Will must be taken/ posted to witnesses
\item
  Witnesses must physically sign the will in the virtual presence of the
  testator.
\item
  Witnesses will date with the date on which they are signing
\item
  Execution process complete once both witnesses have signed.
\item
  If testator dies before all signatures added - tough.
\end{itemize}

\hypertarget{proof-and-presumptions-2}{%
\paragraph{Proof and Presumptions}\label{proof-and-presumptions-2}}

\begin{Shaded}
\begin{Highlighting}[]
\NormalTok{title: Attestation clause}
\NormalTok{If the will includes a clause which recites that the s 9 formalities were observed ("attestation clause"), a presumption of due execution is raised. The will is valid unless there is proof that the formalities were not observed.}
\end{Highlighting}
\end{Shaded}

If there is no attestation clause, the district judge will require an
affidavit of due execution from a witness/ affidavit of handwriting
evidence from the testator/ will refer the case to a judge.

\hypertarget{witnesses}{%
\paragraph{Witnesses}\label{witnesses}}

No formal requirements relating to the capacity of witnesses, though
they must be capable of understanding the significance of witnessing a
signature.

If either of the witnesses is a beneficiary under the will or is the
spouse or a civil partner of a beneficiary, the will remains valid but
the gift to the witness or to the witness's spouse or civil partner
fails (Wills Act 1837, s 15).

A solicitor preparing wills should give clear instructions on how to
sign and witness the will. If the will is returned to the solicitor for
storage, the solicitor is under a duty to check the signatures to see
whether ss 9 and 15 appear to\\
have been complied with.

\hypertarget{revocation}{%
\paragraph{Revocation}\label{revocation}}

Testators can revoke a will during their lifetime, provided they have
testamentary capacity. 3 methods:

\begin{enumerate}
\def\labelenumi{\arabic{enumi}.}
\tightlist
\item
  By a later will or codicil

  \begin{itemize}
  \tightlist
  \item
    s 20 Wills Act 1837: a will may be revoked in whole or in part by a
    later will or codicil.
  \item
    Normally, an express clause revoking previous wills and codicils is
    also included.
  \item
    Exceptionally, the court may decide that an express revocation
    clause was conditional upon a previous event.
  \end{itemize}
\item
  By marriage/ civil partnership

  \begin{itemize}
  \tightlist
  \item
    If testator marries/ forms a civil partnership after executing a
    will, the will is automatically revoked (s 18 Wills Act 1837).
  \item
    Does not apply when the testator was expecting to marry/ form a
    civil partnership and appears not to intend the will to be revoked.
  \item
    Where a civil partnership is converted to a marriage, the conversion
    will not revoke an existing will of either party nor affect any
    dispositions in the wills.
  \item
    If a testator makes a will and is later divorced, or if a civil
    partnership dissolved, the will remains valid but

    \begin{itemize}
    \tightlist
    \item
      Provisions appointing (former) spouse/ civil partner as executor
      or trustee take effect as if they died on the date the marriage/
      civil partnership is dissolved/ annulled.
    \item
      Any property/ interest in property left to former spouse/ civil
      partner passes as if they had died on that date.
    \end{itemize}
  \end{itemize}
\item
  By destruction.

  \begin{itemize}
  \tightlist
  \item
    A will may be revoked by `burning, tearing or otherwise destroying
    the same by the testator or by some person in his presence and by
    his direction with the intention of revoking the same' (Wills Act
    1837, s 20).
  \item
    Physical destruction without the intention to revoke is
    insufficient.
  \item
    Symbolic destruction (e.g., crossing words out, writing ``revoked'')
    does not suffice to revoke the whole will, though if a part of the
    will is completely destroyed/ rendered unreadable, may revoke that
    part.
  \item
    Destruction must be in the testator's presence and by their
    direction.
  \end{itemize}
\end{enumerate}

\hypertarget{alterations}{%
\paragraph{Alterations}\label{alterations}}

Basic rule: alterations invalid unless it can be proved they were made
before the will was executed/ unless executed like a will. Initials of
testator/ witness in margin next to alteration sufficient.

\hypertarget{mechanism-for-effect}{%
\subsection{Mechanism for Effect}\label{mechanism-for-effect}}

\begin{itemize}
\tightlist
\item
  People dealing with estate must decide the effect of the will in the
  light of the circumstances at the date of the testator's death.
\item
  If beneficiaries are referred to by description, they must be
  identified.
\item
  Executors = appointed to deal with estate by will, administrators =
  deal with estate if there was no appointment in the will. Personal
  representatives = both.
\end{itemize}

\hypertarget{property-passing}{%
\subsubsection{Property Passing}\label{property-passing}}

Basic rule:

\begin{Shaded}
\begin{Highlighting}[]
\NormalTok{title: s 24 Wills Act 1837 {-} A will shall be construed to speak from the death of the testator.}

\NormalTok{Every will shall be construed, with reference to the real estate and personal estate comprised in it, to speak and take effect as if it had been executed immediately before the death of the testator, unless a contrary intention shall appear by the will. }
\end{Highlighting}
\end{Shaded}

\hypertarget{ademption}{%
\paragraph{Ademption}\label{ademption}}

A specific legacy will fail if the testator no longer owns the property
at death. Gift said to be `adeemed'.

If the nature of an asset has changed since the will was made, ask
whether the asset has changed in name/form or substance. Change in
substance → gift adeemed.

Suppose `my car' is gifted in the will, but subsequently the testator
bought a new car. Then generally the gift is adeemed.

\begin{Shaded}
\begin{Highlighting}[]
\NormalTok{title: Codicil}
\NormalTok{A supplement to a will which, to be valid, must be executed in the same way as a will. A testator may wish to add to or change a will in a minor way and so may execute a supplementary codicil.}
\end{Highlighting}
\end{Shaded}

A codicil has the effect of republishing the will as at the date of the
codicil. So the property referred to is the property at the date of the
codicil (i.e., references to items in the will are references to the
items on the date of republication).

\hypertarget{testator-surviving-beneficiary}{%
\subsubsection{Testator Surviving
Beneficiary}\label{testator-surviving-beneficiary}}

A gift in a will lapses if the beneficiary dies before the testator. If
a legacy lapses, the property falls into residue, unless the testator
has provided for the possibility of lapse by including a substitutional
gift.

Where no conditions to the contrary are imposed in the will, a gift
vests on the testator's death. There is a presumption that, as regards
people, a will is construed at the date it is made.

\hypertarget{s-184-lpa-1925}{%
\paragraph{S 184 LPA 1925}\label{s-184-lpa-1925}}

\begin{Shaded}
\begin{Highlighting}[]
\NormalTok{title: s 184 LPA 1925 {-} Presumption of survivorship in regard to claims to property.}

\NormalTok{In all cases where, after the commencement of this Act, two or more persons have died in circumstances rendering it uncertain which of them survived the other or others, such deaths shall (subject to any order of the court), for all purposes affecting the title to property, be presumed to have occurred in order of seniority, and accordingly the younger shall be deemed to have survived the elder. }
\end{Highlighting}
\end{Shaded}

\hypertarget{survivorship-clauses}{%
\paragraph{Survivorship Clauses}\label{survivorship-clauses}}

Commonly, gifts in will are made conditional on the survival of the
beneficiaries for a specific period, e.g., 28 days.

\hypertarget{lapse-of-joint-gifts}{%
\paragraph{Lapse of Joint Gifts}\label{lapse-of-joint-gifts}}

\begin{itemize}
\tightlist
\item
  A gift by will to two or more people as joint tenants will not lapse
  unless all the donees die before the testator.
\item
  A class gift (e.g., ``to all my nieces'') does not lapse unless all
  members of the class predecease the testator.
\end{itemize}

\hypertarget{s-33-wills-act-1837}{%
\paragraph{S 33 Wills Act 1837}\label{s-33-wills-act-1837}}

Where a will contains a gift to the testator's child or remoter
descendant and that beneficiary dies before the testator, leaving issue
of their own who survive the testator, the gift does not lapse but
passes instead to the beneficiary's issue. Section 33 does not apply if
the will shows a contrary intention.

(so if testator's kid dies before testator, gift reverts to grandkids).

\hypertarget{other-reasons-for-failure}{%
\subsubsection{Other Reasons for
Failure}\label{other-reasons-for-failure}}

\hypertarget{divorce-dissolution}{%
\paragraph{Divorce/ Dissolution}\label{divorce-dissolution}}

Wills Act 1837, s 18A: where after the date of the will the testator's
marriage or civil partnership is dissolved or annulled or declared void,
'any property which, or an interest in which, is devised or bequeathed
to the former spouse or civil partner \textbf{shall pass as if the
former spouse or civil partner had died'} on the date of the dissolution
or annulment of the marriage or civil partnership.

\hypertarget{beneficiary-witnesses-will}{%
\paragraph{Beneficiary Witnesses
Will}\label{beneficiary-witnesses-will}}

s 15: if the beneficiary, their spouse or civil partner witnesses the
will, a gift by will fails.

\hypertarget{disclaimer}{%
\paragraph{Disclaimer}\label{disclaimer}}

Beneficiaries need not accept gifts given to them by will. If they
disclaim the gift, it then falls into residue. A beneficiary who has
received a benefit from a gift is taken to have accepted the gift and
may no longer disclaim.

\hypertarget{forfeiture}{%
\paragraph{Forfeiture}\label{forfeiture}}

Forfeiture Act 1982: A person is not able to inherit from a person they
have been convicted of unlawfully killing (as a matter of public
policy). Applies to murder or manslaughter (or death by dangerous
driving), but not where the killer was insane. Unlawful killing includes
aiding, abetting, counselling or procuring the death of another person.

The court has a discretion as to whether or not to grant relief. There
is a 3-month time limit to grant relief (s 3(2) Forfeiture Act 1982),
starting at the point of sentence.

\hypertarget{estates-of-deceased-persons-forfeiture-rule-and-law-of-succession-act-2011}{%
\paragraph{Estates of Deceased Persons (Forfeiture Rule and Law of
Succession) Act
2011}\label{estates-of-deceased-persons-forfeiture-rule-and-law-of-succession-act-2011}}

Section 2 inserts a new s 33A into the Wills Act 1837 which provides,
subject to contrary intention in the will, that a person who disclaims
or forfeits an entitlement under a will is to be treated for the
purposes of the Wills Act as having predeceased the testator.

\hypertarget{identifying-beneficiaries}{%
\paragraph{Identifying Beneficiaries}\label{identifying-beneficiaries}}

References to ``children'' means only biological children and not
step-children, unless there is sufficient evidence to the contrary.
Adopted children normally treated as children of the adopted parents.

Persons who have obtained a full gender recognition certificate from the
Gender Recognition Panel are legally recognised in the acquired gender.

\begin{Shaded}
\begin{Highlighting}[]
\NormalTok{title: s 15 Gender Recognition Act 2004 }
\NormalTok{The fact that a person’s gender has become the acquired gender under the Act does not affect the disposal or devolution of property under a will or other instrument made before 4 April 2005.}
\end{Highlighting}
\end{Shaded}

But it will do if the will is made after that date. s 18 GRA 2004: where
the disposition or evolution of any property under a will or other
instrument (made on or after the appointed day) is different from what
it would be but for the fact that a person's gender has become the
acquired gender under the Act, an application may be made to the High
Court where expectations have been defeated.

Trustees and PRs are not liable for failing to check a gender
recognition certificate etc., though a wronged party can still follow
the property and claim it from other beneficiaries.

\hypertarget{intestacy-rules}{%
\subsection{Intestacy Rules}\label{intestacy-rules}}

Intestacy rules contained in Administration of Estates Act 1925. Can
apply when there is no will (person has died interstate), or only a
partial will (partial intestacy). Applies only to property which is
capable of being left by will.

\hypertarget{statutory-trust}{%
\subsubsection{Statutory Trust}\label{statutory-trust}}

The intestacy rules impose a trust over all the property (real and
personal) in respect of which a person dies intestate (AEA 1925, s 33).
This trust is similar to the usual express trust found in a will and
includes a power of sale by the PRs. The remaining balance is the
`residuary estate' to be shared among family (s 46 AEA 1925).

\hypertarget{spouse-civil-partner}{%
\subsubsection{Spouse/ Civil Partner}\label{spouse-civil-partner}}

A spouse is the person to whom the deceased was married at their death,
whether or not they were living together. Civil partners are treated in
the same way as spouses. Former spouses excluded.

\begin{Shaded}
\begin{Highlighting}[]
\NormalTok{title: Issue}
\NormalTok{Includes all direct descendants, as well as adopted children and remoter descendants. Step children not included. }
\end{Highlighting}
\end{Shaded}

\hypertarget{entitlement}{%
\paragraph{Entitlement}\label{entitlement}}

Where the intestate is survived by both spouse or civil partner and
issue:

\begin{itemize}
\tightlist
\item
  Spouse/ civil partner receives personal chattels absolutely

  \begin{itemize}
  \tightlist
  \item
    Tangible moveable property
  \item
    Excludes money, investments and items used mainly for business
    purposes.
  \end{itemize}
\item
  Spouse/ civil partner receives ``statutory legacy'' tax-free, and
  costs + interest from death until payment. This is £270,000.
\item
  Residuary estate divided in half.

  \begin{itemize}
  \tightlist
  \item
    Half held on trust for spouse/ civil partner absolutely
  \item
    Half held for the issue on the statutory trusts.
  \end{itemize}
\end{itemize}

The intestate's spouse or civil partner must survive the intestate for
28 days in order to inherit. The Law Reform (Succession) Act 1995
provides that, where the intestate's spouse or civil partner dies within
28 days of the intestate, the estate is distributed as if the spouse or
civil partner has not survived the intestate.

\hypertarget{applying-statutory-trusts}{%
\paragraph{Applying Statutory Trusts}\label{applying-statutory-trusts}}

Statutory trusts determine membership of the class of beneficiaries:

\begin{itemize}
\tightlist
\item
  Primary beneficiaries are the children of the intestate who are living
  at the intestate's death
\item
  Interests of children conditional on reaching 18/ marrying/ forming
  civil partnership before that age.
\item
  If a child predeceased the intestate, any children of the deceased who
  are living at the intestate's death take their deceased parent's share
  equally between them, conditional on reaching 18 or marrying/ civil
  partnership before that.
\item
  If children or issue survive the intestate but die without attaining a
  vested interest, their interest normally fails, and the estate is
  distributed as if they had never existed.

  \begin{itemize}
  \tightlist
  \item
    Unless they leave issue, in which case, issue may be substituted.
  \end{itemize}
\end{itemize}

\hypertarget{appropriation-of-matrimonial-home}{%
\paragraph{Appropriation of Matrimonial
Home}\label{appropriation-of-matrimonial-home}}

\begin{itemize}
\tightlist
\item
  The spouse/ civil partner can require PR's to appropriate the
  matrimonial home in full or partial satisfaction of any absolute
  interest in the estate (Intestates' Estates Act 1952, s 5).
\item
  If the property is worth more than the entitlement, the spouse or
  civil partner may still require appropriation provided they pay the
  difference, `equality money', to the estate.
\item
  The election must be made in writing to the PRs within 12 months of
  the grant of representation.
\end{itemize}

\hypertarget{no-issue}{%
\paragraph{No Issue}\label{no-issue}}

Where the intestate leaves a surviving spouse or civil partner but no
issue, the whole estate, however large, passes to the spouse or civil
partner absolutely. The spouse or civil partner must survive the
intestate for 28 days in order to take.

\hypertarget{distribution-where-no-surviving-spouse-civil-partner}{%
\paragraph{Distribution Where No Surviving spouse/ Civil
Partner}\label{distribution-where-no-surviving-spouse-civil-partner}}

The residue estate is divided between:

\begin{enumerate}
\def\labelenumi{\arabic{enumi}.}
\tightlist
\item
  issue on the `statutory trusts', but if none,
\item
  parents, equally if both alive, but if none,
\item
  brothers and sisters of the whole blood on the `statutory trusts', but
  if none,
\item
  brothers and sisters of the half blood on the `statutory trusts', but
  if none,
\item
  grandparents, equally if more than one, but if none,
\item
  uncles and aunts of the whole blood on the `statutory trusts', but if
  none,
\item
  uncles and aunts of the half blood on the `statutory trusts', but if
  none,
\item
  the Crown, Duchy of Lancaster, or Duke of Cornwall (\emph{bona
  vacantia}).
\end{enumerate}

Each category except parents and grandparents takes ``on the statutory
trusts'', meaning the members of the class share the estate equally.
Relatives not mentioned in s 46 may inherit on intestacy if their
parents died before the intestate.

\hypertarget{adopted-and-illegitimate-children}{%
\paragraph{Adopted and Illegitimate
Children}\label{adopted-and-illegitimate-children}}

Adopted children are treated for intestacy purposes as the children of
their adoptive parents and not of their natural parents. The intestacy
rules are applied regardless of whether or not a particular individual's
parents were married to each other.

On the intestacy of an individual whose parents were not married to each
other, it is presumed that the individual has not been survived by their
father or by any person related to them through their father unless the
contrary is shown, or the father is named on the birth certificate
(Family Law Reform Act 1987, s 18(2)).

Upon adoption, vested interests of a child are preserved, and contingent
interests may be preserved.

\hypertarget{bona-vacantia}{%
\paragraph{Bona Vacantia}\label{bona-vacantia}}

Where an estate passes \emph{bona vacantia}, the Crown, Duchy of
Lancaster or Duke of Cornwall has a discretion to provide for dependants
of the intestate, or for other persons for whom the intestate might
reasonably have been expected to make provision.

\hypertarget{inheritance-provision-for-family-and-dependents-act-1975}{%
\subsection{Inheritance (Provision for Family and Dependents) Act
1975}\label{inheritance-provision-for-family-and-dependents-act-1975}}

I(PFD)A 1975 allows certain categories of people who may be aggrieved
because they have been left out of a will, or are not inheriting on an
intestacy, or are dissatisfied with the amount they are inheriting, to
apply for a benefit from the estate following the testator's or
intestate's death.

\hypertarget{time-limit}{%
\subsubsection{Time Limit}\label{time-limit}}

An application must be brought within \textbf{6 months} of the date of
issue of the grant of representation to the deceased's estate (s 4),
though the court has discretion to extend.

\hypertarget{who-can-claim}{%
\subsubsection{Who Can Claim}\label{who-can-claim}}

s 1(1):

\begin{itemize}
\tightlist
\item
  Spouse or civil partner
\item
  Former spouse/ civil partner who has not remarried
\item
  Child
\item
  Step-child/ child of cohabitee
\item
  Person being maintained
\item
  Person living in the same house as the deceased for the whole 2 years
  before death.
\end{itemize}

\hypertarget{proof}{%
\subsubsection{Proof}\label{proof}}

The only ground for a claim is that `the disposition of the deceased's
estate effected by his will or the law relating to intestacy, or a
combination of his will and that law, is not such as to make reasonable
financial provision for the applicant'. Section 1(2) sets out two
standards for judging `reasonable financial provision'.

\begin{longtable}[]{@{}
  >{\raggedright\arraybackslash}p{(\columnwidth - 2\tabcolsep) * \real{0.1880}}
  >{\raggedright\arraybackslash}p{(\columnwidth - 2\tabcolsep) * \real{0.8120}}@{}}
\toprule()
\begin{minipage}[b]{\linewidth}\raggedright
Standard
\end{minipage} & \begin{minipage}[b]{\linewidth}\raggedright
Details
\end{minipage} \\
\midrule()
\endhead
Surviving spouse standard (s 1(2)(a)) & Allows a surviving spouse/ civil
partner such financial provision as is reasonable in all the
circumstances. \\
Ordinary standard (s 1(2)(b)) & For other categories of applicant,
allows `such financial provision as it would be reasonable in all the
circumstances\ldots for the applicant to receive for his maintenance' \\
\bottomrule()
\end{longtable}

So a person able to pay for their own maintenance will not obtain any
award.

\hypertarget{factors}{%
\subsubsection{Factors}\label{factors}}

s 3(1) provides guidelines of factors which the court considers:

\begin{itemize}
\tightlist
\item
  Financial resources and needs of the applicant
\item
  Deceased's moral obligations
\item
  Size and nature of estate
\item
  Physical/ mental disabilities
\item
  Conduct of applicant and anything else relevant.
\end{itemize}

\hypertarget{orders-of-the-court}{%
\subsubsection{Orders of the Court}\label{orders-of-the-court}}

The court has wide powers to make orders against the `net estate' of the
deceased. The `net estate' against which an order can be made includes
not only property which the deceased has, or could have, disposed of by
will or nomination, but also the deceased's share of joint property
passing by survivorship if the court so orders.

Any order taken into account for IHT purposes.

\hypertarget{protecting-prs}{%
\subsubsection{Protecting PRs}\label{protecting-prs}}

Personal representatives should be advised not to distribute the estate
until six months have elapsed from the issue of the grant. Must not
distribute once they have notice of a possible claim. If PRs do
distribute within the six-month period and an applicant subsequently
brings a successful claim, the PRs will be personally liable to satisfy
the claim if insufficient assets remain in the estate.

\hypertarget{probate}{%
\section{Probate}\label{probate}}

\hypertarget{introduction-1}{%
\subsection{Introduction}\label{introduction-1}}

It is necessary for assets to be transferred upon death into the names
of beneficiaries. Normally necessary to obtain a court document
authorising the deceased's PRs to transfer assets. Known as a
\textbf{grant of representation}.

The process of applying for a grant is governed by the Non-contentious
Probate Rules 1987 (NCPR 1987) as amended.

\hypertarget{digital-probate}{%
\subsubsection{Digital Probate}\label{digital-probate}}

Professionals are obliged to apply for grants of probate online, subject
to certain exceptions.

Where applications are made online, supporting documentation, such as
the death certificate, original will and HMRC forms have to be sent
separately to HMCTS Probate at Harlow where they are scanned.

\hypertarget{fees}{%
\subsubsection{Fees}\label{fees}}

The application fee is £273 where the estate exceeds £5,000. There is no
fee if the estate is £5,000 or less.

\hypertarget{prs-1}{%
\subsection{PRs}\label{prs-1}}

\begin{itemize}
\tightlist
\item
  If the will is valid and contains an effective appointment of
  executors of whom one or more is willing and able to prove the
  deceased's will, a grant of probate will be issued to the executor(s)
  willing to act.
\item
  If there is a valid will, but there are no persons willing or able to
  act as executors, then the next persons entitled to act are
  administrators with the will annexed.
\item
  If a deceased left no will, or no valid will, the estate will be
  administered in accordance with the law of intestacy by administrators
  appointed by the application of NCPR 1987, r 22.
\end{itemize}

\begin{longtable}[]{@{}
  >{\raggedright\arraybackslash}p{(\columnwidth - 4\tabcolsep) * \real{0.0328}}
  >{\raggedright\arraybackslash}p{(\columnwidth - 4\tabcolsep) * \real{0.6338}}
  >{\raggedright\arraybackslash}p{(\columnwidth - 4\tabcolsep) * \real{0.3333}}@{}}
\toprule()
\begin{minipage}[b]{\linewidth}\raggedright
PR
\end{minipage} & \begin{minipage}[b]{\linewidth}\raggedright
Number of PRs required
\end{minipage} & \begin{minipage}[b]{\linewidth}\raggedright
Authority
\end{minipage} \\
\midrule()
\endhead
Executor & One executor may obtain a grant and act alone, even if the
estate contains land which may be sold during the administration. Note
the difference with a trust, where two trustees are needed. & An
executor derives authority to act in the administration of an estate
from the will. The grant of probate confirms that authority. \\
Administrator & In the case of administrators (with or without the
will), it is normally sufficient for one to act in the administration of
the estate. However, where the will or intestacy creates a life or
minority interest, two administrators are normally required. & An
administrator (with or without the will) has very limited powers before
a grant is made. Their authority stems from the grant which is not
retrospective to the date of death. \\
\bottomrule()
\end{longtable}

\hypertarget{first-steps}{%
\subsubsection{First Steps}\label{first-steps}}

\begin{longtable}[]{@{}
  >{\raggedright\arraybackslash}p{(\columnwidth - 2\tabcolsep) * \real{0.1164}}
  >{\raggedright\arraybackslash}p{(\columnwidth - 2\tabcolsep) * \real{0.8836}}@{}}
\toprule()
\begin{minipage}[b]{\linewidth}\raggedright
Step
\end{minipage} & \begin{minipage}[b]{\linewidth}\raggedright
Details
\end{minipage} \\
\midrule()
\endhead
Will & If there is a will, ensure all executors named receive copies. \\
Funeral directions & Check any special directions as to cremation/ use
of body for medical research, etc. \\
Assets and liabilities & Obtain details of the deceased's property and
of any debts outstanding at the date of death. \\
Beneficiaries & Identify beneficiaries and entitlements, whether gifts
are part of estate, and whether beneficiaries are alive. Inform
beneficiaries that personal data is held (GDPR). \\
\bottomrule()
\end{longtable}

\hypertarget{missing-unknown-creditors-and-beneficiaries}{%
\subsubsection{Missing/ Unknown Creditors and
Beneficiaries}\label{missing-unknown-creditors-and-beneficiaries}}

PRs are responsible for administering the estate correctly. If the PRs
fail to pay someone who is entitled either as a creditor or as a
beneficiary, they will be personally liable to that person.

PRs can protect themselves against unknown claims by advertising for
claimants complying with the requirements of the Trustee Act 1925, s 27.
Provided they wait for the time period specified in the section (at
least two months), the PRs will be protected from liability if an
unknown claimant later appears. However, the claimant will have the
right to claim back assets from the beneficiaries who received them.

\hypertarget{advertising}{%
\paragraph{Advertising}\label{advertising}}

\begin{itemize}
\tightlist
\item
  Advertise as early as possible
\item
  Can use the following methods:

  \begin{itemize}
  \tightlist
  \item
    Advertisement in the London Gazette
  \item
    Advertisement in a local newspaper of the area
  \item
    Such other like notices.
  \end{itemize}
\item
  Include time limit in notice for persons to come forward, which must
  be not less than 2 months from date of notice.
\item
  Make searches in the case of land, to check any benefits or burdens.
\end{itemize}

After notice has expired, PRs should distribute the estate.

\hypertarget{liability}{%
\paragraph{Liability}\label{liability}}

Trustee Act 1925, s 27 exempts trustees from liability from unknown
claims provided they have advertised as above. But does not protect PRs
who knows there is a person with a claim but cannot find them.

\hypertarget{missing-known-creditors-beneficiaries}{%
\paragraph{Missing Known creditors/
Beneficiaries}\label{missing-known-creditors-beneficiaries}}

Options:

\begin{enumerate}
\def\labelenumi{\arabic{enumi}.}
\tightlist
\item
  Retain some assets in case claimant appears
\item
  Take an indemnity from beneficiaries that they will meet any claims if
  claimant reappears (dangerous for the PR)
\item
  Take out insurance to provide funds. Can be expensive
\item
  Apply to court for a \emph{Benjamin} order: an order for distribution
  on the basis that the claimant is dead. Protects PR from liability,
  though claimant retains the right to recover assets from
  beneficiaries.
\end{enumerate}

\hypertarget{necessity-of-grant-of-representation}{%
\subsection{Necessity of Grant of
Representation}\label{necessity-of-grant-of-representation}}

Not always necessary to obtain a grant of representation.

\hypertarget{assets-passing-to-prs-without-a-grant}{%
\subsubsection{Assets Passing to PRs without a
Grant}\label{assets-passing-to-prs-without-a-grant}}

\begin{itemize}
\tightlist
\item
  Orders made under Administration of Estates (Small Payments) Act 1965
  permits payments \(\leq £5,000\) to persons appearing to be
  beneficially entitled to the assets without formal proof of title.
\item
  Chattels
\item
  Cash
\end{itemize}

\hypertarget{assets-not-passing-through-pr}{%
\subsubsection{Assets Not Passing Through
PR}\label{assets-not-passing-through-pr}}

\begin{itemize}
\tightlist
\item
  Beneficially joint property passes by survivorship to the surviving
  joint tenant.
\item
  But transferring tenants in common property requires a grant of
  representation.
\end{itemize}

\hypertarget{property-not-forming-part-of-estate}{%
\subsubsection{Property Not Forming Part of
Estate}\label{property-not-forming-part-of-estate}}

\begin{itemize}
\tightlist
\item
  A life insurance policy held on trust for another. The proceeds make
  tax-free provision for dependents of the deceased.
\item
  Death in service benefits under a pension scheme. Tax-free provision
  for dependants.
\end{itemize}

\hypertarget{applying-for-grant}{%
\subsection{Applying for Grant}\label{applying-for-grant}}

A grant of representation is an order of the {[}{[}High Court{]}{]}.

Documents to submit:

\begin{itemize}
\tightlist
\item
  HMRC Form IHT421 (confirming payment of IHT) or Form IHT205 if the
  estate was `excepted'.
\item
  Deceased's will and codicil, plus 2 photocopies.
\item
  Any supporting evidence required
\item
  Probate court fees (£173 if estate \(\geq £5,000\)).
\item
  Form PA1P/ PA1A (will/ no will) if paper application.
\end{itemize}

\hypertarget{admissibility-of-will}{%
\subsubsection{Admissibility of Will}\label{admissibility-of-will}}

To ensure the will is admissible, check:

\begin{itemize}
\tightlist
\item
  The will is the last will
\item
  Will has not been validly revoked
\item
  Executed in accordance with s 9 Wills Act 1837
\item
  Contains an attestation clause raising a presumption of due execution.
\end{itemize}

\hypertarget{additional-requirements}{%
\subsubsection{Additional Requirements}\label{additional-requirements}}

Registrar may require further evidence, in the form of a statement of
truth/ affidavit. This may include:

\begin{itemize}
\tightlist
\item
  Evidence of due execution and/or capacity

  \begin{itemize}
  \tightlist
  \item
    If there is no attestation clause/ other defect, some sort of
    attesting witness statement will be required.
  \item
    If mental capacity is doubted, a witness statement from a doctor may
    be necessary.
  \end{itemize}
\item
  Evidence as to knowledge and approval

  \begin{itemize}
  \tightlist
  \item
    Registrar may doubt whether the testator was aware of the contents
    of the will on execution.
  \item
    Affidavit/ witness statement will be requested.
  \end{itemize}
\item
  Evidence as to remote witnessing
\item
  Evidence as to plight and condition.

  \begin{itemize}
  \tightlist
  \item
    If the state of the will suggests it has been interfered with since
    execution, further evidence will be required.
  \end{itemize}
\end{itemize}

A will known to have been in the testator's possession but which cannot
be found following death will be presumed to have been destroyed by the
testator with the intention of revocation. If the will is lost/
destroyed accidentally, a copy/ reconstruction can be used, together
with suitable witness affidavits and witness statements.

\hypertarget{completing-iht-account}{%
\subsection{Completing IHT Account}\label{completing-iht-account}}

\begin{itemize}
\tightlist
\item
  If the estate is not an excepted estate, PRs will prepare an IHT400.
\item
  The IHT400 is an inventory of the assets to which the deceased was
  beneficially entitled and of their liabilities, and is the form for
  claiming reliefs and exemptions and calculating the IHT payable.
\item
  It should be delivered within 12 months of the end of the month in
  which the death occurred.
\item
  Where IHT is payable, it is necessary to apply for a reference number
  before submitting the IHT400 (schedule IHT422).
\item
  To claim the unused nil rate band of a spouse, PRs of the survivor
  must make a claim using schedule 402.\\
\item
  To claim the unused RNRB of a spouse, PRs of the survivor must make a
  claim using schedule 436.
\end{itemize}

\hypertarget{paying-iht}{%
\subsubsection{Paying IHT}\label{paying-iht}}

\begin{itemize}
\tightlist
\item
  IHT on property without the right to pay by instalments must be paid
  within 6 months of the end of the month in which the death occurred.
  Failure to do so causes interest to become payable.
\item
  Where property qualifies for the right to pay by instalments, no tax
  is due until the expiry of the first 6-month period. Interest runs on
  all tax not paid at the due date.
\item
  Tax payable on the estate is apportioned between the instalment and
  non-instalment option property.
\end{itemize}

\hypertarget{valuations}{%
\subsubsection{Valuations}\label{valuations}}

Assets in the estate are valued at `the price which the property might
reasonably be expected to fetch if sold in the open market' immediately
before the death (IHTA 1984, s 160).

\hypertarget{jointly-owned-assets}{%
\paragraph{Jointly-owned Assets}\label{jointly-owned-assets}}

Where the deceased was a co-owner of the land at their death, the market
value of the land will normally be discounted by 15\% (residential
property) or 10\% (investment property), to reflect the difficulty of
selling a part interest in land.

The discount is not available for co-ownership of assets other than
land. Where the co-owners of land are spouses/ civil partners, generally
no discount in value is allowed.

\hypertarget{funding-iht}{%
\subsubsection{Funding IHT}\label{funding-iht}}

Where there is tax to pay on delivery of an IHT400, the PRs must arrange
for the money to be sent to HMRC. HMRC sends the probate summary
(IHT421) to the Probate Service, so grant can be issued. Funding the tax
bill may be difficult, since the deceased's assets vesting in PRs are
`frozen' until the grant issues. There are some options.

\begin{itemize}
\tightlist
\item
  Direct payment scheme

  \begin{itemize}
  \tightlist
  \item
    There is a procedure allowing PRs to arrange payment of IHT to HMRC
    directly from the deceased's accounts. The scheme is voluntary on
    the part of institutions.
  \item
    Procedure:

    \begin{enumerate}
    \def\labelenumi{\arabic{enumi}.}
    \tightlist
    \item
      PRs provide identification to bank
    \item
      PRs complete and send IHT423 form to bank, as well as sending
      IHT400 and supporting schedules to HMRC.
    \item
      Bank sends money direct to HMRC.
    \item
      HMRC sends receipted Probate Summary (IHT421) to Probate Service.
    \end{enumerate}
  \item
    This process is usually slow. Solicitors can make private
    arrangements with the bank, for the transfer of funds from the
    deceased's account.
  \end{itemize}
\item
  Life assurance

  \begin{itemize}
  \tightlist
  \item
    A life assurance company may be willing to release funds to pay IHT
    directly to HMRC.
  \end{itemize}
\item
  Assets realisable without grant

  \begin{itemize}
  \tightlist
  \item
    See Administration of Estates (Small Payments) Act 1965.
  \end{itemize}
\item
  Loans from beneficiaries

  \begin{itemize}
  \tightlist
  \item
    Wealthy beneficiaries may be prepared to fund the tax from their own
    resources, on condition that they will be repaid from the deceased's
    estate once the grant issues.
  \end{itemize}
\item
  National savings and government stock

  \begin{itemize}
  \tightlist
  \item
    Payment of tax may also be made from National Savings Bank accounts
    or from the proceeds of National Savings Certificates/ investments.
  \end{itemize}
\item
  Bank borrowing

  \begin{itemize}
  \tightlist
  \item
    Banks will often lend against an undertaking to repay the loan given
    by the PRs, along with (often) an undertaking from the solicitor to
    repay the loan from the proceeds of the estate.
  \end{itemize}
\item
  Heritage property

  \begin{itemize}
  \tightlist
  \item
    Taxpayers can offer HMRC an asset in lieu of tax (IHTA 1984, s
    230(1)). The Secretary of State must agree to accept such assets and
    the standard required of such objects is very high.
  \end{itemize}
\item
  Obtaining a grant on credit

  \begin{itemize}
  \tightlist
  \item
    The grant of probate can be obtained on credit in exceptional
    circumstances where PRs can demonstrate that it is impossible to pay
    IHT in advance.
  \end{itemize}
\end{itemize}

\begin{Shaded}
\begin{Highlighting}[]
\NormalTok{An undertaking should be limiited to "such proceeds as come into the solicitor\textquotesingle{}s control".}
\end{Highlighting}
\end{Shaded}

\hypertarget{excepted-estates}{%
\subsection{Excepted Estates}\label{excepted-estates}}

\hypertarget{inheritance-tax-delivery-of-accounts-excepted-estates-regulations-2004}{%
\subsubsection{Inheritance Tax (Delivery of Accounts) (Excepted Estates)
Regulations
2004}\label{inheritance-tax-delivery-of-accounts-excepted-estates-regulations-2004}}

If the estate is excepted, PRs do not now submit any IHT form to HMRC.
Instead, information as to value of the estate is included on the forms
submitted to the Probate Service which the Probate Service must pass on
to HMRC within one month.

HMRC has \textbf{60 days} from the issue of the grant of representation
to ask for additional information. If no such request is made, the
estate receives \textbf{automatic clearance}.

The requirements for qualifying as an excepted estate have been relaxed
in response to the criticisms of the Office of Tax Simplification,
allowing many more estates to qualify.

There are 3 categories of excepted estates.

\begin{longtable}[]{@{}
  >{\raggedright\arraybackslash}p{(\columnwidth - 2\tabcolsep) * \real{0.1150}}
  >{\raggedright\arraybackslash}p{(\columnwidth - 2\tabcolsep) * \real{0.8850}}@{}}
\toprule()
\begin{minipage}[b]{\linewidth}\raggedright
Category
\end{minipage} & \begin{minipage}[b]{\linewidth}\raggedright
Details
\end{minipage} \\
\midrule()
\endhead
Category 1 - small estates & Estates where the gross value on death for
IHT purposes + chargeable value of any specified transfers does not
exceed nil rate threshold. Nil rate threshold is increased to take
account of any transferred nil rate band. \\
Category 2 - `exempt' estates & Estates where the bulk of the estate
attracts the spouse/ civil partner or charity exemption. The gross value
of the estate + specified transfers must not exceed £3 million. Net
chargeable estate after deduction of liabilities and exemptions must not
exceed nil rate threshold. \\
Category 3 - `non-domiciled' estates & Where the deceased was never
domiciled or treated as domiciled in the UK, and owned only limited
assets in the UK. \\
\bottomrule()
\end{longtable}

\begin{Shaded}
\begin{Highlighting}[]
\NormalTok{title: Specified transfers}
\NormalTok{Chargeable transfers of cash, personal chattels, tangible moveable property, quoted shares, or an interest in land, made in the 7 years before death. When valuing specified transfers, business and agricultural property relief is ignored. }
\end{Highlighting}
\end{Shaded}

\begin{Shaded}
\begin{Highlighting}[]
\NormalTok{title: Specified exempt transfers}
\NormalTok{Transfers of value made in 7 years before death which fall under one of the exceptions:}
\NormalTok{{-} s 18 (transfers between spouses (or civil partners));}
\NormalTok{{-} s 23 (gifts to charities);}
\NormalTok{{-} s 24 (gifts to political parties);}
\NormalTok{{-} s 24A (gifts to housing associations);}
\NormalTok{{-} s 27 (maintenance funds for historic buildings, etc); or}
\NormalTok{{-} s 28 (employee trusts).}
\end{Highlighting}
\end{Shaded}

\hypertarget{procedure}{%
\subsubsection{Procedure}\label{procedure}}

On the application of grant, PRs provide details of the deceased, the
value of their estate and details of any nil rate band.

\begin{Shaded}
\begin{Highlighting}[]
\NormalTok{title: Estate}
\NormalTok{For this purpose, estate means the total of the death estate + specified transfers + specified exempt transfers made $\textbackslash{}leq 7$ years before death.}
\end{Highlighting}
\end{Shaded}

HMRC will select a random sample to review within 60 days of the
application for grant.

\hypertarget{iht400}{%
\subsection{IHT400}\label{iht400}}

Must be used whenever the deceased dies domiciled in the UK and the
estate is not an excepted estate. PRs must complete Form IHT400 and
relevant supporting schedules and sign a declaration of truth. Schedules
required will vary according to the composition of the estate.

IHT400 required \(\implies\) IHT421 (Probate Summary) must be completed
and submitted alongside. Sets out the gross and net values of the estate
for probation purposes.

\begin{itemize}
\tightlist
\item
  Probate value is not reduced by IHT exemptions and reliefs.
\item
  Probate value does not include the value of the property passing by
  survivorship or under the terms of a trust.
\end{itemize}

HMRC will, if satisfied, receipt the IHT421 and forward it to the
Probation Service. This is proof that the relevant IHT has been paid.
HMRC recommends allowing 20 working days between sending the IHT400 and
applying for grant.

\hypertarget{pa1p-and-pa1a}{%
\subsection{PA1P And PA1A}\label{pa1p-and-pa1a}}

Applications for a grant of representation can be verified against a
statement of truth and without the will having to be marked with the
signature of the applicant.

\begin{longtable}[]{@{}ll@{}}
\toprule()
Is there a will? & Form to use \\
\midrule()
\endhead
Yes & PA1P \\
No & PA1A \\
\bottomrule()
\end{longtable}

Online applications are made using one form, with different sections for
the different options.

\hypertarget{purpose-of-application-forms}{%
\subsubsection{Purpose of Application
Forms}\label{purpose-of-application-forms}}

\begin{enumerate}
\def\labelenumi{\arabic{enumi}.}
\tightlist
\item
  Establish basis of the applicants' claim to be entitled to take the
  grant
\item
  Confirm that PRs will carry out their statutory duties
\item
  Identify the will and any codicils to it.
\end{enumerate}

\hypertarget{points-in-common}{%
\subsubsection{Points in Common}\label{points-in-common}}

Both forms have sections for:

\begin{itemize}
\tightlist
\item
  Identifying the applicants

  \begin{itemize}
  \tightlist
  \item
    Name
  \item
    Address
  \item
    Identification documents
  \end{itemize}
\item
  Identifying the deceased

  \begin{itemize}
  \tightlist
  \item
    Name
  \item
    Marital status at death
  \item
    Total of foreign assets
  \item
    Any legally adopted/ adopted out relatives.
  \end{itemize}
\item
  Applications as attorney

  \begin{itemize}
  \tightlist
  \item
    Common for persons to apply for a grant as an attorney on behalf of
    the person entitled to take out the grant.
  \item
    Where no attorney has been appointed, a grant will be made to
    another person for use and benefit of the incapacitated person.
  \end{itemize}
\item
  Foreign domicile

  \begin{itemize}
  \tightlist
  \item
    To obtain a grant for a person domiciled abroad, extra information
    about the entitlement to take out a grant is needed.
  \end{itemize}
\item
  IHT estate and probate estate

  \begin{itemize}
  \tightlist
  \item
    Applicant must state whether an IHT400 or IHT421 was completed.
  \item
    Provide details on excepted estate if relevant.
  \end{itemize}
\item
  Legal statements

  \begin{itemize}
  \tightlist
  \item
    Applicants confirm that the will \& codicils is last will and
    testament of the person who died.
  \item
    Applicants will collect the whole estate
  \item
    Applicant will provide full details of the estate and how it has
    been distributed.
  \item
    Applicant understands that application will be rejected if
    information not provided.
  \item
    Applicant understands that proceedings for contempt of court will be
    brought if there is evidence of dishonesty.
  \end{itemize}
\end{itemize}

\hypertarget{pa1p}{%
\subsection{PA1P}\label{pa1p}}

Completed whenever there is a will, and irrespective of whether the
grant sought is a grant of probate or a grant of letters of
administration with the will.

\begin{itemize}
\tightlist
\item
  Identification of will

  \begin{itemize}
  \tightlist
  \item
    Original will, and any codicils must accompany the application must
    be provided.
  \item
    Must include date of will, and details of any foreign wills.
  \item
    Question of whether there are any minor beneficiaries (in which case
    two administrators must be appointed, if the application is by
    non-executors).
  \end{itemize}
\end{itemize}

\hypertarget{executors}{%
\subsubsection{Executors}\label{executors}}

\begin{itemize}
\tightlist
\item
  Entitlement of executors to act

  \begin{itemize}
  \tightlist
  \item
    Executors have the best right to take a grant of probate. One
    executor may obtain a grant and act alone.
  \end{itemize}
\item
  Executor lacking capacity

  \begin{itemize}
  \tightlist
  \item
    Capacity to act as executor is judged at the time of the application
    for the grant. A person appointed as an executor who lacks capacity
    to make decisions cannot apply for the grant.
  \item
    In the case where all executors lack capacity, the court can
    intervene/ give executor power to the attorney/ give executor power
    to the person entitled to the residuary estate.
  \end{itemize}
\item
  Minors

  \begin{itemize}
  \tightlist
  \item
    An executor who is a minor at the testator's death cannot act as
    executor/ obtain a grant of probate.
  \item
    If only a minor is appointed, a grant of letters of administration
    with will annexed for the use and benefit of the minor can be made
    (?)
  \end{itemize}
\item
  Spouse/ civil partner

  \begin{itemize}
  \tightlist
  \item
    If spouse appointed and marriage/ civil partnership fails,
    appointment will fail unless the testator has shown a contrary
    intention in the will.
  \end{itemize}
\item
  Renunciation

  \begin{itemize}
  \tightlist
  \item
    Persons appointed as executors may renounce their right to take the
    grant, provided that they have not `intermeddled' (e.g., notifying
    the deceased's bank of the death).
  \item
    Renunciation must be made using Form PA15, signed by the renouncer
    and filed with the Probate Service.
  \item
    Executors also appointed trustees will remain trustees despite
    renouncing executorship.
  \end{itemize}
\item
  Power reserved

  \begin{itemize}
  \tightlist
  \item
    No limit on how many executors can be appointed
  \item
    Probate will be granted to a maximum of 4 executors in respect of
    the same property.
  \item
    No need for every executor to act.
  \item
    Possible to obtain a grant limited to only part of the estate.
  \end{itemize}
\end{itemize}

\hypertarget{administrators}{%
\subsubsection{Administrators}\label{administrators}}

\begin{itemize}
\tightlist
\item
  Entitlement of administrators to act.

  \begin{itemize}
  \tightlist
  \item
    If there is a valid will but no executor able or willing to act, the
    appropriate grant is still letters of administration with will
    annexed.
  \item
    The order of priority of person(s) entitled to a grant of letters of
    administration with will annexed is governed by \textbf{NCPR 1987, r
    20}:

    \begin{enumerate}
    \def\labelenumi{\arabic{enumi}.}
    \tightlist
    \item
      Executor
    \item
      Anyone entitle to residue also holding it in trust for another
    \item
      Any other residuary legatee
    \item
      PR of any of the above
    \end{enumerate}
  \item
    If there is more than one person of equal rank and they both apply
    separately, the court will prefer an application from a beneficiary
    with a vested, as opposed a contingent interest.
  \end{itemize}
\item
  Clearing off

  \begin{itemize}
  \tightlist
  \item
    A person in a lower-ranked category may apply only if there is
    nobody in a higher category willing and able to take the grant.
  \end{itemize}
\item
  Minors

  \begin{itemize}
  \tightlist
  \item
    A minor cannot act as administrator with will annexed, nor can they
    apply for grant.
  \item
    Parent/ guardian may apply for a grant ``for the minor's use and
    benefit''.
  \end{itemize}
\item
  Number of administrators

  \begin{itemize}
  \tightlist
  \item
    Maximum of 4 in relation to the same property (s 114 SCA 1981).
  \item
    If there are multiple people entitled to act as administrator, a
    grant can be made on the application of any one of them without
    notice to the other(s) (r 27(4) NCPR 1987).
  \end{itemize}
\item
  Need for 2 administrators

  \begin{itemize}
  \tightlist
  \item
    If there is a life interest or property passing to a minor, the
    court normally requires a minimum of 2 administrators to apply for
    the grant (s 114 SCA 1981).
  \item
    But the court has discretion to only require 1
  \end{itemize}
\item
  Renunciation

  \begin{itemize}
  \tightlist
  \item
    Any person entitled to apply for a grant of letters of
    administration with will annexed can renounce in the same way as an
    executor (though using a Form PA16), except that an administrator
    does not lose the right to renounce by intermeddling.
  \end{itemize}
\end{itemize}

\hypertarget{pa1p-1}{%
\subsection{PA1P}\label{pa1p-1}}

Completed where the deceased died without a valid will.

\begin{Shaded}
\begin{Highlighting}[]
\NormalTok{title: r 22 NCPR 1987}
\NormalTok{(1) Where the deceased died on or after 1 January 1926, wholly intestate, the person or persons having a beneficial interest in the estate shall be entitled to a grant of administration in the following classes in order of priority, namely—}
\NormalTok{{-} (a) the surviving husband or wife;}
\NormalTok{{-} (b) the children of the deceased and the issue of any deceased child who died before the deceased;}
\NormalTok{{-} (c) the father and mother of the deceased;}
\NormalTok{{-} (d) brothers and sisters of the whole blood and the issue of any deceased brother or sister of the whole blood who died before the deceased;}
\NormalTok{{-} (e) brothers and sisters of the half blood and the issue of any deceased brother or sister of the half blood who died before the deceased;}
\NormalTok{{-} (f) grandparents;}
\NormalTok{{-} (g) uncles and aunts of the whole blood and the issue of any deceased uncle or aunt of the whole blood who died before the deceased;}
\NormalTok{{-} (h) uncles and aunts of the half blood and the issue of any deceased uncle or aunt of the half blood who died before the deceased.}
\end{Highlighting}
\end{Shaded}

\hypertarget{clearing-off}{%
\subsubsection{Clearing Off}\label{clearing-off}}

An applicant under r 22 must make clear that there is no one in a higher
category able to apply for the grant (again, this is called `clearing
off') and describe their own relationship to the deceased.

\hypertarget{beneficial-interest}{%
\subsubsection{Beneficial Interest}\label{beneficial-interest}}

Unless the application is made by the Treasury Solicitor or a creditor,
the applicant must have a beneficial interest in the estate (or would
have such an in interest if there was an accretion to the estate) by
virtue of the intestacy rules.

\hypertarget{minors}{%
\subsubsection{Minors}\label{minors}}

A minor cannot act as administrator and cannot apply for grant.

\hypertarget{renunciation}{%
\subsubsection{Renunciation}\label{renunciation}}

A person entitled to a grant under NCPR 1987, r 22 can renounce their
right to the grant in the same way as an administrator with the will
annexed. If they are the only relative of the deceased with a beneficial
entitlement, the grant will be made to a creditor of the deceased.

Common for a creditor to take the grant if the estate is insolvent.

\hypertarget{number-of-administrators}{%
\subsubsection{Number of
Administrators}\label{number-of-administrators}}

\begin{itemize}
\tightlist
\item
  The grant will issue to a maximum of four administrators. If there are
  more than four people with an equal entitlement, it is not possible to
  have `power reserved' to a non-proving administrator.
\item
  Where two or more people are entitled in the same degree, a grant can
  be made on the application of any one of them without notice to the
  other(s).
\item
  A minimum of two administrators is generally required where the
  intestacy creates minority interests through property being held for
  minors on the `statutory trusts'.
\end{itemize}

\hypertarget{effect-of-grant}{%
\subsection{Effect of Grant}\label{effect-of-grant}}

\begin{longtable}[]{@{}
  >{\raggedright\arraybackslash}p{(\columnwidth - 2\tabcolsep) * \real{0.2690}}
  >{\raggedright\arraybackslash}p{(\columnwidth - 2\tabcolsep) * \real{0.7310}}@{}}
\toprule()
\begin{minipage}[b]{\linewidth}\raggedright
Type of grant
\end{minipage} & \begin{minipage}[b]{\linewidth}\raggedright
Effect
\end{minipage} \\
\midrule()
\endhead
Grant of {[}\protect\hyperlink{probate}{Probate}{]} & Confirms the
authority of the executors which stems from the will and arises from the
time of death of the testator's estate. \\
Grant of administration (with or without will) & Confers authority on
the administrator and vests the deceased's property in the
administrator. Until grant issued, the administrator has no authority
and property is vested in the President of the Family Division. \\
\bottomrule()
\end{longtable}

\hypertarget{limited-grant}{%
\subsection{Limited Grant}\label{limited-grant}}

A grant of representation is normally general. Can be limited:

\begin{itemize}
\tightlist
\item
  To a specific part of a deceased's property
\item
  To settled land
\item
  To a special purpose (e.g., for a minor's use and benefits).
\end{itemize}

\hypertarget{death-of-pr}{%
\subsection{Death of PR}\label{death-of-pr}}

If there are several proving PRs administering an estate and one dies
after taking the grant but before the administration has been completed,
the surviving PRs continue to act. Where the death leaves a sole
surviving PR, the court may exercise its powers to appoint an additional
PR.

If a person entitled to be the PR (either as the executor under a will,
or by virtue of NCPR 1987, r 20 or r 22) survives the deceased but then
dies themselves without taking out a grant of representation, the AEA
1925, s 5 provides that their rights concerning the grant die with them.

\hypertarget{chain-of-representation}{%
\subsubsection{Chain of Representation}\label{chain-of-representation}}

The office of executor is personal to the executor appointed by the
testator in their will.\\
Because it is an office of personal trust an executor cannot assign that
office (although they can appoint an attorney).

\begin{Shaded}
\begin{Highlighting}[]
\NormalTok{title: s 7 AEA 1925}
\NormalTok{(1) An executor of a sole or last surviving executor of a testator is the executor of that testator.}

\NormalTok{This provision shall not apply to an executor who does not prove the will of his testator, and, in the case of an executor who on his death leaves surviving him some other executor of his testator who afterwards proves the will of that testator, it shall cease to apply on such probate being granted.}

\NormalTok{(2) So long as the chain of such representation is unbroken, the last executor in the chain is the executor of every preceding testator.}

\NormalTok{(3) The chain of such representation is broken by—}
\NormalTok{{-} (a) an intestacy; or}
\NormalTok{{-} (b) the failure of a testator to appoint an executor; or}
\NormalTok{{-} (c) the failure to obtain probate of a will;}

\NormalTok{but is not broken by a temporary grant of administration if probate is subsequently granted.}

\NormalTok{(4) Every person in the chain of representation to a testator—}
\NormalTok{{-} (a) has the same rights in respect of the real and personal estate of that testator as the original executor would have had if living; and}
\NormalTok{{-} (b) is, to the extent to which the estate whether real or personal of that testator has come to his hands, answerable as if he were an original executor.}
\end{Highlighting}
\end{Shaded}

\hypertarget{grant-de-bonis-non-administratis}{%
\subsubsection{\texorpdfstring{Grant \emph{de Bonis Non
administratis}}{Grant de Bonis Non administratis}}\label{grant-de-bonis-non-administratis}}

In situations where the chain of representation does not apply because
there are no successive proving executors, a \emph{grant de bonis non
administratis} must be obtained to the original estate.

It is issued in estates where the sole, or sole surviving, PR has died
after obtaining the grant but without having completed the
administration, and it relates only to the un-administered part of the
estate.

Two requirements apply:

\begin{enumerate}
\def\labelenumi{\arabic{enumi}.}
\tightlist
\item
  there must have been a prior grant of probate or letters of
  administration to a PR who has now died; and
\item
  the chain of representation must not apply.
\end{enumerate}

\hypertarget{caveats-and-citations}{%
\subsection{Caveats and Citations}\label{caveats-and-citations}}

Available to assist in a dispute over the right to take out a grant of
representation to an estate.

\hypertarget{caveats-ncpr-1987-r-44}{%
\subsubsection{Caveats (NCPR 1987, R 44)}\label{caveats-ncpr-1987-r-44}}

The effect of a caveat is to prevent the issue of a grant of
representation. The person lodging or entering a caveat is called a
`caveator'. A caveat might be used, for example, where a beneficiary
believes the executor named in the will lacks the mental capacity to
act, or where the validity of the will is questioned.

An application for a caveat can be made online or using paper form PA8A.

\hypertarget{citations-ncpr-1987-r-46}{%
\subsubsection{Citations (NCPR 1987, R
46)}\label{citations-ncpr-1987-r-46}}

If the person initially entitled to take the grant refuses to do so and
also refuses to renounce, the estate would remain un-administered and
the beneficiaries would be left waiting indefinitely for their
inheritance. In such circumstances, a citation provides a remedy.

\begin{longtable}[]{@{}
  >{\raggedright\arraybackslash}p{(\columnwidth - 2\tabcolsep) * \real{0.1651}}
  >{\raggedright\arraybackslash}p{(\columnwidth - 2\tabcolsep) * \real{0.8349}}@{}}
\toprule()
\begin{minipage}[b]{\linewidth}\raggedright
Type of citation
\end{minipage} & \begin{minipage}[b]{\linewidth}\raggedright
Details
\end{minipage} \\
\midrule()
\endhead
Citation to take probate & May be used where an executor has lost their
right to renounce probate by intermeddling in the estate, but has not
applied for a grant of probate and shows no signs of doing so. Once
cited, the executor must proceed with an application for the grant of
probate. If they do not, the citor can apply to the court for an order
allowing the executor to be passed over. \\
Citation to accept or refuse a gift & Standard method of clearing off a
person with a prior right to any type of grant who has not applied, and
shows no intention of applying, for a grant. \\
\bottomrule()
\end{longtable}

\hypertarget{alternative}{%
\subsection{Alternative}\label{alternative}}

If a person is unwilling to act as executor in the administration of an
estate, it is often preferable to apply to the Probate Service under the
Senior Courts Act 1981, s 116 for an order passing over that person in
favour of someone else.

\hypertarget{administration-of-estate}{%
\section{Administration of Estate}\label{administration-of-estate}}

\hypertarget{introduction-2}{%
\subsection{Introduction}\label{introduction-2}}

After PRs have obtained the grant, they must administer the estate.
Includes:

\begin{itemize}
\tightlist
\item
  Collecting deceased's assets
\item
  Paying funeral and testamentary expenses and debts
\item
  Distribute the legacies
\item
  Complete administration and distribute residual estate.
\end{itemize}

\hypertarget{administration-period}{%
\subsection{Administration Period}\label{administration-period}}

MERMAID1

But PR holds office for life, so must deal with further assets and
liabilities if they are later discovered.

\hypertarget{pr-duties}{%
\subsection{PR Duties}\label{pr-duties}}

\begin{Shaded}
\begin{Highlighting}[]
\NormalTok{title: s 25, Administration of Estates Act 1925}
\NormalTok{The personal representative of a deceased person shall be under a duty to collect and get in the real and personal estate of the deceased and administer it according to law.}
\end{Highlighting}
\end{Shaded}

PR \textbf{personally liable} for losses to the estate resulting from
any breach of duty they commit.

\begin{Shaded}
\begin{Highlighting}[]
\NormalTok{title: s 61 Trustee Act 1925 {-} Power to relieve trustee from personal liability.}
\NormalTok{If it appears to the court that a trustee, whether appointed by the court or otherwise, is or may be personally liable for any breach of trust, whether the transaction alleged to be a breach of trust occurred before or after the commencement of this Act, but has acted honestly and reasonably, and ought fairly to be excused for the breach of trust and for omitting to obtain the directions of the court in the matter in which he committed such breach, then the court may relieve him either wholly or partly from personal liability for the same. }
\end{Highlighting}
\end{Shaded}

There may be a clause in the will providing protection from liability
for mistakes made in good faith.

\hypertarget{protection-against-liability}{%
\subsection{Protection Against
Liability}\label{protection-against-liability}}

\begin{itemize}
\tightlist
\item
  s 27 TA 1925 protects against personal liability to unknown
  beneficiaries when requirements are complied with.
\item
  But does not protect against claim from missing known beneficiaries:

  \begin{itemize}
  \tightlist
  \item
    Use a Benjamin order. The court will require evidence that full
    enquiries were made to find the missing person.
  \item
    Or take out insurance (may be cheaper).
  \end{itemize}
\item
  Inheritance (Provision for Family and Dependants) Act 1975

  \begin{itemize}
  \tightlist
  \item
    PRs will be personally liable if a successful application for
    ``reasonable financial provision'' is made.
  \item
    Protect against liability by waiting \textbf{\(>6\) months from the
    date of grant of representation} before distributing assets/
    retaining sufficient assets to deal with any such claim after 6
    months.
  \end{itemize}
\end{itemize}

\hypertarget{administrative-powers}{%
\subsection{Administrative Powers}\label{administrative-powers}}

TA 2000 prescribes powers to invest trust property, appoint agents of
agents and nominees, remuneration of trustees and to insure trust
property. A duty of care is imposed, having regard to special knowledge/
skills of the trustee.

\hypertarget{powers-granted-by-will}{%
\subsubsection{Powers Granted by Will}\label{powers-granted-by-will}}

\begin{itemize}
\tightlist
\item
  Many statutory powers can be varied by express provision in the will.
\item
  Will can grant additional powers.
\item
  Will drafting

  \begin{itemize}
  \tightlist
  \item
    Good practice to fully set out the powers of the executors
  \item
    \(\exists\) STEP (Society of Trust and Estates Practitioners)
    Standard Provisions
  \end{itemize}
\end{itemize}

\hypertarget{administration-of-estate-provisions}{%
\paragraph{Administration of Estate
Provisions}\label{administration-of-estate-provisions}}

\begin{longtable}[]{@{}
  >{\raggedright\arraybackslash}p{(\columnwidth - 4\tabcolsep) * \real{0.1277}}
  >{\raggedright\arraybackslash}p{(\columnwidth - 4\tabcolsep) * \real{0.6021}}
  >{\raggedright\arraybackslash}p{(\columnwidth - 4\tabcolsep) * \real{0.2702}}@{}}
\toprule()
\begin{minipage}[b]{\linewidth}\raggedright
Topic
\end{minipage} & \begin{minipage}[b]{\linewidth}\raggedright
Statutory provisions
\end{minipage} & \begin{minipage}[b]{\linewidth}\raggedright
Possible will provision
\end{minipage} \\
\midrule()
\endhead
Power to appropriate assets without consent of beneficiaries & s 41 AEA
1925: PRs have power to appropriate any part of the estate in/ towards
satisfaction of a legacy/ share of residential estate, provided the
appropriation does not prejudice any specific beneficiary. Beneficiary
must consent to the appropriation. & Relieve trustees of the duty to ask
consents of beneficiaries. A general power to re-appropriate assets for
trustees is useful. \\
Power to insure & s 19 TA 1925: PRs and trustees have power to insure
trust property against any risks, to the full value of the property, and
to pay premiums out of capital or income. & Can give trustees the power
to insure the life of the beneficiary/ settlor. \\
Power to accept receipts from/ on behalf of minors & Generally,
unmarried minor cannot ive a good receipt for capital or income. Married
minor can give good receipt for income (s 21 LPA 1925). s 3 Children Act
1989 provides that parents with parental responsibility have the right
to receive/ recover money for the benefit of the child. & Allow PRs to
accept receipt from a child over 16/ leave legacy to trustees for the
benefit of a minor. \\
\bottomrule()
\end{longtable}

\hypertarget{administration-of-trusts-provisions}{%
\paragraph{Administration of Trusts
Provisions}\label{administration-of-trusts-provisions}}

In many cases, PRs will hold the residue of the estate as trustees:

\begin{enumerate}
\def\labelenumi{\arabic{enumi}.}
\tightlist
\item
  Where the beneficiary has a contingent interest
\item
  Where the interests in the property are divided (e.g., between income
  and capital).
\end{enumerate}

\begin{longtable}[]{@{}
  >{\raggedright\arraybackslash}p{(\columnwidth - 4\tabcolsep) * \real{0.0855}}
  >{\raggedright\arraybackslash}p{(\columnwidth - 4\tabcolsep) * \real{0.5573}}
  >{\raggedright\arraybackslash}p{(\columnwidth - 4\tabcolsep) * \real{0.3573}}@{}}
\toprule()
\begin{minipage}[b]{\linewidth}\raggedright
Topic
\end{minipage} & \begin{minipage}[b]{\linewidth}\raggedright
Statutory provisions
\end{minipage} & \begin{minipage}[b]{\linewidth}\raggedright
Possible will provision
\end{minipage} \\
\midrule()
\endhead
Power to invest trust funds & s 3 TA 2000: general power of investment
(though excludes investments in land, other than by mortgage) Trustees
required to take proper advice and review investments periodically. Must
have regard to standard investment criteria. & Express clause not
necessary \\
Power to purchase land & s 8 TA 2000:: trustees have the power to
acquire freehold or leasehold land in the UK. Does not allow foreign
purchases of land/ purchasing interest in land with someone else. &
Express clause to widen powers \\
Power to sell personalty & There is doubt over whether trustees who do
not hold land have an implied power of sale & Many wills continue to
impose an express trust for sale over residue. \\
Power of maintenance & s 31 TA: power to use income for minor's
maintenance, education or benefit. Income paid to beneficiary once they
reach 18. & Can change the relevant age from 18 to 21. \\
Power to apply capital & s 32 TA 1925: allows trustees to permit a
beneficiary with an interest in capital to have the capital applied for
their benefit before they are entitled to receive it. The amount applied
can be up to the level of the beneficiary's vested share. Advance made
must be brought into account on the final distribution (s 32(1)(b)). &
Can give trustees discretion over how to bring into account any payments
received by beneficiaries. Can give trustees the power to give or lend
capital from the fund to someone with an interest only in income. \\
Power to accept receipts from/ on behalf of minors & Statutory powers of
maintenance and advancement as above. & Not needed \\
Control of trustees by beneficiaries & s 19 TLATA 1996 provides that
beneficiaries of full age and capacity together entitled to the whole
fund may direct trustees to retire and appoint new trustees. & s 19
TLATA can be excluded. \\
Trusts of land: duty to consult beneficiaries & s 11 TLATA: trustees
dealing with land must consult any beneficiary who is of full age and
beneficially entitled to an interest in possession in the land. Must
give effect to the wishes of such a beneficiary, so far as is consistent
with the general interest of the trust. & Often excluded in the will. \\
Trusts of land: beneficiary's rights of occupation & s 12: Beneficiary
with a beneficial interest in possession has the right to occupy land in
some circumstances & No power to exclude, but can make a declaration
that the purpose of the trust is not for the occupation of land. \\
Balance between beneficiaries & Trustees are under a duty to ensure a
fair balance between the interests of beneficiaries. & Authorise
trustees to treat the interests of one beneficiary as of paramount
importance. \\
\bottomrule()
\end{longtable}

\hypertarget{maintenance-and-advancement}{%
\subsubsection{Maintenance and
Advancement}\label{maintenance-and-advancement}}

When considering their dispositive powers and obligations it is
important for trustees to distinguish between the trust capital and the
income generated by that capital.

\begin{itemize}
\tightlist
\item
  In some cases different beneficiaries will be entitled to the income
  and the capital. A good example is a life interest trust.
\item
  In other cases, the same beneficiary may be entitled to both capital
  and income but not necessarily at the same time:

  \begin{itemize}
  \tightlist
  \item
    A beneficiary may have a right to receive capital upon the
    occurrence of a particular event, such as reaching a specified age.
    In the meantime, they may be entitled to receive the income as it
    arises.
  \item
    Alternatively, the trustees may have a power or obligation to
    accumulate the income (i.e.~add it to the capital). If the income is
    accumulated, the beneficiary will not receive anything until it is
    time for the capital to be distributed.
  \end{itemize}
\end{itemize}

Can a beneficiary ever make an early claim to the capital or accumulated
income?

\hypertarget{entitlement-to-income}{%
\paragraph{Entitlement to Income}\label{entitlement-to-income}}

Adult beneficiaries with vested interests in trust property will usually
have a right to receive the trust income as it arises.

Adult beneficiaries with contingent interests have a statutory right
under
\href{https://www.legislation.gov.uk/ukpga/Geo5/15-16/19/section/31}{s
31(3) Trustee Act 1925} (`TA 1925') to receive income if the trust
`carries the intermediate income' (i.e., the income produced between
creation of the contingent interest and satisfaction of the
contingency). The circumstances in which a trust `carries the
intermediate income' are beyond the scope of this module. If it is
relevant to the analysis, a question will specify whether a contingent
interest carries the intermediate income.

Minor beneficiaries will not usually be entitled to receive trust income
until they reach the age of 18.

However, s 31(1) TA 1925 gives the trustees a \textbf{power of
maintenance}. This allows the trustees to pay trust income to
beneficiaries who would benefit from receiving it immediately, instead
of waiting for their interest to vest in possession.

Note that the provisions of s 31 TA 1925 may be excluded or varied by a
trust instrument:
\href{https://www.legislation.gov.uk/ukpga/Geo5/15-16/19/section/69}{s
69(2) TA 1925}.

\hypertarget{power-of-maintenance}{%
\paragraph{Power of Maintenance}\label{power-of-maintenance}}

\begin{Shaded}
\begin{Highlighting}[]
\NormalTok{title: s 31 TA 1925 {-}  Power to apply income for maintenance and to accumulate surplus income during a minority.}

\NormalTok{(1) Where any property is held by trustees in trust for any person for any interest whatsoever, whether vested or contingent, then, subject to any prior interests or charges affecting that property—}
\NormalTok{{-} (i) during the infancy of any such person, if his interest so long continues, the trustees may, at their sole discretion, pay to his parent or guardian, if any, or otherwise apply for or towards his maintenance, education, or benefit, the whole or such part, if any, of the income of that property as the trustees may think fit, whether or not there is—}
\NormalTok{    {-} (a) any other fund applicable to the same purpose; or}
\NormalTok{    {-} (b) any person bound by law to provide for his maintenance or education; and}
\NormalTok{{-} (ii) if such person on attaining the age of eighteen years has not a vested interest in such income, the trustees shall thenceforth pay the income of that property and of any accretion thereto under subsection (2) of this section to him, until he either attains a vested interest therein or dies, or until failure of his interest:}

\NormalTok{(2) During the infancy of any such person, if his interest so long continues, the trustees shall accumulate all the residue of that income by investing it, and any profits from so investing it from time to time in authorised investments, and shall hold those accumulations as follows:—}
\NormalTok{{-} (i) If any such person—}
\NormalTok{    {-} (a) attains the age of eighteen years, or marries under that age or forms a civil partnership under that age, and his interest in such income during his infancy, or until his marriage or his formation of a civil partnership, is a vested interest or;}
\NormalTok{    {-} (b) on attaining the age of eighteen years or on marriage, or formation of a civil partnership, under that age becomes entitled to the property from which such income arose in fee simple, absolute or determinable, or absolutely, or for an entailed interest;}
\NormalTok{{-} the trustees shall hold the accumulations in trust for such person absolutely, but without prejudice to any provision with respect thereto contained in any settlement by him made under any statutory powers during his infancy, and so that the receipt of such person after marriage or formation of a civil partnership, and though still an infant shall be a good discharge, and}
\NormalTok{{-} (ii) In any other case the trustees shall, notwithstanding that such person had a vested interest in such income, hold the accumulations as an accretion to the capital of the property from which such accumulations arose, and as one fund with such capital for all purposes, and so that, if such property is settled land, such accumulations shall be held upon the same trusts as if the same were capital money arising therefrom;}

\NormalTok{but the trustees may, at any time during the infancy of such person if his interest so long continues, apply those accumulations, or any part thereof, as if they were income arising in the then current year.}

\NormalTok{(3) This section applies in the case of a contingent interest only if the limitation or trust carries the intermediate income of the property, but it applies to a future or contingent legacy by the parent of, or a person standing in loco parentis to, the legatee, if and for such period as, under the general law, the legacy carries interest for the maintenance of the legatee, and in any such case as last aforesaid the rate of interest shall (if the income available is sufficient, and subject to any rules of court to the contrary) be five pounds per centum per annum.}

\NormalTok{(4) This section applies to a vested annuity in like manner as if the annuity were the income of property held by trustees in trust to pay the income thereof to the annuitant for the same period for which the annuity is payable, save that in any case accumulations made during the infancy of the annuitant shall be held in trust for the annuitant or his personal representatives absolutely.}

\NormalTok{(5) This section does not apply where the instrument, if any, under which the interest arises came into operation before the commencement of this Act.}
\end{Highlighting}
\end{Shaded}

Where a minor has a vested interest in trust property (or a contingent
interest which carries the intermediate income), the trustees have a
wide statutory power under
\href{https://www.legislation.gov.uk/ukpga/Geo5/15-16/19/section/31}{s
31 TA 1925} to pay the income as they think fit. The trustees may:

\begin{itemize}
\tightlist
\item
  Pay the income to the child or their parent or guardian.
\item
  Apply the income directly for the child's `maintenance, education or
  benefit'.
\end{itemize}

This power may be used even if there is another fund available for the
same purpose or other persons legally obliged to provide for the child.

Income which is not paid out under s 31 must be accumulated but can
still be paid out subsequently using the power of maintenance as long as
the beneficiary remains under 18.

Once the beneficiary reaches 18, the accumulated income is added to the
capital and can no longer be accessed by the beneficiary until they are
entitled to receive the capital.

Income generated after the beneficiary is 18 must be paid directly to
them. Although the power of maintenance gives the trustees a broad
discretion, the following points must be noted:

\begin{enumerate}
\def\labelenumi{\arabic{enumi}.}
\tightlist
\item
  The power of maintenance is a \textbf{fiduciary power}. The trustees
  must consciously consider the exercise of the power and, if they
  choose to exercise it, must act in good faith in the interests of the
  beneficiary.
\item
  The income must be used for the primary benefit of the minor
  beneficiary, but it does not matter that it may therefore indirectly
  benefit their parent or guardian.
\item
  It is an improper exercise of the power to unquestioningly pay it to
  the minor's parent or guardian See e.g.~{[}{[}Wilson v Turner (1883)
  22 Ch. D. 521{]}{]}
\end{enumerate}

Because the power of maintenance only applies during the minority of a
beneficiary, it is good practice for trustees to consider exercising the
power shortly before the beneficiary turns 18, particularly if the
beneficiary has a contingent interest to the trust capital.

If it is not exercised at this time, all accumulated income will become
part of the trust capital and the beneficiary will not be able to access
it until their interest in the capital vests.

\begin{Shaded}
\begin{Highlighting}[]
\NormalTok{A testator leaves an estate on trust for A (26), B (19) and C (16) in equal shares upon their reaching 25.}

\NormalTok{A has reached the age of 25 so their share has vested in possession and should be distributed.}

\NormalTok{B’s share has vested in interest. Until B reaches 25, the income generated by B’s share must be paid to B. (Note that B has \_Saunders v Vautier\_ rights so could choose to collapse their share before this).}

\NormalTok{C is a minor and until C reaches the age of 18 the trustees have the power (but not the obligation) to pay income to C’s parent or guardian (or otherwise apply it for C’s maintenance, education and benefit e.g., by paying school fees directly to the school). Income not used is accumulated.}
\end{Highlighting}
\end{Shaded}

S 31 is commonly expressly amended to defer a beneficiary's right to
income, meaning they become entitled to both income and capital at the
same time.

\hypertarget{power-to-advance-capital}{%
\paragraph{Power to Advance Capital}\label{power-to-advance-capital}}

\begin{Shaded}
\begin{Highlighting}[]
\NormalTok{title: s 32 TA 1925 {-} Power of advancement}

\NormalTok{(1) Trustees may at any time or times pay or apply any capital money subject to a trust, or transfer or apply any other property forming part of the capital of the trust property, for the advancement or benefit, in such manner as they may, in their absolute discretion, think fit, of any person entitled to the capital of the trust property or of any share thereof, whether absolutely or contingently on his attaining any specified age or on the occurrence of any other event, or subject to a gift over on his death under any specified age or on the occurrence of any other event, and whether in possession or in remainder or reversion, and such payment, transfer or application may be made notwithstanding that the interest of such person is liable to be defeated by the exercise of a power of appointment or revocation, or to be diminished by the increase of the class to which he belongs:}

\NormalTok{Provided that—}

\NormalTok{{-} (a) property (including any money) so paid, transferred or applied for the advancement or benefit of any person must not, altogether, represent more than the presumptive or vested share or interest of that person in the trust property; and}
\NormalTok{{-} (b) if that person is or becomes absolutely and indefeasibly entitled to a share in the trust property the money or other property so paid, transferred or applied shall be brought into account as part of such share; and}
\NormalTok{{-} (c) no such payment, transfer or application shall be made so as to prejudice any person entitled to any prior life or other interest, whether vested or contingent, in the money or other property paid, transferred or applied unless such person is in existence and of full age and consents in writing to such payment or application.}
\NormalTok{{-} (1A) In exercise of the foregoing power trustees may pay, transfer or apply money or other property on the basis (express or implied) that it shall be treated as a proportionate part of the capital out of which it was paid, transferred or applied, for the purpose of bringing it into account in accordance with proviso (b) to subsection (1) of this section.}

\NormalTok{(2) This section does not apply to capital money arising under the M1 Settled Land Act 1925.}

\NormalTok{(3) This section does not apply to trusts constituted or created before the commencement of this Act.}
\end{Highlighting}
\end{Shaded}

A beneficiary who expects to receive capital from a trust at a future
date may wish to receive their capital before it vests in possession.

\href{https://www.legislation.gov.uk/ukpga/Geo5/15-16/19/section/32}{s
32 TA 1925} gives trustees the power to use capital for the `advancement
or benefit' of a beneficiary before the beneficiary becomes absolutely
entitled to the property.

The power of advancement:

\begin{itemize}
\tightlist
\item
  May be used by both adult and minor beneficiaries
\item
  Applies to both vested and contingent interests
\item
  Can be modified or excluded by the trust instrument:
  \href{https://www.legislation.gov.uk/ukpga/Geo5/15-16/19/section/69}{s
  69(2) TA 1925}
\end{itemize}

The trustees may use the power of advancement to pay up to 100\% of a
beneficiary's entitlement before it vests at all (if the interest is
contingent) or before it vests in possession (if the interest is vested
in interest).

Because the exercise of the power may prejudice other beneficiaries, the
power may only be exercised with the written consent of beneficiaries
with a prior interest.

For example, if a trustee holds property on trust for A for life,
remainder to B, the trustees can only exercise the power of advancement
in favour of B with A's written consent. This is because providing B
with their capital early will clearly prejudice A.

\begin{itemize}
\tightlist
\item
  In the most extreme case, if B requests 100\% of their capital, A's
  life interest will be extinguished.
\item
  Even if B requests less than 100\% of their capital, this will reduce
  the value of A's interest as there will be smaller capital fund from
  which to generate A's income.
\end{itemize}

Consent can only be provided by beneficiaries who are of full age and
sound mind.

\hypertarget{meaning-of-advancement}{%
\paragraph{Meaning of `advancement'}\label{meaning-of-advancement}}

A key issue for trustees when deciding whether to exercise the power of
advancement is whether this action will result in the advancement or
benefit of the beneficiary. Over time the case law in this area has
developed to recognise a broader meaning of advancement than its
previous limitations to the areas of education, career and marriage.

Advancement has now been recognised to provide for an immediate
financial benefit for a beneficiary, such as to avoid an inheritance tax
liability, {[}{[}Pilkington v Inland Revenue Commissioners {[}1964{]} AC
612{]}{]}. In this case Viscount Radcliffe defined advancement as `any
use of the money which will improve the material situation of the
beneficiary'.

Subsequent case law indicates that the advancement of the beneficiary
can include the improvement of the beneficiary's moral well-being by
giving the money for charitable purposes, but only to the extent that
the beneficiary would have otherwise used their own resources for such
purposes (see {[}{[}Re Clore's Settlement Trusts {[}1966{]} 1 WLR
955{]}{]} and {[}{[}X v A {[}2006{]} 1 WLR 741{]}{]})\_.

Trustees have a duty following the exercise of the power of advancement
to ensure that the money is being used for the purposes that it was
provided. Should the beneficiary be found to be spending the money on
something else, the trustees should not pay any further money to the
beneficiary. However, they may instead pay money directly to a third
party for the advancement of the beneficiary instead.

In {[}{[}Re Pauling's Settlement Trusts {[}1964{]} Ch 303{]}{]} the
trust was managed by a bank with a power to advance property to the
children of a marriage. The bank made a number of advancements to the
children. These advancements were used for the benefit of the children's
parents rather than their own benefit, including the purchase of a house
in the parents' names. It was found that the trustees were obliged to
check that the money had been applied for the purpose that it was
advanced and not leave the recipients free to spend the money as they
wished. The bank's failure to do this was a breach of trust.

\hypertarget{bringing-the-payment-into-account}{%
\paragraph{Bringing the Payment into
Account}\label{bringing-the-payment-into-account}}

The trustees may use the power of advancement to pay up to 100\% of a
beneficiary's {[}{[}beneficial entitlement{]}{]}.

Any such payment must be brought into account when the beneficiary
becomes absolutely entitled. In other words, the amount that the
beneficiary will receive when their interest vests will be reduced
proportionately to reflect the proportion of the capital that they
received early.

Trustees have a choice between treating the share advanced as a
\textbf{proportionate share of the overall trust value or its strict
monetary value}. The choice they make could have significant
consequences for both the beneficiary who receives the advancement and
all the other beneficiaries. This is best illustrated by way of example.

\begin{Shaded}
\begin{Highlighting}[]
\NormalTok{Trustees hold a trust fund worth £20,000 on trust for A, B, C and D in equal shares when they reach 18. The trustees exercise the power of advancement to pay £5,000 to A. The trustees must choose whether this is to be treated as A\textquotesingle{}s proportionate share of the fund. This could make a significant difference to A as £5,000 would amount to 100\% of A\textquotesingle{}s share in the trust fund at the date it was paid.}

\NormalTok{If the £5,000 is treated as A’s proportionate share, A will no longer be entitled to anything from the trust fund, regardless of how much the fund is worth when A reaches 18.}

\NormalTok{For example, if the fund grows to £25,000 this will be shared equally between B, C and D (£8,333 each).}

\NormalTok{If the £5,000 is not treated as A’s proportionate share, the trustees will take A’s payment into account before making their distributions, bringing the total trust value to £30,000. That is then shared equally between all four beneficiaries. B,C and D each receive £7,500. A has already had £5,000 so is entitled to an additional £2,500.}
\end{Highlighting}
\end{Shaded}

\begin{quote}
{[}!summary{]} - Adult beneficiaries with vested interests in trust
property will usually have a right to receive the trust income as it
arises. Those with contingent interests will only have a right to income
if their interest `carries the intermediate income'. - Minor
beneficiaries will not usually have a right to income but the trustees
have a statutory power under s 31 TA 1925 to pay the income to the
beneficiary's parent or guardian, or apply it directly for the
beneficiary's maintenance, education or benefit. This gives the trustees
a wide discretion - Income which is not paid to the beneficiaries before
the age of 18 is accumulated and added to the capital. - Trustees also
have a statutory power under s 32 TA 1925 to advance capital. - The
power of advancement can be exercised in favour of any beneficiary, even
if their interest is contingent, but requires the consent of
beneficiaries with a prior interest. - The power of advancement must be
used for their advancement or benefit. - Both powers may be modified or
excluded by the trust instrument.
\end{quote}

\hypertarget{collecting-deceaseds-assets}{%
\subsection{Collecting Deceased's
Assets}\label{collecting-deceaseds-assets}}

\begin{itemize}
\tightlist
\item
  PRs produce grant of representation to whoever is holding the various
  asset, and then collect them.
\item
  For some assets, a grant is not required (e.g., realising assets under
  Administration of Estates (Small Payments) Act 1965).
\end{itemize}

Some properties do not devolve on the PRs:

\begin{itemize}
\tightlist
\item
  Life interest
\item
  Joint tenancy
\item
  Policy held in trust for others
\item
  Pension schemes
\end{itemize}

\hypertarget{funeral-and-testamentary-expenses}{%
\subsection{Funeral and Testamentary
Expenses}\label{funeral-and-testamentary-expenses}}

\begin{itemize}
\tightlist
\item
  PR should pay outstanding debts and the funeral account as soon as
  monies can be collected/ realised
\item
  May need to repay a loan from deceased's bank to pay IHT to obtain the
  grant.

  \begin{itemize}
  \tightlist
  \item
    Such a loan will probably have come with a `first proceeds'
    undertaking.
  \end{itemize}
\item
  When considering which assets to sell, PRs should consider:

  \begin{enumerate}
  \def\labelenumi{\arabic{enumi}.}
  \tightlist
  \item
    Provisions of deceased's will
  \item
    Beneficiaries' wishes
  \item
    Tax consequences.
  \end{enumerate}
\end{itemize}

\hypertarget{funeral-expenses}{%
\subsubsection{Funeral Expenses}\label{funeral-expenses}}

Reasonable funeral expenses are payable from the deceased's estate.

\hypertarget{testamentary-expenses}{%
\subsubsection{Testamentary Expenses}\label{testamentary-expenses}}

These are expenses incident to the proper performance of the duties of a
PR. Includes:

\begin{itemize}
\tightlist
\item
  Costs of obtaining grant
\item
  Costs of collecting and preserving deceased's assets
\item
  Costs of administering deceased's estate (solicitor and valuer fees)
\item
  IHT payable on death of deceased's property in the UK which vests in
  the PRs (s 211 IHTA 1984).
\end{itemize}

\hypertarget{solvent-estate}{%
\subsubsection{Solvent Estate}\label{solvent-estate}}

The statutory order for payment of debts is set out in AEA 1925, Sch 1,
Part II (s 34(3) AEA 1925).

\begin{Shaded}
\begin{Highlighting}[]
\NormalTok{title: General rule}
\NormalTok{When choosing assets to pay funeral/ testamentary expenses and debts, assets forming part of the reside are to be used before property given to specific legatees. }
\end{Highlighting}
\end{Shaded}

\hypertarget{subject-to}{%
\paragraph{Subject to}\label{subject-to}}

The above order is subject to:

\begin{itemize}
\tightlist
\item
  s 35 AEA 1925: a beneficiary who takes an asset which is security for
  a debt takes the asset subject to the debt.
\item
  The will can vary ss 34(3) \& 35 -- check for any provisions.
\end{itemize}

\hypertarget{insolvent-estate}{%
\subsubsection{Insolvent Estate}\label{insolvent-estate}}

\begin{Shaded}
\begin{Highlighting}[]
\NormalTok{An estate is insolvent if the assets are insufficient to discharge in full the funeral, testamentary and administration expenses, debts and liabilities.}
\end{Highlighting}
\end{Shaded}

In doubtful cases, PRs should administer the estate as if it is
insolvent. For an insolvent estate being administered out of court,
follow the order of distribution in Administration of Insolvent Estates
of Deceased Persons Order 1986.

\hypertarget{paying-legacies}{%
\subsection{Paying Legacies}\label{paying-legacies}}

Step 2; consider discharging the gifts arising on death, other than
gifts of the residuary estate. Consider making interim distributions to
the residuary beneficiaries.

\hypertarget{specific-legacies}{%
\subsubsection{Specific Legacies}\label{specific-legacies}}

\begin{itemize}
\tightlist
\item
  Methods of transferring the property to the beneficiary/ trustee will
  vary

  \begin{itemize}
  \tightlist
  \item
    The legal estate in a house or flat should be vested in a
    beneficiary by an `assent'.
  \end{itemize}
\item
  In the case of specific gifts (only), the vesting of the asset in the
  beneficiary is retrospective to the date of death, so that any income
  produced by the property.

  \begin{itemize}
  \tightlist
  \item
    Beneficiary is not entitled to the income as it arises, but must
    wait for the PR to vest the property in the beneficiary.
  \end{itemize}
\item
  Costs of transferring the property to a legatee and cost of insurance
  cover taken are the responsibility of the legatee, who should
  reimburse the PR.
\end{itemize}

\hypertarget{pecuniary-legacies}{%
\subsubsection{Pecuniary Legacies}\label{pecuniary-legacies}}

The will may specify an intention to pay the legacy, e.g., from the
residuary estate. Where the will makes no such provision for pecuniary
legacies, they are paid primarily from residuary personalty.

\hypertarget{time-for-payment-1}{%
\paragraph{Time for Payment}\label{time-for-payment-1}}

\begin{itemize}
\tightlist
\item
  General rule

  \begin{itemize}
  \tightlist
  \item
    Pecuniary legacy payable at the end of ``the executor's year'': one
    year after testator's death.
  \item
    s 44 EA 1925: PRs are not bound to distribute the estate before the
    expiration of one year from the death.
  \end{itemize}
\item
  4 occasions where interest is payable on a pecuniary legacy from the
  date of death:

  \begin{enumerate}
  \def\labelenumi{\arabic{enumi}.}
  \tightlist
  \item
    Legacy payable in satisfaction of debt owed to a creditor
  \item
    Legacy charged on land owned by testator
  \item
    Legacy payable to the testator's minor child
  \item
    Legacy payable to any minor where the intention is to provide for
    the maintenance of that minor.
  \end{enumerate}
\end{itemize}

\hypertarget{administration-and-distribution}{%
\subsection{Administration and
Distribution}\label{administration-and-distribution}}

Before drawing up estate accounts and making a final distribution of
residue, the PRs should deal with IHT liability and taxes.

\hypertarget{adjusting-iht-assessment}{%
\subsubsection{Adjusting IHT
Assessment}\label{adjusting-iht-assessment}}

The amount of IHT payable may have to be adjusted since the IHT account
was submitted:

\begin{itemize}
\tightlist
\item
  Discovery of additional assets/ liabilities
\item
  Discovery of lifetime transfers
\item
  Agreement of provisionally estimated values
\item
  Agreement between PRs and HMRC of a tax liability or repayment
\item
  ``Loss relief'' from sales made by PRs after deceased's death.
\end{itemize}

\hypertarget{iht-loss-relief}{%
\paragraph{IHT Loss Relief}\label{iht-loss-relief}}

PRs often force to sell assets to raise cash to pay for debts, tax
liabilities or legacies. If PRs end up selling assets for less than
their value at the date of death, they can claim loss on sale relief.

\begin{Shaded}
\begin{Highlighting}[]
\NormalTok{Where ‘qualifying investments’ are sold within 12 months of death for less than their market value at the date of death (ie ‘probate value’) then the sale price may be substituted for the market value at death and the IHT liability adjusted accordingly (IHTA 1984, ss 178–189).}
\end{Highlighting}
\end{Shaded}

\begin{itemize}
\tightlist
\item
  `Qualifying investments' are shares or securities which are quoted on
  a recognised stock exchange at the date of death, and also holdings in
  authorised unit trusts.
\item
  Relief is not automatic; must be claimed.
\item
  Available only when PR makes the sale, not when the beneficiary does.
\item
  If relief claimed, value of qualifying investments at death must be
  reduced for CGT purposes (s 187).

  \begin{itemize}
  \tightlist
  \item
    So never claim the relief if the estate is passing to a surviving
    spouse/ civil partner.
  \end{itemize}
\end{itemize}

\hypertarget{limitation}{%
\paragraph{Limitation}\label{limitation}}

To stop PRs selling shares to create a loss and then either buying them
back (`bed and breakfasting') or reinvesting the proceeds in other
shares, s 180 restricts the relief where the appropriate person buys new
qualifying investments within the period starting with death and ending
two months after the date of the last sale.

ss 190-198: provisions allowing loss relief in relation to the sale of
land within 4 years of a death at a loss.

\hypertarget{continuing-iht-liability}{%
\subsubsection{Continuing IHT
Liability}\label{continuing-iht-liability}}

\hypertarget{iht-instalments}{%
\paragraph{IHT Instalments}\label{iht-instalments}}

\begin{itemize}
\tightlist
\item
  If PRs opted to pay IHT by instalments, they continue to be liable
  until all instalments have been paid.
\item
  So don't transfer all the assets to the beneficiaries.
\end{itemize}

\hypertarget{iht-on-lifetime-transfers}{%
\paragraph{IHT on Lifetime Transfers}\label{iht-on-lifetime-transfers}}

\begin{itemize}
\tightlist
\item
  General rule: donees of lifetime transfers are primarily liable for
  the tax. P
\item
  Rs of the donor's estate may become liable if the tax remains unpaid
  by the donees 12 months after the end of the month in which the donor
  died.
\item
  PRs' liability is limited to the extent of the deceased's assets which
  they have received, or would have received in the administration of
  the estate, but for their neglect or default.
\item
  If the deceased gave away property during their lifetime but reserved
  a benefit in that property, such property is treated as part of their
  estate on death.

  \begin{itemize}
  \tightlist
  \item
    The same procedure applies: PRs liable if tax remains unpaid after
    12 months.
  \end{itemize}
\end{itemize}

\hypertarget{corrective-account}{%
\subsubsection{Corrective Account}\label{corrective-account}}

Submit Form C4 to HMRC to report all variations to assets/ liabilities
and reliefs.

\hypertarget{iht-clearance}{%
\subsubsection{IHT Clearance}\label{iht-clearance}}

PRs will want to obtain confirmation from HMRC that there is no further
claim to IHT. Effects:

\begin{itemize}
\tightlist
\item
  Discharges the PRs and all other persons from further liability to IHT
  (unless there is fraud/ non-disclosure of material facts).
\item
  Extinguishes any charge imposed by HMRC on deceased's property.
\end{itemize}

HMRC must apply for a clearance certificate using form IHT30. They will
reply with a letter certifying the discharge.

\hypertarget{income-tax-and-cgt}{%
\subsubsection{Income Tax and CGT}\label{income-tax-and-cgt}}

\hypertarget{deceaseds-liability}{%
\paragraph{Deceased's Liability}\label{deceaseds-liability}}

\begin{Shaded}
\begin{Highlighting}[]
\NormalTok{Immediately following the death, the PRs must make a return to HMRC of the income and capital gains of the deceased for the period starting on 6 April before the death and ending with the date of death.}
\end{Highlighting}
\end{Shaded}

\begin{itemize}
\tightlist
\item
  Any liability to tax is a debt of the deceased, to be paid by PRs
  during administration.
\item
  Any tax refund represents an asset.
\end{itemize}

\hypertarget{administration-period-1}{%
\paragraph{Administration Period}\label{administration-period-1}}

For each income tax year (or part) during the administration period, the
PRs must calculate their income tax and CGT liability on assets disposed
of for administration purposes.

\hypertarget{paying-income-tax-cgt}{%
\paragraph{Paying Income Tax \& CGT}\label{paying-income-tax-cgt}}

PRs can make a \textbf{one-off informal payment} of all income tax and
CGT at the end of administration, unless the estate is classified as a
complex estate.

\begin{Shaded}
\begin{Highlighting}[]
\NormalTok{title: Complex estate}
\NormalTok{An estate is considered complex if either: }
\NormalTok{1. the value of the estate exceeds £2.5 million; or }
\NormalTok{2. tax due for the whole of the administration period exceeds £10,000; or }
\NormalTok{3. the value of assets sold in a tax year exceeds £500,000.}
\end{Highlighting}
\end{Shaded}

\hypertarget{exception}{%
\subparagraph{Exception}\label{exception}}

UK residential land must be paid within a short period of completion of
the sale.

\hypertarget{income-tax}{%
\paragraph{Income Tax}\label{income-tax}}

The rules of income tax for PRs are special:

\begin{itemize}
\tightlist
\item
  No higher rate of income tax
\item
  No personal savings and dividend allowance
\item
  There is still a personal allowance.
\end{itemize}

\begin{longtable}[]{@{}ll@{}}
\toprule()
Asset & Rate of tax \\
\midrule()
\endhead
Dividends & 8.75\% \\
Other income & 20\% \\
\bottomrule()
\end{longtable}

\hypertarget{isas}{%
\subparagraph{ISAs}\label{isas}}

Investments held in an ISA wrapper will not be liable to income tax (or
CGT) during the administration of the deceased's estate, or three years
from the death, whichever is the shorter. PRs will, therefore, pay no
tax during that period on the deceased's ISA investments.

\hypertarget{concession}{%
\subparagraph{Concession}\label{concession}}

If the only income is interest and the interest does not exceed £500,
PRs pay no income tax.

\hypertarget{relief}{%
\paragraph{Relief}\label{relief}}

If PRs take out a bank loan to pay the IHT on the deceased's personal
property in the UK in order to obtain a grant, relief can be claimed on
any interest paid on the bank loan.

\hypertarget{beneficiarys-income-tax-liability}{%
\paragraph{Beneficiary's Income Tax
Liability}\label{beneficiarys-income-tax-liability}}

After PR's income tax position settled, net income will be paid to the
beneficiary. The beneficiary must report the grossed up amount of this
income in their income tax return.

PRs must supply beneficiary with Form R185 - certificate of deduction of
tax. Beneficiaries whose income, including the estate income, is not
above the level of the personal allowance or the savings or dividend
allowances can reclaim the tax paid.

\hypertarget{capital-gains-tax}{%
\subsubsection{Capital Gains Tax}\label{capital-gains-tax}}

\begin{itemize}
\tightlist
\item
  On death, there is no disposal for CGT purposes, so no liability
  arises.
\item
  If PRs dispose of chargeable assets during the administration of the
  estate to raise cash, they are liable to CGT on an chargeable gains
  they make (except ISAs)
\item
  Rate is 20\%, or 28\% for residential property.
\item
  PRs may also deduct from the disposal consideration the incidental
  cost of the disposal, and a proportion of the cost of valuing the
  deceased's estate for probation purposes.
\item
  PRs may claim the annual exemption for the tax year in which the
  deceased's died, and the following two years (if the administration
  lasts this long).
\item
  Exemption of £12,300/ year

  \begin{itemize}
  \tightlist
  \item
    The PRs could plan asset sales carefully to realise gains in stages
    for each tax year it is available.
  \end{itemize}
\end{itemize}

\hypertarget{sale-at-a-loss}{%
\paragraph{Sale at a Loss}\label{sale-at-a-loss}}

An allowable loss for CGT will arise if PRs sell assets for less than
their value at death. The loss may be relieved by setting it against
gains in the same or any future tax year by the PRs. Plan sales
carefully.

\hypertarget{accelerated-cgt-payment-on-uk-residential-land}{%
\paragraph{Accelerated CGT Payment on UK Residential
Land}\label{accelerated-cgt-payment-on-uk-residential-land}}

Finance Act 2019, s 14 \& Sch 2: accelerated reporting and payment dates
for CGT on UK land disposed of by non-residents. Return and payment must
be submitted within 60 days of the disposal.

The charge extends to all disposals of UK residential land on/ after 6
April 2020, but only where tax is due. The tax is paid using an
approximation of the level of income expected during the year. The
payment is made on account of the tax due at the end of the year.

\hypertarget{transfer-of-assets-to-legatees}{%
\paragraph{Transfer of Assets to
Legatees}\label{transfer-of-assets-to-legatees}}

If PRs vest the assets in the legatees rather than selling them, no
chargeable gain or allowable loss arises. The beneficiary or trustee is
assumed to acquire the asset transferred at its probate value.

\hypertarget{transfer-of-assets-to-residuary-beneficiaries}{%
\subsubsection{Transfer of Assets to Residuary
Beneficiaries}\label{transfer-of-assets-to-residuary-beneficiaries}}

If there have been interim distributions to certain beneficiaries, these
should be taken into account when determining the transfer of remaining
assets to the residuary beneficiaries .

\begin{longtable}[]{@{}
  >{\raggedright\arraybackslash}p{(\columnwidth - 4\tabcolsep) * \real{0.0347}}
  >{\raggedright\arraybackslash}p{(\columnwidth - 4\tabcolsep) * \real{0.3028}}
  >{\raggedright\arraybackslash}p{(\columnwidth - 4\tabcolsep) * \real{0.6625}}@{}}
\toprule()
\begin{minipage}[b]{\linewidth}\raggedright
Entitlement
\end{minipage} & \begin{minipage}[b]{\linewidth}\raggedright
Adult
\end{minipage} & \begin{minipage}[b]{\linewidth}\raggedright
Minor
\end{minipage} \\
\midrule()
\endhead
Vested & Entitlement can be transferred to them. & Property usually held
on trust until age of majority reached. May be possible to transfer
minor's beneficial entitlement if expressly authorised in the will, or
to parents/ guardians on the behalf of the minor. \\
Contingent & Property must be transferred to trustees to hold on their
behalf until contingency is satisfied. & Property held on trust until
age of majority reached and/or contingency satisfied. \\
\bottomrule()
\end{longtable}

\hypertarget{personal-property}{%
\paragraph{Personal Property}\label{personal-property}}

The PRs indicate that they no longer require property for administration
purposes when they pass title to it by means of an assent. Note that the
beneficiary's title to the property derives form the will. The assent
just gives effect to the gift by the PRs.

Company shares are transferred y share transfer form, unless held by a
nominee company (as is often the case).

\hypertarget{freehold-leasehold-property}{%
\paragraph{Freehold/ Leasehold
Property}\label{freehold-leasehold-property}}

Personal representatives vest the legal estate in land in the person
entitled (whether beneficially or as trustee) by means of an assent,
which will then become a document of title to the legal estate.

\begin{Shaded}
\begin{Highlighting}[]
\NormalTok{If PRs continue to hold the property in their changed capacity as trustees, PRs should \textquotesingle{}assent\textquotesingle{} to this to formally vest the legal estate in themselves as trustees. }
\end{Highlighting}
\end{Shaded}

\hypertarget{assent-formalities}{%
\paragraph{Assent Formalities}\label{assent-formalities}}

\begin{Shaded}
\begin{Highlighting}[]
\NormalTok{title: s 36(4) AEA 1925}
\NormalTok{An assent to the vesting of a legal estate shall be in writing, signed by the personal representative, and shall name the person in whose favour it is given and shall operate to vest in that person the legal estate to which it relates; and an assent not in writing or not in favour of a named person shall not be effectual to pass a legal estate.}
\end{Highlighting}
\end{Shaded}

The Land Registration Rules 2003 specify the form of an assent further.

Any person in whose favour the PRs make an assent or conveyance may
require notice of it to\\
be endorsed on the original grant of probate or administration. It is
good practice to make this endorsement routinely at the same time as
giving the assent.

If title to the land is registered, 2 options for the PRs:

\begin{itemize}
\tightlist
\item
  Apply to be registered as proprietor in place of the deceased.

  \begin{itemize}
  \tightlist
  \item
    PRs produce grant of representation when making application.
  \end{itemize}
\item
  Transfer the property by assent without being registered as
  proprietor.

  \begin{itemize}
  \tightlist
  \item
    Beneficiary must be given a certified copy of the grant of
    representation, to present with their application for registration.
  \end{itemize}
\end{itemize}

\hypertarget{estate-accounts}{%
\subsubsection{Estate Accounts}\label{estate-accounts}}

PRs must produce estate accounts for the residuary beneficiaries.
Purpose: to show all the assets of the estate, the payment of debts,
administration expenses and legacies and balance remaining for residuary
beneficiaries.

The residuary beneficiaries sign the accounts to indicate approval
-\textgreater{} releases PRs from further liability.

\end{document}
